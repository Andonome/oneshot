\renewcommand{\headingtype}{APPENDIX}

\titleformat{\chapter}[display]
{\bfseries}
{\begin{tikzpicture}
\node[minimum width=\textwidth, text=black!25, fill=black!25, inner sep=1, outer sep=0, anchor=south] (rectinit) {\huge CHAPTER};
\node[minimum width=.8\textwidth, text=white, inner sep=1, outer sep=0, anchor=south west, text width=.8\textwidth, align=right] at (rectinit.south west) (chapname) {\huge APPENDIX~~};
\node[minimum width=.2\textwidth, inner sep=0, outer sep=0, anchor=south west, text width=.2\textwidth, align=left] at (chapname.south east) {\chapnumfont\textcolor{chapnumcol}{\thechapter}};
\end{tikzpicture}}
{0pt}
{\Huge}

\chapter{Creatures \& Items}

\section{Portal Scrolls}

\begin{multicols}{2}

\begin{boxtext}

  As you pick up the scroll you suddenly sense a distant world.
  Little insects dance about walls with hard crystals protruding from them.
  You cannot see the area, instead you sense it, blindly.

  As you drop the scroll on the ground, the vision fades.

\end{boxtext}

These scrolls open a magical doorway to another world known as the Realm of Shifting Corridors, which people can sense as soon as they pick the scroll up.%
\footnote{Most portal scrolls are made with a Clairvoyance spell, but this one was made with a \textit{Ranged} Clairvoyance spell.}
The gnomes made these scrolls as escape routes in case of sudden danger, and as a way to gather the gems which spill from the walls of that realm.
All portal scrolls lead to places nearby each other in the Realm of Shifting Corridors, so if anyone is lost in that realm, they may be seen again the next time a scroll is cast.%
\footnote{See \textit{Adventures in Fenestra},%
\iftoggle{aif}{%
   page \pageref{shiftingcorridors}, for more on \nameref{shiftingcorridors}.%
}{%
  for more on this realm.%
}}

Any time a character grasps the scroll, they see some new scene from the other side.
If they speak the scroll's command word, a portal opens, and an encounter should be rolled to see what is happening on the other side.
Here are some pre-rolled encounters:

\needspace{4em}
\begin{itemize}

  \item{Encounter 1}
  \begin{itemize}
    \item{14 maze-dwarves spill out of the portal, fleeing the poisonous gas of a watcher.}
    \item{The dwarves have no languages in common with the PCs, but they will jump out, kill any nura in sight, then try to return to their labyrinthine realm with goblin heads as trophies.}
  \end{itemize}
  \item{Encounter 2}
  \begin{itemize}
    \item{A single dark corridor stretches out.  At the other end, rests a watchman, spilling poisonous gas slowly into the single, trapped area.}
    \item{Any PC caught here makes a Wits + Vigilance check, TN 10. If they pass, they still live the next time a portal scroll opens (having found their way to the next encounter). If they fail, they have died from the poisonous gas.}
  \end{itemize}
  \item{Encounter 3}
  \begin{itemize}
    \item{A long, dark corridor beckons. At the other end, 10 maze dwarves discuss what to do about the injured archmage in front of them. It has 5 MP left, and will retaliate if approached.}
    \item{Since these maze dwarves are trapped in their current room, they follow any PCs back through the dark corridor.  They will fight any nura present, before returning.}
    \item{Before returning, they gift the players the magical item: \lootMagic.
\iftoggle{aif}{\footnote{See Adventures in Fenestra, page \pageref{magicalitems} for details on magical items.}}{}}
  \end{itemize}
  \item{Encounter 4}
  \begin{itemize}
    \item{A wide corridor with nothing but an umber hulk opens. It crashes through and fights the first thing it can eat, before returning to its own realm.}
  \end{itemize}
\end{itemize}

\dwarvensoldier[\npc{\M\T}{Maze Dwarves}]

\umberhulk

\iftoggle{core}{%
  For rules on large battles between NPCs, see the core book, page \pageref{npcfights}.}{}

\end{multicols}

\section{Potential Creatures}

\begin{multicols}{2}

\label{undeadstats}

\npc{\D\N}{Gnomish Ghoul}
\animal{-2}% STRENGTH
{-1}% DEXTERITY 
{-1}% SPEED
{-2}% WITS
{2}% DR
{2}% AGGRESSION
{None}% SKILLS
{Undead}% ABILITIES
{}

\undeadgoblin[\npc{\D\N}{Goblin Ghoul}]

\ghoul[\npc{\D\N}{Human Ghoul}]

\iftoggle{oneshot}{}{
  \npc{\D\N}{Nura Woodspy Ghoul}
  \animal{4}% STRENGTH
  {-1}% DEXTERITY 
  {-1}% SPEED
  {-2}% WITS
  {4}% DR
  {2}% AGGRESSION
  {Vigilance 1}% SKILLS
  {Undead}% ABILITIES
  {}
}

\undeadogre[\npc{\D\N}{Ogre Ghoul}]

\end{multicols}

\chapter{Riddles}
\label{riddles}

\begin{multicols}{2}

The rules for riddles are simple -- any question which someone has the knowledge to answer is a fair riddle.
Asking `how many letters in the Greek word for ``mushroom''?', is not a fair riddle, because someone may not know.

Any possible answer to a riddle is `the correct one'.
If someone asks `what is black and white and read all over', anything which fits all descriptions must be accepted as an answer.

\ifnum\month<5
  \ifnum\month>3
    \begin{quotation}
    
    I have a little house in which I live all alone.
    It has no doors or windows, and if I want to go out I must break through the wall.
    What am I?
    \end{quotation}
  
    \textbf{Answer: A chick in an egg.}
  \fi

\fi

\ifnum\month>11

  \begin{quotation}

  I'm dry at night, but wet in day

  I tend to stop where I can't stay

  My speed goes up as I slow down

  I wear naught but a silver crown

  I don't rely, I don't depend

  But cannot grow without a friend

  I can't stand rain, but need the sky

  So I ask you: what am I?

  \end{quotation}

  \textbf{Answer: Snow.}

\fi


\ifnum\month=10

  \begin{quotation}
    I have a name but it isn't mine

    You don't think about me while in your prime

    People cry when I'm in their sight

    Others lie with me all day and night.

    What am I?
  \end{quotation}

  \textbf{Answer: a tombstone}

  \begin{quotation}
    I have a head, yet no face.

    I have a mouth, but don't say grace.

    I have two legs, but cannot walk.

    When I'm clean, I'm white as chalk.

    What am I?
  \end{quotation}

  \textbf{Answer: A skeleton.}
\fi

\ifnum\month>11

\begin{quotation}
I have a golden head. I have a golden tail. I have no body.
What am I?
\end{quotation}

\textbf{Answer: A gold coin.}


\begin{quotation}
I am not alive, but I grow: I don't have lungs, but I need air; I don't have a mouth, but water kills me.
What am I?
\end{quotation}

\textbf{Answer: Fire.}

\begin{quotation}
What kind of ball cannot bounce?
\end{quotation}

\textbf{Answer: A snowball.}

\fi

\begin{quotation}

  We have six legs, but only walk on four.
  What are we?

\end{quotation}

\textbf{Answer: A rider and their mount.}


\begin{quotation}
  The more I take the more I leave behind.
  What am I?
\end{quotation}

\textbf{Answer: Footsteps.}

\begin{quotation}
  David's father has three sons: Snap, Crackle, and  \ldots?
\end{quotation}

\textbf{Answer: David.}

\begin{quotation}
You must give me in order to keep me.
What am I? 
\end{quotation}

\textbf{Answer: Your word.}

\begin{quotation}

I can crush great boulders into fine sand,

But without me you will die as sure as you stand.

I rise when I'm cold and I soar when I'm hot,

Tell me what I am and what I am not.

\end{quotation}

\textbf{Answer: Water}

\begin{quotation}
I stand on one leg with my heart in my head.
What am I?
\end{quotation}

\textbf{Answer: A cabbage.}

\begin{quotation}
What comes once in a minute, twice in a moment, but never in a thousand years?
\end{quotation}

\textbf{Answer: The letter M.}

\begin{quotation}
Tall I am young, Short I am old, While with life I glow, Wind is my foe. What am I?
\end{quotation}

\textbf{Answer: A candle.}

\begin{quotation}
When one does not know what it is, then it is something. When one knows what it is, then it is nothing.
\end{quotation}

\textbf{Answer: A riddle.}

\end{multicols}

\iftoggle{oneshot}{}{

  \chapter{Handouts}
  \label{handouts}
  \settoggle{sideTab}{false}


  \includesvg{images/Dyson_Logos/upper-handout}

  \iftoggle{hardcore}{
    \includesvg{images/Dyson_Logos/town}
  }{}

}
