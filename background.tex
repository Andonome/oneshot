\chapter[How It Started]{How It Started\ldots}

\begin{center}
\begin{minipage}{0.7\linewidth}
Passages should lead somewhere.
We can almost be logically certain, but not entirely, because the definition admits of some wiggle room.
And in one place (not in the world, still a place), the passages themselves wriggle, very slowly, which stops them leading anywhere, because their movements soon change the landscape, so they have no permanent place to lead \emph{to}.

In this realm, portals to other realms often appear.
Perhaps someone does this for their amusement.
Perhaps the passages themselves just really feel they ought to go somewhere.

\bigLine
\end{minipage}

\end{center}
\vspace{1em}

\begin{multicols}{2}

\subsubsection{In the Goblin Realm}

\begin{exampletext}
  Goblins sing, eat, and occasionally fight.
  They mostly eat plants, because the plants in their caverns grow fast, making them plentiful.
  If you could see them growing, you would think you were watching a sped-up film, until a goblin walked by at a nearly-normal pace, and stripped the plant down to start cooking dinner.
  Not that goblins track mealtimes.
  They consider food regulation a form of torture from glum stories, and not something they would inflict on anyone.

  Their realm has myriad tunnels, leading to magical portals to other realms, so curiosity often drives a band of goblins to convince some ogres to join them to explore some new tunnel.
  On one of those expeditions, a large troupe found a land of endless tunnels where the walls move slowly, forming a slowly writing maze, and occasionally reveal shining gems.
  But goblins don't eat gems, so with the shifting tunnels blocking their route home, that troupe did not return in time for dinner\ldots
\end{exampletext}

\subsubsection{In the Gnomish Warren}

\begin{exampletext}
  Old Nanpa tells the younger ones that he once thought humans only existed in long-beard stories, and they listen closely, because they say Nanpa achieve enlightenment when his third nostril opened.

  ``It seemed implausible that real people might live above the ground all their lives.
  Even badgers and rabbits know enough to dig themselves in, since pulling dirt away makes excellent walls, while felling and constructing wood makes for shoddy walls, which need maintenance, and which any nasty creature might just climb over.''

  ``But once we confirmed humans exist'', Nanpa continued, ``we taught them alchemy, mostly to see if they could learn it''.

  ``And indeed, they can make alchemy spells almost as powerful as elven songs.
  Did I tell you about the time the human called Bilkhide opened a door to the moon?''
\end{exampletext}

\subsubsection{Bilkhide's Telescope}

Some spells can reach the horizon.
Others go farther -- as far as the eye can see.
So Bilkhide pulled out his telescope, and cast a portal spell, which destroyed the space between him and the moon.
He created the magical doorway which stands at the top of his tower.
Stepping through, he found an endless desert, plants made of crystal, and men made of stone.
And being a sensible person, he left immediately to write a various letters about his experience, and gave them to a passing trader a week later, to pass along the road.

Bilkhide's little bearded friends joined him soon after, their arms stuffed with the precious ingredients to make magical items.
They wanted to cheat the system, and make cheap magical scrolls which cast more spells to open portals to that distance realm.
So they created bog-standard a magical scroll which could open a portal only some steps away, and made their target that moon, which at the time they could clearly see only a few feet away.
They taught that scroll to keep its spell ever-ready, just waiting to be cast, on that exact location, then let the scroll sleep as they took it back to their warren.
This trick could only work with spell which made a mockery of distance.

The warren cast that scroll, deep underground, and built a stone arch to protect its boundaries.
Portals break easily when something touches their edges, or if they have no solid ground to stand on, or simply due to a windy day.

\subsubsection{Crystal Flowers}

The warren experienced decades of delight stealing crystal flowers, and using them to create further portal scrolls, with their own (very solid) moon-gate as the prime portal.
They also explored, mapping the strange realm of bright rocks, and the odd stone-people which defend gardens of the crystal flowers.
Gnomes cannot run very fast, but neither can stone men.
And if any of the gnomes felt endangered by the giants, they could instantly whip out a portal scroll, and open a passage home, then turn around and destroy the edges once safely back in the warren.
After many successful raids, none returned injured, and their safety felt assured.

Then one day, they found a portal which had nothing to do with them, just standing in the centre of a stone monolith.
It lead to a realm of misty passages full of fireflies, where the walls glittered with gems.
Pulling their old tricks, they chained more portals, and made plenty more portal scrolls which could take them to the dark passages.

\subsubsection{Here We Go\ldots}

\begin{exampletext}
  The goblins finally see strange creatures to hunt in the endless caves.
  They have all their fur on the front of their head, making them easy to grab, and the rest looks like tender meat.
  As the creatures yelp, another portal opens.
\end{exampletext}

\end{multicols}

