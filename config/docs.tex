\documentclass[a4paper,openany]{book}
\usepackage{layout}
\usepackage{stats}
\usepackage{monsters}
\usepackage{lipsum}

\date{\today}

\begin{document}

\chapter{How to Make Monsters}

\section{Introduction}

\begin{multicols}{2}

\subsection{All about Dragons}

\begin{boxtext}

As you embark upon your first adventure, you summon your first monster with a simple backstroke.
You write down {\tt\textbackslash dragon}, and behold!

\end{boxtext}

\dragon

Each time you conjure the dragon, it will look a little different.
The next one might look like this:

\dragon

If you find it hard to tell the difference between all the dragons, you can give them names in square brackets with the {\tt\textbackslash NPC command}, like this:

{\tt\textbackslash dragon[\textbackslash npc\{\textbackslash M\}\{Bob the dragon\}] }

Which then makes a male dragon called ``Bob'':

\dragon[\npc{\M}{Bob the Dragon}]

The first field can also indicate a female with an {\tt\textbackslash F} (\F), a team of people with a {\tt\textbackslash T} (\T), or undead with \textbackslash D (\D).

With a little study, you can summon dozens of monsters, including {\tt\textbackslash humansoldier}, {\tt\textbackslash basilisk}, and {\tt \textbackslash ghoul}.

\subsection{Individual NPCs}

Individual characters can be created by using the {\tt\textbackslash npc} command then the \textbackslash person command, with its nine arguments:

\begin{verbatim}

\npc{\M}{Alice}

\person{0}% STRENGTH
{1}% DEXTERITY 
{-1}% SPEED
{{2}% INTELLIGENCE
{0}% WITS
{0}}% CHARISMA
{0}% DR
{1}% COMBAT
{Academics 1, Survival 1}% SKILLS
{\longsword, adventuring equipment}% EQUIPMENT
{}

\end{verbatim}

\npc{\M}{Alice}
\person{0}% STRENGTH
{1}% DEXTERITY 
{-1}% SPEED
{{2}% INTELLIGENCE
{0}% WITS
{0}}% CHARISMA
{0}% DR
{1}% COMBAT
{Academics 1, Survival 1}% SKILLS
{\longsword, adventuring equipment}% EQUIPMENT
{}

You can add things for these people to say with a {\tt\textbackslash begin\{speechtext\}} command:

\begin{speechtext}

	``Would you tell me, please, which way I ought to go from here?''

	``That depends a good deal on where you want to get to.''

\end{speechtext}

\subsection{And now for something completely different}

This is a magical item.

\begin{verbatim}

\magicitem{Noodle of Death}% NAME
	{Extinguish}% SPELL
	{Divinity (FSM)}% PATH
	{Instant}% DURATION
	{Pocket Spell}% TYPE
	{2}% Potency
	{5}% MP

\end{verbatim}

\magicitem{Noodle of Death}% NAME
	{Extinguish}% SPELL
	{Divinity (FSM)}% PATH
	{Instant}% DURATION
	{Pocket Spell}% TYPE
	{2}% Potency
	{5}% MP

\subsection{Encounters}

Make encounter tables like this:

\begin{verbatim}

	\begin{encounters}{Wonderland}

		Fields & Gardens & Results \\\hline

		\li & Doormouse \\
		\li & Dodo \\
		\li \lii Unicorn \\
		\li \lii Red Queen \\
		& \lii Black Queen \\
		& \lii Green Queen \\

\end{verbatim}

\begin{encounters}{Wonderland}

	Fields & Gardens & Results \\\hline

	\li & Doormouse \\
	\li & Dodo \\
	\li \lii Unicorn \\
	\li \lii Red Queen \\
	& \lii Black Queen \\
	& \lii Green Queen \\


\end{encounters}

And charts about roll successes like this:

\begin{verbatim}


	\begin{rollchart}

		Roll & Result \\\hline

		12 & Success \\

		11 & Failure \\

	\end{rollchart}

\end{verbatim}

\begin{rollchart}

	Roll & Result \\\hline

	12 & Success \\

	11 & Failure \\

\end{rollchart}

\end{multicols}

\end{document}
