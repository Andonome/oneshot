\chapter{Introduction}

\begin{center}
\begin{boxtext}[width=0.6\linewidth]

\noindent
  You awaken in complete darkness.
  Fuzzy memories return of the goblins raiding your home village, eating live cows, live dogs, and live villagers.
  Despite being small, they were faster than any normal human.
  They knocked you out with a rock.
  You remember being forced to walk towards a mountain, with your hands tied up.
  Your hands are free now, but your head still hurts.

\end{boxtext}
\end{center}

\section{Overview}

\begin{multicols}{2}

\noindent
This short story servers as an introduction to Fenestra.
\iftoggle{hardcore}{%
  It takes 2-3 sessions to play through, and should provide a real challenge even to seasoned tabletop gamers.
}{
  \iftoggle{oneshot}{
    Players will guide their characters to freedom or death within a single session (and, if death, will have to take a new character to guide out).
  }{
    It takes 1-2 sessions to play through, and shouldn't be too challenging for new people.
}
}

\widePic{Roch_Hercka/transformation}

The PCs awaken to find themselves in a cell, deep underground.
Their last memories surface slowly as the players themselves start to make their characters.
Over the course of the adventure, they find out that the gnomes opened a portal to an astral plane where the nura seeped in.
Once inside the warren, the nura transformed the gnomes into more nura.

Transforming into a nura creature makes one tall, strong, fast, and incredibly stupid and hungry.
Deeply hungry, all the time.
Gnomes who turn into nura are known as `goblins', and humans turn into `ogres'.
The size differs, the essence is the same -- hunger.

Once out into the warren, the party may speak to a dragon who guards a hidden passage, grab magical scrolls made by the gnomes before they became so deformed, and avoid the traps, guards, and aberrations of the goblin lair.
They will meet other prisoners, and hear how the nura destroyed all the villages around, taking people back as captives to be eaten or turned into ogres.

\iftoggle{hardcore}{
  Once out, they find Darton town lies under siege, and the surrounding villages have been eaten.
  They will then need to journey out to find the Black Baron, deliver him a pardon for his criminal behaviour, and escort him safely back to town.
}{
  Their mission is simple -- to escape through the exit at the top, and return to civilization.
}

For more on nura and how they act, see \textit{Adventures in Fenestra}%
\iftoggle{aif}{%
  , \autoref{nura}}{}%
.
\subsection{History}
\label{invasionhistory}

\begin{exampletext}

\noindent
Ten years ago, the gnomes of the Whittling Warren made a magical portal to the Realm of Bright Rocks in order to start making excursions there, and steal the precious flowers which produce such excellent magical items.
\iftoggle{aif}{%
  \footnote{See Adventure in Fenestra, page \pageref{brightrocks} for details of this realm.}
}{}

\end{exampletext}

The deep creatures from the Realm of Darkness and Fire had their own portal in the Realm of Bright Rocks.
They don't usually journey to that domain much, as there is no food, and they do not like the Sun.
However, the goblins and ogres of that deep realm soon found the entrance to the gnomish warren.
They entered, grabbed the little gnomes, then ate them.
As they made their way up the warren, filling their stomachs, they started taking captives, and then used their Saurecanta magic to turn the gnomes into more nura.

Many nura died as they fell victim to the various traps the gnomes had laid, but there were always more nura coming to fight and die.
The nura-gnomes (`goblins') remade the warren into a war-ready stronghold.

At this point, a dragon from the Realm of Bright Rocks wandered into the portal.
The dragon -- `Makil' -- cannot fit through the small door to the treasure room it found, so he decided to just sleep there, waiting for someone small enough to help him.
The nura have had to leave the gold and precious magical items of this room well alone.

The next day, the nura went above-ground to raid various human villages close to the warren, and take those humans back to become food, or to be transformed into ogres.
The day after that, they used those ogres to bring back more people, completing their army.

\iftoggle{hardcore}{
  On the third day, they laid siege to the local town.
  They cannot get beyond the tall, stone walls, but that raiding party has plenty of food brought from the villages nearby.
}{}

\iftoggle{oneshot}{
  \subsection{Quick Rules}

  When the players want to do something, tell them the TN (`Target Number')%
  \footnote{6 = easy, 10 = professional, 14 = extreme}
  and have them roll 2D6 + some \textit{Attribute + Skill}.
}{}

\subsection{Magical Items}

The gnomes went to the otherworldly desert to grab ingredients for magical items, and they returned with lots.
As a result, the party can find many scrolls, talismans, and magical rings in the warren.

\subsubsection{Scrolls}

The gnomes did not want anyone to be able to operate the items, except other gnomes, so they wrote riddles on the scrolls, and made the activation word the answer to the riddle.
Some of those gnomes transformed into goblins and remember the passwords to the scrolls, but they have begun to forget them already.

If any of the PCs spend a Story Point to claim they have learnt the Gnomish language, they can guess a riddle in order to use the scroll.
Players must only state the word consciously in order to activate scrolls.
If they are in combat, this costs 8 Initiative, but if the scroll can still activate by accident if someone is holding it and saying the word.

\subsubsection{Talismans and Artefacts}

For all other items, the party can roll Intelligence + Academics to figure out how to activate them, TN 10.
  However, knowing what makes a magical item go, and knowing what it does are different things.
  Figuring out the function, rather than just the activation command, an Academics roll at TN 12.%
\iftoggle{core}{%
\footnote{See the core book, page \pageref{magicidentification} for more on identifying magical items.}
}{}

\end{multicols}

\section{Impending Scenes}

\begin{multicols}{2}

\noindent
These are the scenes which are coming for the PCs, whether they like it or not.
They are based more on time and circumstance than location.

The first introduces the characters, and the next introduces the adventure.
The third means their death approaches, and the fourth may be their salvation.

\widePic{Roch_Hercka/waking}

\subsection{Creation}

\begin{boxtext}

  You awake in darkness with ropes binding your hands behind your back.
  From the shifting and murmuring sounds around, you know others rest in the same room.

\end{boxtext}

\iftoggle{hardcore}{
  \sidebox{
    \begin{rollchart}
      4-6 & Nothing \\
      3   & Partial armour \\
      2   & Flint box \\
      1   & Knife    \\
    \end{rollchart}
  }
  Players should start the adventure first, and make characters second.
  They will awaken in a dark room, and slowly pull together their identity while introducing themselves to each other.

  Each character can also roll 1D6 to see if the goblins left them with any items.
}{
  Turn the character sheets from the handouts upside-down, then have players take one in any order.
  Give them a moment to study their characters while you hide the rest of the character sheets -- you will need them later.
}

\subsection{The Escape}
\label{escape}

Here the PCs awaken to their hopeless situation.
The players may make their characters at this juncture -- everyone's in the dark, so nobody knows who anyone is at this point.

Give the players a moment to get to know each other and their surroundings.
Remember, that everything is pitch black -- at best they can feel out the room a little, but they don't have long.

\paragraph{If anyone tries to wriggle free of their ropes,}
have them roll Dexterity + Larceny,
\iftoggle{hardcore}{%
  TN 11.
  Freeing another character requires a Dexterity + Crafts roll, TN 7, over the space of a round.
}{%
  TN 9.
  Freeing another character requires a full round.
Remember that if a player cannot succeed in a roll, it's possible for them to take a resting action (one die turns to 6 automatically\iftoggle{core}{, see the core rules, \autopageref{restingactions} for more}{}).
}

\paragraph{After a moment,}
a small voice emerges from the corner of the room.
`Chris', the seneschal (`accountant')
He explains to them that many prisoners waited here, until the shamans came to turn one into an ogre, who then ate the others, except for Chris, who hid under the debris.

\begin{boxtext}

  Heavy footsteps pad down the hall, you hear the door's bar being lifted, and a little goblinoid face peeps in with a torch and a massive sack over her shoulder.
  Behind her, an ogre stoops to the height of a man to avoid the low ceiling while she unties the sack.
  The contents are chucked into the room -- apples, pies, raw potatoes, and lettuce, all in a pile on the filthy floor.
  Chris shouts to the ogre `Alf -- don't do this! Help me!', but the ogre only stares back at him before the little goblin retreats, taking the massive, deformed man with him.

\end{boxtext}

Chris bleats for a moment before he crawls over and begins eating the food.
He knows what will happen to him, but he cannot stand to wait, starving in a cell, any more.

\ogre[\npc{\N\M}{Alf the Ogre}]
\label{alf}

\npc{\N\F}{Blara the Goblin}
\person{-1}% STRENGTH
{2}% DEXTERITY 
{2}% SPEED
{{0}% INTELLIGENCE
{0}% WITS
{-4}}% CHARISMA
{0}% DR
{1}% COMBAT
{Athletics 1, Deceit 1, Stealth 2, Tactics 1}% SKILLS
{Torch}% EQUIPMENT
{}

\paragraph{If anyone tries to fight Alf,}
he attacks them.

\iftoggle{oneshot}{
  \begin{itemize}

    \item
    \textit{To make an attack,} players roll 2D6 + Dexterity + Combat at a TN equal to the monster's `{\scshape Att}ack'.
    \item
    If they hit the monster, then they Damage the monster for 1D6 HP, plus their Strength Bonus.
    \item
    \emph{However}, if the player does not succeed in the roll, the enemy hits their character.
    \item
    If an enemy attacks a PC, the players makes the same roll, but at +1 to the TN.
    \item
    Everyone starts combat with some AP (action points) shown on their sheet and spend 1 AP each time they move, each time they attack, or are attacked.
    \arabic{spd}

  \end{itemize}

}{}

\paragraph{If anyone tries to fight Blara,}
she runs away with her torch (the room's only light source), leaving the party stranded in darkness.

Chasing after Blara requires a resisted roll of Speed + Athletics
\iftoggle{oneshot}{%
  (the player rolls 2D6 plus their Speed + Athletics against a TN of 7 + Blara's Speed (2) + Athletics (1) = 10)%
}{}.%
\iftoggle{core}{\footnote{See the core rules, \autopageref{resistedactions}.}}{}

\paragraph{If the PCs kill Chris immediately,}
he will be vulnerable, so they can make a sneak attack.
\iftoggle{core}{\footnote{\nameref{sneakattack}}}{}

\paragraph{If the party take time to escape,}
Chris has already reached his ogre form.

Remember also that if the party get a Vitals Shot in, they can deal Damage, rather than just inflicting Fatigue.

Once you resolve the scene, then check out the cell's description on page \pageref{entrycell}.

\npc{\N\M}{Chris (as an Ogre)}
\person{4}% STRENGTH
{0}% DEXTERITY 
{3}% SPEED
{{-3}% INTELLIGENCE
{-1}% WITS
{-3}}% CHARISMA
{0}% DR
{0}% COMBAT
{Academics 2, Empathy 1}% SKILLS
{Nothing}% EQUIPMENT
{}

\paragraph{If the party somehow stop Chris eating the tainted nura-food,}
he will accompany them out, but his nerves are too shot to be of any use.

\subsection{The Raiding Party Return}
\label{raidingParty}

This scene can take place at any point.
Any time the party decides to rest is a good time to liven things up with a returning party.

A band of nura have departed to grab prisoners from nearby villages.
Many died, but the remaining ogres have returned with 3 prisoners.
\iftoggle{hardcore}{
  5 goblins are at their side, ensuring the ogres don't do anything stupid.
}{}

They enter the warren and make their way down slowly.
These ogres have shed human blood, and have already forgotten about most of their original lives.
As they return and talk about how much they ate, the other ogres begin to accept their new lives, and accept that they must kill humans to live.
\iftoggle{hardcore}{%
From this point onwards, ogres will not hesitate to enter battle.
}{}

Since the PCs are outnumbered, they will most likely attempt to hide from the raiding party.

\ogre[\npc{\N}{2 ogres handling prisoners}]

\iftoggle{hardcore}{
  \ogre[\npc{\T\N}{2 Ogres}]

  \goblin[\npc{\T\N}{5 Goblins}]
}{}

\paragraph{If the PCs encounter the raiding party by the lift}
(room \ref{lift}), remember that the raiding party can only go down a couple of ogres at a time, due to the weight limit.%
\footnote{See page \pageref{lift}.}

The ten prisoners do not have the strength to fight, but will flee if pushed.

\subsection{Captured}

If the party ever lose a fight, do not push this until each lie dead.
Instead, when it becomes obvious they cannot win, have the nura draw back, mock the party, and then tell them to drop their weapons so a group of ogres can escort them down to their cells.

If any of the party have died, more prisoners come in soon, so another character can be rolled and added to the story from this new group once someone in the party spends the necessary Story Points.

Once this is done, the party can attempt to flee again.

\end{multicols}

\section{GM Considerations}

\begin{multicols}{2}

\subsection{Death}

If any character dies, another can be introduced once the party find any prison.
The nura regularly send out raiding parties to capture people, and those people get dumped back into the prisons, so even if the PCs have fled the prisons, any time they return, more prisoners can be found and liberated.

\iftoggle{hardcore}{
  Returning characters should start with the same XP total they have accumulated, rather than the usual, lower totals.
  If a player earns 10 XP during the mission and then dies, they return with 60 XP.
}{}

\subsection{Newly Fledged Nura}

The ogres here remember being human only hours or days ago.
They still haven't accepted their transformation fully, so many will not want to kill or eat humans unless they feel really hungry.

If the party encounter ogres, but have not yet attacked them, they can make a Charisma + Empathy roll (TN 10), to convince the ogres to not attack for a round.
The check has to be made each round if the players are in sight.
This trick can't last forever, but it \emph{can} buy them time to flee.

The gnomes have turned into goblins some time ago, and are quite beyond help.

\iftoggle{hardcore}{
  \subsection{Light}

  Keep careful track of the light sources -- they are rare and exceedingly valuable.
  If only a single PC has a light source, switch all narrative to that person's perspective -- after all, everyone else will be in the dark, so they can only focus on the light-bearer.

  Also, review the core book rules on fighting in the dark.%
  \iftoggle{core}{%
    \footnote{\nameref{darkness}, page \pageref{darkness}.}
  }{}

  \subsubsection{Candles}

  While common, these light sources go out easily.
  Any running will put a candle out, but dropping them will do nothing.
  Wax and mushroom-based candles lay in almost every room in the warren, though they sit unlit in empty rooms.

  \subsubsection{Torches}

  These far more practical light sources are held by most nuramancer goblins in the warren.
  Anyone with a torch can light up the entire room.
}{}

\subsection{Noise}

Nura fill the gnomish warren, and any loud sounds will summon them.
If the party make any noise, whether casting spells, fighting, or just casting spells, the next room hears them.
The next round whoever is in the next room comes to see what all the trouble is.

The ``next room'' is always the room with the next number, so if the party make a noise in room 3, then room 4 hears it, and if they make a noise in room 18, room 19 hears it, and so on.

If the people from room 4 arrive and decide they are outnumbered, and retreat, they go to get people in room 5, and so on.
Travel times to raise the alarm vary, but if the party make noise and don't manage to kill the nura or escape, they will find themselves outnumbered.

\iftoggle{oneshot}{}{
  \subsection{Sanctuaries}

  Throughout the warren, a number of rooms are out of bounds to the nura.
  These are \nameref{dragon} (page \pageref{dragon}), \nameref{fungusGarden} (page \pageref{fungusGarden}), and \nameref{lounge} (page \pageref{lounge}).
  The party can rest in these sanctuaries, but they may face ambushes later.

  Many nura will leave their prey alone once they cannot see them any more, but nuramancer goblins tend to be a little more cunning, and will organize other nura to wait for the party's return if they ever come back.

}

\end{multicols}


