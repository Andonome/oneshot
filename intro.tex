\chapter{Introduction}

\begin{center}
\begin{tcolorbox}[width=35em]

\noindent
	You awaken in complete darkness.
	Fuzzy memories return of the goblins raiding your home village, eating live cows, live dogs, and live villagers.
	Despite being small, they were faster than any normal human.
	They knocked you out with a rock.
	You remember being forced to walk towards a mountain, with your hands tied up.
	Your hands are free now, but your head still hurts.

\end{tcolorbox}
\end{center}

\section{Overview}

\begin{multicols}{2}

\noindent
The PCs awaken to find themselves in a cell, deep underground.
Their last memories surface slowly as the players themselves start to make their characters.

Over the course of the adventure, they find out that the gnomes opened a portal to an astral plane where the nura seeped in.
Once inside, the nura transformed the gnomes into more nura.
These little nura are known as `goblins'.

Transforming into a nura creature makes one tall, strong, fast, and incredibly stupid and hungry.
Deeply hungry, all the time.

When humans become nura, they become tall and fast;
people know them as `ogres'.

Once out into the dungeon, they may speak to a dragon who guards a hidden passage, grab magical scrolls made by the gnomes before they became so deformed, and avoid the traps, guards, and aberrations of the dungeon.
They will meet other prisoners, and hear how the nura destroyed all the villages around, taking people back as captives to be eaten or turned into ogres.

\iftoggle{hardcore}{
	Once out, they find the main town lies under siege, and the surrounding villages have been eaten.
	They will then need to journey out to find the Black Baron, deliver him a pardon for his criminal behaviour, and escort him safely back to town.
}{
	Their mission is simple -- to escape through the exit at the top, and return to civilization.
}

For more on nura and how they act, see \textit{Adventures in Fenestra}%
\iftoggle{aif}{%
	, chapter \ref{nura}}{}%
.

\end{multicols}

\section{Oneshot Rules}

\begin{multicols}{2}

\subsection{Creation}

Players should start the adventure first, and make characters second.
They will awaken in a dark room, and slowly pull together their identity while introducing themselves to each other.

The PCs will need to spend Story Points to navigate this dungeon, so remind them that they each have 5 Story Points which can be used to declare that they know a language, that they have known another PC, or to find a friend with some specialized skill.

\iftoggle{hardcore}{}{
Since this short adventure has little opportunity for growing skills, PCs should start with 70 XP.
}

Alongside the characters, James rests in the corner.
He exists to explain that there were more prisoners, and the nura took them all away.

\subsection{Death}

If any character dies, another can be introduced in any of the prisons.
The nura regularly send out raiding parties to capture people, and those people get dumped back into the prisons, so even if the PCs have fled the prisons, any time they return, more prisoners can be found and liberated.

Returning characters should start with the same XP total they have accumulated, rather than the usual, lower totals.
If a player earns 10 XP during the mission and then dies, they return with 80 XP.

\subsection{Newly Fledged Ogres}

The ogres here remember being human only hours or days ago.
They still haven't accepted their transformation fully, so many will not want to kill or eat humans unless they feel really hungry.

If the party encounter ogres, but have not yet attacked them, they can make a Charisma + Empathy roll (TN 10), to convince the ogres to not attack for a round.
The check has to be made each round if the players are in sight.

\subsection{Light}

Keep careful track of the light sources -- they are rare and exceedingly valuable.
If only a single PC has a light source, switch all narrative to that person's perspective -- after all, everyone else will be in the dark, so they can only focus on the light-bearer.

\subsubsection{Candles}

While common, these light sources go out easily.
Any running will put a candle out, but dropping them will do nothing.
Wax and mushroom-based candles lay in almost every room in the warren, though they sit unlit unless someone sits with them.

\subsubsection{Torches}

These far more practical light sources are held my most nuramancer goblins in the warren.
Anyone with a torch can light up the entire room.

\subsection{Narrow Tunnels}

Gnomes built these narrow hallways without regard for taller people.
As a result, any attacking with a weapon which requires 6 or more Initiative inflicts a penalty equal to the number of initiative points over 6 which it requires.
For example, a great sword requires 6 Initiative points, so it would take a -1 penalty while in a hallway.

Massive creatures, such as ogres, also receive a -1 penalty to all actions concerned with movement while in any hallway.
\iftoggle{core}{%
	See the core rules, page \pageref{enclosedcombat}, for more on fighting in enclosed spaces.
}{}

\subsection{Noise}

Nura fill the gnomish warren, and any loud sounds will summon them.
If the party make any noise, whether casting spells, fighting, or just casting spells, the next room hears them.
The next round whoever is in the next room comes to see what all the trouble is.

The ``next room'' is always the room with the next number, so if the party make a noise in room 3, then room 4 hears it, and if they make a noise in room 18, room 19 hears it, and so on.

If the people from room 4 arrive and decide they are outnumbered, and retreat, they go to get people in room 5, and so on.
Travel times to raise the alarm vary, but if the party make noise and don't manage to kill the nura or escape, they will find themselves outnumbered.

\subsection{Sanctuaries}

Throughout the warren, a number of rooms are out of bounds to the nura.
They don't want to disturb the local dragon, nor enter the area full of dangerous jellies.
The party can rest in these sanctuaries, but they may face ambushes later.

Most nura will leave their prey alone once they cannot see them any more, but nuramancer goblins tend to be a little more cunning, and will organize other nura to wait for the party's return if they ever come back.

\subsection{Magical Items}

The gnomes went to the otherworldly desert to grab ingredients for magical items, and they returned with lots.
As a result, the warren boasts many scrolls, talismans, and magical rings.

\subsubsection{Scrolls}

The gnomes did not want anyone to be able to operate the items, except other gnomes, so they wrote riddles on the scrolls, and made the activation word the answer to the riddle.
Some of those gnomes transformed into goblins and remember the passwords to the scrolls, but they have begun to forget them already.

If any of the PCs spend a Story Point to claim they have learnt the Gnomish language, they can guess a riddle in order to use the scroll.
If a PC (or indeed, a \textit{player}) actually states the word, then the scroll activates, regardless of the intention.

\subsubsection{Talismans and Artefacts}

For all other items, the party can roll Intelligence + Academics to figure out how to activate them.
\iftoggle{core}{%
	See the core book, page \pageref{magicidentification} for more on figuring out magical items.
}{
	However, knowing what makes a magical item go, and knowing what it does are different things.
	Figuring out the function, rather than just the activation command, requires a Margin of 2, meaning that the player must roll a 16.
}


\subsubsection{Portal Scrolls}

These scrolls open a doorway to another world known as the Realm of Shifting Corridors.
The gnomes made these scrolls as escape routes in case of sudden danger, and as a way to gather the gems which spill from the walls of this realm.
All portal scrolls lead to places nearby each other in the Realm of Shifting Corridors, so if anyone is lost in that realm, they may be seen again the next time a scroll is cast.
\iftoggle{aif}{%
	See \textit{Adventures in Fenestra}, page \pageref{shiftingcorridors} for more on this realm.
}{}

Any time a portal is opened, an encounter should be rolled to see what is happening on the other side.
Here are some pre-rolled encounters:

\begin{itemize}

	\item{Encounter 1}
	\begin{itemize}
		\item{14 maze-dwarves spill out of a room, fleeing the poisonous gas of a watcher.}
		\item{The dwarves have no languages in common with the PCs, but they will jump out, kill any nura in sight, then try to return to their labyrinthine realm with goblin heads as trophies.}
	\end{itemize}
	\item{Encounter 2}
	\begin{itemize}
		\item{A single dark corridor stretches out.  At the other end, rests a watchman, spilling poisonous gas slowly into the single, trapped area.}
		\item{Any PC caught here makes a Wits + Vigilance check, TN 10. If they pass, they still live the next time a portal scroll opens (having found their way to the next encounter). If they fail, they have died from the poisonous gas.}
	\end{itemize}
	\item{Encounter 3}
	\begin{itemize}
		\item{A long, dark corridor beckons. At the other end, 10 maze dwarves discuss what to do about the injured archmage in front of them. It was 5 MP left, and will retaliate if approached.}
		\item{Since these maze dwarves are trapped in their current room, they follow any PCs back through the dark corridor.  They will fight any nura present, before returning.}
		\item{Before returning, they gift the players one \lootMagic.
\iftoggle{aif}{\footnote{See Adventures in Fenestra, chapter \ref{magicalitems} for details on magical items.}}{}}
	\end{itemize}
	\item{Encounter 4}
	\begin{itemize}
		\item{A wide corridor with nothing but an umber hulk opens. It crashes through and fights the first thing it can eat, before returning to its own realm.}
	\end{itemize}
\end{itemize}

\dwarvensoldier[\npc{\M\T}{Maze Dwarves}]

\iftoggle{core}{%
	For rules on large battles between NPCs, see the core book, page \pageref{npcfights}.}{}

\umberhulk

\end{multicols}

\section{History}
\label{invasionhistory}

\begin{multicols}{2}

\noindent
Ten years ago, the gnomes of Badger's Warren made a magical portal to the Realm of Bright Rocks in order to start making excursions there, and steal the precious flowers which produce such excellent magical items.
\iftoggle{aif}{%
	\footnote{See Adventure in Fenestra, page \pageref{brightrocks} for details of this realm.}
}{}

The deep creatures from the Realm of Darkness and Fire had their own portal in the Realm of Bright Rocks.
They don't usually journey to that domain much, as there is no food, and they do not like the Sun.
However, the goblins and ogres of that deep realm soon found the entrance to the gnomish warren.
They entered, grabbed the little gnomes, then ate them.
As they made their way up the warren, filling their stomachs, they started taking captives, and then used their Saurecanta magic to turn the gnomes into more nura.

Many nura died as they fell victim to the various traps the gnomes had laid, but there were always more nura coming to fight and die.
The nura-gnomes (`goblins') and the old nura together remade the warren into a war-ready stronghold.

At this point, a dragon from the Realm of Bright Rocks wandered into the portal.
The dragon -- `Makil' -- cannot fit through the small door to the treasure room it found, so he decided to just sleep there, waiting for someone small enough to help him.
The nura have had to leave the gold and precious magical items of this room well alone.

The next day, the nura went above-ground to raid various human villages close to the warren, and take those humans back to become food, or to be transformed into ogres.
The day after that, they used those ogres to bring back more people, completing their army.

On the third day, they laid siege to the local town.
They cannot get beyond the tall, stone walls, but that raiding party has plenty of food brought from the villages nearby.

\end{multicols}

\section{Plans}

\begin{multicols}{2}

These are the scenes which play out without much reference to particular rooms.
The Escape plays out at the start, followed by optional scenes.

\subsection{The Escape}
\label{escape}

\begin{boxtext}

	You awake in darkness with ropes binding your hands behind your back.
	From the shifting and murmuring sounds around, you know others rest in the same room.

\end{boxtext}

Here the PCs awaken to their hopeless situation.
The players may make their characters at this juncture -- everyone's in the dark, so nobody knows who anyone is at this point.

Once your players have made their PCs, have everyone roll Dexterity + Larceny against TN 11 to escape.

\begin{boxtext}

	Heavy footsteps pad down the hall, you hear the door's bar being lifted, and a little goblinoid face peeps in, with a massive sack over her shoulder.
	Behind her, an ogre stoops to the height of a man to avoid the low ceiling while she unties the sack.
	The contents are chucked into the room -- apples, pies, raw potatoes, and lettuce, all in a pile on the filthy floor.
	James shouts to the ogre `Alf -- don't do this! Help me!', but the ogre only stares back at him before the little goblin shuts the door.

\end{boxtext}

James bleats for a moment before he crawls over and begins eating the food.
He knows what will happen to him, but he cannot stand to wait, starving in a cell, any more.

Resolve the scene, then check out the cell's description on page \pageref{entrycell}.

\paragraph{If the party kill James immediately}
then this is a good time to introduce the Sneak Attack rules.
James will be very vulnerable to attack in this state.
Remember also that if the party get a Vitals Shot in, they can deal Damage, rather than just inflicting Fatigue.

\goblinnuramancer[\npc{\F}{Blara the Goblin}]
\label{alf}

\ogre[\npc{\M}{Alf the Ogre}]

\subsection{The Raiding Party Return}
\label{raidingParty}

A full 20 ogres and 20 goblins have departed to grab prisoners from nearby villages.
Many died, but the remaining ogres have returned with 10 prisoners.
\iftoggle{hardcore}{
	5 of the goblins have come back -- many are still scouting outside.
}{}

They enter the warren and make their way down slowly.
These ogres have shed human blood, and have already forgotten about most of their original lives.
As they return and talk about how much they ate, the other ogres begin to accept their new lives, and accept that they must kill humans to live.
\iftoggle{hardcore}{%
From this point onwards, ogres will not hesitate to enter battle.
}{}

This scene can take place at any point.
Any time the party decides to rest is a good time to liven things up with a returning party.

Since the PCs are outnumbered, they will most likely attempt to hide from the raiding party.

\ogre[\npc{\T\N}{\arabic{enemyNo} ogres}]

\iftoggle{hardcore}{
\ogre[\npc{\T\N}{5 ogres handling prisoners}]

\goblin[\npc{\T\N}{5 goblins}]
}{}

The ten prisoners do not have the strength to fight, but will flee if pushed.

\subsection{Captured}

If the party ever lose a fight, do not push this until each lie dead.
Instead, when it becomes obvious they cannot win, have the nura draw back, mock the party, and then tell them to drop their weapons so a group of ogres can escort them down to their cells.

If any of the party have died, more prisoners come in soon, so another character can be rolled and added to the story from this new group once someone in the party spends the necessary Story Points.

Once this is done, the party can attempt to flee again.

\end{multicols}
