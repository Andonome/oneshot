\chapter{Introduction}

\epigraph{

  You awaken in complete darkness.
  Fuzzy memories return of the goblins raiding your home village, eating live cows, live dogs, and live villagers.
  Despite being small, they were faster than any normal human.
  They knocked you out with a rock.
  You remember being forced to walk towards a mountain, with your hands tied up.
  Your hands are free now, but your head still hurts.
}{}

\section{Overview}

\begin{multicols}{2}


\subsection{Who You Are (Hopefully)}

This RPG module was made for \glspl{gm} who want to run a game of BIND with new characters.
\iftoggle{hardcore}{%
  It takes 2-3 sessions to play through, and should provide a real challenge even to seasoned tabletop gamers.
  The \gls{gm} is assumed to be familiar with Fenestra's main elements, such as the nura, and the Night Guard.
}{%
  \iftoggle{oneshot}{%
    The module stands alone, with the rules introduce along the way.
    You will be the \gls{gm} for a group of 3-5 players who find themselves trapped in the bowels of a goblin warren, and must fight until they are free, or dead.

  You can pick up the basic rules in the introduction, and a few new rules lie scattered throughout the module, introduced with the `New Rule' header.

  \iftoggle{core}{
    Links to the complete rules are included in footnotes%
    \iftoggle{aif}{ along with a few footnotes for the campaign book, \textit{Adventures in Fenestra}.
    These are entirely optional reading.}{.}%.
  }{}
  }{%
    You will need 3-5 friends to join as players.
    
    While the story takes place in Fenestra, familiarity with the world should not be necessary.
  }
}

\widePic{Roch_Hercka/transformation}

\subsection{Overview}

The \glspl{pc} awaken to find themselves in a cell, deep underground.
Their last memories surface slowly as the players \iftoggle{hardcore}{themselves start to make their characters}{get to know their character}.
Over the course of the adventure, they find out that the gnomes opened a portal to a distant land where the goblins invaded, transforming the gnomes into more goblins.

Once out into the warren, the party may speak to a dragon who guards a hidden passage, grab magical scrolls made by the gnomes before they became so deformed, and avoid the traps, guards, and aberrations of the goblin lair.
They will meet other prisoners, and hear how the goblins destroyed all the villages around with their unending hunger.

\iftoggle{hardcore}{
  Once out, they find Darton town lies under siege, and the surrounding villages have been eaten.
  They will then need to journey out to find the Black Baron, deliver him a pardon for his criminal behaviour, and escort him safely back to town.
}{
  The \glspl{pc}' mission is simple -- to escape through the exit at the top, and return to civilization.
}


\subsubsection{History}
\label{invasionHistory}

\begin{exampletext}
Ten years ago, the gnomes of the Whittling Warren made a magical portal to the Realm of Bright Rocks in order to start making excursions there, and steal the precious flowers which produce excellent magical items.
\iftoggle{aif}{\footnote{See Adventure in Fenestra, \autopageref{brightrocks} for details on the Realm of Bright Rocks.}}{}%
They have made all manner of portal scrolls, which open gates to other far realms (and allow users to jump back in case of emergency), and even a magical lift to transport heavy loads up and down the warren.

The twist here comes with a goblin, who came through the portal from the other realm, and cast a spell to turn the gnomes into goblins.

Goblins are in fact simply gnomes cursed with nura magic, and once transformed they become stronger, faster, stupider, uglier, and very, \emph{very}, hungry.

They started by fighting back, but the goblin ran and hid, and waited for the curse to take its course.
The new goblins ate all the food in the warren, and eventually darted upstairs to eat the other gnomes.

Unable to defeat the caster, and fast losing their minds, they eventually agreed to raid human villages out of hunger.
In the villages, the original nuramancer-goblin started transforming humans.
\iftoggle{hardcore}{}{

  The process works about the same for humans, although we call them `ogres'.
  They grow tall, strong, ugly, and stupid.
  And of course, \emph{hungry}.

  Collectively, we call these creatures `nura',%
  \footnote{Pronounced `noo-rah'.}
  and their essence is always hunger.
}

Back at the warren, the new goblins had begun to forget where some of the traps they had known as gnomes lay, and blundered into them.
But they continued raiding, filling the areas with people to eat or transform into ogres, and planning their next raid.

\iftoggle{hardcore}{
  On the third day, they laid siege to the local town.
  They cannot get beyond the tall, stone walls, but that raiding party has plenty of food brought from the villages nearby.
}{}

\end{exampletext}

\end{multicols}

\section{\Glsfmttext{gm} Considerations}

\begin{multicols}{2}

\subsection{Handouts}

Have a look through the handouts.
The first page is
\iftoggle{oneshot}{
  \iftoggle{core}{a quick summary of the rules (you can ignore Character Creation), and the next is
  }{}
  a \gls{gm} sheet, for recording a few notes about the \glspl{pc} and taking general notes on upcoming encounters, or \glspl{npc} with the group.
}{
  a map of the upper dungeon level (which the \glspl{pc} may find inside the dungeon)%
  \iftoggle{hardcore}{%
    , followed by a map of Dayton town, the last surviving remnants of civilization in the area.
  }{.}
}

Next, six villagers are given as statblocks.
You should cut (or tear) these apart, so you can hand them to players to individually keep track of, in case those villagers join the troupe's fight for freedom.

\iftoggle{hardcore}{}{
  Lastly, you will find a slew of pre-made character sheets.

  \subsection{Understanding Boxtext}

  The boxtext is given as an example to jump-off.
  It show you how a room might appear, but it might not appear this way to your players.

  \begin{boxtext}
    You enter a room, candlelight flickers off the tiny, broken, beds.
  \end{boxtext}

  When you see this description of a room, your \glspl{pc} might not have a single candle, or might have three torches.
  It lays out a picture while reading this module for the first time, but should be modified or forgotten when running it live.

}

\subsection{Creation}

\iftoggle{hardcore}{
  \sidebox{
    \begin{rollchart}
      4-6 & Nothing \\
      3   & Partial armour \\
      2   & Flint box \\
      1   & Knife    \\
    \end{rollchart}
  }
  Players should start the adventure first, and make characters while they (the characters) sit in the dark, getting to know each other.
  They can decide who they are through introductions.
  Go through normal character creation,
  \exRef{core}{core rules}{character_rolls}
  but ignore the section on giving \glspl{pc} items, as the nura have taken most of their items already.
  Instead, each player should roll $1D6$ to see if the goblins left them with something by accident.
}{
  Shuffle the character sheets and hand each player a random one.
  Note any which have spells, and ask the players to put the right number of coins on the circles to keep track of their ability scores.
  Give the players a moment to study their characters while you hide the rest of the character sheets -- you will need them later.
}

\iftoggle{oneshot}{
  \subsection{Quick Rules}

  These rules will provide enough for basic actions.
  For anything else, just go with what seems appropriate and keep the ruling consistent.

  \subsubsection{Basic Actions}

  When the players want to do something, tell them the \gls{tn}%
  \footnote{A \glsentrytext{tn} of 6 = easy, 10 = professional, and 14 = extreme.}
  and have them roll 2D6 + some \textit{Attribute + Skill}.

  If they need to hide quickly, they might roll Wits + Stealth (\gls{tn} 8).
  If they need to bust open a door, they might roll Strength + Crafts (\gls{tn} 10).
  When fleeing from the enemy, they might roll Speed + Athletics (\gls{tn} 7 plus the enemy's Speed + Athletics).

  In all cases, you can make the number up depending on what seems right, but you should tell the players the \gls{tn} before the roll.
  If the nura are acting against them, you don't need to make another roll for the nura -- simply add the nura's Attribute + Skill pair to the roll.

  \subsubsection{Statblocks}
  Monster statblocks look like this:

  \npc{\N\F}{Ogre}
  \person{4}% STRENGTH
  {0}% DEXTERITY 
  {2}% SPEED
  {{-2}% INTELLIGENCE
  {-3}% WITS
  {-5}}% CHARISMA
  {0}% DR
  {1}% COMBAT
  {Crafts 1, Stealth 2, Tactics 1}% SKILLS
  {\Dagger}% EQUIPMENT
  {}

  The top rows are the \textit{Attributes}, and the Skills are marked below.
  Almost all rolls involve adding some Attribute + Skill pairing, and many gain an advantage for having a tool, such as a weapon.

  \subsubsection{Combat}

  \begin{itemize}

    \item
    \textit{To make an attack,} players roll 2D6 + Dexterity + Combat at a \gls{tn} equal to the \gls{npc}'s `{\scshape Att}'ack.
    \item
    If they hit the \gls{npc}, then they deal $1D6$ Damage, plus their Strength Bonus (mark off the \gls{npc}'s \glspl{hp} boxes with a pencil).
    \item
    \emph{However}, if the player does not succeed in the roll, the \gls{npc} removes the \gls{pc}'s \glspl{fp}.
  \begin{itemize}
    \item
    \Glspl{fp} represent a limited store of luck each character has.
    \item
    When a character has no more \glspl{fp}, they must mark off \glspl{hp}.
    \item
    For example, the ogre above deals $2D6+1$ damage.
    If she attacked a \gls{pc} and rolled a `9' for Damage, the player might mark off 5 \glspl{fp} then 4 \glspl{hp}.
  \end{itemize}
    \item
    The aggressor always has the upper hand, so if the \gls{npc} attacks a \gls{pc} (instead of a \gls{pc} attacking the \gls{npc}) then the \gls{npc} gains +1 to their {\scshape Att}.
  \begin{itemize}
    \item
    There is no difference between `attack' and `defence' -- the aggressor simply gains an advantage.
  \end{itemize}
    \item
    Everyone starts combat with some \glspl{ap} shown on their sheet and spends 1 \glspl{ap} each time they move, each time they attack, or are attacked.
  \begin{itemize}
    \item
    If there is any disagreement about who goes first, then whoever has the most \glspl{ap} goes first.
  \end{itemize}

  \end{itemize}

  When combat starts, look at the lower half of the statblock:

  \hrulefill

  {\scshape Att 10, \glspl{ap} 5, Dam $2D6+1$}

  10 \glspl{hp} {\large\Repeat{5}{\sqn}\Repeat{5}{\sqr}}

  Once all \glspl{ap} have been spent, the round ends, and everyone refreshes their \glspl{ap}.

}{}

\subsection{Newly Fledged Nura}

The ogres here remember being human only hours or days ago.
They still haven't accepted their transformation fully, so many will not want to kill or eat humans unless they feel really hungry.

If the party encounter ogres, but have not yet attacked them, they can make a Charisma + Empathy roll with a \gls{tn} of 10, to convince the ogres to not attack for a round.
The check has to be made each round if the players are in sight.
This trick can't last forever, but it \emph{can} buy them time to flee.

The gnomes have turned into goblins some time ago, and are quite beyond help.

\iftoggle{hardcore}{
  \subsection{Light}

  Keep careful track of the light sources -- they are rare and exceedingly valuable.
  If only a single \gls{pc} has a light source, switch all narrative to that person's perspective -- after all, everyone else will be in the dark, so they can only focus on the light-bearer.
  \exRef{core}{core rules}{darkness}

  \subsubsection{Candles}

  While common, these light sources go out easily.
  Any running will put a candle out, but dropping them will do nothing.
  Wax and mushroom-based candles lay in almost every room in the warren, though they sit unlit in empty rooms.

  \subsubsection{Torches}

  These far more practical light sources are held by most nuramancer goblins in the warren.
  Anyone with a torch can light up the entire room.
}{}

\subsection{Captured}

If the party ever lose a fight, do not push this until each lie dead.
Instead, when it becomes obvious they cannot win, have the nura draw back, mock the party, and then tell them to drop their weapons so a group of ogres can escort them down to their cells.

If any of the party have died, more prisoners come in soon, so another character can be rolled and added to the story from this new group once someone in the party spends the necessary Story Points.
Once this is done, the party can attempt to flee again.

\subsection{Death}

If any character dies, another can be introduced once the troupe reaches any prison.
The nura regularly send out raiding parties to capture people, and those people get dumped back into the prisons, so even if the \glspl{pc} have fled the prisons, any time they return, more prisoners can be found and liberated.

\iftoggle{hardcore}{
  Returning characters should start with the same \glspl{xp} total they have accumulated, rather than the usual, lower totals.
  If a player earns 10 \glspl{xp} during the mission and then dies, they return with 60 \glspl{xp}.
}{}

\end{multicols}
