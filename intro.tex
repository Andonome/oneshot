\chapter{Introduction}

\begin{center}
\begin{boxtext}[width=0.6\linewidth]

  \noindent
  You awaken in complete darkness.
  Fuzzy memories return of the goblins raiding your home village, eating live cows, live dogs, and live villagers.
  Despite being small, they were faster than any normal human.
  They knocked you out with a rock.
  You remember being forced to walk towards a mountain, with your hands tied up.
  Your hands are free now, but your head still hurts.

\end{boxtext}
\end{center}

\section{Overview}

\begin{multicols}{2}

\noindent
This short story servers as an introduction to Fenestra.
\iftoggle{hardcore}{%
  It takes 2-3 sessions to play through, and should provide a real challenge even to seasoned tabletop gamers.
  The GM is assumed to be familiar with Fenestra's main elements, such as the nura, and the night guard.
}{
  \iftoggle{oneshot}{
    Players will guide their characters to freedom or death within a single session (and, if death, will have to take a new character to guide out).
  }{
    It takes 1-2 sessions to play through, and shouldn't be too challenging for new people.

    The rules are given throughout the module -- a little here in the introduction, and a few scattered throughout the module as they come up.
    The main rules are summarized in the last page of the handouts.
  }
}

\widePic{Roch_Hercka/transformation}

The PCs awaken to find themselves in a cell, deep underground.
Their last memories surface slowly as the players \iftoggle{hardcore}{themselves start to make their characters}{\iftoggle{core}{get to know their character}{themselves start to make their characters}}.
Over the course of the adventure, they find out that the gnomes opened a portal to a distant land where the goblins invaded, transforming the gnomes into more goblins.

Once out into the warren, the party may speak to a dragon who guards a hidden passage, grab magical scrolls made by the gnomes before they became so deformed, and avoid the traps, guards, and aberrations of the goblin lair.
They will meet other prisoners, and hear how the goblins destroyed all the villages around with their unending hunger.

\iftoggle{hardcore}{
  Once out, they find Darton town lies under siege, and the surrounding villages have been eaten.
  They will then need to journey out to find the Black Baron, deliver him a pardon for his criminal behaviour, and escort him safely back to town.
}{
  The PCs' mission is simple -- to escape through the exit at the top, and return to civilization.
}


\subsection{History}
\label{invasionHistory}

\begin{exampletext}

\noindent
Ten years ago, the gnomes of the Whittling Warren made a magical portal to the Realm of Bright Rocks in order to start making excursions there, and steal the precious flowers which produce excellent magical items.
\iftoggle{aif}{\footnote{See Adventure in Fenestra, \autopageref{brightrocks} for details on the Realm of Bright Rocks.}}{}%
They have made all manner of portal scrolls, which open gates to other far realms (and allow users to jump back in case of emergency), and even a magical lift to transport heavy loads up and down the warren.

The twist here comes with a goblin, who came through the portal from the other realm, and cast a spell to turn the gnomes into goblins.

Goblins are in fact simply gnomes cursed with nura magic, and once transformed they become stronger, faster, stupider, uglier, and very, \emph{very}, hungry.

They started by fighting back, but the goblin ran and hid, and waited for the curse to take its course.
The new goblins ate all the food in the warren, and eventually darted upstairs to eat the other gnomes.

Unable to defeat the caster, and fast losing their minds, they eventually agreed to raid human villages out of hunger.
In the villages, the original nuramancer-goblin started transforming humans.
\iftoggle{hardcore}{}{

  The process works about the same for humans, although we call them `ogres'.
  They grow tall, strong, ugly, and stupid.
  And of course, \emph{hungry}.

  Collectively, we call these creatures `nura',%
  \footnote{Pronounced `noo-rah'.}
  and their essence is always hunger.
}

Back at the warren, the new goblins had begun to forget where some of the traps they had known as gnomes lay, and blundered into them.
But they continued raiding, filling the areas with people to eat or transform into ogres, and planning their next raid.

\iftoggle{hardcore}{
  On the third day, they laid siege to the local town.
  They cannot get beyond the tall, stone walls, but that raiding party has plenty of food brought from the villages nearby.
}{}

\end{exampletext}

\end{multicols}

\section{GM Considerations}

\begin{multicols}{2}

\iftoggle{hardcore}{
  \subsection{Handouts}

  The handouts contain a map which the PCs might be able to access, and an overview of the local city, which they can use for planning the final scene.
}{
  \iftoggle{core}{
    \subsection{Handouts}
    The handouts contain example characters which you can hand out to players.
    There should be more than enough, so if a player's character dies, they can get another one.

    You will also find a few statblocks for villagers who will later become captured and brought to the warren.
  }{}
}

\subsection{Understanding Boxtext}

The boxtext is given as an example to jump-off.
It show you how a room might appear, but it might not appear this way to your players.

\begin{boxtext}
  You enter a room, candlelight flickers off the tiny, broken, beds.
\end{boxtext}

When you see this description of a room, your PCs might not have a single candle, or might have three torches.
It lays out a picture while reading this module for the first time, but should be modified or forgotten when running it live.

\subsection{Newly Fledged Nura}

The ogres here remember being human only hours or days ago.
They still haven't accepted their transformation fully, so many will not want to kill or eat humans unless they feel really hungry.

If the party encounter ogres, but have not yet attacked them, they can make a Charisma + Empathy roll (TN 10), to convince the ogres to not attack for a round.
The check has to be made each round if the players are in sight.
This trick can't last forever, but it \emph{can} buy them time to flee.

The gnomes have turned into goblins some time ago, and are quite beyond help.

\subsection{Creation}

\iftoggle{hardcore}{
  \sidebox{
    \begin{rollchart}
      4-6 & Nothing \\
      3   & Partial armour \\
      2   & Flint box \\
      1   & Knife    \\
    \end{rollchart}
  }
  Players should start the adventure first, and make characters second.
  They will awaken in a dark room, and slowly pull together their identity while introducing themselves to each other.

  Each character can also roll 1D6 to see if the goblins left them with any items.
}{
  Turn the character sheets from the handouts upside-down, then have players take one in any order.
  Give them a moment to study their characters while you hide the rest of the character sheets -- you will need them later.
}

\iftoggle{oneshot}{
  \subsection{Quick Rules}

  When the players want to do something, tell them the TN (`Target Number')%
  \footnote{A TN of 6 = easy, 10 = professional, and 14 = extreme}
  and have them roll 2D6 + some \textit{Attribute + Skill}.

  If they need to hide quickly, they might roll Wits + Stealth (TN 8).
  If they need to bust open a door, they might roll Strength + Crafts (TN 10).

  In all cases, you can make the number up depending on what seems right, but you should tell the players the TN before the roll.

  \subsection{Combat}
  Monster statblocks look like this:

  \person{4}% STRENGTH
  {0}% DEXTERITY 
  {2}% SPEED
  {{-2}% INTELLIGENCE
  {-3}% WITS
  {-5}}% CHARISMA
  {0}% DR
  {1}% COMBAT
  {Crafts 1, Stealth 2, Tactics 1}% SKILLS
  {\Dagger}% EQUIPMENT
  {}

  When combat starts, look at the lower half of the statblock:

  \hrulefill

  {\scshape Att 10, AP 5, Dam $2D6+1$}

  10 HP {\large\Repeat{5}{\sqn}\Repeat{5}{\sqr}}

  \begin{itemize}

    \item
    \textit{To make an attack,} players roll 2D6 + Dexterity + Combat at a TN equal to the enemy's `{\scshape Att}'ack.
    \item
    If they hit the enemy, then they deal $1D6$ Damage, plus their Strength Bonus (mark off the enemy's HP boxes with a pencil).
    \item
    \emph{However}, if the player does not succeed in the roll, the enemy deals Damage to the character ($2D6+1$ in this case).
  \begin{itemize}
    \item
    PCs can mark off Fate Points (FP) instead of HP to avoid Damage.
  \end{itemize}
    \item
    If an enemy attacks a PC, the players makes the same roll, but at +1 to the TN.
    \item
    Everyone starts combat with some AP (action points) shown on their sheet and spend 1 AP each time they move, each time they attack, or are attacked.
    \arabic{spd}

  \end{itemize}

}{}

\iftoggle{hardcore}{
  \subsection{Light}

  Keep careful track of the light sources -- they are rare and exceedingly valuable.
  If only a single PC has a light source, switch all narrative to that person's perspective -- after all, everyone else will be in the dark, so they can only focus on the light-bearer.

  Also, review the core book rules on fighting in the dark.%
  \iftoggle{core}{%
    \footnote{\nameref{darkness}, page \pageref{darkness}.}
  }{}

  \subsubsection{Candles}

  While common, these light sources go out easily.
  Any running will put a candle out, but dropping them will do nothing.
  Wax and mushroom-based candles lay in almost every room in the warren, though they sit unlit in empty rooms.

  \subsubsection{Torches}

  These far more practical light sources are held by most nuramancer goblins in the warren.
  Anyone with a torch can light up the entire room.
}{}

\iftoggle{oneshot}{
  \subsection{Narrow Passages}

  Ogres cannot swing their big clubs indoors very easily, but they keep forgetting.
  Any time someone attempts to use a long weapon in one of the passageways (which were designed for gnomes) they suffer a penalty equal to the weapon's AP.

  Anything with a Weight Rating above 0 counts as a `long weapon'.
}

\iftoggle{oneshot}{}{
  \subsection{Sanctuaries}

  Throughout the warren, a number of rooms are out of bounds to the nura.
  These are \nameref{dragon} (page \pageref{dragon}), \nameref{fungusGarden} (page \pageref{fungusGarden}), and \nameref{lounge} (page \pageref{lounge}).
  The party can rest in these sanctuaries, but they may face ambushes later.

  Many nura will leave their prey alone once they cannot see them any more, but nuramancer goblins tend to be a little more cunning, and will organize other nura to wait for the party's return if they ever come back.

}

\subsection{Captured}

If the party ever lose a fight, do not push this until each lie dead.
Instead, when it becomes obvious they cannot win, have the nura draw back, mock the party, and then tell them to drop their weapons so a group of ogres can escort them down to their cells.

If any of the party have died, more prisoners come in soon, so another character can be rolled and added to the story from this new group once someone in the party spends the necessary Story Points.
Once this is done, the party can attempt to flee again.

\subsection{Death}

If any character dies, another can be introduced once the party find any prison.
The nura regularly send out raiding parties to capture people, and those people get dumped back into the prisons, so even if the PCs have fled the prisons, any time they return, more prisoners can be found and liberated.

\iftoggle{hardcore}{
  Returning characters should start with the same XP total they have accumulated, rather than the usual, lower totals.
  If a player earns 10 XP during the mission and then dies, they return with 60 XP.
}{}

\end{multicols}
