\chapter{Introduction}

\section{Overview}

\begin{multicols}{2}

\noindent
The PCs awaken to find themselves in prison, deep underground.
Their last memories of being taken from their home villages by goblins surface slowly.

Over the course of the adventure, they find out that the gnomes opened a portal to an astral plane where the nura seeped in.
Once inside, the nura transformed the gnomes into more nura.
These little nura are known as `goblins'.

Transforming into a nura creature makes one tall, strong, fast, and incredibly stupid and hungry.
Deeply hungry, all the time.

When humans become nura, they become tall and fast.
People know them as `ogres', but the basic idea remains the same.
For more on nura and how they act, see \textit{Adventures in Fenestra}.

Once out into the dungeon, they may speak to a dragon who guards a hidden passage, grab magical scrolls made by the gnomes before they became so deformed, and avoid the traps, guards, and aberrations of the dungeon.
They will meet other prisoners, and hear how the nura destroyed all the villages around, taking people back as captives to be eaten or turned into ogres.
Finally, they will hear that the local town lies under siege to a growing nura horde.

Their mission is simple -- to escape through the exit at the top.

\end{multicols}

\section{Oneshot Rules}

\begin{multicols}{2}

\subsection{Creation}

Players should start the adventure first, and make characters second.
They will awaken in a dark room, and slowly pull together their identity while introducing themselves to each other.

The PCs will need to spend Story Points to navigate this dungeon, so remind them that they each have 5 Story Points which can be used to declare that they know a language, that they have known another PC, or to find a friend with some specialized skill.

Since this short adventure has little opportunity for growing skills, PCs should start with 70 XP.

Alongside the characters, James rests in the corner.
He exists to explain that there were more prisoners, and the nura took them all away.

\subsection{Death}

If any character dies, another can be introduced in any of the prisons.
The nura regularly send out raiding parties to capture people, and those people get dumped back into the prisons, so even if the PCs have fled the prisons, any time they return, more prisoners can be found and liberated.

Returning characters should start with the same XP total they have accumulated, rather than the usual, lower totals.
If a player earns 10 XP during the mission and then dies, they return with 80 XP.

\subsection{Noise}

Nura fill the gnomish warren, and any loud sounds will summon them.
If the party make any noise, whether casting spells, fighting, or just casting spells, the next room hears them.
The next round whoever is in the next room comes to see what all the trouble is.

The ``next room'' is always the room with the next number, so if the party make a noise in room 3, then room 4 hears it, and if they make a noise in room 30, room 31 hears it, and so on.

If the people from room 4 arrive and decide they are outnumbered, and retreat, they go to get people in room 5, and so on.
Travel times to raise the alarm vary, but if the party make noise and don't manage to kill the nura who arrive in time, they will find themselves outnumbered.

\subsection{Sanctuaries}

Throughout the warren, a number of rooms are out of bounds to the nura.
They don't want to disturb the local dragon, nor enter the area full of dangerous jellies.
The party can rest in these sanctuaries, but they may face ambushes later.

Most nura will leave their prey alone once they cannot see them any more, but nuramancer goblins tend to be a little more cunning, and will organize other nura to wait for the party's return if they ever come back.

\subsection{Magical Items}

The gnomes went to the otherworldly desert to grab ingredients for magical items, and they returned with lots.
As a result, the warren boasts many scrolls, talismans, and magical rings.
However, the PCs will have a difficult time telling what the items do.

Anyone can roll Intelligence + Academics, and the basic TN to use a magical item is 14.
However, character gets a +4 bonus for knowing gnomish, a +1 bonus for knowing dwarvish,%
\footnote{The languages are very similar}
and a +4 bonus for having the \textit{Alchemy} specialization.

However, knowing what makes a magical item go, and knowing what it does are different things.
Figuring out the function, rather than just the activation command, requires a Margin of 4, meaning that the TN is 18.

\subsubsection{Portal Scrolls}

These items were used by the gnomes both as escape routes, and as a way to gather gems.
All portal scrolls lead to places nearby each other in the Realm of Shifting Corridors, so if anyone is lost in that realm, they may be seen again the next time a scroll is cast.

\end{multicols}

