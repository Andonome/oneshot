\chapter{Introduction}

\epigraph{

  You awaken in a dim room, with a fire burning at the far side.
  Fuzzy memories return of the goblins raiding your home village, eating live cows, live dogs, and live villagers.
  Despite being small, they were faster than any normal human.
  They knocked you out with a rock.
  You remember being forced to walk towards a mountain, with your hands tied up.
  Your hands remain tied, and your head hurts.
}{}

\section{Overview}

\begin{multicols}{2}

\subsection{Who You Are (Hopefully)}

This RPG module was made for \glspl{gm} who want to run a game of BIND with new characters.
\iftoggle{hardcore}{%
  It takes 2-3 sessions to play through, and should provide a real challenge even to seasoned tabletop gamers.
  The \gls{gm} is assumed to be familiar with Fenestra's main elements, such as the nura, and the Night Guard.
}{%
  \iftoggle{oneshot}{%
    The module stands alone, with the rules introduce along the way.
    You will be the \gls{gm} for a group of 3-5 players who find themselves trapped in the bowels of a goblin warren, and must fight until they are free, or dead.

  You can pick up the basic rules in the introduction, and a few new rules lie scattered throughout the module, introduced with the `New Rule' header.

  \iftoggle{core}{
    Links to the complete rules are included in footnotes%
    \iftoggle{aif}{ along with a few footnotes for the campaign book, \textit{Adventures in Fenestra}.
    These are entirely optional reading.}{.}%.
  }{}
  }{%
    You will need 3-5 friends to join as players.
    
    While the story takes place in Fenestra, familiarity with the world should not be necessary.
  }
}

%\widePic{Roch_Hercka/transformation}

\subsection{Overview}

The \glspl{pc} awaken to find themselves in a cell, deep underground.
Their last memories surface slowly as the players \iftoggle{hardcore}{themselves start to make their characters}{get to know their character}.

Once they escape, into the once-gnomish warren, the troupe may


\begin{itemize}
  \item
  speak to a dragon who cannot squeeze through the door to the treasure room,
  \item
  grab magical scrolls made by the gnomes before their defeat,
  \item
  avoid goblins raiding parties,
  \item
  dodge the traps left by the gnomes,
  \item
  find and free other prisoners,
  \item
  and perhaps even escape.
\end{itemize}

\iftoggle{hardcore}{
  Once out, they emerge into the hillside at night, without food or water.
  Once night comes, a light appears in the distance, which seems to signal for aid.
  If the \glspl{pc} investigate, they will find the black tower, where goblins have surrounded the proprietor, known as `the alchemist'.
}{
  The \glspl{pc}' mission is simple -- to escape through the exit at the top, and return to civilization.
}

\subsubsection{History}
\label{invasionHistory}

\begin{exampletext}

  When the gnomes heard the hum of mana in the \glspl{deep} below them, they did not fancy making the long journey.
  Instead, they cast a portal spell, to open a magical doorway \emph{down}, right next to the source of all that power.
  Stepping through, they found a plentiful garden of magical plants, tended by goblin druids.
  They picked a few choice plants, and left quietly, without suspecting that the druids had spotted them.

  The gnomes planned to create a series of portal-scroll pairs.
  One would create a portal to the goblin realm, and its twin would open a portal back to their home.

  The goblin druids, with some divination spells, predicted where the portal would appear, and amassed a host of goblins and ogres to defend themselves, and fight back against the strange invaders with hairy faces. 

  Neither plan worked very well.
  The gnomes' plan ended with the start of the goblins' plan.
  The goblins plan went sour once they realized that a) they had no way to return, and b) no source of food.

  The horde emerged from the warren, and journeyed to nearby human towns.
  They had a dozen reasons they could not simply ask for food, and two dozen ogres.

  The rest you might guess.

\end{exampletext}

\end{multicols}

\section{Preparations}

\begin{multicols}{2}

\subsection{Handouts}

Have a look through the handouts.
The first page is
\iftoggle{oneshot}{
  \iftoggle{core}{a quick summary of the rules (you can ignore Character Creation), and the next is
  }{}
  a \gls{gm} sheet, for recording a few notes about the \glspl{pc} and taking general notes on upcoming encounters, or \glspl{npc} with the group.

}{
  a map of the upper dungeon level (which the \glspl{pc} may find inside the dungeon).
}

Next, you will find six villagers statblocks.
You should cut (or tear) these apart, so you can hand them to players to individually keep track of, in case those villagers join the troupe's fight for freedom.

\iftoggle{hardcore}{}{
  Lastly, you will find a slew of pre-made character sheets.

  \subsection{Understanding Boxtext}

  The boxtext is given as an example to jump-off.
  It show you how a room might appear, but it might not appear this way to your players.

  \begin{boxtext}
    You enter a room, candlelight flickers off the tiny, broken, beds.
  \end{boxtext}

  When you see this description of a room, your \glspl{pc} might not have a single candle, or might have three torches.
  It lays out a picture while reading this module for the first time, but should be modified or forgotten when running it live.
}

\subsection{Creation}

\iftoggle{hardcore}{


  \sidebox{
    \begin{tabularx}{\linewidth}{c|L}
      \hline
      4-6 & Nothing \\
      3   & Partial leather armour \\
      2   & Flint box \\
      1   & Knife    \\
      \hline
    \end{tabularx}
  }

  Players should start the adventure first, and make characters while they (the characters) sit in the dark, getting to know each other.
  They can decide who they are through introductions.
  Go through normal character creation,
  \exRef{stories}{Book of Stories}{character_rolls}
  but ignore the section on giving \glspl{pc} items, as the nura have taken most of their items already.
  Instead, each player should roll $1D6$ to see if the goblins left them with something by accident.
}{
  Shuffle the character sheets and hand each player a random one.
  Note any which have spells, and ask the players to put the right number of coins on the circles to keep track of their ability scores.
  Give the players a moment to study their characters while you hide the rest of the character sheets -- you will need them later.
}

\iftoggle{oneshot}{
  \subsection{Quick Rules}

  These rules will provide enough for basic actions.
  For anything else, just go with what seems appropriate and keep the ruling consistent.

  \input{config/rules/attributes.tex}

  \input{config/rules/skills.tex}

  \subsubsection*{Fate}

  \input{config/rules/fate.tex}

  \subsubsection*{\Glsfmtplural{fatigue}}
  \input{config/rules/fatigue.tex}

  \input{config/rules/actions_basic.tex}

  \input{config/rules/actions_resisted.tex}

  \subsection*{Combat}

  \input{config/rules/hitting_things.tex}

  \paragraph{Monster Statblocks}
  look like this:

  \npc{\N\F}{Ogre}
  \person{4}% STRENGTH
  {0}% DEXTERITY 
  {2}% SPEED
  {{-2}% INTELLIGENCE
  {-3}% WITS
  {-5}}% CHARISMA
  {0}% DR
  {1}% COMBAT
  {Crafts 1, Stealth 2, Tactics 1}% SKILLS
  {\Dagger}% EQUIPMENT
  {}

  This ogre has \arabic{ap}~\glspl{ap}, and Attack \arabic{att}, so when she attacks, put a coin on your \gls{gm} sheet at `\arabic{ap}', and remove one for each attack.
  If a \gls{pc} attacks her, they roll $2D6$ plus their Attack, and try to beat her \gls{tn} of \arabic{att}.
  If the \gls{pc} rolls a tie, they can choose to both deal and take Damage,or neither.

}{}

\subsection{Light}

Keep careful track of the light sources -- they are rare and exceedingly valuable.
If only a single \gls{pc} has a light source, switch all narrative to that person's perspective -- after all, everyone else will be in the dark, so they can only focus on the light-bearer.
\exRef{core}{core rules}{darkness}

\subsubsection{Candles}

While common, these light sources go out easily.
Any running will put a candle out, but dropping them will do nothing.
Wax and mushroom-based candles lay in almost every room in the warren, though they sit unlit in empty rooms.

\subsubsection{Torches}

These far more practical light sources are held by most nuramancer goblins in the warren.
Anyone with a torch can light up the entire room.

\subsection{Captured}

If the party ever lose a fight, do not push this until each lie dead.
Instead, when it becomes obvious they cannot win, have the nura draw back, mock the party, and then tell them to drop their weapons so a group of ogres can escort them down to their cells.

If any of the party have died, more prisoners come in soon, so another character can be rolled and added to the story from this new group%
\iftoggle{stories}{once someone in the party spends the necessary \glspl{storypoint}}{}.
Once this is done, the party can attempt to flee again.

\subsection{Death}

If any character dies, another can be introduced once the troupe reaches any prison.
The nura regularly send out raiding parties to capture people, and those people get dumped back into the prisons, so even if the \glspl{pc} have fled the prisons, any time they return, more prisoners can be found and liberated.

\subsubsection{Annihilation}

Everyone in the party dying doesn't have to mean the game ends.
Just pull up some more characters, and have the new group awaken, ready for the goblins to eat them.

\end{multicols}
