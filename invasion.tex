\chapter{Introduction}

\epigraph{

  You awaken in a dim room, with a fire burning at the far side.
  Fuzzy memories return of the goblins raiding your home village, eating live cows, live dogs, and live villagers.
  Despite being small, they were faster than any normal human.
  They knocked you out with a rock.
  You remember being forced to walk towards a mountain, with your hands tied up.
  Your hands remain tied, and your head hurts.
}{}

\section{Overview}

\begin{multicols}{2}

\noindent
\iftoggle{hardcore}{%
  This module provides an introduction to your BIND campaign.
  If provides an opportunity for players to make characters, and then make some more once some characters die.

  It will take around two sessions to play through the entire module, and should provide a challenge even for seasoned gamers.
  Many scenes may come across as `unfair', but the area provides enough tools, talismans, and potential allies that the troupe should prevail as long as they have a little cunning and insight.
}{%
  This one-shot module for the BIND RPG provides a no-stress, simple game, which lasts a single evening.

  It goes like this:

  \begin{itemize}
    \item
    This introduction covers the basic rules in a couple of pages.
    \item
    More rules are covered in footnotes as they come up, so you can pick up a couple of extra resolution mechanics when you need them.
    \item
    Players receive their new characters, with all abilities written on the character sheets.
    \item
    These characters begin without any equipment or weapons, allowing players to pick up those extra rules, just when they need them.
    \item
    Goblins have captured various human prisoners, so once a \gls{pc} dies, the player can find a new character once the troupe liberates some prisoners, or finds themselves captured again.
  \end{itemize}
}

\subsection{Overview}

The \glspl{pc} awaken to find themselves in a cell, deep underground.
Their last memories surface slowly as the players \iftoggle{hardcore}{themselves start to make their characters}{get to know their character}.

Once they escape, into the once-gnomish warren, the troupe may

\begin{itemize}
  \item
  speak to a dragon who cannot squeeze through the door to the treasure room,
  \item
  grab magical scrolls made by the gnomes before their defeat,
  \item
  avoid goblins raiding parties,
  \item
  dodge the traps left by the gnomes,
  \item
  find and free other prisoners,
  \item
  and perhaps even escape.
\end{itemize}

\iftoggle{hardcore}{
  Once out, they emerge into the hillside at night, without food or water.
  Once night comes, a light appears in the distance, which seems to signal for aid.
  If the \glspl{pc} investigate, they will find the black tower, where goblins have surrounded the proprietor, known as `the alchemist'.
}{
  The \glspl{pc}' mission is simple -- to escape through the exit at the top, and return to civilization.
}

\subsubsection{History}
\label{invasionHistory}

\begin{exampletext}

  When the gnomes heard the hum of mana in the \glspl{deep} below them, they did not fancy making the long journey.
  Instead, they cast a portal spell, to open a magical doorway \emph{down}, right next to the source of all that power.
  Stepping through, they found a plentiful garden of magical plants, tended by goblin druids.
  They picked a few choice plants, and left quietly, without suspecting that the druids had spotted them.

  The gnomes planned to create a series of portal-scroll pairs.
  One would create a portal to the goblin realm, and its twin would open a portal back to their home.

  The goblin druids, with some divination spells, predicted where the portal would appear, and amassed a host of goblins and ogres to defend themselves, and fight back against the strange invaders with hairy faces. 

  Neither plan worked very well.
  The gnomes' plan ended with the start of the goblins' plan.
  The goblins plan went sour once they realized that a) they had no way to return, and b) no source of food.

  \subsubsection{The Goblin Tempo}
  Goblins works fast, play fast, eat fast and digest fast.
  This works out well for them, as the deep, \emph{deep} world where they live has lots of plants, which grow very fast.%
  \footnote{Alchemists theorize that goblins once lived on the surface of \gls{fenestra}.
  Their bleach-white skin cannot handle Sunlight for long, but they become very energetic around Sundown.
  Therefore, goblins gain energy from the Sun, like plants.
  Therefore, (the alchemists continue) goblins once looked like green plants, and stood about in the Sunlight.

  Theories like this are why nobody talks to alchemists in the pub.}


  Once the horde reached \gls{fenestra}'s surface, the world appeared as a barren landscape, nearly devoid of sustenance.
  They all needed food.
  They all journeyed to nearby human towns.
  They had a dozen reasons they could not simply ask for food, and two dozen ogres.

  The rest you might guess.

\end{exampletext}

\end{multicols}

\section{Preparations}

\begin{multicols}{2}

\subsection{Handouts}

Have a look through the handouts.
The first page is
\iftoggle{oneshot}{
  \iftoggle{core}{a quick summary of the rules (you can ignore Character Creation), and the next is
  }{}
  a \gls{gm} sheet, for recording a few notes about the \glspl{pc} and taking general notes on upcoming encounters, or \glspl{npc} with the group.

}{
  a map of the upper dungeon level (which the \glspl{pc} may find inside the dungeon).
}

Next, you will find six villagers statblocks.
You should cut (or tear) these apart, so you can hand them to players to individually keep track of, in case those villagers join the troupe's fight for freedom.

\iftoggle{hardcore}{}{
  Lastly, you will find a slew of pre-made character sheets.

  \subsection{Understanding Boxtext}

  The boxtext is given as an example to jump-off.
  It show you how a room might appear, but it might not appear this way to your players.

  \begin{boxtext}
    You enter a room, candlelight flickers off the tiny, broken, beds.
  \end{boxtext}

  When you see this description of a room, your \glspl{pc} might not have a single candle, or might have three torches.
  It lays out a picture while reading this module for the first time, but should be modified or forgotten when running it live.
}

\subsection{Creation \& Introductions}

\iftoggle{hardcore}{

  \sidebox{
    \begin{tabularx}{\linewidth}{c|L}
      \hline
      \textbf{Roll} & \textbf{Item} \\
      4-6 & Nothing \\
      3   & Partial leather armour \\
      2   & Flint box \\
      1   & Knife    \\
      \hline
    \end{tabularx}
  }

  Players should start the adventure first, and make characters while they (the characters) sit in the dark, getting to know each other.
  They can decide who they are through introductions.
  Go through normal character creation,
  \exRef{stories}{Book of Stories}{character_rolls}
  but ignore the section on giving \glspl{pc} items, as the goblins have taken most of their items already.
  Instead, each player should roll $1D6$ to see if the goblins left them with something by accident.
}{
  Shuffle the character sheets and hand each player a random one.
  Note any which have spells, and ask the players to put the right number of coins on the circles to keep track of their ability scores.
  Give the players a moment to study their characters while you hide the rest of the character sheets -- you will need them later.
}

\iftoggle{oneshot}{
  \subsection{Quick Rules}

  These rules will provide enough for basic actions.
  For anything else, just go with what seems appropriate and keep the ruling consistent.

  \input{config/rules/attributes.tex}

  \input{config/rules/skills.tex}

  \subsubsection*{Fate}

  \input{config/rules/fate.tex}

  \subsubsection*{\Glsfmtplural{fatigue}}
  \input{config/rules/fatigue.tex}

  \input{config/rules/actions_basic.tex}

  \input{config/rules/actions_resisted.tex}

  \subsection*{Combat}

  \input{config/rules/hitting_things.tex}

  \paragraph{Monster Statblocks}
  look like this:

  \npc{\N\F}{Ogre}
  \person{4}% STRENGTH
  {0}% DEXTERITY 
  {2}% SPEED
  {{-2}% INTELLIGENCE
  {-3}% WITS
  {-5}}% CHARISMA
  {0}% DR
  {1}% COMBAT
  {Crafts 1, Stealth 2, Tactics 1}% SKILLS
  {\Dagger}% EQUIPMENT
  {}

  This ogre has \arabic{ap}~\glspl{ap}, and Attack \arabic{att}, so when she attacks, put a coin on your \gls{gm} sheet at `\arabic{ap}', and remove one for each attack.
  If a \gls{pc} attacks her, they roll $2D6$ plus their Attack, and try to beat her \gls{tn} of \arabic{att}.
  If the \gls{pc} rolls a tie, they can choose to both deal and take Damage,or neither.

}{}

\subsection{Darkness \& Light}

Keep careful track of the light sources -- they are rare and exceedingly valuable.
If only a single \gls{pc} has a light source, switch all narrative to that person's perspective -- after all, everyone else will be in the dark, so they can only focus on the light-bearer.
\exRef{core}{Core Rules}{darkness}

\subsubsection{Candles}

While common, these light sources go out easily.
Any running will put a candle out, but dropping them will do nothing.
Wax and mushroom-based candles lay in almost every room in the warren, though they sit unlit in empty rooms.

\subsubsection{Torches}

These far more practical light sources are held by many goblins in the warren.
Anyone with a torch can light up the entire room.

\widePic{Roch_Hercka/transformation}

\subsection{Scrolls \& Riddles}
\label{scrollRiddles}

Everyone has their little bigotries, and gnomes are no exception.
When the gnomes of this \textit{Whittling Warren} make magical talismans (such as scrolls), they generally put a riddle on the front, where the answer will activate the talisman's ability.

This serves two functions:

\begin{enumerate}
  \item
  You don't have to write the activation word on the scroll, and risk it being picked up by a human.
  Even if a human grabbed it from you, they couldn't figure out the riddle (or so the gnomes presumed).
  \item
  You can write the activation word on the scroll, so nobody has to do a bunch of paperwork with `scroll number 28' to find out the activation word.%
  \footnote{Imagine finding a tin can without a label in your kitchen.
  Now imagine you feel really hungry, but you're unsure if the can has baked beans or live spiders.
  Finally, imagine you live with thirty other people, who are all scatter-brained, and intensely curious.
  In this situation, writing riddles on all future scrolls seems like a sensible decision.}
\end{enumerate}

The only restriction is that you need to keep the portals well stuffed in a bag.
The moment someone states the correct answer to the riddle, they activate.

\labyrinthScroll

\paragraph{Understanding the scrolls}
requires an \roll{Intelligence}{Academics} roll (\tn[12]) to identify what it targets (somewhere deep underground, with magical energy) and the same again (\tn[14]) to understand what it does (it opens a portal).

\label{readingScrolls}
The writing on the scroll is Gnomish, which forms the basis of the common Trading Tongue, so anyone with Academeics 1 or more can read Gnomish writing.

\paragraph{If goblins ever encounter an open portal to their realm,}
they jump through it immediately!
Their ultimate goal is to return home.

\subsection{Setbacks \& Death}

If any character dies, another can be introduced once the troupe reaches any prison.
The goblins regularly send out raiding parties to capture people, and those people get dumped back into the prisons, so even if the \glspl{pc} have fled the prisons, any time they return, more prisoners can be found and liberated.

\subsubsection{Captured}

If the party ever lose a fight, do not push this until each lie dead.
Instead, when it becomes obvious they cannot win, have the horde draw back, mock the party, and then tell them to drop their weapons so a group of ogres can escort them down to their cells.

\subsubsection{Annihilation}

Everyone in the party dying doesn't have to mean the game ends.
Just pull up some more characters, and have the new group awaken, ready for the goblins to eat them.
Perhaps Bellcut still waits for them, this time more cynical than ever!

\subsection{Gnomes \& Complications}
\label{saving_the_children}

As the goblin horde invaded, one gnome ran to save the children.
Before she could leave with them, she found herself trapped, so her next step was to put them all in a storage room, and lock the key.
She fell to the horde, shortly after.

The tiny gnomish children had a little food with them, but now starve in room \ref{nursery}.
She cast off her cloak, hat, and the storage key before dying, and it still lies in room \ref{slugHall}.

\subsubsection{The Spy}
\label{kalama}

One gnome -- a powerful skin-changer named Kalama -- managed to escape.
He used \textit{Life} spells to determine that gnomes still live within the warren, and decided to return.
He morphed his body to look like a goblin, then descended.

Unfortunately the spells make him extremely hungry.
His state of exhaustion leave him confused, and he knows that any spells he casts will make the situation much worse.

\end{multicols}
