\chapter{Introduction}

\epigraph{

  You awaken in a dim room, with a fire burning at the far side.
  Fuzzy memories return of the goblins raiding your home village, eating live cows, live dogs, and live villagers.
  Despite being small, they were faster than any normal human.
  They knocked you out with a rock.
  You remember being forced to walk towards a mountain, with your hands tied up.
  Your hands remain tied, and your head hurts.
}{}

\section{Overview}

\begin{multicols}{2}

\noindent
\iftoggle{hardcore}{%
  This module provides an introduction to your BIND campaign.
  It provides an opportunity for players to make characters, and then make some more once some characters die.

  It will take around two sessions to play through the entire module, and should provide a challenge even for seasoned gamers.
  Many scenes may come across as `unfair', but the area provides enough tools, \glspl{talisman}, and potential allies that the troupe should prevail as long as they have a little cunning and insight.
}{%
  This one-shot module for the BIND RPG provides a no-stress, simple game, which lasts a single evening.

  It goes like this:

  \begin{itemize}
    \item
    This introduction covers the basic rules in a couple of pages.
    \item
    More rules are covered in footnotes as they come up, so you can pick up a couple of extra resolution mechanics when you need them.
    \item
    Players receive their new characters, with all abilities written on the character sheets.
    \item
    These characters begin without any equipment or weapons, allowing players to pick up those extra rules, just when they need them.
    \item
    Goblins have captured human prisoners, so once \pgls{pc} dies, the player can find a new character once the troupe liberates some prisoners, or finds themselves captured again.
  \end{itemize}

  You should read through the module twice before running it.
  Some people read carefully, practising each boxtext fragment as they find it, then revise with a quick scan through the pages.
  Others flip through the pages randomly to check for interesting ideas before they commit to reading anything more than a sentence, then begin making notes in pen around the edges.

  However you read, this module's rooms don't stand in isolation.
  It has a handful of parts, like the scrolls, the dragon, and the goblinoid horde; and you should feel comfortable moving these people to a nearby room, or finding out what the \glspl{pc} get up to when they move alchemical scrolls and prisoners about.
}

\subsection{The Premise}

The \glspl{pc} awaken to find themselves in a cell, deep underground.
Their last memories surface slowly as the players \iftoggle{hardcore}{themselves start to make their characters}{get to know their character}.

Once they escape into the once-gnomish \gls{warren}, the troupe might:

\begin{itemize}
  \item
  speak to a dragon who cannot squeeze through the door to the treasure room,
  \item
  grab magical scrolls made by the gnomes before their defeat,
  \item
  avoid goblins raiding parties,
  \item
  dodge the traps left by the gnomes,
  \item
  find and free other prisoners,
  \item
  and perhaps even escape.
\end{itemize}

\iftoggle{hardcore}{
  Once out, they emerge into the hillside at night, without food or water.
  Once night comes, a light appears in the distance, which seems to signal for aid.
  If the \glspl{pc} investigate, they will find the black tower, where goblins have surrounded the proprietor, known as `\gls{alchemist}'.
}{
  The \glspl{pc}' mission is simple -- to escape through the exit at the top, and return to civilization.
}

\widePic{Roch_Hercka/transformation}

\subsection{History}
\label{invasionHistory}

\begin{exampletext}

  When the gnomes heard the hum of mana in the \glspl{deep} below them, they did not fancy making the long journey.
  Instead, they cast a gateway spell, to open a magical doorway \emph{down}, right next to the source of all that power.
  Stepping through, they found a plentiful garden of magical plants, tended by goblin druids.
  They picked a few choice plants, and left quietly, without suspecting that the druids had spotted them.

  The gnomes planned to create a series of \gls{deep}-scroll pairs.
  One would create a gateway to the goblin realm, and its twin would open a gateway back to their home.

  The goblin druids, with some divination spells, predicted where the gateway would appear, and amassed a host of goblins and \glspl{ogre} to defend themselves, and fight back against the strange invaders with hairy faces. 

  Neither plan worked.
  The gnomes' plan ended with the start of the goblins' plan.
  The goblins plan went sour once they realized that a) they had no way to return, and b) no source of food.

  \subsubsection{The Goblin Tempo}
  beats quicker than any other creature's.
  Goblins works fast, play fast, eat fast and digest fast.
  This works out well for them, as the deep, \emph{deep} world where they live has lots of plants, which grow so rapidly that you can see the change.%
  \footnote{Alchemists theorize that goblins once lived on the surface of \gls{fenestra}.
  Their bleach-white skin cannot handle Sunlight for long, but they become energetic around Sundown.
  Therefore, goblins gain energy from the Sun, like plants.
  Therefore, (the alchemists continue) goblins once looked like green plants, and stood about in the Sunlight.

  Pointless theories like this are why nobody talks to alchemists in the pub.}

  Once the horde reached \gls{fenestra}'s surface, the world appeared as a barren landscape, nearly devoid of sustenance.
  They all needed food.
  They all journeyed to nearby human towns.
  They had a dozen reasons they could not simply ask for food, and two dozen \glspl{ogre}.

  The rest you might guess.

\end{exampletext}

\widePic{Roch_Hercka/waking}

\subsubsection{Gnomes in the Cupboard}
\label{saving_the_children}

\begin{exampletext}
  As the goblin horde invaded, one gnome ran to save the children.
  Before she could leave with them, she found herself trapped, so her next step was to put them all in a storage room, and lock the door with a key.
  She fell to the horde, shortly after.

  The tiny gnomish children had a little food with them, but now starve in room \ref{nursery}.
  She cast off her cloak, hat, and the storage key before dying, and it still lies in room \ref{slugHall}.
\end{exampletext}

\subsubsection{The Spy}
\label{kalama}

\begin{exampletext}
  One gnome -- a powerful skin-changer named \gls{kalama} -- managed to escape.
  He used \textit{Life} spells to determine that gnomes still live within the \gls{warren}, and decided to return.
  He morphed his body to look like a goblin, then descended.

  Unfortunately the spells make him ravenous.
  His state of exhaustion leave him confused, and he knows that any spells he casts will make the situation much worse.
\end{exampletext}

\end{multicols}

\section{Preparations}

\begin{multicols}{2}

\subsection{Handouts}

Have a look through the handouts.
The first page is
\iftoggle{oneshot}{
  \pgls{gm} sheet, for recording notes about the \glspl{pc}, upcoming encounters, and \glspl{npc}.

}{
  a map of the upper dungeon level (which the \glspl{pc} may find inside the dungeon).
}

Next, you will find six farmer \glspl{statblock}.
You should cut (or tear) these apart, so you can hand them to players to individually keep track of, in case those villagers join the troupe's fight for freedom.

\iftoggle{hardcore}{}{
  Lastly, you will find a slew of pre-made character sheets.
  Go through the first couple of areas, and try a few practice rolls on your own.
  You'll quickly develop a feel for the \gls{attribute} plus \gls{skill} system.
}
\iftoggle{oneshot}{

  \subsection{Understanding Boxtext}

  The boxtext is given as an example to jump-off.
  It show you how a room might appear, but it might not appear this way to your players.

  \begin{boxtext}
    You enter a room, candlelight flickers off the child-sized, broken, beds.
  \end{boxtext}

  When you see this description of a room, your \glspl{pc} might not have a single candle, or might have three torches.
  It lays out a picture while reading this module for the first time, but should be modified or forgotten when running it live.
}{}

\subsection{Creation \& Introductions}

\iftoggle{hardcore}{

  \sidebox{
    \begin{dlist}
      \ifnumcomp{\value{temperature}}{=}{0}{\item Warm clothing (\glsfmttext{weight} 2)}{}
      \item
      Partial leather armour
      \item
      Flint box
      \item
      Knife
    \end{dlist}
  }

  Players should start the adventure first, and make characters while they (the characters) sit in the dark, getting to know each other.
  They can decide who they are through introductions.
  Go through normal character creation,
  \exRef{stories}{Book of Stories}{character_rolls}
  but ignore the section on giving \glspl{pc} items, as the goblins have taken most of their items already.
  Instead, each player should roll $1D6$ to see if the goblins left them with something by accident.
}{
  Shuffle the character sheets and hand each player a random one.
  Give the players a moment to study their characters while you hide the rest of the character sheets -- you will need them later.
  Ask them to put the right number of coins on the circles to keep track of their \glspl{hp}, \glspl{fp}, and so on.

  Character sheets with spells on the back should also use coins to mark 3~\glspl{mp} per elemental Sphere listed.
}

\iftoggle{oneshot}{
  \subsection{Quick Rules}

  \paragraph{Monster Statblocks}
  look like this:

  \goblincaster[\npc{\F\N}{Goblin Druid}]

  This goblin has \arabic{ap}~\glspl{ap}, and Attack \arabic{toHit}, so when she attacks, put a coin on your \gls{gm} sheet at `\arabic{ap}', and remove one for each attack.
  If \pgls{pc} attacks her, they roll $2D6$ plus their Attack, and try to beat her \gls{tn} of \arabic{toHit}.
  If the \gls{pc} rolls a tie, they can choose to both deal and take Damage, or neither.

  You won't normally need to track \gls{weight}-carried for \glspl{npc}, but they appear just-in-case, as a half-filled \gls{hp}-marker.
  In this case, the dagger has \gls{weight}~1.
  Creatures can also carry an effective \gls{weight} due to their thick skin, or from \glspl{ep}.

  \paragraph{Spells}
  explain themselves, but do pay attention to the `\textit{Roll}' and `\textit{Resistance by}' text.
  The `roll' indicates the spellcaster's Bonus, while you set the \gls{tn} depending on what makes the spell easier or harder to cast.
  But if you see `resisted by', then this spell might target \pgls{pc} directly, so the player would roll against the \gls{tn} beside the spell.

  These spells look like this:

  \begin{exampletext}
    \showStdSpells
  \end{exampletext}

  \Glspl{talisman} work the same way, showing either how \pgls{pc} resists, or just showing their Bonus to cast.
}{}

\subsection{Darkness \& Light}

Keep careful track of the light sources -- they are rare and valuable.
If only a single \gls{pc} has a light source, switch all narrative to that person's perspective -- after all, everyone else will be in the dark, so they can only focus on the light-bearer.%
\exRef{core}{Core Rules}{darkness}

\subsubsection{Candles}
are common, these light sources go out easily.
Any running will put a candle out, but dropping them will do nothing.
Wax and mushroom-based candles lay in almost every room in the \gls{warren}, though they sit unlit in empty rooms.

\subsubsection{Torches}
can light up an entire room, although characters will struggle to put them out quickly.
You don't want to put out a torch with your hands!

\subsubsection{Scrolls \& Riddles}
\label{scrollRiddles}
litter the \gls{warren}.
Everyone has their little bigotries, and gnomes are no exception.
When the gnomes of this \gls{warren} make magical \glspl{talisman} (such as scrolls), they generally put a riddle on the front, where the answer will activate the \gls{talisman}'s ability.

This serves two functions:

\begin{enumerate}
  \item
  You don't have to write the activation word on the scroll, and risk it being picked up by a human.
  Even if a human grabbed it from you, they couldn't figure out the riddle (or so the gnomes presumed).
  \item
  You can write the activation word on the scroll, so nobody has to do a bunch of paperwork with `scroll number 28' to find out the activation word.%
  \footnote{Imagine finding a tin can without a label in your kitchen.
  Now imagine you feel really hungry, but you're unsure if the can has baked beans or live spiders.
  Finally, imagine you live with thirty other people, who are all scatter-brained, intensely curious, and secretive.
  In this situation, writing riddles on all future scrolls seems like a sensible decision.}
\end{enumerate}

The only restriction is that you need to keep the \Gls{deep}~Scrolls well stuffed in a bag.
The moment someone states the correct answer to the riddle within `earshot', the scrolls activate.%
\footnote{Naturally, scrolls cannot actually hear, since they have no ears, nor anything to process sound with.  They only seem to have good hearing because they can lip-read well.}

\labyrinthScroll

\showTalisman

\begin{boxtext}
  The scroll burns itself to cinders with a flash!
  The ashes and flame spread across the hallway, blocking the hall with a patch of darkness which shows a distant orange glow.

  You can see the silhouettes of some scene on the other side, but can't understand if those twisted shapes come from plants, rocks, or something else.
  Whatever the source, the shapes sit somewhere along a distant horizon, in some faraway darkness.
  The empty gash in the world must lead somewhere \emph{other}.
\end{boxtext}

\paragraph{Understanding the scrolls}
requires an \roll{Intelligence}{Academics} roll (\tn[12]) to identify what it targets (somewhere deep underground, with magical energy) and the same again (\tn[14]) to understand what it does (it opens a gateway).

\label{readingScrolls}
The writing on the scroll is Gnomish, which forms the basis of the common \gls{tradeTongue}, so anyone with Academics~1 or more can read Gnomish writing.

\paragraph{If goblins ever encounter an open gateway to their realm,}
they jump through it immediately!
Their ultimate goal is to return home.

\paragraph{If anyone goes through the gateway,}
\iftoggle{hardcore}%
  {they are lost forever}%
  {they will return the next time \pgls{deep} scroll opens, with a strange story of a magma-filled landscape.

  Have the player mark $1D6$ \glspl{ep} on the character sheet.}

\subsection{Setbacks \& Death}

If any character dies, you can introduce another once the troupe reaches any prison (rooms \ref{entrycell} and \ref{secondPrison}).
The goblins regularly send out raiding parties to capture people, then return and dump them into the prisons, so even if the \glspl{pc} have passed a prison, any time they return, they can find and liberate more prisoners.

\subsubsection{Captured}
\glspl{pc} are better-off than dead \glspl{pc}.
So if the party ever lose a fight, do not push this until each lie dead.
Instead, when it becomes obvious they cannot win, have the horde draw back, mock the party, and then tell them to drop their weapons so a group of \glspl{ogre} can escort them down to their cells.

\subsubsection{Annihilation}
doesn't have to mean the game ends.
If you have at least a couple of hours more time for the game's night, just pull up some more characters, and have the new group awaken, ready for the goblins to eat them.
Perhaps \gls{sadMan} still waits for them, this time more cynical than ever!

\end{multicols}
