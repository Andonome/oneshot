\chapter{The Siege}
\epigraph{
  As you cross the next hill, you see the stone walls of Darton town ahead, and the smoke coming up from the hundreds of chimneys within.
  Climbing higher, you can see ladders poking at the town's tall walls.
  A man runs along the edge, pushing them aside.

  You climb higher, and see Darton surrounded by energetic, deformed, creatures of all sizes.
  The nura have multiplied more than you thought.
  They have laid siege to the entire town.
}{}

\begin{multicols}{2}

\noindent
The party probably thought they were out of the woods, but you can continue the adventure with the siege.
Darton has been completely surrounded by the nura, but a single outlaw alchemist can help.
If they can find him, and convince him to return to the town, he may be able to turn the tides on the siege.

\end{multicols}

\section{Journey to the Black Alchemist}

\begin{multicols}{2}

\subsection{The Last Messenger}

\begin{exampletext}

  Laith and his men were charged with escaping the city, and delivering a message to `Baron Quenn' (as he likes to call himself), otherwise known as the `Black Alchemist'.
  The message would pardon him of all wrongdoings, and recognize his current lands as an official domain, if only he approaches the town to save it.

  Unfortunately, Laith and his men were then attacked by dozens of goblins.
  All died, except a heavily wounded Laith, who is bleeding heavily.

\end{exampletext}

\begin{boxtext}

  You see a man wandering towards you, wearing thick, black, leather armour.
  He seems to be a member of the Night Guard.
  As he stumbles closer, you notice the trickle of blood leaking a river behind him.

\end{boxtext}

The moment Laith approaches, he explains his mission, and gives the party directions.
\iftoggle{aif}{%
  See \textit{Adventures in Fenestra}, \autoref{nightguard} for more on the Night Guard.
}{}

\begin{speechtext}

  You must walk East until the Sun sets, find the place in the forest where the trees neither grow, nor die, then scout for the Black Alchemist who lives there in his tower, and hand him this message.
  Darton needs him.
  He is our only hope.

\end{speechtext}

\humansoldier[\npc{\M}{Laith the Scout}]

\paragraph{At this point,}
the party spot a goblin sniffing along the stream of blood Laith left behind, far in the distance.
He shrieks in delight, then runs back.

\paragraph{If the party leave Laith behind immediately,}
they can run to safety, but will still hear Laith being torn apart.

\paragraph{If they wait to speak with him,}
a horde of goblins and ogres run fast behind them, but they should get the opportunity to ask any questions they wish.

\paragraph{If they insist on saving Laith,}
they will have to carry him, with all the encumbrance that entails.

\paragraph{If the party end up fighting with Laith,}
he has only 1 \glspl{hp} left, but can still make one last stand.

\paragraph{When the party flee,}
they will find that the nura stop to feast on Laith before following, so they have plenty of time to get away.

\paragraph{If the players decide they'd rather go directly to the town,}
skip ahead to \autoref{siege}: \nameref{siege}.
They will find the entire situation hopeless, and will not survive without a lot of sneaking.
Make sure to impress upon them just how many nura wander around the outskirts freely.

\iftoggle{hardcore}{%
  \deepogre[\npc{\N\T[2]}{2 Ogres}]
}{
  \deepogre[\npc{\N\M}{Ogre}]
}

\iftoggle{hardcore}{%
  \goblin
}{
  \goblin[\npc{\N\T[1]}{6 Goblins}]
}

\subsection{Finding the Tower}

\begin{boxtext}

  \noindent
  You walk East for hour upon hour.
  Civilized roads are far behind you -- you can see only the heavy foliage of the forest.

\end{boxtext}

\noindent
The tower is 15 miles away, which is a reasonable walk for most humans, but any non-humans in the group will need a rest before that.
Have the players roll Intelligence + Survival, TN 11.
Each failure margin adds another 2 miles to the journey.%
\exRef{core}{the core rules}{fatigue}

\subsection{The Chitincrawler}

En route to the tower, the party run into a chitincrawler's web.
Have them roll Wits + Vigilance, TN 10.
This is a group roll, so everyone who fails has been caught in the web.

\chitincrawler

\subsection{Footprints}

Have the party make a Teamwork Roll of Wits + Survival, TN 10.
Success indicates that one has spotted nura footprints nearby.

\begin{boxtext}

  The dark forest has no roads, but despite this you notice the smallest hint of a trampled path.
  Wandering over, you find two distinct types of footprints -- one very large, and another very small.

\end{boxtext}

The characters guess at least a dozen goblins, and half a dozen ogres, given the footprints.

\end{multicols}

\section{The Black Tower}

\begin{multicols}{2}

\begin{exampletext}

  The Black Tower began as a lonely outpost deep in wild lands.
  It was planned as the first part of a new place for civilization, but the plan was interrupted once the Rex of Fenestra banned townmasters from keeping their own guards, leaving only the Night Guard to defend against monsters.

  When the local lawmakers outlawed Theodore Quenn for learning Necromancy, and for refusing to reply to various summons from the College of Alchemy, he ventured into the wilderness and made his home here in the tower.
  Since then, the local declared theoretical war on him, and he declared himself theoretically `baron' of his own domain, and everyone in practice left each other well-alone.
  After all, `the Black Baron', can summon enough raging fire to wipe out an army, and he managed to explain this politely to any of the Night Guard who came to his gates, before offering his sympathy with their position.

  Since then, things have been good for Baron Quenn, until today.
  Nura invaded his home, so he burned them to death them with alchemical fire.
  Then more came and killed his servant, so now he finds himself stuck on his own roof with three apples and yesterday's pie.

\end{exampletext}

\smolMapPic{Dyson_Logos/black_tower_base}{
  \ref{towerBridge}/24/45,
  \ref{towerPortcullis}/46/66,
}

\subsection{The Approach}

\begin{boxtext}

  Ahead, a wide, raging river stands in front of the black tower, which is in fact a grey colour typical of the slate in the area.

\end{boxtext}

\mapentry[towerBridge]{The Bridge}

The bridge is a simple stone construction, which stays put.
Below, it is supported by wood, and can be taken apart with a Strength + Crafts roll, TN 12.
Proper tools, such as a saw, can add up to a +4 Bonus.

\mapentry[towerPortcullis]{The Portcullis}

\begin{boxtext}

  Approaching the portcullis, two ogres inside race forward, then pull it up with their hands, keeping eye-contact with you the whole time.

\end{boxtext}

\noindent
By the time the party stand 10 steps away, the ogres start to pull the gates up.
It takes the ogres 8 Initiative points to raise the portcullis.
\iftoggle{core}{%
\footnote{See the core book, page \pageref{movement} for rules on maximum movement.}
}{}

\paragraph{If someone in the group kills an ogre,}
they drop the portcullis.

\paragraph{Raising the portcullis}
requires a total of 6 Strength points, so the party will have a difficult time doing this on their own.
However, if one ogres remains alive inside, tricking it into opening the gate will not pose much difficulty.

\paragraph{If anyone ends up in the river,}
have them roll Speed + Athletics, TN 9.
Failure indicates they have been washed downstream, and must take \pgls{interval} trudging back.
\iftoggle{core}{%
\footnote{Larger party members carrying small ones will gain the standard penalties for carrying a heavy item (or in this case, `party member').  See the core book, page \pageref{weightrating}.}
}{}

\ogre[\npc{\N\T[2]}{2 Ogres}]

\smolMapPic{Dyson_Logos/black_tower_f1}{
  \ref{towerPortcullis}/23/44,
  \ref{exStable}/42/62,
}

\mapentry[exStable]{The Stable}

\begin{boxtext}

  Opening the door, you find a room covered in straw.
  At the far side lies a horse-corpse, covered in goblins, swarming like maggots.

\end{boxtext}

\paragraph{If the party have any sense,}
they will close the door.

If they draw attention to themselves, the goblins pick up their weapons, then attack.

\goblin

\mapentry[towerArmoury]{The Armoury}

\begin{boxtext}

  Outside, a full moon glimpses in through the arrow slits in the walls.
  Goblins have clearly vandalized this armoury, but the armour itself remains undamaged.
  To your left, a spiral staircase leads upward.

  Down the hall, the tower continues round.
  Human feet can be seen lying there.

\end{boxtext}

\smolMapPic{Dyson_Logos/black_tower_f2}{
  \ref{towerArmoury}/59/64,
  \ref{towerMechanism}/25/43,
}

The party can pick up three suits of chainmail for anyone with Strength 2-3, and three longswords.

The human feet across the room belonged to Baron Quenn's servant, who kept the place tidy, and helped with the portcullis.
He has been eaten, and little remains except his boots and bones.

\mapentry[towerMechanism]{The Portcullis Mechanism}

Here, the cogs which control the portcullises sit unused.
They require only a Strength Bonus of +1 to operate.

\mapentry[towerKitchen]{Kitchen}

\begin{boxtext}

  Traipsing up the steps, you heard the sound of angry mastication.
  At the top, you see an open door, with an ogre inside a kitchen, pulling down fruits with one hand and salted meats with the other.

  He looks up at you, but does not stop eating.

\end{boxtext}

\paragraph{If the party look formidable but unthreatening,}
the ogre leaves them be.
Otherwise, he attacks.

\ogre[\npc{\N\M}{Hungry Ogre}]

\smolMapPic{Dyson_Logos/black_tower_f3}{
  \ref{towerServants}/25/47,
  \ref{towerKitchen}/25/75,
  \ref{towerStorage}/70/44,
}

\mapentry[towerServants]{Servant's Room}

\begin{boxtext}

  A high-pitched voice gives a lecture in the common tongue, but it's so fast the contents make little sense.

\end{boxtext}

\paragraph{If the party have not opened the door, but merely listen,}
have them make a Wits + Vigilance check.
Success indicates that they can pass unnoticed, or stay and listen.

Understanding the conversation from outside the room requires a Wits + Empathy check, TN 9.
If the roll succeeds, tell the players that the Black Alchemist is hiding on the roof, and the head goblin is having a hard time explaining that if anyone goes up there, they will have their head incinerated by magic.

\goblinnuramancer

\goblin[\npc{\N\T[6]}{6 Goblins}]

\mapentry[towerStorage]{Storage Room}

This room contains spare clothing and all manner of long-life food.
The nura have not managed to figure out the lock yet.
It has no key -- instead, there is a hole with a series of ropes inside which one must lift in the correct way.
An Intelligence + Larceny check (TN 10) is required to figure out how to open it.

\mapentry[towerLibrary]{Library}

\begin{boxtext}

  Stellar maps cover the walls, full of numbers etched into the side.
  The desks have etches of Sunlit deserts, with strange geometrical shapes.
  Scrolls have calculations concerning abstract coordinates, such as `area A', with no mention of what that area may be.

\end{boxtext}

\smolMapPic{Dyson_Logos/black_tower_f4}{
  \ref{towerLibrary}/59/46,
  \ref{towerBaron}/25/72,
}

\noindent
Players may spot at this point that the maps pertain to the Realm of Bright Rocks lying on a table.
What they may not spot, is a little goblin who came up to look for food, who ended up hiding behind that table, in order to avoid being caught by the party.

\paragraph{If the players succeed on a Wits + Vigilance roll}
(TN 10), they find the hiding goblin.
If not, the last one to exit the room has a nasty surprise in store as the goblin attacks from behind, prompting a Sneak Attack.%
\iftoggle{core}{%
\footnote{See the core rules, page \pageref{sneakattack}.}%
}{}

\goblin[\npc{\N\F}{Hiding Goblin}]

\paragraph{Once the \glspl{pc} enter}
they notice a trapdoor in the roof.

If any goblins enter, they will be fried by the Black Alchemist, but the players only need to shout up in a friendly voice, and say they mean no harm to come up without being incinerated.

\mapentry[towerBaron]{Baron Quenn's Room}

The baron's room is a mess, but one made of fine quilts and quality pillows.
Anyone familiar with gnomish culture can roll Wits + Crafts (TN 8) to notice that the bedding and most of the room's contents is Gnomish in origin.

The quilts have come through gnomish contacts, but the Baron will not wish to discuss this, as he is aware that gifting the gnomes with knowledge of the Realm of Bright Rocks lead to the current predicament.

Around the room are:

\begin{enumerate}

  \item{A jar with the head of a Nura Chitincrawler. It once had 12 \glspl{hp}, so it would be worth a lot to priests of Ohta.}
  \item{A rare green gem, worth 40 \glspl{gp}.}
  \item{The preserved eyeballs of Darren the seer. He had 20 \glspl{fp}, so this could be worth a lot to followers of Ohta.}
  \item
  \lootMagic hidden in a drawer.

\end{enumerate}

\mapentry[towerRoof]{The Roof}

Here, the Black Alchemist has stayed safe from the nura horde by first taking the ladder up, and secondly by blasting any nura who come up in the face with magical fire.

\paragraph{Once the party enter,}
the Baron allows them to come up, greets them politely, and listens to anything they have to say.

\smolMapPic{Dyson_Logos/black_tower_f5}{
  \ref{towerStorage}/60/54,
}

\paragraph{If any party members have died,}
a new \gls{pc} can be found here.
This might be someone in the employ of the Barron, or a random person who was in the deep forest and ran away from the nura, then took shelter in the Barron's keep.

\paragraph{Once the party give over the letter,}
the Barron reads it slowly, and agrees to return with the party.

\paragraph{If the party asks why such a powerful alchemist requires escorts,}
he explains that while he could defeat an entire army, a single goblin javelin could still kill him.
He will require the party not only to guard him on the way back, but especially to guard him at the siege while he prepares his spells.

\npc{\M}{Baron Quenn}
\person{1}% STRENGTH
{0}% DEXTERITY 
{0}% SPEED
{{3}% INTELLIGENCE
{1}% WITS
{0}}% CHARISMA
{0}% DR
{1}% COMBAT
{Projectiles~1, Academics~2, Crafts~1, Deceit~1, Medicine~2, Vigilance 1
\knacks{\alchemist, \bloodCaster}
\Path{\force~3, \enchantment~2, \invocation~4, \necromancy~2}
}% SKILLS
{\shortsword, 3 daggers, \partialleather}% EQUIPMENT
{\addtocounter{fp}{10}\lockedmana{1}}

\end{multicols}

\section{The Journey Back}
\label{siege}

\begin{multicols}{2}

\noindent
Keep a tally of Fatigue, and have the players make proper decisions.
If they decide to return home immediately, they have one encounter with a basilisk.
If they wish to rest and recover, throw in another encounter.

\subsection{Discovering the Theft}

If the players have stolen anything precious from Baron Quenn, he might notice if they do not hide it well enough.
Have the player roll their Intelligence + Larceny, TN 9.
Failure indicates that while rummaging through a bag to take out food or equipment, they have revealed the stolen item for a moment.

If the Baron sees a stolen item, he will remember, and attempt to kill the character later.
Make a note of his intentions.
Once pardoned, he will launch an official complaint, and if nobody arrests the character and returns the item, he will swear an oath of vengeance.

\subsection{The Basilisk}

\begin{boxtext}

  A nasty stench approaches as you hear branches and trees being pushed aside and flattened.

\end{boxtext}

The party will notice the basilisk coming a mile away, and have time to prepare.
This gives a great opportunity to see what the Black Alchemist can do, as he rains hell-fire on the basilisk, with a \textit{\textbf{Raging}, Raging Fireball} for $2D6+3$ Damage.

\basilisk

If seriously damaged, the basilisk retreats.

\subsection{The Goblins}

By this point, there is a good chance the party have become too tired to move, and will need to rest.
The journey back poses no navigation problems, as Baron Quenn knows exactly how to return to Darton.
However, the party will still have to decide during each slice of the day if they wish to rest or continue marching.
\paragraph{If they decide to recover Fatigue Points by resting,}
they will need to put up with the additional encounter of goblins riding nura wolves.

\goblin[\npc{\N\T[7]}{7 Goblin Riders}]

\nurawolf[\npc{\N\A\T[7]}{7 Nura Wolves}]

If, on the other hand, they do not rest, they will not recover Fatigue Points from the journey.

\pic{Decky/shaman2}

\end{multicols}

\section{The Siege Breaks}

\begin{multicols}{2}

\noindent
The party must break the siege, by allowing the Black Barron to cast a spell over a massive section of the nura horde surrounding it.

Specifically, the baron plans to cast a \textit{Massive, Raging Fireball}, which will cover 5 areas around the town, inflicting $2D6+1$ Damage on everything inside.
These areas may be any areas around the town which he can see, so he can simply cast fire over any half of the town which he can see.

Alternatively, he might cast a \textit{Ranged, Massive Fireball}, but this would inflict only $1D6 + 3$ Damage on everyone inside.

This final part of the adventure leaves everything open to player planning and plotting.
They have time to deliberate, but not too long.
You may wish to set a ten minute timer, and tell them that nightfall comes once the timer ends (of course, this is artificial `game-time', as Sunset does not necessarily come in ten minutes).

\paragraph{At the start of the encounter,}
show the players the town map and explain their options.%
\footnote{See \autoref{handouts}, for the map handouts.}

The party have a number of questions to answer before they begin this exercise:

\begin{itemize}

  \item{Where do they begin the spell, such that the Barron can see everything he needs to, but will not be overrun by nura?}
  \item{If surviving nura rush towards them, what will they do?}
  \item{Do they want to orchestrate something inside the town walls to coincide with the attack?}

\end{itemize}

\paragraph{If the party scout around the area,}
have them make a Group Roll of Dexterity + Stealth, TN 9.

    \paragraph{Failure}
    means the scout is caught, and will be eaten, publicly, within about ten minutes (the nura want to start a fire and gather some spices).

\setcounter{list}{-1}
\begin{list}{\addtocounter{list}{1}\textbf{Margin \arabic{list}}:}{\raggedleft}

  \item{only allows them to return safely with the knowledge that Darton town is indeed surrounded.}
  \item{tells them that the majority of the nura horde around them are ogres.}
  \item{tells them that the Southern portion of the wall has fewer nura around it.}
  \item{tells the group that a number of goblins go diving around the East section of town, and seem to have found some secret entrance to town.}
\end{list}

\paragraph{If the party wait until nightfall,}
the goblins enter town through a sewage system which leads to the river at the Eastern side.
Darton's night guard spends the rest of the night dealing with these dozens of goblins, and by the morning the drawbridge has been lowered, ogres enter, and the town is doomed.

\paragraph{If the party get caught wandering,}
then some ogres will chase after them.
The Barron will probably be able to save them with his magic, but will afterwards be depleted, forcing them to hide and rest.

\ogre

\end{multicols}

\section{The End}

\begin{multicols}{2}

\noindent
If the party are successful, the Night Guard in the town rush out and chase away the remaining nura.
They want to make a good show of bravery so nobody calls them cowards.
If the party left the nura portal open, the location will forever have a nura blight around it, but nobody could really blame them.%
\iftoggle{aif}{%
  \footnote{See page \pageref{blight} in Adventures in Fenestra.}
}

The townsfolk then exit and give three cheers to Baron Quenn for saving the day.
A local crier also requests thanks to the party for bringing him, but by this point the townsfolk will have started dispersing back to their villages to see if anything remains of them.

A lone man in a black uniform wanders out to ask if they would like to join the Night Guard.

\end{multicols}
