\chapter{The Siege}

\begin{center}
\begin{tcolorbox}[width=35em]

	As you cross the next hill, you see the stone walls of the town ahead, and the smoke coming up from the hundreds of chimneys within.
	Climbing higher, you can see ladders poking at the town's tall walls.
	A man runs along the edge, pushing them aside.

	You climb higher, and see the town surrounded by energetic, deformed, creatures of all sizes.
	The nura have multiplied more than you thought.
	They have laid siege to the entire town.

\end{tcolorbox}
\end{center}

\section{Journey to the Black Alchemist}

\begin{multicols}{2}

\noindent
The party probably thought they were out of the water, but you can continue the adventure with the siege.

\subsection{The Last Messenger}

\begin{exampletext}

	Laith and his men were charged with escaping the city, and delivering a message to `Baron Quenn' (as he likes to call himself), otherwise known as the `Black Alchemist'.
	The message would pardon him of all wrongdoings, and recognize his current lands as an official domain, if only he approaches the town to save it.

	Unfortunately, Laith and his men were then attacked by dozens of goblins.
	All died, except a heavily wounded Laith, who is bleeding heavily.

\end{exampletext}

\begin{boxtext}

	You see a man wandering towards you, wearing thick, black, leather armour.
	He seems to be a member of the Night Guard.
	As he stumbles closer, you notice the trickle of blood leaking a river behind him.

\end{boxtext}

The moment Laith approaches, he explains his mission, and gives the party directions.

\begin{speechtext}

	You must walk East until the Sun sets, find the place in the forest where the trees neither grow, nor die, then scout for the Black Alchemist who lives there in his tower, and hand him this message.
	The town needs him.
	He is our only hope.

\end{speechtext}

\paragraph{At this point,}
the party spot a goblin sniffing along the stream of blood Laith left behind, far in the distance.
He shrieks in delight, then runs back.

\paragraph{If the party leave Laith behind immediately,}
they can run to safety, but will still hear Laith being torn apart.

\paragraph{If they wait to speak with him,}
a horde of goblins and ogres run fast behind them, but they should get the opportunity to ask any questions they wish.

\paragraph{If they insist on saving Laith,}
they will have to carry him, with all the encumbrance that entails.

\paragraph{If the party end up fighting with Laith,}
he has only 1 HP left, but can still make one last stand.

\paragraph{When the party flee,}
they will find that the nura stop to feast on the dead before following, so they have plenty of time to get away.

\paragraph{If the players decide they'd rather go directly to the town,}
skip ahead to page \pageref{siege}.
They will find the entire situation hopeless, and will not survive without a lot of sneaking.
Make sure to impress upon them just how many nura wander around the outskirts freely.

\humansoldier[\npc{\M}{Laith}]

\deepogre

\goblin

\subsection{Finding the Tower}

\begin{boxtext}

	\noindent
	You walk East for hour upon hour.
	Civilized roads are far behind you -- you can see only the heavy foliage of the forest.

\end{boxtext}

\noindent
Have the players roll Intelligence + Survival, TN 8.
Failure indicates that they have lost their way for a full day, inflicting 5 Fatigue Points.
Success indicates that they have found the location, but also receive 5 Fatigue Points.
Each margin on the roll reduces the total Fatigue Points by 1.

\subsection{The Chitincrawler}

En route to the tower, the party run into a chitincrawler's web.
Have them roll Wits + Vigilance, TN 10.
This is a group roll, so everyone who fails has been caught in the web.

\chitincrawler

\subsection{Footprints}

Have the party make a Teamwork Roll of Wits + Survival, TN 10.
Success indicates that one has spotted nura footprints nearby.

\begin{boxtext}

	The dark forest has no roads, but despite this you notice the smallest hint of a trampled path.
	Wandering over, you find two distinct types of footprints -- one very large, and another very small.

\end{boxtext}

The characters guess at least a dozen goblins, and half a dozen ogres, given the footprints.

\end{multicols}

\section{The Black Tower}

\begin{figure*}[t!]
	\includesvg{images/Dyson_Logos/black_tower}
\end{figure*}

\begin{multicols}{2}

\begin{exampletext}

	When Baron Quenn first came to the black tower, it was a lonely outpost, hoping to be the first part of a grander civilisation.
	Unfortunately, once men built the tower, funding fell away.

	When the local lawmakers outlawed Theodore Quenn for learning Necromancy, and for refusing to reply to various summons from the College of Alchemy, he ventured into the wilderness and made his home here.
	Since then, the local declared theoretical war on him, and he declared himself theoretically `baron' of his own domain, and everyone in practice left each other well-alone.
	After all, `the Black Baron', can summon enough raging fire to wipe out an army, and he managed to explain this politely to any of the Night Guard who came to his gates, before offering his sympathy with their position.

	Since then, things have been good for Baron Queen, until today.
	Nura have invaded his home, which he killed.
	Then more nura invaded, so he captured some in a room before locking it with magic.
	Then more came, and now he finds himself stuck on his own roof with three apples and yesterday's pie.

\end{exampletext}

\subsection{The Approach}

\setcounter{list}{0}

\begin{boxtext}

	Ahead, a wide, raging river stands in front of the black tower, which is in fact a grey colour typical of large stones.

\end{boxtext}

\mapentry{The Bridge}

\includesvg[width=\linewidth]{images/Dyson_Logos/black_tower_base}

\mapentry{The Portcullis}

\begin{boxtext}

	Approaching the portcullis, two ogres inside race forward, then pull it up with their hands, keeping eye-contact with you the whole time.

\end{boxtext}

\noindent
By the time the party stand 13 squares away, the ogres start to pull the gates up.
It takes the ogres 8 Initiative points to raise the portcullis.
\iftoggle{core}{%
\footnote{See the core book, page \pageref{movement} for rules on maximum movement.}
}{}

\paragraph{If someone in the group kills an ogre,}
the portcullis is lowered.
More ogres stand inside, but they will have to be enticed outside to raise the bridge again.

\paragraph{If some of the party dart inside,}
the ogres will go to whichever side has more people.

\paragraph{Raising the portcullis}
requires a total of 6 Strength points, so the party will have a difficult time doing this on their own.

\ogre[\npc{\N\T}{2 Ogres}]

\mapentry{The Stable}

\begin{boxtext}

	Opening the door, you find a room covered in straw.
	At the far side lies a horse-corpse, covered in goblins, swarming like maggots.

\end{boxtext}

\includesvg[width=\linewidth]{images/Dyson_Logos/black_tower_f1}

\paragraph{If the party have any sense,}
they will close the door.

If they draw attention to themselves, the goblins pick up their weapons, then attack.

\goblin

\mapentry{The Armoury}

\begin{boxtext}

	Outside, a full moon glimpses in through the arrow slits in the walls.
	Goblins have clearly vandalized this armoury, but the armour itself remains undamaged.
	To your left, a spiral staircase leads upward.
	Down the hall, the tower continues round.

	Human feet can be seen lying there.

\end{boxtext}

\sidepic[4]{Dyson_Logos/black_tower_f2}

The party can pick up three suits of chainmail for anyone with Strength 2-3, and three longswords.

The human feet across the room belonged to Baron Queen's servant, who kept the place tidy, and helped with the portcullis.
He has been eaten, and little remains except his boots and bones.

\mapentry{The Portcullis Mechanism}

Here, the cogs which control the portcullises sit unused.
They require only a Strength Bonus of +1 to operate.

\mapentry{Kitchen}

\begin{boxtext}

	Traipsing up the steps, you heard the sound of angry mastication.
	At the top, you see an open door, with an ogre inside a kitchen, pulling down fruits with one hand and salted meats with the other.

	He looks up at you, but does not stop eating.

\end{boxtext}

\paragraph{If the party look formidable but unthreatening,}
the ogre leaves them be.
Otherwise, he attacks.

\ogre[\npc{\M\N}{Hungry Ogre}]

\mapentry{Servant's Room}

\begin{boxtext}

	A high-pitched voice gives a lecture in the common tongue, but it's so fast the contents make little sense.

\end{boxtext}

\sidepic[4]{Dyson_Logos/black_tower_f3}

\noindent
Have the party make a Group roll of Dexterity + Stealth, TN 6.
Success indicates that they can pass unnoticed, or stay and listen.

Understanding the conversation from outside the room requires a Wits + Empathy check, TN 9.
If the roll succeeds, tell the players that the Black Alchemist is hiding on the roof, and the head goblin is having a hard time explaining that if anyone goes up there, they will have their head incinerated by magic.

\mapentry{Storage Room}

This room contains spare clothing and all manner of long-life food.
The nura have not managed to figure out the lock yet.
It has no key -- instead, there is a hole with a series of ropes inside which one must lift in the correct way.
An Intelligence + Larceny check (TN 10) is required to figure out how to open it.

\mapentry{Library}

\begin{boxtext}

	Stellar maps cover the walls, full of numbers etched into the side.
	The desks have etches of Sunlit deserts, with strange geometrical shapes.
	Scrolls have calculations concerning abstract coordinates, such as `area A', with no mention of what that area may be.

\end{boxtext}

\sidepic[4]{Dyson_Logos/black_tower_f4}

\noindent
Players may spot at this point that the maps pertain to the Realm of Bright Rocks.
What they may not spot, is a little goblin who came up to look for food, who ended up hiding behind a table.

\paragraph{If the players succeed on a Wits + Vigilance roll}
(TN 10), they find the hiding goblin.
If not, the last one to exit the room has a nasty surprise in store as the goblin attacks from behind, prompting a Sneak Attack.%
\iftoggle{core}{%
\footnote{See the core rules, page \pageref{sneakattack}.}%
}{}

\goblin[\npc{\N}{Hiding Goblin}]

\paragraph{Once the PCs enter}
they notice a trapdoor in the roof.

If any goblins enter, they will be fried by the Black Alchemist, but the players only need to shout up in a friendly voice, and say they mean no harm to come up without being incinerated.

\paragraph{If the PCs scour the room,}
they find magical item: \lootMagic.%
\iftoggle{aif}{%
\footnote{See \textit{Adventures in Fenestra}, page \pageref{magicalitems}, for magical items.}
}{}

\mapentry{Baron Quenn's Room}

The baron's room is a mess, but one made of fine quilts and quality pillows.
Anyone familiar with gnomish culture can roll Wits + Crafts (TN 8) to notice that the bedding and most of the room's contents is Gnomish in origin.
The Baron will not speak about this in any way.

\mapentry{The Roof}

\sidepic[4]{Dyson_Logos/black_tower_f5}

Here, the Black Alchemist has stayed safe from the nura horde by first taking the ladder up, and secondly by blasting any nura who come up in the face with magical fire.

\paragraph{Once the party enter,}
the Baron allows them to come up, greets them politely, and agrees to journey with them back to the city.

\npc{\M}{Baron Quenn}
\person{1}% STRENGTH
{0}% DEXTERITY 
{0}% SPEED
{{3}% INTELLIGENCE
{1}% WITS
{0}}% CHARISMA
{0}% DR
{1}% COMBAT
{Projectiles 1, Academics 2, Crafts 1, Deceit 1, Medicine 2
\Path{Blood \& Alchemy}{Force 3, Enchantment 2, Invocation 4, Necromancy 2}
}% SKILLS
{\shortsword, 3 daggers, \partialleather}% EQUIPMENT
{\mana{5}}

\end{multicols}

\section{The Siege Breaks}
\label{siege}

\begin{multicols}{2}

\subsection{The Journey Back}

Keep a tally of Fatigue, and have the players make proper decisions.
If they decide to return home immediately, they have one encounter with a basilisk.
If they wish to rest and recover, throw in another encounter.

\subsubsection{The Basilisk}

The party will notice the basilisk coming a mile away, and have time to prepare.
This gives a great opportunity to see what the Black Alchemist can do.

\basilisk

\subsubsection{The Goblins}

By this point, there is a good chance the party have become too tired to move, and will need to rest.
If they decide to recover Fatigue Points by resting, they will need to put up with the additional encounter of goblins riding nura wolves.

\goblin[\npc{\N\T}{7 Goblin Riders}]

\nurawolf[\npc{\A\N\T}{Nura Wolves}]

\end{multicols}
