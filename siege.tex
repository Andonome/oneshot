\chapter{The Siege}

\begin{center}
\begin{tcolorbox}[width=35em]

	As you cross the next hill, you see the stone walls of the town ahead, and the smoke coming up from the hundreds of chimneys within.
	Climbing higher, you can see ladders poking at the town's tall walls.
	A man runs along the edge, pushing them aside.

	You climb higher, and see the town surrounded by energetic, deformed, creatures of all sizes.
	The nura have multiplied more than you thought.
	They have laid siege to the entire town.

\end{tcolorbox}
\end{center}

\section{Journey to the Black Alchemist}

\begin{multicols}{2}

The party probably thought they were out of the water, but you can continue the adventure with the siege.

\subsection{The Last Messenger}

\begin{boxtext}

	You see a man wandering towards you, wearing thick, black, leather armour.
	He seems to be a member of the Night Guard.
	As he stumbles closer, you notice the trickle of blood leaking a river behind him.

\end{boxtext}

\begin{exampletext}

	Laith and his men were charged with escaping the city, and delivering a message to `Baron Quenn' (as he likes to call himself), otherwise known as the `Black Alchemist'.
	The message would pardon him of all wrongdoings, and recognize his current lands as an official domain, if only he approaches the town to save it.

	Unfortunately, Laith and his men were then attacked by dozens of goblins.
	All died, except a heavily wounded Laith, who is bleeding heavily.

\end{exampletext}

The moment Laith approaches, he explains his mission, and gives the party directions.

\begin{speechtext}

	You must walk East until the Sun sets, find the place in the forest where the trees neither grow, nor die, then scout for the Black Alchemist who lives there in his tower, and hand him this message.
	The town needs him.
	He is our only hope.

\end{speechtext}

\paragraph{At this point,}
the party spot a goblin sniffing along the stream of blood Laith left behind, far in the distance.
He shrieks in delight, then runs back.

\paragraph{If the party leave Laith behind immediately,}
they can run to safety, but will still hear Laith being torn apart.

\paragraph{If they wait to speak with him,}
a horde of goblins and ogres run fast behind them, but they should get the opportunity to ask any questions they wish.

\paragraph{If they insist on saving Laith,}
they will have to carry him, with all the encumbrance that entails.

\paragraph{If the party end up fighting with Laith,}
he has only 1 HP left, but can still make one last stand.

\paragraph{When the party flee,}
they will find that the nura stop to feast on the dead before following, so they have plenty of time to get away.

\paragraph{If the players decide they'd rather go directly to the town,}
skip ahead to page \pageref{siege}.
They will find the entire situation hopeless, and will not survive without a lot of sneaking.
Make sure to impress upon them just how many nura wander around the outskirts freely.

\humansoldier[\npc{\M}{Laith}]

\deepogre

\goblin

\subsection{Finding the Tower}

\begin{boxtext}

	\noindent
	You walk East for hour upon hour.
	Civilized roads are far behind you -- you can see only the heavy foliage of the forest.

\end{boxtext}

Have the players roll Intelligence + Survival, TN 8.
Failure indicates that they have lost their way for a full day, inflicting 5 Fatigue Points.
Success indicates that they have found the location, but also receive 5 Fatigue Points.
Each margin on the roll reduces the total Fatigue Points by 1.

\subsection{The Chitincrawler}

En route to the tower, the party run into a chitincrawler's web.
Have them roll Wits + Vigilance, TN 10.
This is a group roll, so everyone who fails has been caught in the web.

\chitincrawler

\subsection{Footprints}

Have the party make a Teamwork Roll of Wits + Survival, TN 10.
Success indicates that one has spotted nura footprints nearby.

\begin{boxtext}

	The dark forest has no roads, but despite this you notice the smallest hint of a trampled path.
	Wandering over, you find two distinct types of footprints -- one very large, and another very small.

\end{boxtext}

The characters guess at least a dozen goblins, and half a dozen ogres, given the footprints.

\end{multicols}

\section{The Black Tower}

\begin{figure*}[b!]
	\includesvg{images/Dyson_Logos/black_tower}
\end{figure*}

\begin{multicols}{2}

\begin{exampletext}

	When Baron Quenn first came to the black tower, it was a lonely outpost, hoping to be the first part of a grander civilisation.
	Unfortunately, once men built the tower, funding fell away.

	When the local lawmakers outlawed Theodore Quenn for learning Necromancy, and for refusing to reply to various summons from the College of Alchemy, he ventured into the wilderness and made his home here.
	Since then, the local declared theoretical war on him, and he declared himself theoretically `baron' of his own domain, and everyone in practice left each other well-alone.
	After all, `the Black Baron', can summon enough raging fire to wipe out an army, and he managed to explain this politely to any of the Night Guard who came to his gates, before offering his sympathy with their position.

	Since then, things have been good for Baron Queen, until today.
	Nura have invaded his home, which he killed.
	Then more nura invaded, so he captured some in a room before locking it with magic.
	Then more came, and now he finds himself stuck on his own roof with three apples and yesterday's pie.

\end{exampletext}

\subsection{The Approach}

\setcounter{list}{0}

\begin{boxtext}

	Ahead, a wide, raging river stands in front of the black tower, which is in fact a grey colour typical of large stones.

\end{boxtext}

\mapentry{The Bridge}

\mapentry{The Portcullis}

\begin{boxtext}

	Approaching the portcullis, two ogres inside race forward, then pull it up with their hands, keeping eye-contact with you the whole time.

\end{boxtext}

By the time the party stand 13 squares away, the ogres start to pull the gates up.
It takes the ogres 8 Initiative points to raise the portcullis.
\iftoggle{core}{%
\footnote{See the core book, page \pageref{movement} for rules on maximum movement.}
}{}

\paragraph{If someone in the group kills an ogre,}
the portcullis is lowered.
More ogres stand inside, but they will have to be enticed outside to raise the bridge again.

\paragraph{If some of the party dart inside,}
the ogres will go to whichever side has more people.

\paragraph{Raising the portcullis}
requires a total of 6 Strength points, so the party will have a difficult time doing this on their own.

\ogre[\npc{\N\T}{2 Ogres}]

\mapentry{The Stable}

\begin{boxtext}

	Opening the door, you find a room covered in straw.
	At the far side lies a horse-corpse, covered in goblins, swarming like maggots.

\end{boxtext}

\paragraph{If the party have any sense,}
they will close the door.

If they draw attention to themselves, the goblins pick up their weapons, then attack.

\goblin

\mapentry{The Armoury}

\begin{boxtext}

	Outside, a full moon glimpses in through the arrow slits in the walls.
	Goblins have clearly vandalized this armoury, but the armour itself remains undamaged.
	To your left, a spiral staircase leads upward.
	Down the hall, the tower continues round.

	Human feet can be seen lying there.

\end{boxtext}

The party can pick up three suits of chainmail for anyone with Strength 2-3, and three longswords.

The human feet across the room belonged to Baron Quenn's servant, who kept the place tidy, and helped with the portcullis.
He has been eaten, and little remains except his boots and bones.

\mapentry{The Portcullis Mechanism}

Here, the cogs which control the portcullises sit unused.
They require only a Strength Bonus of +1 to operate.

\mapentry{Kitchen}

\mapentry{Servant's Room}

\mapentry{Storage Room}

\mapentry{Baron Quenn's Room}

\mapentry{Library}

\mapentry{The Roof}

\npc{\M}{Baron Quenn}
\person{1}% STRENGTH
{0}% DEXTERITY 
{0}% SPEED
{{3}% INTELLIGENCE
{1}% WITS
{0}}% CHARISMA
{0}% DR
{1}% COMBAT
{Projectiles 1, Academics 2, Crafts 1, Deceit 1, Medicine 2
\Path{Blood \& Alchemy}{Force 3, Enchantment 2, Invocation 4, Necromancy 2}
}% SKILLS
{\shortsword, 3 daggers, \partialleather}% EQUIPMENT
{\mana{5}}

\end{multicols}

\section{The Siege Breaks}
\label{siege}

\begin{multicols}{2}

\end{multicols}
