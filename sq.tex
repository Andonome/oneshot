
 
% - Premise
% - Handouts? Does this still make sense?
% 
% # SQ
% 
% - (Roads) Gnomes Yan and Lawa sell Ingredients.
% - (Town) Complaints about gnomish Ingredients.
% - (Forest/ squashed) Kalama et c. intervene with their magical scrolls (explanation of riddles)
% - (Town) Complaints about gnomish Ingredients.
% - (Roads) Kickoff!  The troupe see a goblin raiding troupe from a distance (they're obvious and noisy) going to a bailey.
% 
% The 'gnomes' here may include Kalama, to introduce him early.
% 
% 


\sidequest[Town,Roads,Forest]{\Glsfmttext{warren}}
\label{gnomeWarrenPrelude}

\noindent
When the gnomes of the \gls{warren} find alchemical plant-\glspl{ingredient} in the \gls{deep}, they don't ask if anyone owns them -- they just start making \glspl{talisman}!
The \glspl{pc} encounter groups of gnomes selling their wares a few times before trouble strikes.
In a flash, the goblins who owned those powerful plants, which grow in the \gls{deep}, follow the gnomes and slay them.
But once trapped in the \gls{warren}, they realize they will soon starve.
They cannot speak the \gls{tradeTongue}, and cannot barter for food, which leaves only one option; they must devour the local \glspl{village}.

If the \glspl{pc} manage to fend off the horde's raiding party, the next session begins with terrible news: another raiding party has captured their allies (or someone form the \glspl{pc}'s \gls{characterPool}).
The \glspl{pc} will have to descend into the warren to rescue their loved ones.

If, on the other hand, the \glspl{pc} are defeated by the raiding party, they will find the \glspl{ogre} do not kill them, but prefer to grab them and stuff them into sacks.
In this case, they begin in the \textit{bowels} of the \gls{warren}, without their weapons, and must fight their way out.

\sqpart{Roads}% AREA
{Timid Sales-People}% NAME
{Gnomes quietly watch the \glsfmtplural{pc}, hoping to sell \Glsfmtplural{ingredient}}% SUMMARY

\begin{exampletext}
		Yan, Lawa, and a few other gnomes have five (yes, five!) alchemical \glspl{ingredient} encased in two phials of fungal-oil.
		The first phial has two green-leaved plants, usable as Earth \glspl{ingredient}, and the second has three red-leaved plant, useable as Fire \glspl{ingredient}.
		They want to sell these phials at rock-bottom prices, and scamper back to the \gls{warren}.

		Unfortunately, the gnomes don't know if the \glspl{pc} will try to kill them, and simply take the \glspl{ingredient}.
		They listen to the \glspl{pc} for a while to gauge their character, before negotiationing.

\end{exampletext}


\begin{boxtext}
	A harsh wind rattles the \gls{bothy}'s door and makes the forest shake.
	Taking a look outside, you find the peephole completely black -- the night doesn't show even the feintest hint of moonlight.

	What do you all talk about before sleep?
\end{boxtext}

If the \glspl{pc} rest in \pgls{bothy}, a gnome listens by pushing her ear up to the peephole, which blocks it entirely.
Or if the \glspl{pc} have a camp by the roadside, the gnomes will listen nearby while Lawa casts a Fire spell to warm the them, so they don't need to light their own fire (which would alert the \glspl{pc}).

\paragraph{If the \glspl{pc} sound like ruffians,}
the gnomes leave, quietly.

\paragraph{If the \glspl{pc} sound like alright people,}
Yan approaches and introduces himself with a friendly wave, or a knock on their \gls{bothy} door and begins bartering (though he does not carry the \glspl{ingredient}).

\paragraph{While bartering,}
Yan will not let the \glspl{pc} inspect the goods; instead he says that they should simply tell him the prices they \textit{would pay} (assuming the \glspl{ingredient} check out), and that he will then take one phial at a time, then take payment.

The gnomes will sell the phial with two Earth \glspl{ingredient} for 12~\glspl{sp}, and the second phial, with three Fire \glspl{ingredient} for 15~\glspl{sp}.

\paragraph{If the \glspl{pc} inspect the goods,}
they cannot identify the plants without an \roll{Intelligence}{Xenomology} roll (\tn[12]).
They also cannot barter for individual \glspl{ingredient} -- they can only buy phials.

Each phial has \pgls{weight} of 1.

\sqpart{Town}% AREA
{\squash~Flooding the Market}% NAME
{The price of \Glsfmtplural{ingredient} has plummeted as gnomes sell for cheap}% SUMMARY

\begin{exampletext}
  As gnomes from the \gls{warren} have been selling cheap \glspl{ingredient}, everyone in town has an opinion.
  Everyone at the \gls{templeOfFrost} loves it, as they can buy cheaply, while any \glspl{guard} in town speak bitterly about the lower prices, as they can't sell the bodies of \glspl{monster} for the usual high prices.
\end{exampletext}

Earth, Fire, and Water \glspl{ingredient} have been sold for around 7~\glspl{sp} each, but people sell them for 14~\glspl{sp} (`\textit{low prices can't last forever!}').
Each \gls{ingredient} is a phial, encased in fungal oil to preserve it, and has \pgls{weight} of 1.

\sqpart{Forest}% AREA
{\squash~Gnomes to the Rescue}% NAME
{\Glsentrytext{kalama} helps out the \glspl{pc} with his \glsfmtplural{talisman}, then wants paid}% SUMMARY

En route to town (to sell \glspl{ingredient}) this band of gnomes run into the \glspl{pc} (at the same time as the next \gls{segment}).
The gnomes immediately hide at the roadside (as gnomes usually do).

\paragraph{If the \glspl{pc} seem friendly,}
they may try to sell their goods for 5~\glspl{sp} per \glspl{ingredient}.

\paragraph{If the \glspl{pc} have run into trouble,}
the gnomes will help, but will also ask for reimbursement: 10~\glspl{sp} for each \glspl{talisman} used.
If the \glspl{pc} cannot pay, they ask them to pay later, and begin drawing up a contract.
The gnomes then begin arguing loudly about whether or not humans are capable of writing.

\sqpart{Roads}% AREA
{\Glsentrytext{afternoon}~The Hoard Emerge}% NAME
{The \glsfmtplural{pc} witness a raiding party of \glsfmtplural{ogre} and goblins}% SUMMARY

\begin{exampletext}
  The hoard have eaten every gnome in the \gls{warren}, and must feed again.
  Without the ability to speak the \gls{tradeTongue}, they journey to \pgls{village}, to take what they can.
\end{exampletext}

The \glspl{pc} should roll \roll{Wits}{Vigilance} (\tn[5]) to hear the hoard coming.
Each margin on the roll affords them another 100~\glspl{step}.

\ogre

\ogre

\ogre

The \glspl{ogre} will attempt to capture the \glspl{pc} and put them in sacks.
To do this, they simply need to grab the character,%
\exRef{core}{core rules}{grab}
then complete two more grab actions with a -2 Penalty to the usual roll.

If a couple of players fail that resisted \roll{Strength}{Brawl} roll, the goblins will run \emph{past} the remaining \glspl{pc} and attempt to corner or flank them.%
\exRef{core}{core rules}{flank}

\goblin

Outrunning the goblins requires a resisted \roll{Speed}{Athletics} roll at \tn\ for these goblins\ldots

\goblin

and a \roll{Speed}{Athletics} at \tn\ for these goblins, as each group run at different speeds.

\goblin

\paragraph{If the horde manage to capture the \glspl{pc},}
they return to the \gls{warren} immediately, placing them in the depths (room \vpageref{entrycell}).
They take the troupe's weapons and backpacks, but leave their armour on, because \glspl{ogre} dislike faff.

\Glspl{pc} in the \gls{warren} means you can skip the next \gls{segment}, and end the session immediately (unless you have a few more hours for the players to fight their way out of the horde's clutches).
If the game needs to stop here, and different players arrive for the next session, then anyone without an established \gls{pc} in the \gls{warren} should receive a new one (or perhaps use someone from their \gls{characterPool}).
Meanwhile, if the horde capture \pgls{pc} whose player cannot make the next session, the \gls{pc} may end up in the upper warren (\vpageref{upperPrison}), or perhaps the \glspl{ogre} simply eat them first.



