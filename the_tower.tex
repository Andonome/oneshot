\chapter{The Black Tower}
\epigraph{
  Emerging into the wild hills, you see the Sun beginning its descent.
  Clouds gather overhead.
  A nasty roaring sound comes from your stomachs.

  The hill ridges all around lie empty, with a cold wind.
  Below, the forest offers food, darkness, and danger.

  Where do you go?
}{}

\noindent
The party probably thought they were out of the woods, but in fact they will have to enter the woods, before salvation.
\ifnum\value{temperature}=0
  To make matters worse, those woods are white with snow.
  The land lies barren, and frozen.
  Each character without warm clothing receives 3 \glspl{fatigue} per \gls{interval}.
\fi

\section{Journey to \Glsfmttext{alchemist}}

\begin{multicols}{2}

\begin{exampletext}
  The black tower began as a half-way house for the \gls{guard}, as they travelled between \gls{dayton} and Grungefell.
  The wardens of Grungefell never posted enough guards, spending the town's money on games and statues instead, and it soon fell.
  The tower lay unoccupied, until \gls{alchemist} -- a warden who took to the study of alchemy -- set himself up there, along with a few servants and a hidden store of treasure.
\end{exampletext}

\subsection{Lights in the Night}

\begin{boxtext}
  Once the sky goes black, you perceive little lights in the distance.
  Once lies low, glowing yellow.
  The other lies between the ground and the stars, and flickers faintly blue.

  Where do you go?
\end{boxtext}

Night's darkness allows the \glspl{pc} to see two fires.
The first lies 3 miles away, North West.
The fire hosts a camp of cowardly \gls{guard} who fled their posts when the goblinoid horde descended.

The second lies 4 miles North, and comes from \gls{alchemist}'s tower.

\paragraph{If the troupe head towards either light,}
they will have to descend into the forest, at which point they can no longer see the lights.

Finding the lights requires an \roll{Intelligence}{Wyldcrafting} roll (\tn[9]).
Each Margin of failure on the roll adds an extra mile to the journey.%
\exRef{core}{Core Rules}{marching}

\paragraph{Remaining on the hill}
means no food, and goblins skulking around the area at night.

However, the goblin horde will be easy to avoid as long as the troupe don't light a fire or make any noise.
Still, nobody likes hearing goblin shrieks in the night\ldots

\subsection{The Cowardly Guards}

\begin{speechtext}
  ``There's no use fighting giants, you know?

  Like, we can't expect to survive, so no use in getting eaten, you know?

  If we stayed, we'd just be in some giant's belly right now, and that would actually be worse for everyone.

  You understand, right?''
\end{speechtext}

\subsubsection{Introductions}

The \glspl{guard} have a massive pot of soup, stolen from household supplies just before they fled the towns.
Six meals -- enough for themselves and a few more.
They saved a single donkey from being eaten.

\donkey[\npc{\A}{Chrysippus}]

They and the donkey have no idea that goblins have taken an interest in the smell of their food.

Find the \glspl{guard} stats in the handouts.

\paragraph{If \pgls{pc} has died,}
you can introduce a new one here.

Allow the players a moment to take stock of their situation.
Have any of the original group survived?
Everyone has come so far!

And while they talk, little goblins watch the fire, hungrily, then return to the warren to report.

\goblin[\npc{\N\T[4]}{\arabic{noAppearing} Goblin Scouts}]

\paragraph{Spotting the goblin scouts}
requires a \roll{Wits}{Vigilance} roll against the goblins' \roll{Dexterity}{Stealth} (\tn).

\paragraph{If the troupe decide to rest,}
they will have to designate watchmen for the night.
Keeping watch inflicts 3 \glspl{fatigue}, which the troupe can divide between themselves in any way they wish.

\paragraph{If the party befriend the \glspl{guard}}
they will divide the watch phase between the two groups.
Otherwise, the \glspl{guard} will keep watch, whether or not the \glspl{pc} do so.

Have the party roll \roll{Charisma}{Empathy} (\tn[10]).

\subsubsection{The Ogre Ambush}

If the party spotted and killed the goblins, all is well.
Otherwise, the goblins leave to fetch some ogres, to add the guards to their own soup.

\randomfour

\ogre

\paragraph{Spotting the ogres}
requires a \roll{Wits}{Vigilance} roll, against the ogres' \roll{Dexterity}{Stealth} (\tn).

\subsection{The Chitincrawler}

En route to the tower, the party run into a chitincrawler's web.
Have them roll \roll{Wits}{Vigilance}, \tn[10].
This is a group roll, so the check counts against the first three characters (probably those with the highest \roll{Speed}{Athletics}).

\chitincrawler

\paragraph{If the chitincrawler takes anyone by surprise,}
it attacks!
Otherwise, it withdraws backwards, into the forest, but nobody can see how far\ldots

\subsection{Footprints}

\begin{boxtext}
  The dark forest has no roads, but despite this you notice the smallest hint of a trampled path.
  Wandering over, you find two distinct types of footprints -- one very large, and another very small.
\end{boxtext}

\noindent
Have the party make a Teamwork Roll of \roll{Wits}{Survival}, \tn[10].
Success indicates that one has spotted goblin footprints nearby.

The characters guess at least a dozen goblins, and half a dozen ogres, from the footprints.

\subsubsection{Consumed Villages}

If the troupe approach any villages, they find them already consumed by the horde.
If the \glspl{guard} joined them, they will refuse to approach any villages.

The villages sometimes have some of the goblin horde remaining.
Roll to find the inhabitants:

\begin{itemize}
  \item
  $(1D6 - 3)\times 5$ goblins.
  \item
  $(1D6 - 3)\times 3$ ogres.
  \item
  Any village with goblins still around has $1D6$ farmers, tied up and waiting to be eaten.
\end{itemize}

\end{multicols}

\section{The Black Tower}

\begin{multicols}{2}

\smolMapPic{Dyson_Logos/black_tower_base}{
  \ref{towerBridge}/24/45,
  \ref{towerPortcullis}/46/66,
}

\begin{exampletext}
  \noindent
  The rampant goblin horde, having eaten their way through the villages, have invaded the black tower.
  If they can secure this outpost, they can travel farther, eventually reaching farther villages.

  The bridge here provides the only place to cross the massive river, so if the troupe stop the horde progressing, they will have limited its reach substantially.
\end{exampletext}

\subsection{The Approach}

\Gls{alchemist} sits at the top of the tower, and will wave to the troupe.

\begin{boxtext}

  Ahead, a wide, raging river stands in front of the black tower, which is in fact a grey colour typical of the slate in the area.
  At the top, a man wearing a long, dark robe, watches your approach.

\end{boxtext}

\paragraph{If anyone shouts towards \gls{alchemist},}
he silently points to the danger in the tower below him, as he does not want to draw attention to himself.

\mapentry[towerBridge]{The Bridge}

The bridge is a simple stone construction, which stays put.
Below, it is supported by wood, and can be taken apart with a \roll{Strength}{Crafts} roll, \tn[14].
Proper tools, such as a saw, can add up to a +4 Bonus.

\paragraph{If the troupe approach during daylight,}
a goblin spellcaster, watching from \nameref{towerServants} (room \ref{towerServants}) spots them, and begins casting offensive spells.

\mapentry[towerPortcullis]{The Portcullis}

Two ogres stand inside, happily eating a horse.
They have lowered the gates, so the troupe cannot rush in to attack.

\ogre[\npc{\N\T[2]}{2 Ogres}]

\paragraph{Raising the portcullis}
requires a total of 6 Strength points, so the party will have a difficult time doing this on their own.

\paragraph{Going around the back of the tower}
means edging along the narrow earthy edge.
This demands a \roll{Dexterity}{Athletics} (\tn[10]).%
\footnote{Remember, anyone wants to go around the outside, start by asking \emph{all} the players if they want to go around the outside, before the roll occurs.}

\paragraph{If anyone ends up in the river,}
have them roll \roll{Speed}{Athletics}, \tn[9].
Failure indicates they have been washed downstream, and must take \pgls{interval} trudging back.%
\iftoggle{core}{%
  \footnote{Larger party members carrying small ones will gain the standard penalties for carrying a heavy item (or in this case, `party member').
  See the core rules, \autopageref{weight}.}
}{}

\smolMapPic{Dyson_Logos/black_tower_f1}{
  \ref{towerPortcullis}/23/44,
  \ref{exStable}/42/62,
}

\begin{boxtext}
  Opening the door, the blood-sodden stable only hosts a swarm of goblins, consuming a horse corpse, and competing with a mass of flies to see how quickly they can eat.
\end{boxtext}

\mapentry[exStable]{The Stable}

Further goblins hide in here, eating a horse the ogres left for them.

\goblin

\begin{boxtext}

  Outside, a full moon glimpses in through the arrow slits in the walls.
  Goblins have clearly vandalized this armoury, but the armour itself remains undamaged.
  To your left, a spiral staircase leads upward.

  Down the hall, the tower continues round.
  Human feet can be seen lying there.

\end{boxtext}

\mapentry[towerArmoury]{The Armoury}

\begin{exampletext}
  Whenever the \gls{guard} came to the tower, \gls{alchemist} had excellent weapons waiting for them.
  This ensured they kept on good terms with him, and that he had a rotation of well-armed guards.
\end{exampletext}

\boxPair{
  \smolMapPic{Dyson_Logos/black_tower_f2}{
    \ref{towerArmoury}/59/64,
    \ref{towerMechanism}/25/43,
  }
}{
  \smolMapPic{Dyson_Logos/black_tower_f3}{
    \ref{towerServants}/25/47,
    \ref{towerKitchen}/25/75,
    \ref{towerStorage}/70/44,
  }
}

The party can pick up three suits of chainmail for anyone with Strength 2-3, and three longswords.

The human feet across the room belonged to \gls{alchemist}'s servant, who kept the place tidy, and helped with the portcullis.
He has been eaten, and little remains except his boots, blood, and bones.

\mapentry[towerMechanism]{The Portcullis Mechanism}

Here, the ropes and wheels which control the portcullises sit unused.
They require only a Strength Bonus of +1 to operate.

\begin{boxtext}
  Traipsing up the steps, you hear the sound of angry mastication.
  At the top, you see an open door, with an ogre inside a kitchen, pulling down fruits with one hand and salted meats with the other.

  He looks up at you, but does not stop eating.
\end{boxtext}

\mapentry[towerKitchen]{Kitchen}

The kitchen still holds 8 meals' worth of food, but the ogre will eat all of it soon enough.

\paragraph{If the party look formidable but unthreatening,}
the ogre leaves them be.
Otherwise, he attacks.

\ogre[\npc{\N\M}{Hungry Ogre}]

\mapentry[towerServants]{Servant's Room}

A goblin spellcaster stands here, watching for anyone approaching.

\goblincaster

\showStdSpells

\mapentry[towerStorage]{Storage Room}

This room contains spare clothing and all manner of long-life food.
The goblins have not managed to figure out the lock yet.
It has no key -- instead, there is a hole with a series of ropes inside which one must lift in the correct way.
An \roll{Intelligence}{Larceny} check (\tn[10]) lets \pgls{pc} figure out how to open it.

\boxPair[t]{
  \smolMapPic{Dyson_Logos/black_tower_f4}{
    \ref{towerLibrary}/59/46,
    \ref{towerBaron}/25/72,
  }\vspace{-3em}%
}{
  \smolMapPic{Dyson_Logos/black_tower_f5}{
    \ref{towerRoof}/60/54,
  }\vspace{-3em}%
}

\begin{boxtext}

  Stellar maps cover the walls, full of numbers etched into the side.
  The desks have etches of Sunlit deserts, with strange geometrical shapes.
  Scrolls have calculations concerning abstract coordinates, such as `area C', with no mention of what that area may be.

\end{boxtext}

\mapentry[towerLibrary]{Library}

The library contains myriad books:

\begin{itemize}
  \item
  \textit{The Practical Witch, and Other Tall Tales}
  \item
  \textit{Essays on the Benefits of Good Monarchs}
  \item
  \textit{Elemental Metaphysics}
  (on the metaphysics of the five elements)
  \item
  \textit{Collected Recounts of the Nine Worlds of the Dead}
  \item
  \textit{Gnomish Anatomy, volumes I-VII}
  (number VI is missing)
\end{itemize}

\noindent
A single goblin hides underneath a table, perusing Volume VI of \textit{Gnomish Anatomy}, which contains naked gnome ladies.
It feels scared by the big people approaching, and waits to stab a player from hiding, then bolt downstairs.

\goblin[\npc{\N\F}{Hiding Goblin}]

\paragraph{If the \glspl{pc} fail to spot the hiding goblin,}
it attempts a sneak attack, then bolts down the stairs.

\paragraph{Searching for particularly valuable books}
requires an \roll{Intelligence}{Academics} roll (\tn[12]).

\paragraph{Once the \glspl{pc} enter}
they notice a trapdoor in the roof.

If any goblins enter, they will be fried by \gls{alchemist}, but the players only need to shout up in a friendly voice, and say they mean no harm to come up without being incinerated.

\mapentry[towerBaron]{\Glsfmttext{alchemist}'s Room}

The room is a mess, but a mess made of fine quilts and quality pillows.

Around the room are:

\begin{enumerate}

  \item{A jar with the web-spinning organ of a chitincrawler, soaked in wine.}
  \item{A rare green gem, worth 40 \glspl{gp}.%
  \footnote{Finding a buyer will prove difficult.}}
  \item{The preserved eyeballs of Coltrank the seer.
  It could make an excellent \gls{ingredient} for a Fate spell with a bit of work.}
  \item
  One \lootMagic, hidden in a drawer.

  \showTalisman

\end{enumerate}

\mapentry[towerRoof]{The Roof}

\begin{exampletext}
  \Gls{alchemist} has lit a bonfire, made from a broken bookshelf.
  He has enchanted it to flicker blue in order to call for aid.
  When goblins poke their heads up, he uses his magical abilities to blast them off the side of the tower, or have the bonfire consume them.
\end{exampletext}

\theAlchemist

\paragraph{Once the party enter,}
the Baron allows them to come up, greets them politely, and listens to anything they have to say.

\paragraph{If any party members have died,}
a new \gls{pc} can be found here.
This might be someone in the employ of \gls{alchemist}, or a random person who was in the deep forest and ran away from the horde, then took shelter in the tower.

\paragraph{While the troupe make their introductions,}
the goblins below feel hungry again, and ascend the tower.
List out all enemies the troupe have not eliminated: all of them join together and plan their attack.

\end{multicols}

\subsection{The Final Stretch}

\begin{multicols}{2}

\subsection{A Conversation on Civilization}

At this point, most of the \glspl{pc} have little energy left.
Perhaps some had a meal with the \glspl{guard}, earlier, and perhaps they saved some food from room \ref{towerKitchen}.
But if you have reminded the players about each \gls{fatigue} their characters had to mark down, the troupe will find themselves in a tight spot -- they must take a long journey to \gls{dayton} (the only surviving civilization in the area), but they have no food.

\paragraph{If the \glspl{pc} want any food from inside the Tower,}
\gls{alchemist} refuses.
All the food belongs to him, and he wants it.

\paragraph{If \gls{kalama} still accompanies the troupe,}
\ifnum\value{temperature}>0
  he can use his spells to summon a little food while in the forest.
  However, a basilisk will be upon them soon\ldots
\else%
  his spells cannot grow fresh food in a frozen forest.
\fi

\paragraph{If the players seem unconcerned about their \glspl{pc}' hunger,}
tell them simply.
Explain that they feel starving, but need to travel a long way.

\ifnum\value{temperature}>0%
  \subsection{The Basilisk's Stench}

  The very next interval, a nasty stench arrives in the wind -- a basilisk has arrived.

  \paragraph{If the portcullis lies open,}
  it enters and eats what it can.

  If anyone hurts it, the beast wanders away.
  Otherwise, it remains, stinking up the tower (which inflicts further \glspl{fatigue}).
  \basilisk
\else
  Remind them of the freezing weather outside.
  It has already begun to snow again\ldots
\fi

\end{multicols}
