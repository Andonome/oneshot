\chapter{The Top Floor}
\section{Quiet Rooms}
\label{upper}

\setcounter{list}{0}

\begin{multicols}{2}

\mapentry{Entrance Hallway}

	\toppic{Roch_Hercka/spiral_staircase}{\label{roch:ogres}}

At the other side of this door sits the lonely guard, Bobble.
He only turned into a goblin recently, and he doesn't like it.
He is also extremely hungry.

\paragraph{If the party attempt to trick Bobble,}
it's not terribly difficult.

\paragraph{If the party simple enter,}
then Bobble will run as fast as he can, away from the door, in order to alert the rest of the upper warren.

\goblin[\npc{\N\M}{Bobble}]

\mapentry{Banquet Hall}

\begin{boxtext}

	The door at the top of the stairs reveals a large banquet hall.
	Human bones lie on the table, all polished clean.
	The cups and cutlery lie smashed and broken.
	At the far end of the hallway, a door stands ajar.
	On the right, a passageway with a hint of sunlight.

\end{boxtext}

\paragraph{If the players search the table,}
they can find a knife which functions as a dagger.

\mapentry{The Spiral Staircase Exit}

\begin{boxtext}

	In the sunlit intersection of four hallways,
	\iftoggle{hardcore}%
		{four ogres sit playing a game of dice.  Two are clad in black leather armour, with a massive sword by their side.
		Another sucks on a horse's uncooked head, while the third goes for a piss behind the staircase.}%
		{three ogres sit playing dice.
		Two are clad in black, leather armour, apparently pieced together from multiple suits.
		The third sits watching them play some dice game.}%

\end{boxtext}

The party can see the exit, but cannot escape without being attacked by three ferocious ogres.
In order to progress, they could do any number of things:

\begin{itemize}

	\item{Trick one of the ogres into chasing them (but only one).}
	\item{Find a missile weapon in order to shoot from afar, forcing the guardian ogres to chase them.}
	\item{Distract the ogres with illusions of something, while running up the stairs.}

\end{itemize}

Exactly what plans they come up with and how well they work depends entirely on your players and how their situation emerges.

\paragraph{If the party attempt to get the ogres to chase them,}
the armour-wearing ogres remain where they are, as they have strict orders not to leave the base of the staircase.

\npc{\N\T}{2 Armoured Ogres}

\person{6}% STRENGTH
{0}% DEXTERITY 
{4}% SPEED
{{-3}% INTELLIGENCE
{0}% WITS
{-3}}% CHARISMA
{0}% DR
{1}% COMBAT
{Crafts~1, Vigilance~2}% SKILLS
{\greatsword, \partialleather}% EQUIPMENT
{}

\npc{\N\M}{Sitting Ogre}

\label{chasingogre}

\person{4}% STRENGTH
{0}% DEXTERITY 
{4}% SPEED
{{-3}% INTELLIGENCE
{0}% WITS
{-3}}% CHARISMA
{0}% DR
{1}% COMBAT
{Crafts~1, Vigilance~2}% SKILLS
{\greatclub}% EQUIPMENT
{}

\iftoggle{hardcore}{%
	\ogre[\npc{\M\N}{Hidden Ogre}]
}{}

\mapentry{The Trapped Hallway}

\begin{boxtext}

	The door swings open with a loud creek, revealing a hallway covered wall-to-wall with sticky webs.
	Two goblin corpses stand upright, pierced in multiple places, with a pool of their own blood lying on the ground.

\end{boxtext}

\begin{exampletext}

	When the nura invaded, three little gnomes found themselves trapped in this little side-room.
	They got together enough food for a few days, and have been waiting since then for someone to come and rescue them.

\end{exampletext}

\paragraph{If anyone tries to wriggle past the webbing,}
have them roll Dexterity + Athletics, TN 12.

\paragraph{If they get caught,}
three gnomes will rush out and try to kill them while they're trapped and unable to defend themselves.
Have them make a Wits + Empathy roll, TN 8, to avoid the initial attack.

\paragraph{If the party use torches to destroy the webs,}
it works okay, but moving through quietly isn't easy.
Have them roll Intelligence + Stealth to plan a route through, without upsetting the goblin corpses (which could fall and clatter), or otherwise make a racket.

\mapentry{The Lounge}
\label{lounge}

\begin{boxtext}

	Row after row of beds, little hammocks, and tables with old, cold pipes sitting on them stretch across the room.
	In the distance, little voices come through an open doorway.
	They talk lazily, quite unlike the goblinoid chatter you normally hear.

\end{boxtext}

If the party entered secretly, have them roll Wits + Vigilance, TN 8, to notice a dozing gnome.
If the party move through without explaining themselves, he wakes, then sounds the alarm, and Tar\'ik in the far room incinerates the place with a \textit{Wide Fireball}, inflicting $1D6 +3$ Damage.

\toppic{Dyson_Logos/upper}{}

\mapentry{Trapped Gnomes}

What happens here depends entirely on how the party have come through the previous rooms.

\paragraph{If they shouted in a loud, friendly manner,}
they get a friendly greeting as all the gnomes here thank them for their presumed rescue, and hope to join them in the escape.

These gnomes know the entire warren, and can tell them about the contents of every room.

\paragraph{If the party entered secretly,}
then as soon as the gnomes spot them, they will draw their weapons while Tar\'ik throws invocation spells at them.

\paragraph{If the party ask what has happened here,}
the gnomes tell them everything.
See page \pageref{invasionhistory} for the complete history.

\paragraph{If the gnomes join the party,}
remember to keep track of their Fate Points.
Each scene they will regenerate 2.

\setcounter{enc}{\value{list}}
\addtocounter{enc}{1}

\paragraph{Before the party leave,}
the gnomes request they nip out to the kitchen, (room \arabic{enc}) and get them some food.

With proper food, the party and the gnomes can rest, regain any Fatigue Points they may have accumulated, and the gnomes can do the same.

\gnomishsoldier[\npc{\M}{Derkel}]

\gnomishsoldier[\npc{\F}{Merkel}]

\gnomishsoldier[\npc{\M}{Klaus}]

\NPC{\F}{Tar\'ik the Alchemist}{Nosey}{Picks nose}{Tribe}
\person{-3}% STRENGTH
{1}% DEXTERITY 
{-2}% SPEED
{{2}% INTELLIGENCE
{1}% WITS
{-1}}% CHARISMA
{0}% DR
{0}% COMBAT
{Projectiles~1, Academics~2, Empathy~1, Tactics~2
\Path{Alchemy}{\force~2, \invocation~2, \illusion~3}}% SKILLS
{Mana stone with 4 MP}% EQUIPMENT
{\lockedmana{4}\addtocounter{sp}{8}}

\mapentry{The Kitchen}

\begin{boxtext}

	This Spartan kitchen has pans, pots, and boxes strewn all over the place.
	A fireplace with ventilation through the roof sits to your right.
	Amid the mess of sacks and broken plates, you can see that a little food remains, apparently dumped here.

\end{boxtext}

A recent raiding party managed to get enough food to save, and one of the goblins convinced the ogres to leave it here for later.

If the party want the food, they will have to make a Dexterity + Stealth roll to pilfer it without making any noise, TN 8.
Each margin on the roll gives them enough food to heal 4 Fatigue Points.
They can roll as many times as they like, but each roll carries the danger of being spotted.

\paragraph{If the party make any noise,}
the chasing ogre from the central room rushes after them (Page \pageref{chasingogre}).

\mapentry{The Storage Room}

\begin{boxtext}

	Inside the next room, you find weapons, leather armour, rope, mining supplies, scrolls, and a desk with a full map on the wall.

	A portal at the other end leads out into a long, dark hallway.

\end{boxtext}

The room contains every item which could be called `Adventuring Equipment',
\iftoggle{core}{%
\footnote{See the core book, page \pageref{start_equipment}.}
}{}%
in addition to:

\begin{itemize}

	\item{10 shortswords}
	\item{2 suits of partial leather (for someone with Strength 2 or 3)}
	\iftoggle{hardcore}{%
	\item{1 suit of partial leather (for someone with Strength 1 or 2)}
	}{
	\item{2 suits of partial chain (for someone with Strength 1 or 2)}
	}
	\item{A longbow which deals 1D6+2 Damage (requires at least Strength +2)}
	\item{3 quivers of 20 arrows}
	\item{A portal scroll}

\end{itemize}

If the party examine the map on the wall, they find it to be a map of the current level of the warren.
See Appendix \ref{handouts} for the map handout.

\mapentry{The Woodspy Pool}

\begin{boxtext}

	In the distance, you can see a massive pool of water shimmering in the dark room.
	It seems to move a little, as if it had its own subtle tides.

	Close by, you can see a room full of goblins arguing with each other, surrounded by candlelight.

\end{boxtext}

\begin{exampletext}

	The gnomes once bread woodspies in this little pool.
	They would chuck in various leftover food and organic waste from the surface world, and as soon as a woodspy laid some eggs, the gnomes would catch it and eat it.
	They never presented a danger, so long as they stayed small.

	Since then, the nuramancers decided to cast a spell on one, turning it into a nura.
	It ate all the others, and now sits hungrily, at the bottom of the pool.

\end{exampletext}

\nurawoodspy[\npc{\A\N}{Nura Woodspy}]

\paragraph{If any non-nura attempt to run past the pool of water,}
the woodspy within reaches up and try to grab them.

\paragraph{If the PCs attempt to shoot at the woodspy,}
then it retreats into the water, where it remains safe from missile weapons.

\mapentry{The Duelling Room}

\begin{exampletext}

	When gnomes practice magic, things can get dangerous, so they set aside a room where all the dangerous fireballs and shield spells could go off, without people accidentally getting in the way.
	The traditional alchemist-battles here involved each participant placing a statue on a plinth, and then trying to protect their own statue, and knocking the other one over.

\end{exampletext}

Currently the room hosts a rabble of goblins, arguing over battle-plans in the common tongue, while a nuramancer in the corner tries to get them to calm down.

\iftoggle{hardcore}{%
	\goblin
}{
	\goblin[\npc{\N\T}{4 Bickering Goblins}]
}

\goblinnuramancer

\paragraph{If the PCs make any noise,}
the prisoners in the next room shout for help.

\mapentry{The Upper Prison}

\begin{boxtext}

	Entering another room which once housed gnomes, but now houses prisoners, it suddenly dawns on you just how far you have come -- how many problems you have solved, and how many adversaries you have faced.
	The people in the room rush towards you, eagerly asking about the possibility of escape.
	They all appear starved and cold.
	Half of them have torn the little gnomish tapestries on the wall down to provide extra layers of clothing.

\end{boxtext}

\begin{exampletext}

When gnomes slept here, some of the cheeky young ones wanted to get out and play, so they poked away at the walls and eventually dug an exit for themselves behind a tapestry.

\end{exampletext}

This room has a shoddily-made bar on the \textit{outside}, so the prisoners cannot exit.

This is another point for the group to spend Story Points in order to bring NPCs into the group, or add another PC if someone has lost their character.

The local farmers imprisoned here need food badly.
They will be little use in a fight, but will agree to try if given proper weapons and at least a snack.

\paragraph{If the players end up captured in any areas nearby,}
they are imprisoned here, rather than down stairs.

\paragraph{If the PCs search around the room,}
they will soon find the secret exit.
If they don't search, have them roll a Wits + Vigilance check, TN 10 (Teamwork roll).%
\footnote{See the core book, page \pageref{teamwork} for Teamwork rolls.}

The secret hole in the wall is, unfortunately, rather narrow.
It is not possible for anyone with a Strength bonus greater than 0 to get through, but a Strength + Crafts check, TN 11, might allow someone to chisel away at just enough to let someone of Strength +1 through.

\npc{\M}{Andrew}
\person{2}% STRENGTH
{0}% DEXTERITY 
{0}% SPEED
{{-1}% INTELLIGENCE
{-1}% WITS
{1}}% CHARISMA
{0}% DR
{0}% COMBAT
{Crafts~2, Empathy~1, Vigilance~1}% SKILLS
{Nothing}% EQUIPMENT
{}

\npc{\F}{Ronda Marsh}
\person{1}% STRENGTH
{0}% DEXTERITY 
{0}% SPEED
{{0}% INTELLIGENCE
{-1}% WITS
{0}}% CHARISMA
{0}% DR
{0}% COMBAT
{Beast~Ken~1, Crafts~1, Empathy~1}% SKILLS
{Nothing}% EQUIPMENT
{}

\mapentry{The Last Room}

This nasty little room holds a couple of beds and the remnants of gnomish board games.
Among the wreckage, the party can find
\iftoggle{aif}%
	{three Spider Arrows in a quiver (see Adventures in Fenestra, page \pageref{spiderarrows}).}%
	{yet another portal scroll.}

\end{multicols}

\iftoggle{hardcore}{}{

\section{The Sunlight}

\begin{multicols}{2}

\noindent
Once up the stairs, the party can leave.

\begin{boxtext}

	Past the trees and down the hill, the Sunlight shows burning villages in the distance, but farther, the great walled town stands strong.

\end{boxtext}

\iftoggle{hardcore}{

\paragraph{If the party were running from the nura,}
then the chase continues down the rocky, mountain path.

\paragraph{If the party have escaped with the hiding gnomes,}
they gnomes go their own way.
They thank the party for their help, but refuse to travel with them further.
}{

\paragraph{If the party have escaped with the hiding gnomes,}
the gnomes offer to put them up in a nearby gnomish warren which (they assume) has not been invaded by the nura.

\paragraph{Once the party get back to the major town,}
they find the nura hordes have been pushed back, and so long as the locals stay inside their walled town for a day or two more, all the nura in the local area should starve to death.

\paragraph{Whatever happens,}
ask each player what their characters are returning to, just to sum the adventure up on a positive note.
Perhaps this ends the story, or perhaps it marks the beginning of a new group, ready to purge the nura from the land.

}


\end{multicols}

}
