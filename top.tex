\chapter{The Top Floor}
\epigraph{
  \iftoggle{hardcore}{
    I have township, yet no houses.
    Forests, but no trees.
    Rivers, but no water.
  }{
    Never resting, never still.

    Moving silently from hill to hill.

    It does not walk, run or trot.

    All is cool where it is not.
  }
}{}

\glsresetall

\begin{exampletext}
  \noindent
  When the horde invaded, three little gnomes found themselves trapped in this little side-room.
  They grabbed a few days' worth of food from the adjacent kitchen (room \ref{upperKitchen}),
  barred the door to the main hall (room \ref{spiral}), then blocked off room \ref{webTrap} by spinning water into webs, to give the appearance of giant, fresh webs, hanging from floor to ceiling.
  When two goblins entered, they used their crossbows to kill them in an instant, then retrieved the crossbow bolts.

  The intimidating spell worked -- the goblin horde decided some monstrous spider must wait somewhere in that darkness, behind the webs.
  The gnomes still rest in room \vref{gnomes}, with two half-sized barrels of salted mushrooms to sustain them.

  When \gls{kalama}
  returned to the \gls{warren}, disguised with magic as a goblin, he noted a woodspy%
  \exRef{judgement}{Judgement}{woodspy}
  creeping in after him
  (see \autopageref{kalama} for \gls{kalama}).
  Seizing the opportunity for mischief, he transformed the woodspy with the same spell he used to transform himself.
  It swelled into a tidal wave of tentacles, chased after him, and ate a number of goblins on the way.
  Now it lies in a pool of water in room \ref{woodspies}, slowly digesting the goblin bodies.
\end{exampletext}

\begin{multicols}{2}

\subsection{Quiet Rooms}
\label{upper}

\mapentry[entrance]{Entrance Hallway}

\widePic{Roch_Hercka/spiral_staircase}

At the other side of this door sits the lonely goblin guard, Rubble.

\paragraph{Once the \glspl{pc} enter}
Rubble flees, then watches them from afar.
He reports their every move to others in the \gls{warren}, and tries to get round them, go back down the stairs, and summon the rest of the horde to come and eat them.

\goblin[\npc{\N\M}{Rubble}]

\mapentry[banquet]{Banquet Hall}

If the players search the table,
they can find a knife which functions as a dagger.

\begin{boxtext}
  The door at the top of the stairs reveals a large banquet hall.
  Human bones lie on the table, all polished clean.
  The cups and cutlery lie smashed and broken.
  At the far end of the hallway, a door stands ajar.
  On the right, a passageway shows a shaft of Sunlight in the far distance.
\end{boxtext}

\mapentry[spiral]{The Spiral Staircase Exit}

The party can see the exit, but cannot escape without three ferocious ogres attacking them.
Exactly what plans they come up with and how well they work depends entirely on your players and how their situation emerges.

\begin{boxtext}
  In the Sunlit intersection of four hallways,
  \iftoggle{hardcore}%
    {four ogres sit playing a game of dice.  Two are clad in black leather armour, with a massive sword by their side.
    Another sucks on a horse's uncooked head, while the third goes for a piss behind the staircase.}%
    {three ogres sit playing dice.
    Two are clad in black, leather armour, apparently pieced together from multiple suits.
    The third sits watching them play some dice game.}%
\end{boxtext}

\paragraph{If the party attempt to get the ogres to chase them,}
the armour-wearing ogres remain where they are, as they have strict orders not to leave the base of the staircase.


\Person{\npc{\T[2]\N}{2 Armoured Ogres}}%
  {{6}{0}{1}}% BODY
  {{-3}{0}{-3}}% MIND
  {%
    \set{Combat}{1}
    \set{Brawl}{1}
    \set{Crafts}{1}
    \set{Vigilance}{2}
    \greatsword
    \partialleather
  }% SKILLS
  {}% KNACKS
  {%
      \lootGoblin/ \lootGoblin%
  }% EQUIPMENT
  {}% ABILITIES

\label{chasingogre}

\Person{\npc{\N\M}{Sitting Ogre}}%
  {{4}{0}{0}}% BODY
  {{-3}{0}{-3}}% MIND
  {%
    \set{Combat}{1}
    \set{Brawl}{3}
    \set{Caving}{2}
    \set{Vigilance}{2}
    \greatclub
  }% SKILLS
  {}% KNACKS
  {%
      \lootGoblin%
  }% EQUIPMENT
  {}% ABILITIES

\iftoggle{hardcore}{%
  \ogre[\npc{\M\N}{Hidden Ogre}]
}{}

\mapentry[webTrap]{The Trapped Hallway}

The entire contents of this room are illusory, left by the gnomes to scare the horde from entering.

\begin{boxtext}
  The door swings open with a loud creek, revealing a hallway covered wall-to-wall with sticky webs.
  Two goblin corpses stand upright, pierced in multiple places, with a pool of their own blood lying on the ground.
\end{boxtext}

\paragraph{If anyone tries to wriggle past the webbing,}
have them roll \roll{Dexterity}{Athletics} (\tn[12]).

\paragraph{If they get caught,}
three gnomes will rush out to kill with crossbows.
Trying the missiles with \roll{Dexterity}{Athletics} takes a -4 Penalty from the magical webs, but quickly shouting that they are not the enemy (\roll{Wits}{Empathy})  has no penalty.

In either case, they roll at \tn[10].

\paragraph{If the party use a torch to destroy the webs,}
they vanish, silently.

\mapPic{b}{Dyson_Logos/upper}{
  \ref{entrance}/85/36,
  \ref{banquet}/67/31,
  \ref{spiral}/37/66,
  \ref{webTrap}/51/13,
  \ref{lounge}/28/33,
  \ref{gnomes}/08/51,
  \ref{upperKitchen}/45/33,
  \ref{storage}/72/51,
  \ref{woodspies}/85/84,
  \ref{duelling}/51/99,
  \ref{upperPrison}/24/96,
  \ref{finalRoom}/08/84,
}

\mapentry[lounge]{The Lounge}

If the party entered secretly, have them roll \roll{Wits}{Vigilance} (\tn[8]), to notice Yan, dozing gently.
If the party move through without explaining themselves, he wakes, then sounds the alarm, and all of them come out attacking, frantically.

\mapentry[gnomes]{Trapped Gnomes}

What happens here depends entirely on how the party have come through the previous rooms.

\begin{boxtext}
  Row after row of beds, little hammocks, and tables with old, cold pipes sitting on them stretch across the room.
  In the distance, little voices come through an open doorway.
  They talk lazily, quite unlike the goblinoid chatter you've heard for too long.
\end{boxtext}

\paragraph{If they shouted in a loud, friendly manner,}
they get a friendly greeting as all the gnomes here thank them for their presumed rescue, and hope to join them in the escape.

These gnomes know the entire \gls{warren}, and can tell them about the contents of every room.

\paragraph{If the party entered secretly,}
then as soon as the gnomes spot them, they will draw their weapons while Lawa throws invocation spells at them.

\paragraph{If the gnomes join the party,}
they will enter combat behind the \glspl{pc}.

\paragraph{Before the \glspl{pc} leave,}
the gnomes request they nip out to the kitchen, (room \vref{lounge}) and get them some food.

With proper food, the party and the gnomes can rest, lose some \glspl{ep}, and the gnomes can do the same.

\gnome[\npc{\Gn\M}{Yan}]

\gnomishsoldier[\npc{\Gn\F}{Nanpa}]

\gnomishsoldier[\npc{\Gn\M}{\"{O}pen}]

\gnomishillusionist[\NPC{\Gn\F}{Lawa the Alchemist}{nosey}{picks nose}{cake \& freedom}]

\secondRaidingParty

\mapentry[upperKitchen]{The Kitchen}

\begin{exampletext}
  A recent raiding party managed to get enough food to save, and one of the goblins convinced the ogres to leave it here for later.
\end{exampletext}

If the party want the food, they will have to make a \roll{Dexterity}{Stealth} roll to pilfer it without making any noise (\tn[8]).
Each Margin on the roll gives them enough food to heal 4 \glspl{ep}.
They can roll as many times as they like, but each roll carries the danger of being spotted by a mobile ogre at the stairs.

\begin{boxtext}
  This Spartan kitchen has pans, pots, hip-bones, skulls, rib-bones, and boxes strewn all over the place.
  A fireplace with ventilation through the roof sits to your right.
  Amid the mess of sacks and broken plates, you can see that a little food remains, apparently dumped here.
\end{boxtext}

\paragraph{If the party make any noise,}
the chasing ogre from the central room rushes after them (\vref{chasingogre}).

\mapentry[storage]{The Storage Room}

The room contains myriad items, left over from the gnomes, or taken from raids on nearby people.

\columnbreak

\begin{itemize}
  \item
  Chalk
  \item
  Portable writing equipment (\gls{weight} 1)
  \item
  10 shortswords
  \item
  2 suits of partial leather (for someone with Strength 2 or 3)
    \iftoggle{hardcore}{%
    \item
    1 suit of partial leather (for someone with Strength 1 or 2)
  }{
    \item
    2 suits of partial chain (for someone with Strength 1 or 2)
  }
  \item
  A longbow which deals $1D6+2$ Damage (requires at least Strength +2)
  \item
  3 quivers of 20 arrows
\end{itemize}

If the party examine the map on the wall, they find it to be a map of the current level of the \gls{warren}.
Hand them the map from the handouts if they don't have a copy already.

\begin{boxtext}
  Inside the next room, you find weapons, leather armour, rope, mining supplies, scrolls, and a desk with a large map on the wall.

  A portal at the other end holds a patch of darkness, hiding safely away from the Sunlight.
\end{boxtext}

\null
\mapentry[woodspies]{The Woodspy Pool}

\begin{exampletext}

  A small stream of water from above once provided the gnomes here with a place to bathe and fetch water.

  Since then, a woodspy has entered, and \gls{kalama} the spell-casting, trouble-making gnome, has transformed the cephalopod into an even more monstrous creature, using Life magic.

  It ate a few goblins who were putting away prisoners in room \vref{upperPrison}, and the other goblins retreated to room \vref{duelling}, where they remain, trapped by the giant creature.

\end{exampletext}

\begin{boxtext}
  In the distance, you can see a massive pool of water shimmering in the darkness.
  It seems to move a little, as if it had its own subtle tides.

  Past the pool, the next room hosts goblins arguing with each other, surrounded by candlelight.
\end{boxtext}

\morphwoodspy[\npc{\N\A}{Gargantuan Woodspy}]

\paragraph{If anyone attempts to run past the pool of water,}
the woodspy within reaches up and try to grab them.

\paragraph{If the \glspl{pc} attempt to shoot at the woodspy,}
then it retreats into the water, where it remains safe from missile weapons.

\paragraph{Giving the creature a goblin}
(dead or alive)
leaves it entertained for a few minutes.
After that, it will become excited by any more motion, and continue grabbing anything nearby.

If it receives any injury, and has something to eat, it remains passive for a full \gls{interval}.

\mapentry[duelling]{The Duelling Room}

\begin{exampletext}

  When gnomes practice magic, things can get dangerous, so they set aside a room where all the dangerous fireballs and shield spells could go off, without people accidentally getting in the way.
  The traditional alchemist-battles here involved each participant placing a statue on a plinth, and then trying to protect their own statue, and knocking the other one over.

\end{exampletext}

Currently the room hosts a rabble of goblins, trapped by the monstrous woodspy in room \vref{woodspies}.
They occasionally shout to the ogres in room \vref{spiral}, but the ogres don't want to fight the monster, so they stick to their duty of guarding.

The display cabinet at the side of the room contains the following:

\begin{itemize}
  \item
  3 Fire \gls{boon} powders.
  \item
  1 Earth \gls{boon} powder.
  \item
  1 \lootTalisman, in a drawer.

  \showTalisman
  \item
  1 \lootTalisman, in a drawer.

  \showTalisman
\end{itemize}

\iftoggle{hardcore}{%
  \goblin
}{
  \goblin[\npc{\N\T[4]}{4 Bickering Goblins}]
}

\goblincaster

\mapentry[upperPrison]{The Upper Prison}

\begin{exampletext}
When gnomes slept here, some of the cheeky young ones wanted to sneak out and play unnoticed, so they poked away at the walls and eventually dug a secret exit for themselves behind a tapestry.
\end{exampletext}

\begin{boxtext}
  Up the stairs, the passage shifts to the right, where you see a wide, low, door.
  Its handle has been shoddily wedged shut with a small chair.
\end{boxtext}

\paragraph{If the \glspl{pc} don't shout in a loud, friendly voice,}
the prisoners will think they are more goblins, here to eat them or put in more prisoners.

Take three farmers from the handout.
The first farmer tries to grab the first \gls{pc} in the door,%
\exRef{core}{core rules}{grab}
and the other two attack them while they are trapped.%
\exRef{core}{core rules}{trapped}

\iftoggle{hardcore}{
  Each farmer here begins with 7 \glspl{ep}, so they will quickly accrue penalties if they exert themselves in any way.
}{}

\begin{boxtext}
  Opening the door, you see rows of tiny gnome-sized beds, mostly broken, all abandoned, then a hand reaches round the door for your face.
\end{boxtext}

\paragraph{If any \gls{pc} has died,}
this is yet another place to find a new character.

\paragraph{If the players end up captured in any areas nearby,}
they are imprisoned here, rather than down stairs.

\paragraph{Searching the room}
reveals the dark, little exit with a \roll{Wits}{Crafts} roll (\tn[10]).

Unfortunately, the little hole can only fit people with 6~\glspl{hp} or fewer.
But a \roll{Strength}{Crafts} check (\tn[11]), might allow someone to chisel away at just enough to let someone of Strength +1 through (while making a lot of noise).

\mapentry[finalRoom]{The Last Room}

This nasty little room holds a couple of beds and the remnants of gnomish board games.
Among the wreckage, the party can find a hidden item:

\lootTalisman

\showTalisman

\iftoggle{hardcore}{
  \paragraph{If the escapees exit while running from pursing ogres and goblins,}
  then the chase continues down the rocky, mountain path.

  \paragraph{If the party have escaped with the hiding gnomes,}
  they gnomes go their own way.
  They thank the \glspl{pc} for their help, but refuse to travel with them further.
}{
  \subsection{The Sunlight}

  \noindent
  Once up the stairs, the party can leave.

  \begin{boxtext}
    Past the trees and down the hill, the Sunlight shows burning villages in the distance, but farther, the great walled town stands strong.
  \end{boxtext}

  \paragraph{If the party have escaped with the hiding gnomes,}
  the gnomes offer to put them up in a nearby gnomish \gls{warren} which (they hope) has not been invaded by the horde.

  \paragraph{Whatever happens,}
  ask each player what their characters are returning to, just to sum the adventure up on a positive note.
  Perhaps this ends the story, or perhaps it marks the beginning of a new group, ready to purge the goblins from the land.
}

%\pic{Decky/shaman}

\end{multicols}

