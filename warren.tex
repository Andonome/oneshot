\chapter{The \Glsfmttext{warren}}
\epigraph{
  \iftoggle{hardcore}{%
    The more I take the more I leave behind.
    What am I?
  }{%
    There is something I seek.

    While it is bound, it chooses kings and peasants.
    When it is freed, it foretells war or woe.

    While it bound, it propels men's lusts and furies.
    When it is freed, it tumbles, falls, and fades.

    While it is bound, life will often thrive.
    When it is freed, death will often follow.
  }%
}{}

\begin{multicols}{2}

\subsection{The Lowest Level}

\newRule{Team Rolls}{

  \input{config/rules/banding.tex}

  \paragraph{Example 1:}
  the troupe are sneaking down a hallway, and have to roll \roll{Stealth}{Dexterity} (\tn[9]).
  One player rolls $2D6$, and gets a `7'.
  Characters with a total bonus of +2 were walking quietly, but if even a single \gls{pc} had a bonus of +1 or less, they made a noise, and the entire group is given away.

  \paragraph{Example 2:}
  The characters try to persuade \gls{sadMan} to help them, and one player rolls the dice, getting a `7'.
  Their \roll{Charisma}{Empathy} total is -1, so they fail, but another character tries to speak with him and has a +3 total, so they succeed (without rolling the dice again).
}

\mapentry[entrycell]{Cells}

\begin{exampletext}

  Gnomes erected cots, cribs and hammocks here to be used as a communal sleeping area.
  Since then, goblins have placed a bar over the outside of the door to house prisoners.

\end{exampletext}

\noindent
Here the \glspl{pc} awaken to their hopeless situation.

\begin{itemize}
  \iftoggle{hardcore}{%
    \item
    Ask the players to mark down $1D6$ \glspl{ep} on their character sheets from the trials they've experienced getting here, and of course from their hunger.
  }{}%
  \item
  Give the players a moment to get to know each others' characters and their surroundings.
\end{itemize}

\paragraph{If anyone tries to wriggle free of their ropes,}
have them roll \roll{Dexterity}{Larceny},
\iftoggle{hardcore}{%
  \tn[11].
  Freeing another character requires a \roll{Dexterity}{Crafts} roll, \tn[7], over the space of a round.
}{%
  \tn[9].
  Freeing another character requires a full round.
}

\paragraph{After a moment,}
a small voice from the corner of the room says

\begin{quotation}
  ``No use in struggling.
  They will eat us soon no matter what we do''.
\end{quotation}

\noindent
The ogres captured \gls{sadMan} from a nearby village, along with others.
They ate the others, while \gls{sadMan} hid under debris at the side of the room.
He's learnt that cowardice means survival quickly, but won't unlearn it as fast.

\firstPrisoner

\paragraph{If anyone asks \gls{sadMan} for help,}
he refuses, and explains why they are powerless, and may as well wait to be eaten.
If they say they want to fight, he explains how large ogres stand, and how many wait above.
If they say that he should untie them so they can die fighting, then he explains that he can't trust them, and that they might throw him out the door first.
He would rather take his chances hiding in the dark.

\paragraph{Every time the \glspl{pc} give \gls{sadMan} a reason to hope,}
have them roll \roll{Charisma}{Empathy} at \iftoggle{oneshot}{\tn[9]}{\tn[11]}.
Success means that \gls{sadMan} will untie them.
A tie means he asks a follow-up question (allowing another re-roll if they can answer him well).
A failure means he disengages, muttering ``no, it won't work''.

\subsubsection*{Shortly after,}
Blara the goblin druid, and an ogre walk towards the cell.
The players should have around 5 actions (or dice-rolls) before trouble arrives.

\begin{boxtext}
  Heavy footsteps pad down the hall, you hear the door's bar being lifted, and a little goblinoid face peeps in with a torch.
  Behind her, an ogre stoops to the height of a man to avoid the low ceiling.
\end{boxtext}

The ogre then tries to pick up a character, and take it up to the kitchen.

\paragraph{If anyone wants to fight while still tied up,}
they can do so with a -4 penalty to the roll.




\Person{\npc{\N\F}{Blara the Goblin Druid}}%
  {{-2}{2}{2}}% BODY
  {{0}{0}{-4}}% MIND
  {
    \setcounter{Athletics}{1}
    \setcounter{Deceit}{1}
    \setcounter{Stealth}{2}
    \setcounter{Tactics}{1}
    \setcounter{Wyldcrafting}{2}
    \setcounter{Fire}{2}
    \setcounter{Air}{2}
    \Dagger
  }% SKILLS
  {\snapcaster}% KNACKS
  {torch, human foot\iftoggle{oneshot}{}{, map}}% EQUIPMENT
  {\ifnum\value{noAppearing}<3
    \mutation{r4}%
  \fi}% ABILITIES

\showStdSpells

\ogre[\npc{\N\M}{Olf the Ogre}]

\paragraph{If anyone attacks Blara,}
the ogre immediately turns his attention to them.

Blara will attempt to run away, but must wait for the ogre to move out of the doorway.

\iftoggle{oneshot}{
  Chasing after Blara requires a resisted roll of \roll{Speed}{Athletics}
  (the player rolls 2D6 plus their \roll{Speed}{Athletics} against Blara's \roll{Speed}{Athletics}, \tn[10]).%
}{}%

\paragraph{If Blara flees,}
she runs out, taking her torch with her.
\iftoggle{oneshot}{%
  If the fire has gone out and Blara leaves with the torch, the room becomes pitch-black.
  This gives the \glspl{pc} a bonus to any attack roll equal to their \roll{Wits}{Vigilance} +3 (the ogre cannot coordinate well in the dark).
}{}

\paragraph{If the party win the fight,}
he will accompany them out, but his nerves are too shot to be of much use.
He will not join any fights, but can hold a torch.

\iftoggle{oneshot}{}{
  \paragraph{If the \glspl{pc} search Blara's body,}
  they find a map of the upper level (see the handouts).
}

\newRule{End of \Glsentrytext{interval} Regeneration}{
  \input{config/rules/interval.tex}

  Being underground, the \glspl{pc} will only regenerate 1 \gls{mp} per interval, which goes to whichever \gls{pc} has the most \glspl{mp} lost.

  Of course the \glspl{pc} won't know exactly how long they've spent underground, but they will at least be able to count the number of resting periods they take.
}

\boxPair[t]{
  \goblin[\npc{\N\M}{Hungry Goblin}]
}{
  \morphrat[\npc{\N\A}{Morph Rat}]
}

\mapentry[escapeShaft]{The Chimney}

\begin{exampletext}
  The gnomes tunnelled down this shaft for mining, then made a ladder to easily reach the lower levels.

  More recently, the magical lift (room \ref{lift}) made it redundant, so they've placed a forge in front of the entrance to the workshop (room \ref{workshop}), since this shaft ascends above, as a chimney.
\end{exampletext}

The \glspl{pc} will probably encounter the ladder, carved into the stone, while feeling around in the dark.
Perhaps one or more will find themselves at the top of the shaft when Blara enters.

\paragraph{Anyone at the top of the ladder,}
will hear an argument between a goblin and an ogre.

\paragraph{Pushing the furnace aside}
demands a \roll{Strength}{Crafts} roll (\tn[6]), but doing so quietly requires a \roll{Strength}{Stealth} roll (\tn[13]).
A tie means the horde stop bickering to focus on the noise, while failure indicates that they realize that the forge has a tunnel behind it, and two ogres come out to investigate (and likely try to eat a character).

\paragraph{If someone wants to continue climbing up,}
they can't.
The rest of the ascent has nothing to hold onto, and becomes smaller as it ascends.

\mapentry[diningRoom]{Dining Room}

Every morsel has been licked clean from this little dining room.
Pots, smashed plates, forks, and candlesticks litter the floor.
A goblin and a giant rat bicker over a human bone.%
\footnote{\Gls{kalama} cast a spell upon the rat, transforming it into a beast, in order to distract the goblins.
  Unfortunately, it just fit right in.}

On the table rest two large knives which can be used as \emph{daggers}.
\iftoggle{oneshot}{%
  Daggers grant +1 Damage in combat, but no Bonus to hit.
}{}

\begin{boxtext}
  Smashed-up chairs and broken cutlery surround a low dining table, lit by a dribbling candle.
  On the table, a little bleach-white goblin wrestles with a rat-like creature, the size of a dog.
  The ugly pair fight over a human leg.
\end{boxtext}

\paragraph{If \pgls{pc} sneaks up quietly,}
have them roll \roll{Dexterity}{Stealth} (\tn[9]).
They should get a -2 Penalty for taking a torch.

A tie means they must retreat, while the goblin investigates their room.
Failure means the goblin and rat attack together.

\iftoggle{oneshot}{
  \paragraph{If the \glspl{pc} fight,}
  they take 1 \gls{ep} for the strenuous activity.
}{}

\mapentry[spellCasters]{Spellcaster Arguments}

\begin{exampletext}
  A single gnome has returned, disguised by magic as a goblin.
  Unfortunately, he does not speak the goblin language, and everyone soon noticed.
\end{exampletext}

Two druids interrogate \gls{kalama} while he can only shrug in confusion.

The scene either begins with the goblins slowly realizing that the \glspl{pc} have escaped, or by interrogating \gls{kalama}.
If the \glspl{pc} don't intervene, the druids will kill him.

\begin{boxtext}
  From the top of the stairs, you see a goblin, clothed only in leather satchels, holding another goblin on the ground, and yelling at him.
  The goblin on the ground looks round, and\ldots winks at you?
\end{boxtext}

\paragraph{If the \glspl{pc} have already made a ruckus,}
the spellcaster will run away, then cast spells from a short distance; \gls{kalama}, the disguised gnome, will flee immediately; and every goblin in the kitchen will wake up, and run through, at a rate of two per round.

\iftoggle{hardcore}{
  \goblin[\npc{\T[3]\N}{\arabic{noAppearing} Goblins}]
}{}

\goblincaster[\npc{\F\N}{Hunch, Goblin Druid}]

\iftoggle{oneshot}{
  \paragraph{If the \glspl{pc} have not made a noise, they can gain a surprise attack,}
  they make a Group Roll of \roll{Dexterity}{Stealth}, against the goblin druid's \roll{Wits}{Vigilance} (\tn).
  As before, one player rolls $2D6$, and each \gls{pc} who wants to sneak out adds their own Bonuses to that roll -- but if a single \gls{pc} fails, everyone fails.
}{}

\iftoggle{oneshot}{
  \paragraph{If the \glspl{pc} fight,}
  give each \pgls{ep}.
}{}

\warrenMap

\paragraph{If \gls{kalama} finds a peaceful moment to speak,}
he explains his whole situation, and his mission to find the children (in room \ref{nursery}).
He has become so exhausted that he will be of little use to anyone without a meal and a rest.

\spyGnome{2}

\paragraph{If the characters investigate the table,}
they find the \gls{talisman}, `\lootTalisman', left by the goblin druids.

\showTalisman

\labyrinthScroll

\showTalisman

\paragraph{If the \glspl{pc} want to use a scroll,}
they will have to answer its riddle.
Go to \autoref{riddles} and select a riddle which activates that scroll.

\paragraph{If the \textit{\Gls{deep} Scroll} is activated,}
it begins to shimmer with golden flecks, then blocks a section of the hall, turning into a gateway to a deep underground world.
See \nameref{readingScrolls} (\vpageref{readingScrolls}) for details on the calamities which ensue after reading the \Gls{deep} Scrolls.

\mapentry[kitchen]{Kitchen}

\begin{exampletext}
  Goblins with a curious side tried to operate the kitchen, and cook a feast.
  Half-cooked human limbs testify that goblins like their men rare.
\end{exampletext}

\begin{boxtext}
  Peering into the darkness in the alcove, you can just make out the four figures sprawled over tables and chairs or curled up on the oven, snoring contentedly.
  From the embers of the hearth you can see the ovens and cooking utensils that make up a substantial kitchen, and silhouettes of human femurs and skulls.

  Entering further, you see that the four figures are more goblins with little fat bellies.
  At the end of the kitchen you can see a large door to a cold store with a lock on it and two large cleavers stuck into a butchers block.
\end{boxtext}

\paragraph{If the \glspl{pc} have avoided making a lot of noise nearby,}
then the goblins are asleep in the kitchen, and they can sneak in with a \roll{Dexterity}{Stealth} roll (\tn[8]).%
\exRef{core}{Core Rules}{sneakattack}

\iftoggle{oneshot}{%
  As before, the Group Roll only requires one player to roll, and any character which enters the room produces a different result.
}{}

\newRule{Weapons}{

  \input{config/rules/weight}

  Without a backpack, \glspl{pc} can only carry two items -- one in each hand.

  \input{config/rules/weapons.tex}

  The \glspl{pc} can find all these items in room \ref{kitchen}.
  They don't make the best weapons, but the \glspl{pc} have little choice!

  \begin{boxtable}[YlYYYY]

    \textbf{No.} & \textbf{Name} & \textbf{Attack Bonus} & \textbf{Damage Bonus} & \textbf{\glspl{ap} Cost} & \textbf{Weight} \\\hline

    1 & Broom & 1 & 0 & 2 & 1 \\
    2 & Cleaver & 0 & 2 & 2 & 2 \\
    4 & Small \showWeapon{\chair} \\
    2 & \showWeapon{\Dagger}  \\
    2 & \showWeapon{\skillet} \\
    4 & \showWeapon{\Log} \\
    1 & \showWeapon{\woodaxe} \\

  \end{boxtable}
}

\goblin[\npc{\N}{Goblin on the table}]

\iftoggle{hardcore}{
  \goblin[\npc{\T[4]\N}{4 Goblins on the floor}]
}{
  \goblin[\npc{\N}{Goblin on the floor}]
  \goblin[\npc{\T[2]\N}{2 Goblins on the oven}]
}

\paragraph{If the roll fails,}
all goblins wake up with hungry stomachs.

\paragraph{If the players try to find weapons,}%
\iftoggle{oneshot}{
  they can find plenty of make-shift weapons around the kitchen.

  These weapons may not be good quality, but they can still improve the \glspl{pc}' situation immensely.

  \paragraph{If the players select weapons for their characters,}
  check out the Kitchen Weapons table and have them write down the stats on their character sheet one at a time.
 }{
  they can use either of the two cast iron skillets.
}

\widePic[t]{Roch_Hercka/dragon}

\paragraph{If the party raid the room for food,}
they'll find a few canteens of water, one of wine, and sacks vegetables (enough for 4 meals).

The players can unlock the larder with a \roll{Dexterity}{Larceny} roll at \tn[10] to pick the lock, at which point they discover that the larder is still full of food.

\mapentry[lift]{The Gnomish Lift}

The gnomes created the lift with the Force Sphere.
It can lift a combined \gls{weight} of 17, and safely descend with a total \gls{weight} of 34 standing on it.%
\footnote{A creature's \gls{weight} equals its \glspl{hp}, plus half the value of its equipment.}
If the party step on it with a greater \gls{weight} than this, each additional point inflicts 2 \glspl{ep} on everyone in the lift upon impact with the ground.

The lift responds to magical passwords -- the gnomish words for `farm' (for the top), `work' (for the middle), and `cook' (for the bottom).
The ogres and goblins who use the lift only know the passwords for the bottom and middle sections (Blara extracted this information from a gnome before he died).%
\footnote{One goblin druid extracted the password for the top, but has kept this secret.}

The lift responds to any password spoken while standing on it.

\begin{boxtext}
  The great double doors swing open, revealing a wide, empty room.
  Your torchlight stretches far above, and well out of reach you can see a wooden ceiling.
  The room appears otherwise empty.
\end{boxtext}

\paragraph{If the \glspl{pc} ask \gls{kalama} for the password to the top,}
he refuses to help until the \glspl{pc} save the gnome-children (room \vref{nursery}).

\paragraph{If anyone tries to climb the walls,}
they will find less purchase than a Sun-screen salesman in a Scotland.

\paragraph{If the \glspl{pc} dawdle too long here,}
a `raiding party' return with more prisoners to place in the lower cell (room \ref{entrycell}).
See \autopageref{raidingParty} for details on the raiding party.

\newRule{Projectiles}{
  \begin{itemize}
    \item
    Players roll \roll{Dexterity}{Projectiles} against \tn[6] to hit targets.
    \item
    Every 5 steps adds +1 to the \gls{tn}.
    \item
    When a player hits the \gls{tn} precisely, they miss their first target, but hit any other target behind.
    \item
    Shortbows deal only $1D6-1$ Damage.
  \end{itemize}
}

\mapentry[dragonApproach]{Dragon's Approach}

\begin{exampletext}
  When the horde attacked, one gnome decided to activate a \Gls{deep} Scroll in order to flee.
  Like the others, it targeted an unknown area, full of powerful magic, deep underground.
  Unlike the others, it held a sleeping dragon, who woke, and sauntered through the magical gateway.
  He stomped on a couple of goblins, then followed the smell of gold and magic, to the treasure room.
  However, the massive monster could not fit through the door.

  When goblins came up the stairs to investigate, the dragon incinerated them with his fiery breath.
  So now he waits, with the legendary patience of a dragon.
  He has no intention of fighting, since he could get hurt, and the goblin horde have no intention of bothering him, so they just pass by that corridor.

  The dragon and the horde have ended in a kind of stalemate.
  And just like the horde, he cannot leave, as he does not know how to read the \Gls{deep} Scrolls\ldots
\end{exampletext}

\begin{boxtext}
  Ahead, three charred goblin corpses lie on the ground.
  A strange scent wanders down from above, something like a chicken cooked in rotten eggs.
\end{boxtext}

\paragraph{If the \glspl{pc} loot the bodies}
they find one \Gls{deep} Scroll wrapped safely in a scroll case.

\mapentry[dragon]{The Dragon's Lair}

The dragon will happily talk with anyone who approaches%
\iftoggle{hardcore}{}{
  The dragon's eventual goal is to obtain the rest of the treasure, then leave the \gls{warren}, and go somewhere he can spread his wings.
  Like all dragons, he has more cunning than ferocity, and \emph{plenty} of both.
  He will look for opportunities to turn the \glspl{pc} against each other, and reframe all conversations around the assumption that he will soon leave, with most of the treasure, through a \gls{deep} scroll, and that the \glspl{pc} should view this as their best possible outcome.

  Despite his cunning, he will agree to worse terms if he has to.
}

\paragraph{Revealing a \gls{deep} scroll}
gains no reaction.
\Gls{makil} knows he has to play it cool, and not let anyone know that he really needs that spell to leave this place.
He \emph{might} fit through the narrow passages up, but he would be in a dangerous situation, being cramped in a narrow corridor, with ogres everywhere.

\paragraph{If the party request he kill goblins for them}
then he agrees to kill any in the current area, but will not journey to another floor.
In return, he wants all the treasure out of the treasure room.

\paragraph{If the party offer to split the treasure,}
he refuses, unless they give him a good reason.

\iftoggle{hardcore}{
  The treasure chinks noisily of course, so the party will then receive a -1 penalty to all Stealth checks while carrying it.}{}

\paragraph{If the party push for more treasure,}
the dragon asks if they would like to challenge him to a game of riddles.
Each point anyone scores allows them to demand a single item, such as a chest, or quiver.

If they say `yes', then he accepts their challenge, and asks what their riddle is.%
\footnote{Use of the internet is prohibited by trans-dimensional law, common sense, basic decency, and the Geneva Convention.}
If they can think of none,
then the dragon declares that he has won the first round.

\paragraph{The rules for riddles}
are simple -- any question which someone has the knowledge to answer is a fair riddle.
Asking `how many letters in the Greek word for ``mushroom''?', is not a fair riddle, because someone may not know.

Any possible answer to a riddle is `the correct one'.
If someone asks `what is black and white and read all over', anything which fits all descriptions must be accepted as an answer.

See \autoref{riddles} for riddles.

\dragonMakil

\showStdSpells[
  \setcounter{diceNo}{0}
  \input{config/spells/Mind3.tex}
]

\paragraph{If the players ask why he wants treasure,}
he explains that he wants to attract a mate; when his flame becomes hot enough to melt the gold, he will carve a golden statue of the most deadly dragon in his area in order to attract her attention.
He will then decorate the statue with magical items.%
\footnote{Should the ideal mate be pretty, or have the power and aggression to destroy an entire town?
Dragons think the answer is obvious.}

\paragraph{If the dragon parts on good terms,}
he blesses them all, restoring any lost \glspl{fp}.

\iftoggle{oneshot}{
  \paragraph{If the \glspl{pc} attack the dragon,}
  he kills the first to attack.

  His {\scshape \gls{dr} 5} means he reduces all Damage by 5, unless the attacker hits 5 over the \gls{tn} to attack him (a total of 15), achieving a `Vitals Shot'.
}{}

\noteRaidingParty

\mapentry[treasureRoom]{Treasure Room}

\begin{itemize}
  \item
  A chest containing 432 \glspl{cp}%
  \footnote{Every 100 coins has a \gls{weight} of 1.}
  \item
  A chest containing 300 \glspl{sp}
  \item
  An ivory short bow (with string, but no arrows)
  \item
  A small backpack (can hold up to a \gls{weight} of 3)
  \item
  A buckler shield made of pure silver, worth 30~\glspl{sp} (it breaks after one use)
  \item
  Two gem-encrusted shortswords (worth 4~\glspl{gp})
\end{itemize}

\begin{boxtext}
  Through the door, two locked chests lie on the ground.
  Above them, a shortbow and two beautiful short swords stand affixed to the wall, with a quiver of arrows with gemstones used as arrow tips.
\end{boxtext}

\subsection{Mid Levels}

\iftoggle{hardcore}{
  \mapentry[goblinSentry]{The Sentry}

  Svart, the goblin, stands at the door to room \ref{nursery} (below), attempting to pick the lock.
  \paragraph{When the \glspl{pc} exit the lift door,}
  he runs over, hoping to beg some food from the ogres he expects to emerge.
  Once he sees the \glspl{pc}, he runs for the stairs, shouting for help.

  \goblin[\npc{\M\N}{Svart}]
}{}

\mapentry[nursery]{Nursery}

\begin{exampletext}
  When the horde arrived, a gnome locked the children in here with enough food for a few days.
\end{exampletext}

The door is locked, but can be picked with a \roll{Dexterity}{Larceny} roll (\tn[9]) if anyone has some lock-picking tools.
It's far too strong to be broken into by force.

If the \glspl{pc} enter the room, they may think these little gnomes are little goblins and kill them.
If any of them try to do so, have them roll a \roll{Intelligence}{Medicine} (\tn[8]) check to realize their mistake.
Of course, any gnome can identify the children immediately.

\begin{boxtext}
  The sound of crying emanates from the door as you swing it open.
  In this cramped room, a dozen infants with fat little noses lie in a crib with hay, looking up at you in terror.
  The tiny room stinks of shit.
\end{boxtext}

\paragraph{The moment the \glspl{pc} approach the nursery door,}
a goblin falls onto room \ref{lift}'s lift and states the command word to bring it up to the top floor.
This goblin is Grank, and he is the only goblin who knows the password for the top.

\paragraph{If the \glspl{pc} take the children,}
they know to know to keep quiet and follow, but will only keep quiet if the party succeed on an \roll{Intelligence}{Empathy} roll (\tn[10]).

\swarm{Gnomish Children}{5}{-2}{-3}{-5}

There are five children in total, and they each have a \gls{weight} between 1 and 2, but can generally be treated as a single `swarm'.%
\exRef{core}{Core Rules}{swarms}

None can walk, so the troupe must carry an additional \gls{weight} of 5, distributed as they please.
To further complicate matters, the gnome-children must go \emph{somewhere}, such as a bag.
Characters might also construct a baby-wrap out of clothes with an \roll{Intelligence}{Crafts} roll (\tn[10]).
Rolling a tie indicates the wrap does not work, while rolling a failure means the wrap works until the character tries to run\ldots

\mapentry[slugHall]{Slug Hall}

\begin{exampletext}
Gnomes grew mushrooms throughout this room in order to grow slugs, which fed fireflies.
While torches work better than fireflies, having omnipresent little lights wandering around the \gls{warren} made sure that people could get about easier.

Once the goblin druids arrived, they cast life-engorging spells to grow the slugs to monstrous proportions, in order to ensure that prisoners don't escape past this point.
Salt covers the stairs, preventing the slugs from moving upwards.
The fireflies continue nipping at them, and have quadrupled their population.
The slugs don't make good guardians, but at least the halls have plenty of light.
\end{exampletext}

\boxPair{
  \label{workshopGoblins}
  \goblin[\npc{\T[\arabic{enemyNo}]\N}{\arabic{enemyNo} Goblins}]

  \addtocounter{enemyNo}{-2}

  \ogre[\npc{\T[\arabic{enemyNo}]\N}{\arabic{enemyNo} Ogres}]
}{
  \pic{Decky/armoury}

}

\begin{boxtext}
  The doorway reveals a massive hallway of sparkling, floating, gently-buzzing, lights.
  Over to the left, a massive staircase leads up into the darkness.
  And ahead, the little lights gently illuminate giant slugs, feasting on corpses and torn-up books.

  The moment you enter, the slugs' eye-stalks perk up, and they begin to slide off the corpses they were feasting on, and approach\ldots
\end{boxtext}

\paragraph{If the \glspl{pc} run up the stairs,}
have them roll \roll{Speed}{Athletics} (\gls{tn} \arabic{track}) to avoid the acidic spray form the slugs.

From that point until the slugs loose sight of them, they are in combat.
The slugs will spray acid at them, follow them up the stairs, and pester them for as long as they remain in sight.

\morphslug[\npc{\T[9]\A\N}{20 Morph Slugs}]

\paragraph{If the \glspl{pc} throw in some food,}
the morph-slugs head\ldots slowly\ldots towards it rather than fight.

\paragraph{If the \glspl{pc} want to investigate the corpses,}
they find a dead gnome with a \gls{deep} Scroll, and a discarded, half-eaten cloak, with a key to room \ref{nursery} in the pocket.%
\footnote{See \autopageref{saving_the_children} for how the key arrived here.}
\iftoggle{oneshot}{%
  As with the others, the \Gls{deep} Scroll requires a riddle to be answered, takes four rounds before it activates, and vanishes once used.
}{}

\paragraph{If the \glspl{pc} remove all the salt from the stairs,}
the slugs ascend\ldots slowly.
However, the resulting battle will kill three ogres, and all morph-slugs.

\mapentry[greatHall]{The Great Hallway}

If the \glspl{pc} have indeed been quiet enough in the previous room to not raise an alarm, they find everyone in the neighbouring rooms napping.
A single sound means they will be in serious trouble.

The greasy floor results from a mixture of faeces, drool, blood and leftover mushroom-juice.

\begin{boxtext}
  At the top of the stairs, this massive chamber lies empty, except for the fireflies darting about, and some human bones on the filthy floor.
  To the right, there two short tunnels reverberate with assorted snoring sounds.
  Ahead of you, two grand staircases lead up into a dim but unwavering light.
  Below you, the entire floor is sticky and greasy.
\end{boxtext}

\paragraph{If the \glspl{pc} have come from room \ref{slugHall},}
they will not see the staircase on their left immediately, but \emph{will} see it after doing literally anything (fighting, searching, et~c.).

\paragraph{If the \glspl{pc} tarry or talk,}
have them roll \roll{Intelligence}{Stealth} (\tn[8]), to get across the sticky floor without squelching too much.
\iftoggle{oneshot}{%
  They make this as a \textit{Group Roll}, so a single roll counts for the whole group.
  Each margin on the roll allows them an additional round before the horde wakes.
}{}

\iftoggle{hardcore}{
  If a single \emph{player} clearly gives in-character advice to another, the horde awaken.
}{}

\mapentry[workshop]{The Workshop}

Picks, shovels, backpacks, wood, short swords, shortbows, and all manner of crafting and mining equipment litter the room.

\begin{boxtext}
  Some goblins and three ogres lie sleeping on the floor between workbenches.
  The place is so full, you can't make out how many lie here, but the snoring indicates more than you can see.
  On the benches, most of the equipment lies broken, but delicate gnomish hands once used these tables to polish gems, craft magical items, and forge digging equipment.
  On one table, you can see a pile shortswords and spears.
\end{boxtext}

Find the goblins stats \vpageref{workshopGoblins}.

\paragraph{If the \glspl{pc} have come up from room \ref{escapeShaft},}
the reaction depends entirely on how much noise they made while pushing the great forge aside.

\paragraph{If the \glspl{pc} attempt to take either a short sword or a spear,}
each attempt requires a \roll{Dexterity}{Stealth} roll, \tn[7].
Failure will awaken the entire horde, while a tie allows them a good head start.

\mapentry[grandLibrary]{The Grand Library}

The goblin druid had been investigating \pgls{talisman} -- \lootTalisman.
It remains on the ground beside him.

\showTalisman

\goblincaster

\labyrinthScroll 
The goblin druid holds a \spellName\ in her sleepy hands.

\iftoggle{hardcore}{
  \ogre[\npc{\N\T}{2 Docile Ogres}]
}{
  \ogre[\npc{\N}{Sleeping Ogre}]
}

\begin{boxtext}
  At the top of the stairs you find the ruins of a massive library.
  Book cases lie in a smashed heap on the ground, others appear to be used as a makeshift bed for an ogre.
  The books themselves are gone, except for a scroll, now tightly clutched by a goblin in a black cowl.
\end{boxtext}

\paragraph{If any of the \glspl{pc} attempt to sneak in,}
have them roll \roll{Dexterity}{Stealth}, \tn[8].

Failure will, of course, spell disaster, but success will allow them to steal a magical item.
If the player wants to steal multiple magical items, describe them and see how many they decide to take.
Each item taken increases the roll's \gls{tn} by 1, so taking 3 items would mean a \gls{tn} of 11.
The player should not roll again -- the original roll remains, but increasing the \gls{tn} may well turn success into awful failure.

\iftoggle{oneshot}{%
  \columnbreak
  \subsection{Last Level}
}{
  \subsection{Upper Warren}
}

\mapentry[windingStairs]{Winding Stairs}

A single ogre guards the prisoners here (the door has no lock).

\begin{boxtext}
  As you round the stairs' third turn, you see a massive ogre crouching by a door, blocking the path upwards.
\end{boxtext}

\paragraph{If the party have made a reasonable attempt at staying quiet,}
they can avoid alerting this ogre with a \roll{Wits}{Stealth} roll, \tn[9].
Whoever is at the front makes the roll.
If it's unclear who's at the front, the character with the highest \roll{Speed}{Athletics} is in the lead.
With a successful roll,
\iftoggle{hardcore}{
  the ogre is resting, and must take a round to gather what's left of his wits, but it will still wake if approached.
}{%
 the party find the ogre sleeping.
}

\ogre[\npc{\N\M}{Rick, the Ogre Guard}]

\mapentry[secondPrison]{Second Prison}

\begin{exampletext}
  This little room once housed a full family of gnomes, but now serves only as another prison.
\end{exampletext}

The prisoners require no locks or handcuffs -- the ogre waiting outside suffices to terrify them into staying put.

\paragraph{If any of the \glspl{pc} have died,}
introduce another \gls{pc} here%
\iftoggle{hardcore}{.}{, from the pool.}

\iftoggle{hardcore}{%
  \paragraph{If the party get a moment to ask about the outside world,}
  the ex-prisoners inform them that all the \pgls{village} for some miles have fallen, and the \gls{dayton} lies unprotected.
  Even if the goblins die, the town will be in trouble.
}{}

\mapentry[armoury]{Armoury}

\begin{exampletext}

  The gnomes once stashed their little weapons here.
  The horde have added to it considerably.

\end{exampletext}

The \glspl{pc} see the following:

\begin{itemize}

  \item{3 buckler shields}
  \item{1 crossbow (unstrung but usable with an \roll{Wits}{Crafts} roll, \tn[7])}
  \item{3 quivers, each with 20 arrows (for a shortbow)}
  \iftoggle{hardcore}{%
    \item{3 crossbow bolts}
    \item{2 shortbows (also unstrung)}
  }{
    \item{8 crossbow bolts}
    \item{2 shortbows}
  }
  \item{3 shortswords}
  \item{7 wood axes}

\end{itemize}

\begin{boxtext}
  At the top of the stairway, three dying fireflies wander pointlessly.
  Behind them, a cluster of shadow in an alcove holds metallic glints.
  To the left, dirty little footprints lead out a little wooden door.
\end{boxtext}

\iftoggle{oneshot}{
  \paragraph{Bucklar Shields}
  These shields work like a weapon, except that they deal no damage.
  They require only 2 \glspl{ap} to use in combat, but add a +2 Bonus.

  \paragraph{The crossbow}
  (if repaired) deals $1D6+3$ Damage, but requires at least 4 rounds to reload.

  \paragraph{Shortbows}
  Require 1 \glspl{ap} to loose an arrow, and 1 \glspl{ap} to reload.
  However, they deal only $1D6-1$ Damage.

  \begin{boxtable}[XYYYY]
  \label{armouryWeapons}

  \textbf{Name} & \textbf{Attack Bonus} & \textbf{Damage Bonus} & \textbf{\glspl{ap} Cost} & \textbf{\gls{weight}} \\\hline

  Bucklar Shield & +2 & None & 1 & 1 \\
  \showWeapon{\shortsword} \\
  \showWeapon{\spear} \\
  \showWeapon{\woodaxe} \\

  \end{boxtable}
}{}

\newRule{\Glsfmtlong{dr} \& Vitals Shots}{

  \input{config/rules/armour.tex}

  \Gls{dr} reduce Damage taken due to armour, or just a creature's thick hide.
  The morph-wolves have `{\scshape \gls{dr} 2}', so they remove 2 from any Damage Taken.

  We assume everyone is trying to target sensitive areas, like the throat, groin, and eyes, whenever they can, to hit more exposed areas.
  If anyone hits 5 points over what they need to hit the wolves, they get a `\textit{Vitals Shot}', which ignores the \gls{dr} entirely, as the attack hits a sensitive area.
}

\mapentry[inDoor]{The Two-Way Doors}

\begin{exampletext}
  When gnomes rushed busily around the \gls{warren}, taking mushrooms to the kitchen, chamber-pots to the mushrooms, and secrets to fellow conspirators, they sometimes bumped into each other.
  To stop this happening again, they designated this door as `out', and the other as `in'.
\end{exampletext}

The \glspl{pc} will see the lift's double-doors straight ahead of them (assuming they have any light at all).

\begin{boxtext}
  The dark hall reveals one passage to the far right.
  On your left, smooth, man-sized, double-doors stand without any handle, lock or other feature.
\end{boxtext}

\paragraph{Opening either door from the wrong side}
requires a \roll{Strength}{Larceny} roll, \tn[10].
Once it shuts, it shuts.

\mapentry[outDoor]{The Out Door}

This door opens the opposite way from the last, and also demands a roll of \roll{Strength}{Larceny} (\tn[10]) to open from the wrong side.

\mapentry[topShaft]{The Top of the Shaft}

\widePic{Roch_Hercka/garden}

Grank the goblin druid has heard the \glspl{pc} coming, and has no intention of fighting them alone.
He knows he has the only key to the exit door (room \vref{lowerExit}), so he intends to loose the morph-wolves he has tied up at the room's side before fleeing into the fungal gardens.

\paragraph{If the party attempt to run through the lift's double doors,}
they might suddenly find an empty lift-shaft.
In this case, have them roll \roll{Wits}{Athletics} (\tn[7]) to back off before they fall.
A drop from this height inflicts $1D6+1$ Damage, plus their Strength Bonus.%
\footnote{Larger creatures receive more Damage from falls}

\morphwolf[\npc{\T[4]\A\N}{\arabic{noAppearing} Morph Wolves}]

\paragraph{If the party fall into the lift,}
they end up in the mid-section of the \gls{warren}.

\mapentry[fungusGarden]{Fungal Gardens}

\begin{exampletext}
  This beautiful fungal garden took dripping rain from above, and sieved it through the roof then the soil below, until it distributed nutrients for a forest of mushrooms, big and small.
  The fungal garden was regularly invaded by oozes which can creep into small cracks when young, and grow massive quickly.
  The goblins never really kept up with the garden's maintenance, so the room festered with dangerous jellies.
\end{exampletext}

While the place looks serene, it is inhabited by
\iftoggle{hardcore}{%
  dangerous oozes.
}{%
  a dangerous ooze.
}

\paragraph{Once the players enter the room,}
\iftoggle{hardcore}{%
  the oozes begin to stalk them.
  If multiple oozes chase them, the smaller ones will always back away from the larger ones, so no more than one ooze should follow them at a time (always the largest one).
}{%
  the ooze begins to stalk them.

  \paragraph{Investigating the green glows}
  reveals little patches of fluorescent mushrooms.%
  \exRef{judgement}{Judgement}{glowshroom}
}

\paragraph{If the \glspl{pc} approach Grank,}
he will hide while casting spells.

\paragraph{If Grank ever feels like his life is under threat,}
then he will taunt the \glspl{pc} with the key to the outside world he has in is position, and throw it into the nearest ooze.
He then lets out a giggle and dashes off into the fungal undergrowth, leaving the players to face the hulking, pulsating, mass.

\goblincaster[\npc{\M\N}{Grank}\label{grank}]

\setcounter{diceNo}{1}
\showStdSpells

\iftoggle{oneshot}{}{
  \jelly
}

\iftoggle{hardcore}{
  \jelly
}{}

\jelly

\mapentry[lowerExit]{The Exit}

\iftoggle{oneshot}{
  The \glspl{pc} have found the exit, and can finally leave the nightmare behind.

  \begin{boxtext}
    As the key turns, the door swings open, and daylight floods in.
    Green trees cover the road down the hill, and in the far distance, chimney-fires from little hamlets wander into the sky.

    You have finally escaped from the \gls{warren}.
  \end{boxtext}
}{
  This is where the party exit the lower portion of the \gls{warren}, and enter the upper.

  \paragraph{If the \glspl{pc} don't have a key,}
  they might try tricking the goblin on the other side, who has a key to let people in and out.
  See room \vref{entrance}.

  \paragraph{If the \glspl{pc} examine the door,}
  they can see the goblin in the next area by a glimmer of candle-light.
}


\iftoggle{oneshot}{
  \subsection{Rewind}

  Remember to congratulate your players on a tough journey.
  Give them a moment to breathe.
  Summarize any clever plans or unexpected outcomes that happened.

  And if you've just finished reading the module for the first time, remember to come back later, and give the module a second read-through before running it.%
  \footnote{Did you spot the key hiding in an image?}
  \pic{Decky/screech}
}{}

\end{multicols}
