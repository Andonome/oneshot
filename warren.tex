\chapter{The Warren}

\begin{figure*}[t]
	\includesvg{images/Dyson_Logos/lower}
\end{figure*}

\section{The Lowest Level}

\begin{multicols}{2}
\setcounter{list}{0}

\mapentry{Cells}
\label{entrycell}

\begin{exampletext}

	Gnomes erected cots, cribs and hammocks here to be used as a communal sleeping area.
	Since then, goblins have placed a bar over the outside of the door to house prisoners.

\end{exampletext}

This is the room where the PCs awaken.
Run the first event in the time line -- \textit{the Escape} (page \pageref{escape}).

\paragraph{If the PCs make a lot of noise}
then the goblins down the hall are coming for them, and they had better move fast.
Make it clear the party have little chance of success in a head-to-head combat, so they should run the other way.

\paragraph{If the PCs exit quietly,}
they have free reign to explore.

\paragraph{If the party do not manage to escape}
then add 3 Fatigue Points for their time starving in prison, and throw in a load of new prisoners.

Once new prisoners are in, the players can spend Story Points to introduce old friends, and have \textit{them} roll to escape from bondage.

\mapentry{Exit Shaft}

\begin{exampletext}

	This exit shaft was used by gnomes to travel up quickly.
	Aware of the dangers below, they trapped the ladder and hid the door at the top.
	Since then, the goblins have forgotten how the ladder works, and stopped using it.

\end{exampletext}

If the players attempt to climb, they find that every rung of a multiple of five (the fifth, tenth, fifteenth, and so on), are coated in a slipper substance made from the fungal gardens.
The rungs which are a multiple of 7 are set to break once anything with a Weight Rating 7 or more will break the rung.%
\footnote{A creature's Weight Rating is equal to its HP.}
These breakable sections are actually solid illusions, so they appear again shortly after breaking, but will resist any markings such as paint, as those marking disappear as the illusion resets every so often.
Meanwhile, any rung which is both a multiple of 5 and 7 indicates that the next section is invisible and has switch to the wall on the left.

Despite the ladder switching places, it appears to be one solid piece, reaching upwards.

The nura have since abandoned this path, and have forgotten the door at the top.

\paragraph{If the players climb up the staircase without understanding the trap,}
have them roll Dexterity + Athletics, TN 10.
If they fall, they receive $1D6-2$ Damage.

If they continue climbing, give additional Dexterity rolls.
The character can then make an Intelligence + Academics roll to understand the trap, TN 12.
Whatever number they roll, keep that result; it will tell you how long the character takes to figure out the trap.
Every time they make a new roll:

\begin{enumerate}

	\item{The Damage they get for failing increases by 1.}
	\item{The roll to understand the trap decreases by 1.}

\end{enumerate}

Of course, if the player figures out the trap, all rolls can be dispenses with, and the player can ascend to room at the top, on page \pageref{laddertop}.

\mapentry{Dining Room}

\begin{boxtext}

	As you climb the stairs away from your cell you enter a room full of benches and tables, all overturned and broken. 
	The floor is strewn with pots and plates, knives, forks and candlesticks, but no candles and not of scrap of food remains.
	It looks like a pack of ravenous dogs swept through here and licked ever morsel from every surface.
	In the middle of the room you can see upon a table two creatures -- a goblin and the other you would call a rat, but only if the rat had been mated with a Great Dane.
	The two creatures seem to be fighting over something and as you focus on the scene before you it becomes clear that what they are fighting over is a bone (possibly human).
	It looks at first like they are pulling at it, one on each end, but then you hear the sound of grinding and snapping and you realise that they are both just chewing at either end.

\end{boxtext}

\paragraph{If the players make a sound}
the goblin will stop fighting with the rat for the ownership of the bone and run out of the room shouting for the shaman who had left them in the cell previously.
There are two knives in this room that can be wielded as daggers.

\paragraph{Whatever happens,}
the rat will swallow the remaining bone whole, then attack.

\goblin[\npc{\F\N}{Hungry Goblin}]

\nurarat[\npc{\A\N}{Nura Rat}]

\mapentry{Magic Portal}

\begin{boxtext}

	In this room there stands before you a Magical portal 8' high and 4' wide.
	A warm wind blows through the portal and on the other side you can see an endless desert of bright, yellow rocks.
	Runic writing sits above the portal.
	The room seems empty except for the scuff marks leading to and from the portal, and the massive double-doorway behind you. 

	The passageway continues, but becomes quickly lost in darkness as it loses the Sunlight through the portal.

\end{boxtext}

\paragraph{If any of the PCs are literate and speak the Gnomish tongue,}
they can read the runes above the magical portal; they read ``Desert Realm''.

\paragraph{If the PCs enter the portal and journey through the Realm of Bright Rocks,}
they won't find much -- not even a place to hide, given the desert is expansive and flat.
If they stay there long, give them the usual 4 Fatigue Points and roll an encounter for that Realm.%
\footnote{See Adventures in Fenestra for more on the Realm of Bright Rocks.}

\mapentry{Spellcasters}

\begin{boxtext}

	Two goblins wearing long black robes bicker with each other over a little table.
	Other goblins stand at the side, apparently bored of the argument.

\end{boxtext}

\paragraph{If the players try to sneak past,}
the roll is Dexterity + Stealth, TN 10.
They will also have to put out their torch, if they have one.

\paragraph{If the characters kill the nuramancers and search the bodies,}
they find a Portal Scroll,%
\footnote{This scroll opens a portal to the Realm of Shifting corridors. See Adventures in Fenestra for more on it.}%
 and a Scroll of Uprising.

\magicitem{Scroll of Uprising}{Ghoul Calling}{Saurecanta}{Instant}{Pocket Spell}{+3}{7}

Once the spell is cast, all creatures with 11 HP or fewer in the surrounding 3 areas, raise from the dead as ghouls.
See Appendix \ref{undeadstats} for stats on the various ghouls.

\magicitem{Portal Scroll}{Unrestrained, Open Teleport}{Alchemy}{2 Scenes}{Pocket Spell}{4}{5}

The scroll takes 5 rounds to speak properly, and opens a doorway to the Realm of Shifting Corridors.%
\footnote{See Adventures in Fenestra for more on this realm.}
The portal remains open for 2 scenes.


\goblinnuramancer[\npc{\F}{Screamer}]

\goblinnuramancer[\npc{\M}{Brock}]

\goblin[\npc{\T}{3 Goblins}]

Traps here were trashed.

Supplies:

- Portal scroll
- Staff (wand) of blinding light held in paper to limit the light
- Bowl of Water (summons water)
- Scroll of Insight (feel the entire dungeon level around you)
- Scroll of Locking
- Ring of wishes (multiple conjuration spells)
- Cloak of shadows (illusion of matching nearby object)

\goblinnuramancer

\goblin

\mapentry{Kitchen}

\begin{boxtext}

The door creaks open and as you peer into the darkness you can just make out the 5 figuers sprawled over tables and chairs or curled up on the floor snooring quietly. 
From the embers of the hearth you can see the ovens and cooking utensils that make up a substantial kitchen. 
Opening the door further, the light from the hall's sconces fills the room, and you can see that the five figures are more goblins with little fat bellies.
At the end of the kitchen you can see a large door to a cold store with a lock on it and some large cleavers stuck into a butchers block.

\end{boxtext}

\noindent
In the room the players can find two cleavers which would be the equivalent to short swords and two large pot lids which could be used as bucklers.
There is also a sixth goblin in the room hiding in the far corner in the dark; players can make a Wits + Vigilance roll at TN 8 to spot him.
Should they make to much noise or light up the room, all goblins will wake up and attack.
The players can unlock the cold store
\footnote{A cold store is an old fashioned fridge.}%
with a Dexterity + Larceny roll at TN 10 to pick the lock, at which point they discover that the larder is still full of food.


\goblin[\npc{\T\N}{2 Goblins on the table}]

\goblin[\npc{\T\N}{2 Goblins on the floor}]

\goblin[\npc{\T\N}{2 Goblins on the oven}]

\mapentry{The Gnomish Lift}

\begin{boxtext}

	The great double doors swing open, revealing a wide, empty room.
	Your torchlight stretches far above, and well out of reach you can see the ceiling, with a small hole in it, perhaps a foot square.

\end{boxtext}

This lift is operated by a simple pulley system with a counterweight.
When left alone, the lift rises to the top as the great weight goes down.
Not even the gnomes are sure what kind of metal that weight it made from, but it's heavy indeed.

When enough weight pushes down the elevator, the weight rises.
The weight's rope is as long as the entire shaft, so it has to pass through a small hole in the lift's floor.
However, nobody with a weight rating above 4 could possibly fit through the gap.

\begin{itemize}

	\item{When the lift is at the top, the weighted rope rests at the bottom.}
	\item{When the rope is half way down, a key can be turned, locking it so that it cannot go up or (depending on the key position) cannot go down). The weighted rope sits inside the lift at this point.}
	\item{When the lift is at the bottom, the weighted rope sits at the top, and will pull the lift back up if less than a total of weight rating of 14 sits in the lift.}
\end{itemize}

When the lift is half way up, and mobile, characters can add their Strength Bonus to their effective Weight Rating by pulling on the rope.

When descending, it is fairly easy to limit the rate of descent by simply holding onto the rope until the half way point.
After that, the rope rises above, so there's little use in holding onto it.

\mapentry{Dragon's Approach}

\begin{boxtext}

	Ahead, three charred goblin corpses lie on the ground.
	A strange scent wanders down from above, something like a chicken cooked in sulphur.

\end{boxtext}

\begin{exampletext}

	These three goblins came up the stairs with a plan to use a portal scroll next to the dragon, then throw javelins at it until it went away.
	The dragon, however, was faster.
	It incinerated them without a pause.

\end{exampletext}

\paragraph{If the party loot the bodies}
they find only the charred remains of a portal scroll.
They will be able to figure out that it has the same markings as any other portal scroll they find, but will probably not figure out what this scroll does just from its remains (TN 22).

\mapentry{The Dragon's Lair}

\begin{exampletext}

	Shortly after the nura arrived from the Realm of Bright Rocks, a dragon followed.
	He wandered up the stairs towards the treasure room, but could not fit through the little gnomish door.
	He managed to pull a chest full of copper coinage through the door, but could not get anything else as he could not reach.

	The nura wandered up the stairs, and found out too late they could not get any treasure.
	The dragon wanted to barter with them, and make a full army, but found that all the goblins and ogres responded to him either by trying to fight him, or by running away.

	The cautious beast won't approach the nura to attach when he has no idea how many there are around him.
	He may be powerful, but a room full of ogres alongside a spell caster could seriously injure or kill him, and he doesn't want to take the chance.
	Similarly, the nura don't want to take any chances, so they've decided to just leave him where he is.

	The nura and dragon have ended with a stalemate.

\end{exampletext}

The dragon will happily talk with anyone who approaches, but only speaks Quenya (the Elvish language).
Remind the players that they can spend a Story Point to say that they know another language, as long as they say when they learnt the language.

The dragon's eventual goal is to obtain the rest of the treasure, then return through the portal to the Realm of Bright Rocks.

\NPC{\M}{Makil the Dragon}{Inquisitive}{Drums Fingers}{Acquisition}
\person{7}% STRENGTH
{2}% DEXTERITY 
{5}% SPEED
{{4}% INTELLIGENCE
{3}% WITS
{-2}}% CHARISMA
{5}% DR
{2}% COMBAT
{Aggression 2, Projectiles 2, Academics 3, Athletics 1, Deceit 3, Tactics 2, Vigilance 4\Path{Blood}{Enchantment 2, Fate 3, Invocation 4}}% SKILLS
{Assorted treasures}% EQUIPMENT
{\mana{8}}

\paragraph{If the party request he kill nura for them}
then he obliges, in return for getting all treasure out of the treasure room, and leaving it with him.
\paragraph{If the party offer to split the treasure,}
he agrees to letting them take half the coinage, while he takes the other half and all other items.
\paragraph{If the party push for more treasure,}
the dragon asks if they would like to challenge him to a game of riddles.
Each point anyone scores allows them to demand a single item, such as a chest, or quiver.

If they say `yes', then he accepts their challenge, and asks what their riddle is.%
\footnote{Use of the internet is prohibited by trans-dimensional law, common sense, basic decency, and the Geneva Convention.}
If they can think of none,
then the dragon declares that he has won the first round.

See Appendix \ref{riddles} for riddles.

If the dragon parts on good terms, he blesses them all with \textit{Fortune}, allowing them to receive +1 to their Combat Skill for the remainder of the adventure.

\mapentry{The Treasure Room}

\begin{itemize}

	\item{A short bow}
	\item{A buckler shield made of pure silver, worth 30sp (it breaks after one use)}
	\item{Two gem-encrusted shortswords (each are mana stones carrying 2 MP)}
	\item{Four magical arrows which explode, dealing $2D6$ damage upon striking}
	\item{A chest containing 432cp}
	\item{A chest containing 300sp}
	\item{A scroll of stone}

\end{itemize}

\magicitem{Scroll of Stone}{Transmutation}{Alchemy}{1 Scene}{Pocket Spell}{6}{1}
The spell rolls at a TN of 1 plus the caster's Weight Rating (which is equal to their HP).
If the spell succeeds, the reader turns to stone.

This spell was designed to evade oozes or other unintelligent creatures by becoming an uninteresting object for a short while.

\magicitem{Lock Scroll}{Lock}{Alchemy}{2 Scenes}{Pocket Spell}{4}{3}



\end{multicols}


\section{Mid Levels}

\begin{multicols}{2}

\mapentry{Nursery}

\begin{exampletext}

	The gnomes locked this door tightly, while leaving enough food for their children to survive for days, then destroyed the key.
	Unfortunately, those gnomes have turned into goblins, and if they entered the room they would probably eat their own children.

\end{exampletext}

\begin{boxtext}

	The sound of crying emanates from the door as you swing it open.
	A dozen baby creatures with fat little noses lie in a crib with hay, looking up at you in terror.
	The entire crib and the floor around stink of shit.

\end{boxtext}

If the PCs enter the room, they may think these little gnomes are little goblins and kill them.
If any of them try to do so, have them roll a Wits + Medicine check to realize their mistake, TN 8.
Any gnomes in the party will automatically pass this check.

\mapentry{Slug Hall}

\begin{boxtext}

	The doorway reveals a room with an earthy floor, packed with giant slugs, all crawling over each other in a massive pile in the distance.
	The moment you enter, their eye-stalks perk up, and they begin to slide off the corpses they were feasting on.

\end{boxtext}

Gnomes grew mushrooms throughout this room in order to grow slugs, so that they could feed fireflies.
While torches work best, having omnipresent fireflies around the warren makes sure that people can coordinate between rooms without worrying about light.

Since then, the nura turned those slugs into nura slugs.
The fireflies have survived on the enlarged slug, which the nura keep alive as they prefer the human flesh and fresher mushrooms, rather than the grass, corpses, and faeces which the slugs will eat.

\paragraph{If the PCs run up the stairs,}
have them roll initiative against the slugs.

From that point until the slugs loose sight of them, they are in combat.
The slugs will spray acid at them, follow them up the stairs, and pester them for as long as they remain in sight.

\paragraph{If the PCs throw in some food,}
the nura slugs chase it rather than them.

\paragraph{If the PCs manage to investigate the corpses somehow,}
they find two dead gnomes, one with a torch, the other with a Portal Scroll.

\nuraslug[\npc{\T\N}{20 Nura Slugs}]

\mapentry{Armoury \& Equipment}

\label{laddertop}

Picks, shovels, wood, short swords.

\mapentry{The Grand Bedroom}

\end{multicols}

\section{Upper Levels}

\begin{multicols}{2}

\mapentry{Fungal Gardens}

Jellies have taken over most of the area -- ogres simply grab what crawls out.

\jelly

\jelly

\mapentry{Trapped Hallway}

\end{multicols}

\section{Upper Levels}

\begin{multicols}{2}

\mapentry{Rich Prisoners}

Richer prisoners rest here.

One of the prisoners is a mage who helped summon the nura.

\humandiplomat

\mapentry{The Exit}

\umberhulk

\end{multicols}

