\chapter{The Warren}

\includesvg{images/Dyson_Logos/lower}

\section{The Lowest Level}

\begin{multicols}{2}
\setcounter{list}{0}

\mapentry{Rooms}

\begin{exampletext}

	Gnomes erected cots, cribs and hammocks here to be used as a communal sleeping area.
	Since then, goblins have placed a bar over the outside of the door to house prisoners.

\end{exampletext}

\begin{boxtext}

	You awake in darkness with ropes binding your hands behind your back.
	From the shifting and murmuring sounds around, you know others rest in the same room.

\end{boxtext}

Here the PCs awaken to their hopeless situation.
Have everyone roll Dexterity + Larceny against TN 11 to escape.
Run the first event in the time line -- \textit{the Escape} (page \pageref{escape}).

\paragraph{If the PCs make a lot of noise}
then the goblins down the hall are coming for them, and they had better move fast.
Make it clear the party have little chance of success in a head-to-head combat, so they should run the other way.

\paragraph{If the PCs exit quietly,}
they have free reign to explore.

\paragraph{If the party do not manage to escape}
then add 3 Fatigue Points for their time starving in prison, and throw in a load of new prisoners.

Once new prisoners are in, the players can spend Story Points to introduce old friends, and have \textit{them} roll to escape from bondage.

\mapentry{Exit Shaft}

\begin{exampletext}

	This exit shaft was used by gnomes to travel up quickly.
	Aware of the dangers below, they trapped the ladder and hid the door at the top.

\end{exampletext}

If the players attempt to climb, they find that every rung of a multiple of five (the fifth, tenth, fifteenth, and so on), are coated in a slipper substance made from the fungal gardens.
The rungs which are a multiple of 7 are set to break once anything with a Weight Rating 7 or more will break the rung.%
\footnote{A creature's Weight Rating is equal to its HP.}

The nura have since abandoned this path, and are not aware of the door at the top.

\paragraph{If the players climb up the staircase without understanding the trap,}
have them roll Dexterity + Athletics, TN 10.
If they fall, they receive $1D6-2$ Damage.

If they continue climbing, give additional Dexterity rolls.
The character can then make an Intelligence + Tactics roll to understand the trap, TN 12.
Whatever number they roll, keep that result; it will tell you how long the character takes to figure out the trap.
Every time they make a new roll:

\begin{enumerate}

	\item{The Damage they get for failing increases by 1.}
	\item{The roll to understand the trap decreases by 1.}

\end{enumerate}

Of course, if the player figures out the trap, all rolls can be dispenses with, and the player can ascend to room at the top, on page \pageref{laddertop}.

\mapentry{Magic Portal}

\begin{boxtext}

	In this room there stands before you a Magical portal 8' high and 4' wide.
	A warm wind blows through the portal and there are runes above it.
	There are lit Sconces on the walls and the room seems empty accept for the scuff marks leading to and from the portal. 

\end{boxtext}

If any of the players are literate and speak the Gnomish tongue they can read the runes above the magical portal which read Desert Realm.

\mapentry{Dining Room}

\begin{boxtext}

As you climb the stairs away from your cell you enter a room full of benches and tables overturned and broken. 
The floor is strewn with pots and plates, Knives,forks and candlesticks but no candles and not of scrap of food remains it looks like a pack of ravenous dogs swept through here and licked ever morsel from every surface.
In the middle of the room you can see upon a table 2 creatures one you now define as a goblin and the other you would call a rat but only if the rat had been mated with a Great Dane.
The two creatures seem to be fighting over something and as you focus on the scene before you it becomes clear that what they are fighting over is a bone (possibly human).
It looks at first like they are pulling at it one on each end but then you hear the sound of grinding and snapping and you realise that they are both biting into either end of this bone.

\end{boxtext}

\paragraph{If the players make a sound}
the goblin will stop fighting with the rat for the ownership of the bone and run out of the room shouting for the shaman who had left them in the cell previously.
There are 2 knives in this room that can be wielded as daggers 
Whatever happens, the rat will Swollow the reaining bone whole and attack.

\goblin[\npc{\F\N}{Random Goblin}]

\nurarat[\npc{\A\N}{Nura Rat}]

\mapentry{Spellcasters}

\begin{boxtext}

	Two goblins wearing long black robes bicker with each other over a little table.
	Other goblins stand at the side, apparently bored of the argument.

\end{boxtext}

\paragraph{If the players try to sneak past,}
the roll is Dexterity + Stealth, TN 10.
They will also have to put out their torch, if they have one.

\paragraph{If the characters search the bodies,}
they find a Portal Scroll,%
\footnote{This scroll opens a portal to the Realm of Shifting corridors. See Adventures in Fenestra for more on it.}%
 and a Scroll of Uprising.

\magicitem{Scroll of Uprising}{Ghoul Calling}{Saurecanta}{Instant}{Pocket Spell}{+3}{7}

Once the spell is cast, all creatures with 11 HP or fewer in the surrounding 3 areas, raise from the dead.

\magicitem{Portal Scroll}{Unrestrained, Open Teleport}{Alchemy}{2 Scenes}{Pocket Spell}{4}{5}

The scroll takes 5 rounds to speak properly, and opens a doorway to the Realm of Shifting Corridors.%
\footnote{See Adventures in Fenestra for more on this realm.}
The portal remains open for 2 scenes.


\goblinnuramancer[\npc{\F}{Screamer}]
\goblinnuramancer[\npc{\F\N}{Screamer}]

\goblinnuramancer[\npc{\M}{Brock}]
\goblinnuramancer[\npc{\M\N}{Brock}]

\goblin[\npc{\T}{3 Goblins}]

Traps here were trashed.

Supplies:

- Portal scroll
- Staff (wand) of blinding light held in paper to limit the light
- Bowl of Water (summons water)
- Scroll of Insight (feel the entire dungeon level around you)
- Scroll of Locking
- Ring of wishes (multiple conjuration spells)
- Cloak of shadows (illusion of matching nearby object)

\goblinnuramancer

\goblin

\mapentry{Kitchen}

\begin{boxtext}

The door creaks open and as you peer into the darkness you can just make out the 5 figuers sprawled over tables and chairs or curled up on the floor snooring quietly. 
From the embers of the hearth you can see the ovens and cooking utensils that make up a substantial kitchen. 
Opening the door further, the light from the hall's sconces fills the room, and you can see that the five figures are more goblins with little fat bellies.
At the end of the kitchen you can see a large door to a cold store with a lock on it and some large cleavers stuck into a butchers block.

\end{boxtext}

\noindent
In the room the players can find two cleavers which would be the equivalent to short swords and two large pot lids which could be used as bucklers.
There is also a sixth goblin in the room hiding in the far corner in the dark; players can make a Wits + Vigilance roll at TN 8 to spot him.
Should they make to much noise or light up the room, all goblins will wake up and attack.
The players can unlock the cold store
\footnote{A cold store is an old fashioned fridge.}%
with a Dexterity + Larceny roll at TN 10 to pick the lock, at which point they discover that the larder is still full of food.


\goblin[\npc{\T\N}{2 Goblins on the table}]

\goblin[\npc{\T\N}{2 Goblins on the floor}]

\goblin[\npc{\T\N}{2 Goblins on the oven}]

\mapentry{Dragon's Approach}

\begin{boxtext}

	Ahead, three charred goblin corpses lie on the ground.
	A strange scent wanders down from above, something like a chicken cooked in sulphur.

\end{boxtext}

\mapentry{The Dragon's Lair}

\begin{exampletext}

	Shortly after the nura arrived from the Realm of Bright Rocks, a dragon followed.
	He wandered up the stairs towards the treasure room, but could not fit through the little gnomish door.
	He managed to pull a chest full of copper coinage through the door, but could not get anything else as he could not reach.

	The nura wandered up the stairs, and found out too late they could not get any treasure.
	The dragon wanted to barter with them, and make a full army, but found that all the goblins and ogres responded to him either by trying to fight him, or by running away.

	The slumbering beast won't approach the nura to attach when he has no idea how many there are around him.
	He may be powerful, but a room full of ogres alongside a spell caster could seriously injure or kill him, and he doesn't want to take the chance.
	Similarly, the nura don't want to take any chances, so they've decided to just leave him where he is.

	The nura and dragon have ended with a stalemate.

\end{exampletext}

The dragon will happily talk with anyone who approaches, but only speaks Quenya (the Elvish language).
Remind the players that they can spend a Story Point to say that they know another language, as long as they say when they learnt the language.

The dragon's eventual goal is to obtain the rest of the treasure, then return through the portal to the Realm of Bright Rocks.

\NPC{\M}{Makil the Dragon}{Inquisitive}{Drums Fingers}{Acquisition}
\person{7}% STRENGTH
{2}% DEXTERITY 
{5}% SPEED
{{4}% INTELLIGENCE
{3}% WITS
{-2}}% CHARISMA
{5}% DR
{2}% COMBAT
{Aggression 2, Projectiles 2, Academics 3, Athletics 1, Deceit 3, Tactics 2, Vigilance 4\Path{Blood}{Enchantment 2, Fate 3, Invocation 4}}% SKILLS
{Assorted treasures}% EQUIPMENT
{\mana{8}}


\mapentry{The Treasure Room}

\begin{itemize}

	\item{A chest containing 432cp}
	\item{A short bow}
	\item{A buckler shield made of pure silver, worth 30sp (it breaks after one use)}
	\item{Two gem-encrusted shortswords}
	\item{Four magical arrows which explode, dealing $2D6$ damage upon striking.}

\end{itemize}



\mapentry{Beds}

Plenty of goblins, just sleeping and playing games.

\goblin

\mapentry{Nursery}

\begin{exampletext}

	The gnomes locked this door tightly, while leaving enough food for their children to survive for days, then destroyed the key.
	Unfortunately, those gnomes have turned into goblins, and if they entered the room they would probably eat their own children.

\end{exampletext}

\begin{boxtext}

	The sound of crying emanates from the door as you swing it open.
	A dozen baby creatures with fat little noses lie in a crib with hay, looking up at you in terror.
	The entire crib and the floor around stink of shit.

\end{boxtext}

If the PCs enter the room, they may think these little gnomes are little goblins and kill them.
If any of them try to do so, have them roll a Wits + Medicine check to realize their mistake, TN 8.
Any gnomes in the party will automatically pass this check.

\end{multicols}

\mapentry{Armoury}

\label{laddertop}

\section{Mid Levels}

\begin{multicols}{2}

\mapentry{Fungal Gardens}

Jellies have taken over most of the area -- ogres simply grab what crawls out.

\jelly

\jelly

\mapentry{Trapped Hallway}

\end{multicols}

\section{Upper Levels}

\begin{multicols}{2}

\mapentry{Rich Prisoners}

Richer prisoners rest here.

One of the prisoners is a mage who helped summon the nura.

\humandiplomat

\mapentry{The Exit}

\umberhulk

\end{multicols}

