\chapter{The Warren}
\epigraph{
  \iftoggle{hardcore}{%
    The more I take the more I leave behind.
    What am I?
  }{%
    There is something I seek.

    While it is bound, it chooses kings and peasants.
    When it is freed, it foretells war or woe.

    While it bound, it propels men's lusts and furies.
    When it is freed, it tumbles, falls, and fades.

    While it is bound, life will often thrive.
    When it is freed, death will often follow.
  }%
}{}

\section{The Lowest Level}

\begin{multicols}{2}

\newRule{Team Rolls}{

  \input{config/rules/actions_team.tex}

  \paragraph{Example 1:}
  the troupe are sneaking down a hallway, and have to roll \roll{Stealth}{Dexterity} (\tn[9]).
  One player rolls $2D6$, and gets a `7'.
  Characters with a total bonus of +2 were walking quietly, but if even a single \gls{pc} had a bonus of +1 or less, they made a noise, and the entire group is given away.

  \paragraph{Example 2:}
  The characters try to persuade Chris to help them, and one player rolls the dice, getting a `7'.
  Their \roll{Charisma}{Empathy} total is -1, so they fail, but another character tries to speak with him and has a +3 total, so they succeed (without rolling the dice again).
}

\widePic[t]{Roch_Hercka/waking}

\mapentry[entrycell]{Cells}

\begin{exampletext}

  Gnomes erected cots, cribs and hammocks here to be used as a communal sleeping area.
  Since then, goblins have placed a bar over the outside of the door to house prisoners.

\end{exampletext}

Here the \glspl{pc} awaken to their hopeless situation.

\begin{itemize}
  \iftoggle{hardcore}{%
    \item
    Ask the players to mark down $1D6$ \glspl{fatigue} on their character sheets from the trials they've experienced getting here, and of course from their hunger.
  }{}%
  \item
  Give the players a moment to get to know each others' characters and their surroundings.
\end{itemize}

\paragraph{If anyone tries to wriggle free of their ropes,}
have them roll \roll{Dexterity}{Larceny},
\iftoggle{hardcore}{%
  \tn[11].
  Freeing another character requires a \roll{Dexterity}{Crafts} roll, \tn[7], over the space of a round.
}{%
  \tn[9].
  Freeing another character requires a full round.
}

\paragraph{After a moment,}
a small voice from the corner of the room says

\begin{quotation}
  \noindent
  ``No use in struggling.
  They will eat us soon no matter what we do''.
\end{quotation}

The ogres captured Chris from a nearby village, along with others.
When the goblinoid hordes entered, Chris hid in the side of the room while his companions fought.
He's learnt how to survive through cowardice and will keep himself as safe as he can from now on.

\NPC{\Hu\M}{Chris}{Seaweed-like hair half-covers his gloomy eyes}{Sighs}{Tribalist}
\person{2}% STRENGTH
{0}% DEXTERITY 
{1}% SPEED
{{0}% INTELLIGENCE
{-1}% WITS
{0}}% CHARISMA
{0}% DR
{0}% COMBAT
{
  \setcounter{Projectiles}{2}
  Crafts 2, Empathy 1, Wyldcrafting 2
}% SKILLS
{Nothing}% EQUIPMENT
{}

\paragraph{If anyone asks Chris for help,}
he refuses, and explains why they are powerless, and may as well wait to be eaten.
If they say they want to fight, he explains how large ogres stand, and how many wait above.
If they say that they want to die fighting, and to untie them, then he explains that he can't trust them, and that they might throw him out the door first.
He would rather take his chances hiding in the dark.

\paragraph{Every time the \glspl{pc} give Chris a reason to hope,}
have them roll \roll{Charisma}{Empathy} at \iftoggle{oneshot}{\tn[9]}{\tn[11]}.
Success means that Chris will untie them.
A tie means he asks a follow-up question (allowing another re-roll if they can answer him well).
A failure means he disengages, muttering ``no, it won't work''.

\paragraph{Shortly after,}
Blara the goblin nuramancer, and Olf the ogre walk towards the cell.

\begin{boxtext}

  Heavy footsteps pad down the hall, you hear the door's bar being lifted, and a little goblinoid face peeps in with a torch.
  Behind her, an ogre stoops to the height of a man to avoid the low ceiling.

\end{boxtext}

Olf then tries to pick up a character, and take it up to the kitchen.

\paragraph{If anyone wants to fight while still tied up,}
they can do so with a -4 penalty to the roll.

\iftoggle{oneshot}{
  \npc{\N\M}{Olf the Ogre}
  \person{6}% STRENGTH
  {0}% DEXTERITY 
  {1}% SPEED
  {{-2}% INTELLIGENCE
  {-3}% WITS
  {-5}}% CHARISMA
  {0}% DR
  {1}% COMBAT
  {Crafts 1, Wyldcrafting 1}% SKILLS
  {Nothing}% EQUIPMENT
  {}
}{
  \ogre[\npc{\N\M}{Olf the Ogre}]
}

\npc{\N\F}{Blara the Goblin Nuramancer}

\person{-2}% STRENGTH
{2}% DEXTERITY 
{2}% SPEED
{{0}% INTELLIGENCE
{0}% WITS
{-4}}% CHARISMA
{0}% DR
{1}% COMBAT
{
  \setcounter{Athletics}{1}
  \setcounter{Deceit}{1}
  \setcounter{Stealth}{2}
  \setcounter{Tactics}{1}
}% SKILLS
{\Dagger, torch, human foot\iftoggle{oneshot}{}{, map}}% EQUIPMENT
{
  \setcounter{Wyldcrafting}{2}
  \knacks{\snapcaster}
  \setcounter{Fire}{2}
  \setcounter{Air}{2}
}

\showStdSpells

\paragraph{If anyone attacks Blara,}
Olf immediately turns his attention to them.

Blara will attempt to run away, but must wait for Olf to move out of the doorway.

\iftoggle{oneshot}{
  Chasing after Blara requires a resisted roll of \roll{Speed}{Athletics}
  (the player rolls 2D6 plus their \roll{Speed}{Athletics} against Blara's \roll{Speed}{Athletics}, \tn[10]).%
}{}%

\paragraph{If Blara flees,}
she runs out, taking her torch with her.
\iftoggle{oneshot}{%
  If the fire has gone out and Blara leaves with the torch, the room becomes pitch-black.
  This gives the \glspl{pc} a bonus to any attack roll equal to their \roll{Wits}{Vigilance} +3 (Olf cannot coordinate well in the dark).
}{}

\paragraph{If the party win the fight,}
he will accompany them out, but his nerves are too shot to be of much use.
He will not join any fights, but can hold a torch.

\iftoggle{oneshot}{}{
  \paragraph{If the \glspl{pc} search Blara's body,}
  they find a map of the upper level (see the handouts).
}

\newRule{End of \Glsentrytext{interval} Regeneration}{
  \input{config/rules/interval.tex}

  Being underground, the \glspl{pc} will only regenerate 1 \gls{mp} per interval, which goes to whichever \gls{pc} has the most \glspl{mp} lost.

  Of course the \glspl{pc} won't know exactly how long they've spent underground, but they will at least be able to count the number of resting periods they take.
}

\mapentry[escapeShaft]{Fire Room}

\begin{exampletext}
  A fire burns low in this alcove, casting long shadows across the room.%
  \footnote{The gnomes built this shaft to mine below, but once they had a lift to transport them more easily, they stopped using the stone ladder.
  Once abandoned, they decided to make it into a fireplace, by adding a chimney to the surface.}
\end{exampletext}

If the \glspl{pc} look around the fire, they will notice hand-holds carved into the rock on the left.
If they climb while the fire burns, the smoke coming up will make them choke and cough, inflicting \pgls{fatigue}.

Climbing to the top requires a \roll{Speed}{Athletics} roll at \tn[13], and every point rolled below 13 inflicts \pgls{fatigue}.
E.g. a character rolling a total of `5' would gain 8 \glspl{fatigue}.

The \glspl{pc} cannot see how far up this shaft goes, so they will have to track it moment by moment.
Give any player who climbs up one or two \glspl{fatigue} at a time, and each time, ask them if they want to continue (without knowing how far they have to go) or go back down (and receive the same number of \glspl{fatigue} as before, minus 2).

\paragraph{Once at the top of the ladder,}
this chimney has an alcove for a furnace in the workshop, above.
Any \gls{pc} will hear the goblins talking.

\paragraph{Breaking through the furnace}
needs a \roll{Intelligence}{Crafts}, \tn[6].
Failure inflicts $1D6$ Damage as some goof pulls hot coals from the furnace onto the character.

If anyone succeeds in breaking that furnace, they certainly make a lot of noise, and find themselves at room \ref{workshop}.

\paragraph{If someone wants to continue climbing up,}
the chimney becomes smaller and smaller, then eventually breaks out into various hand-sized tunnels, and exits above, almost invisibly.

\mapentry[diningRoom]{Dining Room}

Every morsel has been licked clean from this little dining room.
Pots, smashed plates, forks, and candlesticks litter the floor.
A goblin and a polymorphed rat bicker over a human bone.

\begin{boxtext}

  Smashed-up chairs and broken cutlary surround a low dining table.
  On the table, a little bleach-white goblin wrestles with a rat-like creature, the size of a dog.
  The two creatures are fighting over a human leg.

\end{boxtext}

\paragraph{If a \gls{pc} sneaks up quietly,}
have them roll \roll{Dexterity}{Stealth},  at \tn[9].

They should get a +2 bonus for leaving any lights behind.

\paragraph{If the \glspl{pc} wait patiently and silently,}
the nura rat and goblin will leave.
The goblin will go up the lift, and the rat will wander the hall, searching for more food.

\paragraph{If the \glspl{pc} make a sound,}

\begin{itemize}

  \item
  the goblin grabs the candle on the table (which puts it out),
  \item
  then he runs out of the room, shouting for a nuramancer in area \ref{spellCasters}.
  \item
  The rat attacks the \glspl{pc} immediately.
  \iftoggle{oneshot}{(As before, players roll 2D6 + Dexterity + Combat roll, at \gls{tn} equal to the creature's `{\scshape Att}', and the winner deals Damage)}{}
  \item
  The goblin nuramancers observe them for a moment, then flee to the lift (room \ref{lift}). 
\end{itemize}

\paragraph{If the \glspl{pc} enter combat,}
they take 1 \gls{fatigue}.

\paragraph{Searching the room}
reveals two large knives which can be used as \emph{daggers}.
\iftoggle{oneshot}{%
  Daggers grant +1 Damage in combat.
}{}

\mapPic{b}{Dyson_Logos/lower}{
  \ref{entrycell}/81/04,
  \ref{escapeShaft}/86/15,
  \ref{diningRoom}/72/27,
  \ref{spellCasters}/58/29,
  \ref{desertPortal}/47/32,
  \ref{kitchen}/51/07,
  \ref{lift}/32/33,
  \ref{dragonApproach}/08/33,
  \ref{dragon}/13/47,
  \ref{treasureRoom}/25/53,%
  \ref{nursery}/425/45,
  \ref{slugHall}/73/34,
  \ref{greatHall}/71/69,
  \ref{workshop}/90/62,
  \ref{grandLibrary}/905/81,
  \ref{windingStairs}/58/64,
  \ref{secondPrison}/43/68,
  \ref{armoury}/64/84,
  \ref{trappedHall}/54/84,
  \ref{topShaft}/25/87,
  \ref{fungusGarden}/09/77,
  \ref{lockedDoor}/35/94,
  \ref{lowerExit}/62/97,%
  {\iftoggle{hardcore}{\ref{goblinSentry}}{}}/41/53,
}

\goblin[\npc{\N\F\Gn}{Hungry Goblin}]

\nurarat[\npc{\N\A}{Nura Rat}]

\mapentry[spellCasters]{Spellcaster Arguments}

%! Remove scroll

The two spell casters cannot figure out what the scrolls on the table do, despite the fact that one is a Scroll of uprising, brought here by the nura.

\begin{boxtext}

  At the end of the tunnel, in an alcove to the right, goblins wearing long black robes bicker with each other over a little table.
  \iftoggle{hardcore}{%
    Other goblins stand at the side, apparently bored of the argument.
  }{}%
  Over on the left, you can see a light so strong it must be daylight.

\end{boxtext}

\paragraph{If the \glspl{pc} try to sneak past,}
they make a Group Roll with \roll{Dexterity}{Stealth}, \tn[10].
\iftoggle{oneshot}{As before, one player rolls 2D6, and everyone adds their own \roll{Dexterity}{Stealth}.
If a single \gls{pc} fails, everyone fails.}{}%
\exRef{core}{core rules}{teamwork}

\paragraph{If the goblins spot the party,}
they attempt to flee.
They cannot call the lift quickly enough to escape.

Two rounds later, the goblins in the kitchen join the attack.

\paragraph{If the \glspl{pc} corner the goblins,}
they respond with aggressive spells.

\goblinnuramancer[\npc{\F}{Screamer the Nuramancer}]

\goblinnuramancer[\npc{\M}{Brock the Nuramancer}]

\iftoggle{hardcore}{
  \goblin[\npc{\T}{3 Goblins}]
}{}

\iftoggle{oneshot}{
  \paragraph{After the fight,}
  give each player 1 \gls{fatigue} for each round of combat.
}{}

\paragraph{If the characters investigate the table,}
they find a Portal Scroll, and a Scroll of Uprising.

Once the spell is cast, all creatures with 11 \glspl{hp} or fewer in the surrounding 3 areas, instantly raise from the dead as ghouls.
See \autoref{creaturesAndItems} for stats on the various ghouls.

Unlike other magical items in this area, this scroll was made by a nuramancer (not a gnome), so the command word is written plainly in the Trade Tongue (which many nura speak).
The scroll contains the epic of Logan, an Elvish poem.
Once the last line is spoken, the scroll activates (the rest does not need to be spoken).

\iftoggle{oneshot}{
  \paragraph{If the \glspl{pc} try to examine a scroll,}
  ask if they think their character should speak Gnomish.
  If any say `yes', have them mark off a Story Point (on the back of their character sheet) and explain where they learnt the language.

  Every time players spend a Story Point, the character gains 1 \glspl{xp} for every Story Point marked off so far (so the second gives 2 \glspl{xp}, the third gives 3, and so on).%
  \exRef{stories}{core rules}{stories}

  \paragraph{Understanding the scrolls}
  requires more than being literate.
  They will have to roll 
}{}

\paragraph{If the \glspl{pc} want to identify what a scroll does,}
have them roll \roll{Intelligence}{Academics}, \tn[12].

\paragraph{If the \glspl{pc} want to activate a Portal Scroll,}
they will have to answer its riddle.
Go to \autoref{riddles} and select any riddle to activate the scroll.

\paragraph{Once the Portal Scroll is activated,}
it begins to shimmer with golden flecks, and four rounds later the portal destroys itself, creating a magical portal to another realm.
See \autoref{creaturesAndItems}: \nameref{creaturesAndItems} for details on the calamities which ensue after reading a Portal Scroll.

\paragraph{When leaving this alcove,}
the \glspl{pc} head towards the blinding light.

\begin{boxtext}

  Turning away from the table in the little alcove, the hallway ahead drops into two shadowy openings, and on the right daylight floods in through a passage, almost blinding you.

\end{boxtext}

\mapentry[desertPortal]{Magic Portal}

\begin{boxtext}
  Shielding your eyes, you turn right and head towards the light.

  Just four steps down the passage, you find the Sunlight coming from a massive opening on the right.
  A warm wind blows through the portal and on the other side you can see an endless desert of bright, yellow rocks.
  Strange symbols and gemstones sit around the opening's perimeter.

  On the left, double doors made of wood stare into the light.
  Then passageway continues, but becomes quickly lost in darkness.

\end{boxtext}

\iftoggle{oneshot}{
  \paragraph{Ask the players if they can speak Gnomish,}
  if none got an opportunity to spend Story Points on the scrolls (above, in room \ref{spellCasters}).
}{}

\paragraph{If any of the \glspl{pc} are literate and speak the Gnomish tongue,}
they can read the runes above the magical portal; they read ``Realm of Bright Rocks''.

\paragraph{If the \glspl{pc} enter the portal and journey through the Realm of Bright Rocks,}
they won't find much -- not even a place to hide, given the desert is expansive and flat.
If they stay there long, give them 4 \glspl{fatigue} (due to the heat).
%\footnote{See \textit{Adventures in Fenestra} for more on the Realm of Bright Rocks\iftoggle{aif}{%
%  , page \pageref{brightrocks}}%
%  {}%
%.}


\paragraph{If they enter the portal to flee from the nura,}
\iftoggle{hardcore}{%
  the nura give chase\ldots for a while, however nura without food tire very quickly.
  If they run through 3 or more areas, the nura must turn back due to exhaustion.
}{%
  the nura leave them alone -- their hyperactive metabolism tires too quickly without food.
  They can only journey through that realm with a well-stocked supply line and heavy Sun-proof cloaks.
}

\iftoggle{oneshot}{
  \paragraph{\glspl{fatigue}}
  don't do much at first, but every \gls{fatigue} above the character's \emph{current} \glspl{hp} gives a -1 penalty.
  So a character with 6 \glspl{fatigue} who gets reduced to 3 \glspl{hp} would take a -3 penalty to all actions.
}{}

\mapentry[kitchen]{Kitchen}

\paragraph{If the \glspl{pc} have avoided making a lot of noise nearby,}
then the goblins are asleep in the kitchen.

\begin{boxtext}

The door creaks open and as you peer into the darkness you can just make out the four figures sprawled over tables and chairs or curled up on the oven, snoring quietly.
From the embers of the hearth you can see the ovens and cooking utensils that make up a substantial kitchen, and silhouettes of human femurs and skulls.

Opening the door further, the light from the hall's sconces fills the room, and you can see that the four figures are more goblins with little fat bellies.
At the end of the kitchen you can see a large door to a cold store with a lock on it and some large cleavers stuck into a butchers block.

\end{boxtext}

\paragraph{If the \glspl{pc} try to sneak in,}
have them make a Group Roll of \roll{Dexterity}{Stealth}, \tn[8].
\iftoggle{oneshot}{%
  As before, the Group Roll only requires one player to roll, and any character which enters the room produces a different result.
}{}

\paragraph{If the roll fails,}
all goblins wake up
\iftoggle{oneshot}{
  but must take a morale check before attacking.
  The Morale check only uses the Combat Skill at \tn[5], with some modifiers.

  \begin{itemize}
    \item
    -2 if any \gls{pc} has a higher Strength than the goblin.
    \item
    -2 if there are more \glspl{pc} \emph{in the room} than goblins.
    \item
    -2 if the goblin is wounded.
    \item
    -1 if any \gls{pc} has displayed magical powers.
  \end{itemize}

  Make a Group Roll for all the goblins (one roll, but each goblins adds their own Combat Skill to get a new result).

  \paragraph{If the goblins fail the Morale Check,}
  they attempt to flee into the prison cell the \glspl{pc} just came from.

  Their Morale check remains throughout the combat, without needing to re-roll it.
  If they roll a `7' from the start, then a Goblin with Combat +1 will have a total of `8', and one with Combat +2 would have `9'.
  If the \glspl{pc} outnumber the goblins, and one is stronger than them, then their total would become `5' and `6'.
  If a goblin became wounded, they would flee, but the others might stay.
}{
  and take a morale check.

  Two more goblins are sleeping in a corner, making six in total.
}

\newRule{Weapons}{


  \input{config/rules/weight}

  \input{config/rules/weapons.tex}

  The \glspl{pc} can find all these items in the \nameref{kitchen}.
  They don't make the best weapons, but the \glspl{pc} have little choice!

  \end{multicols}
  \begin{nametable}[YXYYYY]{Kitchen Weapons}

    \textbf{No.} & \textbf{Name} & \textbf{Attack Bonus} & \textbf{Damage Bonus} & \textbf{\glspl{ap} Cost} & \textbf{Weight Rating} \\\hline

    1 & Broom & 1 & 0 & 2 & -1 \\
    4 & Small\showWeapon{\chair} \\
    2 & \showWeapon{\Dagger}  \\
    2 & \showWeapon{\skillet} \\
    4 & \showWeapon{\Log} \\
    1 & \showWeapon{\woodaxe} \\

    \end{nametable}
  \begin{multicols}{2}

}

\goblin[\npc{\T[2]\N}{2 Goblins on the table}]

\iftoggle{hardcore}{
  \goblin[\npc{\T[4]\N}{4 Goblins on the floor}]
}{
  \goblin[\npc{\T[2]\N}{2 Goblins on the oven}]
}

\paragraph{If the players try to find weapons,}%
\iftoggle{oneshot}{
  they can find plenty of make-shift weapons around the kitchen.

  These weapons may not be good quality, but they can still improve the \glspl{pc}' situation immensely.

  \paragraph{If the players select weapons for their characters,}
  check out the Kitchen Weapons table and have them write down the stats on their character sheet one at a time.
 }{
  they can use either of the two cast iron skillets.
}

\widePic[t]{Roch_Hercka/dragon}

\paragraph{If the party raid the room for food,}
they'll find a few canteens of water, one of wine, and some spare sacks of vegetables.

The players can unlock the cold store%
\footnote{A cold store is an old fashioned fridge where people put ice to keep food from going bad.}
  with a Dexterity + Larceny roll at \tn[10] to pick the lock, at which point they discover that the larder is still full of food.

\mapentry[lift]{The Gnomish Lift}

The gnomes created the lift with the Force sphere.
It can lift a combined Weight Rating of 16, and safely descend with a total Weight Rating of 26 standing on it.%
\footnote{A creature's Weight Rating is equal to its \glspl{hp}.
Some items count towards this total, but items with a negative Weight Rating do not.}
If the party step on it with a greater Weight Rating than this, each additional point inflicts 2 \glspl{fatigue} on everyone in the lift upon impact with the ground.

\begin{boxtext}

  The great double doors swing open, revealing a wide, empty room.
  Your torchlight stretches far above, and well out of reach you can see a ceiling covered in gemstones, inlaid into the wood.
  The room appears otherwise empty.

\end{boxtext}

The lift responds to magical passwords -- the gnomish words for `farm' (for the top), `sleep' (for the middle), and `food' (for the bottom).
The various ogres and goblins who use the lift only know the passwords for the bottom and middle section.
Only the nuramancer goblins know the word for the top.

The lift responds to any password within earshot.

\paragraph{If anyone tries to cling onto the walls,}
they will find less purchase than a Sun-screen salesman in a Scotland.

\paragraph{If the \glspl{pc} dawdle too long here,}
a Raiding party return with more prisoners to place in the cell (room \ref{entrycell}).
See \autopageref{raidingParty} for details.

\mapentry[dragonApproach]{Dragon's Approach}

\begin{exampletext}

  Once the gnomes turned into nura, they neglected to de-activate the portal, as it's easier to see when it's open.
  A couple of hours ago, a dragon named `Makil' wandered in, and went up the stairs towards the treasure room, but could not fit through the little gnomish door.
  He managed to pull a chest full of copper coinage through the door, but could not get anything else as he could not reach.

  The Makil wanted to barter with the goblins, but found that they responded to him either by trying to fight him, or by running away, and none spoke elvish (the only language he knows spoken in these parts).

  The cautious beast won't approach the nura to attack when he has no idea how many there are around him.
  He may be powerful, but a room full of ogres alongside a spell caster could seriously injure or kill him, and he doesn't want to take the chance.
  Similarly, the nura don't want to take any chances, so they've decided to just leave him where he is.

  The nura and dragon have ended with a stalemate, so he has taken a nap.

\end{exampletext}

\begin{boxtext}

  Ahead, three charred goblin corpses lie on the ground.
  A strange scent wanders down from above, something like a chicken cooked in sulphur.

\end{boxtext}

\paragraph{If the \glspl{pc} loot the bodies}
they find a portal scroll wrapped safely in a scroll case.

\mapentry[dragon]{The Dragon's Lair}

The dragon will happily talk with anyone who approaches, but only understands Elvish.
\iftoggle{hardcore}{}{%
  Remind the players that they can spend a Story Point to say that they know another language, as long as they say when they learnt the language.
}
Despite his linguistic deficits, he can communicate using the Enchantment sphere to send ideas to characters.

The dragon's eventual goal is to obtain the rest of the treasure, then return through the portal to the Realm of Bright Rocks.

\paragraph{If the party request he kill nura for them}
then he agrees to kill any in the current area, but will not journey to another floor.
In return, he wants all the treasure out of the treasure room.

\NPC{\E}{Makil the Dragon}{Inquisitive}{Drums Fingers}{Acquisition}
\person{7}% STRENGTH
{2}% DEXTERITY 
{3}% SPEED
{{4}% INTELLIGENCE
{3}% WITS
{-2}}% CHARISMA
{}% DR
{1}% COMBAT
{
  Projectiles~2, Academics~3, Athletics~1, Deceit~3, Tactics~2, Vigilance~4
}% SKILLS
{Nothing}% EQUIPMENT
{
  \renewcommand\abilities{\flight}
  \hide{5}
  \knacks{\snapcaster}
  \addtocounter{Brawl}{3}
  \addtocounter{Fate}{2}
  \addtocounter{Fire}{2}
}

\showStdSpells

\paragraph{If the party offer to split the treasure,}
he refuses, unless they give him a good reason.

\iftoggle{hardcore}{
  The treasure chinks noisily of course, so the party will then receive a -1 penalty to all Stealth checks.}{}

\paragraph{If the party push for more treasure,}
the dragon asks if they would like to challenge him to a game of riddles.
Each point anyone scores allows them to demand a single item, such as a chest, or quiver.

If they say `yes', then he accepts their challenge, and asks what their riddle is.%
\footnote{Use of the internet is prohibited by trans-dimensional law, common sense, basic decency, and the Geneva Convention.}
If they can think of none,
then the dragon declares that he has won the first round.

\paragraph{The rules for riddles}
are simple -- any question which someone has the knowledge to answer is a fair riddle.
Asking `how many letters in the Greek word for ``mushroom''?', is not a fair riddle, because someone may not know.

Any possible answer to a riddle is `the correct one'.
If someone asks `what is black and white and read all over', anything which fits all descriptions must be accepted as an answer.

See \autoref{riddles} for riddles.

\paragraph{If the players ask why he wants treasure,}
he explains that he wants to attract a mate; when his flame becomes hot enough to melt the gold, he will carve a golden statue of the most deadly dragon in his area in order to attract her attention.
He will then decorate the statue with magical items.
\footnote{Would you prefer a mate is beautiful, or one that can kill anything in an instant?
Dragons think the answer is obvious.}

\paragraph{If the dragon parts on good terms,}
he blesses them all with a \textit{Wide Blessing} (which heals $2D6$ \glspl{fp}), and casts a \textit{Fortune} spell on the weakest member, allowing them to receive +1 to their Combat Skill for the rest of the adventure.

\iftoggle{oneshot}{
  \paragraph{If the \glspl{pc} attack the dragon,}
  he kills the first to attack.

  His {\scshape \gls{dr} 5} means he reduces all Damage by 5, unless the attacker hits 5 over the \gls{tn} to attack him (a total of 15), achieving a `Vitals Shot'.
}{}


\noteRaidingParty

\mapentry[treasureRoom]{The Treasure Room}
\begin{itemize}

  \item{A short bow}
  \item{A buckler shield made of pure silver, worth 30sp (it breaks after one use)}
  \item{Two gem-encrusted shortswords (each are mana stones carrying 2 \glspl{mp})}
  \item{Four magical arrows which explode, dealing $2D6$ damage upon striking}
  \item{A chest containing 432 \glspl{cp}}
  \item{A chest containing 300 \glspl{sp}}
  \item{Portal Scroll to the Realm of Shifting Corridors}
  \iftoggle{hardcore}{}{
    \item
    Portal Scroll
    \footnote{See \autoref{creaturesAndItems} for these scrolls.}
  }

\end{itemize}


\begin{boxtext}

  Through the door, two locked chests lie on the ground.
  Above them, a shortbow and two beautiful short swords stand affixed to the wall, with a quiver of arrows with gemstones used as arrow tips.
  \iftoggle{hardcore}{%
    The scroll cabinet has two scrolls left.
  }{The scroll cabinet has three scrolls inside.}

\end{boxtext}

\end{multicols}

\section{Mid Levels}

\begin{multicols}{2}

\iftoggle{hardcore}{
  \mapentry[goblinSentry]{The Sentry}

  Svart the goblin stands at the door to room \ref{nursery} (below), attempting to pick the lock.
  \paragraph{If the \glspl{pc} exit the lift door,}
  she runs over, hoping to beg some food from the goblin she expects to emerge.
  Once he sees the \glspl{pc}, he runs for aid.

  \goblin[\npc{\F\N}{Svart}]
}{}

\mapentry[nursery]{Nursery}

\begin{exampletext}

  When the nura arrived, the gnomes put their children in here, then left enough food for a few days, locked the door, and destroyed the key.
  Unfortunately, those gnomes have turned into goblins, and if they entered the room they would probably eat their own children.

\end{exampletext}

\begin{boxtext}

  The sound of crying emanates from the door as you swing it open.
  A dozen baby creatures with fat little noses lie in a crib with hay, looking up at you in terror.
  The entire crib and the floor around stink of shit.

\end{boxtext}

The door is locked, but can be picked with a \roll{Dexterity}{Larceny} check, \tn[9].
It's far too strong to be broken into by force.
If the \glspl{pc} enter the room, they may think these little gnomes are little goblins and kill them.
If any of them try to do so, have them roll a Wits + Medicine check to realize their mistake, \tn[8].
Any gnomes in the party will automatically pass this check.

\iftoggle{oneshot}{
  Anyone following Laiqu\"e who successfully rescues a child from the warren gains 3 \glspl{xp} at the end of the session.
}{}

\paragraph{If the party save the children,}
the children know to know to keep quiet and to follow, but they will be at the back of the party when moving alone.

There are six children in total, and they each have a Weight Rating between 1 and 2.
\iftoggle{oneshot}{
  Characters can only carry a Weight Rating equal or less than their Strength Bonus, or they will take penalties to their \glspl{ap}, and gain \glspl{fatigue} at the end of each \gls{interval} of activity.

  \paragraph{If the \glspl{pc} rest,}
  Remember to have them heal \glspl{fatigue} equal to half their \glspl{hp}.
}{}

\newRule{Projectiles}{
  \begin{itemize}
    \item
    Players \roll{Dexterity}{Projectiles} against \tn[6] to hit targets.
    \item
    Every 5 steps adds +1 to the \gls{tn}.
    \item
    When a player hits the \gls{tn} precisely, they miss their first target, but hit any other target behind.
    \item
    Shortbows deal only $1D6-1$ Damage.
  \end{itemize}
}

\mapentry[slugHall]{Slug Hall}

\begin{exampletext}

Gnomes grew mushrooms throughout this room in order to grow slugs, so that they could feed fireflies.
While torches work best, having omnipresent fireflies around the warren makes sure that people can coordinate between rooms without worrying about light.

Since then, the nura turned those slugs into nura slugs.
The fireflies have survived on the enlarged slug, which the nura keep alive with the scraps they don't want to eat, such as mushrooms and bones and faeces.

\end{exampletext}

\begin{boxtext}

  The doorway reveals a massive hallway of sparkling, floating, lights.
  Across the earthy floor, pieces of paper lie everywhere.
  Giant slugs lazily wander, masticating them.
  Half of the slugs wear the pages, as if someone had thrown the paper in the room from above.
  Around those pages, little grubs eat into the massive slugs.
  The moment you enter, their eye-stalks perk up, and they begin to slide off the corpses they were feasting on.
  To your left, stairs head upwards into darkness.

\end{boxtext}

\paragraph{If the \glspl{pc} run up the stairs,}
have them roll initiative against the slugs.

From that point until the slugs loose sight of them, they are in combat.
The slugs will spray acid at them, follow them up the stairs, and pester them for as long as they remain in sight.

\paragraph{If the \glspl{pc} throw in some food,}
the nura slugs chase it rather than them.

\paragraph{If the \glspl{pc} manage to investigate the corpses somehow,}
they find two dead gnomes, one with a torch, the other with a Portal Scroll.
\iftoggle{oneshot}{%
  As with the others, the Portal Scroll requires a riddle to be answered, takes four rounds before it activates, and vanishes once used.
}{}

\nuraslug[\npc{\T[9]\N}{20 Nura Slugs}]

\mapentry[greatHall]{The Great Hallway}

If the \glspl{pc} have indeed been quiet enough in the previous room to not raise an alarm, they find everyone in nearby rooms napping.
A single sound means they will be in serious trouble.

The greasy floor results from a mixture of faeces, drool, blood and leftover mushroom-juice.
Anyone who can use Conjuration magic will be able to turn the offal on the floor into a slippery substance.

\begin{boxtext}

  At the top of the stairs, this massive chamber lies empty, except for the fireflies darting about, and a few human bones.
  To the right, there are two short tunnels.
  Ahead of you, two grand staircases lead up into a dim but unwavering light.
  Below you, the entire floor is sticky and greasy.
  Somewhere close, you hear snoring.

\end{boxtext}

\paragraph{If the \glspl{pc} have come from room \ref{slugHall},}
they will not immediately see the staircase on their left, but will see it after doing literally anything (fighting, searching, et c.).

\paragraph{If the \glspl{pc} tarry or talk,}
have them make a \roll{Dexterity}{Stealth} check, \tn[6].
They make this as a \textit{Group Roll}, so a single roll counts for the whole group.
Each margin on the roll allows them an additional round before the horde wakes.

\mapentry[workshop]{The Workshop}

\label{laddertop}


\begin{exampletext}

  Frightened gnomes fled their bedrooms from the nura who had rushed through the portal.
  Most were caught, but some managed to run up the trapped ladders, having memorized the sequence perfectly.
  As the last one got to the top of the ladders, he turned to cast an illusion of a solid wall in order to fool anyone coming up behind him.

\end{exampletext}

\begin{boxtext}

  Some goblins and three ogres lie sleeping on the floor between workbenches.
  The place is so full, you can't make out how many lie here, but the snoring indicates more than you can see.
  On the benches, most of the equipment lies broken, but obviously delicate gnomish hands once used these tables to polish gems, craft magical items, and forge digging equipment.
  On one table, you can see a pile of weapons -- short swords and spears -- piled on a table.

\end{boxtext}
Picks, shovels, wood, short swords, shortbows, and all manner of crafting and mining equipment litter the room.

\paragraph{If the \glspl{pc} have come up from room \ref{entrycell},}
they find the exit has been covered by an illusion of a wall.
If they investigate the area at all, the illusion fades, revealing the sleeping nura.

\paragraph{If the \glspl{pc} attempt to take either a short sword or a spear,}
each attempt requires a \roll{Dexterity}{Stealth} roll, \tn[7].
Failure will awaken the entire horde.

\iftoggle{oneshot}{
  You can find stats for the weapons in \autoref{creaturesAndItems}.
}{}

\pic{Decky/armoury}

\goblin[\npc{\T[\arabic{enemyNo}]\N}{\arabic{enemyNo} Goblins}]

\addtocounter{enemyNo}{-2}

\ogre[\npc{\T[\arabic{enemyNo}]\N}{\arabic{enemyNo} Ogres}]

\mapentry[grandLibrary]{The Grand Library}

%! Remove old magical items
In total, the room contains the following magical items:

This bowl is always full of water.
If it empties, then the air around it quickly forms into more water.

This lock scroll will lock any door, increasing the \gls{tn} to break through it by 4.

As with any Portal Scroll, it leads to the Realm of Shifting Corridors.

This ring is actually three pocket spells -- each ruby in the ring is a separate magical item, and each use will make one ruby go dark.

The caster can summon anything available for Conjuration level 4.
With 7 \glspl{mp}, the ring has a Metamagic score of 7, allowing it to cast \textit{Wide} spells, and other Metamagic enhancements.
\iftoggle{oneshot}{%
  The spell can broadly summon any item up to the size of a person.
}{}

\goblinnuramancer

The spell allows the user to feel 4 areas around (meaning `rooms'), though not with much detail.
The effect feels very disorienting at first, as most people are not used to having their consciousness spread apart so far.

This `staff' will probably feel more like a wand in the hands of a human, as it's about the height of a short gnome.
The gnomish word for `morning', activates the light, while the gnomish word for `night' stops it.
The nura do not know, or have forgotten, these activation words, so they simply keep it under blankets as they find the light irritating.

If wrapped up in heavy sheets and suddenly displayed, the light can blind opponents, as usual.

\begin{boxtext}

  At the top of the stairs you find the ruins of a massive library.
  Book cases lie in a smashed heap on the ground, others appear to be used as a makeshift bed for an ogre.
  The books themselves are gone, except for a few scrolls, now tightly clutched by a goblin in a black cowl.
  His hand shines with a thumb-ring, showing three massive gemstones.

\end{boxtext}

\iftoggle{hardcore}{
  \ogre[\npc{\N\T}{2 Docile Ogres}]
}{
  \ogre[\npc{\N\T}{Sleeping Ogre}]
}

\paragraph{If any of the \glspl{pc} attempt to sneak in,}
have them roll \roll{Dexterity}{Stealth}, \tn[8].

Failure will, of course, spell disaster, but success will allow them to steal a magical item.
If the player wants to steal multiple magical items, describe them and see how many they decide to take.
Each item taken increases the roll's \gls{tn} by 1, so taking 3 items would mean a \gls{tn} of 11.
The player should not roll again -- the original roll remains, but increasing the \gls{tn} may well turn success into awful failure.

\end{multicols}

\iftoggle{oneshot}{%
  \section{Last Level}
}{
  \section{Upper Warren}
}

\begin{multicols}{2}

\mapentry[windingStairs]{Winding Stairs}

A single ogre guards the prisoners here (the door has no lock).

\begin{boxtext}

  As you round the stairs' third turn, you see a massive ogre crouching by a door.
  It blocks the path completely.

\end{boxtext}

\paragraph{If the party have made a reasonable attempt at staying quiet,}
they can avoid alerting this ogre with a \roll{Wits}{Stealth} roll, \tn[9].
Whoever is at the front makes the roll.
If it's unclear who's at the front, the character with the highest \roll{Speed}{Athletics} is in the lead.
With a successful roll,
\iftoggle{hardcore}{
  the ogre is resting, and must take a round to gather what's left of his wits, but it will still wake if approached.
}{%
 the party find the ogre sleeping.
}

\ogre[\npc{\N\M}{Rick, the Ogre Guard}]

\paragraph{If the party try to talk with the ogre,}
Rick has only recently been turned into an ogre, and he has eaten his fill of mushrooms, so a \roll{Wits}{Empathy} roll, \tn[10], will allow the party to convince him to let them go.
However, if Rick lets them go, he will insist on joining them so he can be free.
While he genuinely wants to escape and become human again,%
\footnote{Nura who have recently turned can change back to their original forms if they are starved for twice as long as they have been nura.}
the moment the party enter battle with other nura his instincts will kick in, and he will turn on the party.%

\mapentry[secondPrison]{Second Prison}

\begin{exampletext}

  This little room once housed a full family of gnomes, but now serves only as another prison.

\end{exampletext}

The prisoners require no locks or handcuffs -- the ogre waiting outside suffices to terrify them into staying put.

\paragraph{If any of the \glspl{pc} have died,}
another player should spend 2 Story Points to introduce an old friend.
\iftoggle{hardcore}{}%
  \iftoggle{core}{%
    Pick another pre-made character sheet from the pile and hand it to any players without characters.
  }{}%
The players should agree on how they know each other.

Take the last two farmers from the handout to join the troupe.

\iftoggle{hardcore}{%
  \paragraph{If the party get a moment to ask about the outside world,}
  the villagers tell them that they last saw hundreds of nura swarming around town, and she was on her horse to get away.
  They suspect that the entire place will have been overrun.
}{}

\iftoggle{oneshot}{% This weapon chart goes next to the armoury
  \begin{figure*}[t!]
  \begin{nametable}[YXYYYY]{Armoury Weapons}

  \textbf{Quantity} & \textbf{Name} & \textbf{Attack Bonus} & \textbf{Damage Bonus} & \textbf{\glspl{ap} Cost} & \textbf{Weight Rating} \\\hline

  3 & Bucklar Shield & +2 & None & 1 & 0 \\
  3 & \showWeapon{\shortsword} \\
  7 & \showWeapon{\woodaxe} \\

  \end{nametable}
  \end{figure*}
}{}

\mapentry[armoury]{Armoury}

\begin{exampletext}

  The gnomes once stashed their little weapons here.
  The nura horde have added to it considerably.

\end{exampletext}

\begin{boxtext}

  At the top of the stairway, three dying fireflies wander pointlessly.
  Behind them, you see shadows, with a glimpse of metal, lying in an alcove.
  To the left, a wooden door waits, with dirty footprints visible on the floor.

\end{boxtext}

\begin{itemize}

  \item{3 buckler shields}
  \item{1 crossbow (unstrung but usable with an \roll{Wits}{Crafts} roll, \tn[7])}
  \item{3 quivers, each with 20 arrows}
  \iftoggle{hardcore}{%
    \item{3 crossbow bolts}
    \item{2 shortbows (also unstrung)}
  }{
    \item{8 crossbow bolts}
    \item{2 shortbows}
  }
  \item{3 shortswords}
  \item{7 wood axes}

\end{itemize}

\iftoggle{oneshot}{
  \paragraph{Bucklar Shields}
  These shields work like a weapon, except that they deal no damage.
  They require only 2 \glspl{ap} to use in combat, but add a +2 Bonus.

  \paragraph{The crossbow}
  (if repaired) deals $1D6+3$ Damage, but requires at least 4 rounds to reload.

  \paragraph{Shortbows}
  Require 1 \glspl{ap} to loose an arrow, and 1 \glspl{ap} to reload.
  However, they deal only $1D6-1$ Damage.
}{}

\newRule{\Glsfmtlong{dr} \& Vitals Shots}{

  \input{config/rules/armour.tex}

  \Gls{dr} reduce Damage taken due to armour, or just a creature's thick hide.
  The nura wolves have `{\scshape \gls{dr} 2}', so they remove 2 from any Damage Taken.

  We assume everyone is trying to target sensitive areas, like the throat, groin, and eyes, whenever they can, to hit more exposed areas.
  If anyone hits 5 points over what they need to hit the wolves, they get a `\textit{Vitals Shot}', which ignores the \gls{dr} entirely, as the attack hits a sensitive area.
}

\mapentry[trappedHall]{Trapped Hallway}

The squares between the exit door and the party's path can turn anyone stepping upon them into stone.
One goblin was unfortunate enough to trigger the trap.
The second goblin did not believe the original goblins about why the first was there, and ignored the statue, so he was turned to stone too.

This magical trap didn't look quite right when the flagstones had to be inlaid with gems in order to complete the alchemical spell -- that kind of shine gives the game away.
The gnomes' solution was to make sure \textit{all} flagstones had a magical amulet in them, so nobody could tell which might hold a deadly spell.

Anyone standing on the stone rolls their Weight Rating at \tn[15] to avoid being turned to stone.
\iftoggle{oneshot}{%
  Each character's Weight Rating is equal to their \glspl{hp}.
}{}

\begin{boxtext}

  Your lantern illuminates dirty footprints leading out the door, to the left, and down a dark corridor.
  To your right, two statues of ogres stand, like stooping gargoyles, blocking the path.
  All around, the floor glistens with tiny gemstones carved into the centre of flagstones.

\end{boxtext}

\paragraph{If anyone steps into the trap,}
they must roll their Weight Rating at \tn[15], but anyone failing may spend 5 \glspl{fp} to ignore the effects of the trap.
If this happens, explain that their foot or hand begins to harden, then turn to stone for half a second before they retract it.

\paragraph{If the players figure out that the area is trapped and try to jump over it,}
remind them that they cannot see where the trap starts and where it ends.
If they avoid stepping into the trap, they can make a \roll{Speed}{Athletics} roll, \tn[9], to jump over the afflicted area.

\mapentry[topShaft]{The Top of the Shaft}

\widePic{Roch_Hercka/garden}

Grank the nuramancer goblin has heard the \glspl{pc} coming, and has no intention of fighting them alone.
He knows he has the only key to the exit door, so he intends to loose the hell-hounds he has tied up at the room's side before fleeing into the fungal gardens.

\iftoggle{hardcore}{%
  By the time the party reach round the corner, he will have cut the ropes disappeared, leaving unchained nura wolves.
}{
  The party arrive to see him still sawing through the ropes, and have only a single round to stop him before he runs.
}

\begin{boxtext}

  Ahead lies to massive double-doors.
  To the right, the hallway extends into a massive room, where you can hear the sound of sawing.

\end{boxtext}

\paragraph{No matter what the party do,}
the wolves go straight for the kill.

\paragraph{If the party attempt to run through the double doors,}
they will suddenly find an empty lift-shaft, unless they have taken the lift to the top themselves.
Have them make a Wits + Athletics check (\tn[7]) to back off before they fall.

\nurawolf[\npc{\A\N\T[4]}{4 Nura Wolves}]

\paragraph{If the party fall into the lift,}
they end up in the mid-section of the warren.

\mapentry[fungusGarden]{Fungal Gardens}

\begin{exampletext}

  This beautiful fungal garden took dripping rain from above, and sieved it through the roof then the soil below, until it distributed nutrients for a forest of mushrooms, big and small.
  The fungal garden was regularly invaded by oozes which can creep into small cracks when young, and grow massive quickly.
  The nura never really kept up with the garden's maintenance, so the room festered with dangerous jellies.

\end{exampletext}

\begin{boxtext}

  As the doors open, your torchlight shows a crowded mess of fungus.
  Some reach up to your knees, others tower above the light and so far that you cannot see the top.
  The maze of unkempt mushrooms shows random, meandering paths between the taller fungi.

\end{boxtext}

While the place looks serene, it is inhabited by
\iftoggle{hardcore}{%
  dangerous oozes.

}{%
  a dangerous ooze.
}

\paragraph{Once the players enter the room,}
\iftoggle{hardcore}{%
  the oozes begin to stalk them.
  If multiple oozes chase them, the smaller ones will always back away from the larger ones, so no more than one ooze should follow them at a time (always the largest one).
}{%
  the ooze begins to stalk them.
}

\paragraph{If the \glspl{pc} approach Grank,}
he will hide while casting a \textit{Wide, Fireball} (requires 2 rounds to cast, covers 4 steps, and deals $2D6+2$ Damage).
Finding him in the darkness before he finishes his spell requires a Wits + Vigilance roll, \tn[10].
If he is successful, the party make a group roll against \tn[9] to avoid the fireball ($1D6+3$ Damage).

\paragraph{If Grank ever feels like his life is under threat,}
then he will taunt the \glspl{pc} with the key to the outside world he has in is position, and throw it into the nearest ooze.
He then lets out a giggle and dashes off into the fungal undergrowth, leaving the players to face the hulking, pulsating, mass.

\label{grank}
\npc{\M\N}{Grank}
\person{-2}% STRENGTH
{1}% DEXTERITY 
{4}% SPEED
{{1}% INTELLIGENCE
{2}% WITS
{-4}}% CHARISMA
{0}% DR
{1}% COMBAT
{Projectiles~1, Athletics~2, Medicine~1, Stealth~2, Tactics~2}% SKILLS
{\iftoggle{hardcore}{Seeing Stone}{\Dagger}}% EQUIPMENT
{
  \setcounter{Earth}{2}
  \setcounter{Water}{2}
}

\iftoggle{oneshot}{}{
  \jelly
}

\iftoggle{hardcore}{
  \jelly
}{}

\jelly

\mapentry[lockedDoor]{The Locked Door}

The locked door has a bronze border and cannot be broken without a \roll{Strength}{Crafts} roll, \tn[18].
The exit key, however, fits in nicely.

If the party rest here long enough (perhaps because they have failed to get any key), a raiding party of nura return.
See page \pageref{raidingParty} for the returning raiding party.

\mapentry[lowerExit]{The Exit}

\iftoggle{oneshot}{
  \begin{boxtext}

    As the key turns, the door swings open, and daylight floods in.
    Green trees cover the road down the hill, and in the far distance, chimney-fires from little hamlets wander into the sky.

    You have finally escaped from the warren.

  \end{boxtext}
}{
  This is where the party exit the lower portion of the warren, and enter the upper part.

  \paragraph{If the \glspl{pc} don't have a key,}
  they might try tricking the goblin on the other side, who has a key to let people in and out.
  See area 1, \autoref{upper}.

  \paragraph{If the \glspl{pc} examine the door,}
  they can see the goblin in the next area by a glimmer of candle-light.
}

\end{multicols}
