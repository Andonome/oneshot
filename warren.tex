\chapter{The Warren}
\epigraph{
  \iftoggle{hardcore}{%
    The more I take the more I leave behind.
    What am I?
  }{%
    There is something I seek.

    While it is bound, it chooses kings and peasants.
    When it is freed, it foretells war or woe.

    While it bound, it propels men's lusts and furies.
    When it is freed, it tumbles, falls, and fades.

    While it is bound, life will often thrive.
    When it is freed, death will often follow.
  }%
}{}

\section{The Lowest Level}

\begin{multicols}{2}

\newRule{Re-Rolls \& Group Rolls}{
  When players roll $2D6$ to perform an action, the dice tell you if the situation is favourable or not.
  Players do not re-roll dice unless the situation has changed.

  When a group of people want to do the same thing (sneaking, speaking, running), they don't all roll to check the situation -- only one player rolls dice, and all the players use the same result for their characters.

  \paragraph{Example 1:}
  the troupe are sneaking down a hallway, and have to roll Stealth + Dexterity (\gls{tn} 9).
  One player rolls $2D6$, and gets a `7'.
  Characters with a total bonus of +2 were walking quietly, but if even a single \gls{pc} had a bonus of +1 or less, they made a noise, and the entire group is given away.

  \paragraph{Example 2:}
  The characters try to persuade Chris to help them, and one player rolls the dice, getting a `7'.
  Their Charisma + Empathy total is -1, so they fail, but another character tries to speak with him and has a +3 total, so they succeed (without rolling the dice again).

}

\widePic[t]{Roch_Hercka/waking}

\mapentry[entrycell]{Cells}

\begin{exampletext}

  Gnomes erected cots, cribs and hammocks here to be used as a communal sleeping area.
  Since then, goblins have placed a bar over the outside of the door to house prisoners.

\end{exampletext}

Here the \glspl{pc} awaken to their hopeless situation.
\iftoggle{hardcore}{%
  Ask the players to mark down $1D6$ Fatigue Points on their character sheets from the trials they've experience getting here, and of course from their hunger.
}{}%
Give the players a moment to get to know each others' characters and their surroundings.
Remember, that everything is pitch black -- at best they can feel out the room a little, but they don't have long.

\paragraph{If anyone tries to wriggle free of their ropes,}
have them roll Dexterity + Larceny,
\iftoggle{hardcore}{%
  \gls{tn} 11.
  Freeing another character requires a Dexterity + Crafts roll, \gls{tn} 7, over the space of a round.
}{%
  \gls{tn} 9.
  Freeing another character requires a full round.
}

\paragraph{After a moment,}
a small voice from the corner of the room says
`\begin{textit} it's all useless, we're all going to eat each other\end{textit}'.
Chris was captured from a village along with another batch of prisoners.

\NPC{\Hu\M}{Chris}{Despondent}{Sighs}{Tribe}
\person{1}% STRENGTH
{0}% DEXTERITY 
{1}% SPEED
{{0}% INTELLIGENCE
{-1}% WITS
{0}}% CHARISMA
{0}% DR
{0}% COMBAT
{Academics 2, Empathy 1}% SKILLS
{Nothing}% EQUIPMENT
{}

\begin{exampletext}
  Blara the gnome accepted her fate, and her hunger, then began to learn nuramancy -- the sphere of magic which transforms gnomes into goblins and humans into ogres.

  Since then, she routinely enters the make-shift prison cell, and casts a spell on one of the prisoners to turn them into an ogre.
  The target of the spell becomes ravenous, and eventually eats another prisoner (they have no desire to eat other nura).

  She has made a dozen new ogres this way, but the last time one person was left on his own -- Chris.
  Since then, he has sat in the darkness, waiting to be eaten.
  Now the \glspl{pc} have been shunted into the cell with him, he knows his time is short.

\end{exampletext}

Chris explains the situation to the \glspl{pc} (whether or not anyone asks him).

\paragraph{Trying to plea with Alf}
won't prove easy.
The player can attempt a Charisma + Empathy roll (\gls{tn} 12), to have him leave the room out of shame and guilt, but he won't help anyone.

\paragraph{If anyone asks Chris for help,}
he tells that that he already managed to wriggle out of the ropes binding him, but that it's all useless, because you can't fight against a monster.

Convincing Chris to help someone requires a Charisma + Empathy roll (\gls{tn} 9).

\paragraph{Shortly after,}
Blara the goblin nuramancer, and Alf (the last man to transform into an ogre) walk towards the cell.

\begin{boxtext}

  Heavy footsteps pad down the hall, you hear the door's bar being lifted, and a little goblinoid face peeps in with a torch.
  Behind her, an ogre stoops to the height of a man to avoid the low ceiling.

  The little goblin then begins chanting while staring intently at Chris.

\end{boxtext}

\paragraph{If anyone wants to fight while still tied up,}
they can do so with a -4 penalty to the roll.

\paragraph{Anyone attacking Blara}
immediately annoys Alf, the ogre, who then attacks.

Blara will attempt to run away, but must wait for Alf to move out of the doorway.

\iftoggle{oneshot}{
  Chasing after Blara requires a resisted roll of Speed + Athletics
  (the player rolls 2D6 plus their Speed + Athletics against a \gls{tn} of 7 + Blara's Speed (2) + Athletics (1) = 10)%
}{}.%

\paragraph{If Blara flees,}
she takes the only light out of the room.
\iftoggle{oneshot}{%
  This gives the \glspl{pc} a bonus to any attack roll equal to their Wits + Vigilance +3 (Alf cannot coordinate well in the dark).
}{}

\iftoggle{oneshot}{
  \npc{\N\M}{Alf the Ogre}
  \person{6}% STRENGTH
  {0}% DEXTERITY 
  {1}% SPEED
  {{-2}% INTELLIGENCE
  {-3}% WITS
  {-5}}% CHARISMA
  {0}% DR
  {1}% COMBAT
  {Crafts 1, Wyldcrafting 1}% SKILLS
  {Nothing}% EQUIPMENT
  {}
}{
  \ogre[\npc{\N\M}{Alf the Ogre}]
}
\label{alf}

\npc{\N\F}{Blara the Goblin Nuramancer}

\person{-2}% STRENGTH
{2}% DEXTERITY 
{2}% SPEED
{{0}% INTELLIGENCE
{0}% WITS
{-4}}% CHARISMA
{0}% DR
{1}% COMBAT
{Athletics 1, Deceit 1, Stealth 2, Tactics 1
\knacks{\alchemist, \nuraCaster}
\Path{\illusion~1, \conjuration~2,\saurecanta~2}
}% SKILLS
{\Dagger, torch, human foot\iftoggle{oneshot}{}{, map}}% EQUIPMENT
{}

\paragraph{If Blara's spell is interrupted,}
she has to start again.

\paragraph{After 2 rounds,}
\iftoggle{oneshot}{the spell begins to take hold,}{Blara's Brawn-Form spell finishes, and Chris is stricken with the \textit{Hunger Pains} spell,%
\exRef{aif}{Adventures in Fenestra}{saurecanta}}
and Chris starts trying to eat people.

If he has any trouble attacking people, Blara reaches into her bag and pulls out a human foot, then throws it on the ground in front of Chris.
Once he eats it, he gains +1 Strength%
\iftoggle{oneshot}{ and deals +1 Damage}{.}

If he successfully kills someone, he eats them, becomes an ogre, and leaves with Blara (she needs to wait for her mana to recharge).

\paragraph{If the \glspl{pc} manage to throw the human foot towards Alf,}
he stops to eat it for 2 rounds, doing nothing, else unless attacked.

\paragraph{If the \glspl{pc} try to kill Chris immediately,}
\iftoggle{oneshot}{they gain a +2 Attack Bonus and +2 Damage for the `Sneak Attack'.}{he will be vulnerable, so they can make a sneak attack.}

\npc{\N\M}{Chris (as an Ogre)}
\person{4}% STRENGTH
{0}% DEXTERITY 
{1}% SPEED
{{-2}% INTELLIGENCE
{-1}% WITS
{-2}}% CHARISMA
{0}% DR
{1}% COMBAT
{Academics 2, Empathy 1}% SKILLS
{Nothing}% EQUIPMENT
{}

\paragraph{If the party save Chris,}
he will accompany them out, but his nerves are too shot to be of much use.
He will not join any fights, but can hold a torch.

\iftoggle{oneshot}{}{
  \paragraph{If the \glspl{pc} search Blara's body,}
  they find a map of the upper level (see the handouts).
}

\paragraph{If the party do not manage to escape or save Chris,}
then he eats one of them, and joins Blara into the door, leaving the rest in the dark.

She returns some hours later, throws in a new prisoner (a new \gls{pc}!), and the encounter begins again with a new victim.

\newRule{End of Scene Regeneration}{
  Each time the party enter a new situation (often when they simply enter a new room), they regain 40\% of their base Fate Points (\glspl{fp}).
  Those with a base of 5 \glspl{fp} regain 2, and those with a base of 10 regain 4.

  Those with Mana Points (\glspl{mp}) can also regain 2 \glspl{mp} plus their Wits Bonus.

}

\mapentry[escapeShaft]{Exit Shaft}

\iftoggle{oneshot}{%
  \paragraph{New Scenes}
  mean regeneration.
  As the \glspl{pc} enter this new room, they heal 40\% of their \textit{Base \glspl{fp}}, so characters with 5 \glspl{fp} regain 2, and characters with 10 \glspl{fp} regain 4.
  Any characters with Mana Points (\glspl{mp}) regain 2 + their Wits Bonus.
}{}

\begin{exampletext}

  This exit shaft was used by gnomes to travel up quickly.
  Aware of the dangers below, they trapped the ladder and hid the door at the top.
  Since then, the goblins have forgotten how the ladder works, and stopped using it.

\end{exampletext}

If the players attempt to climb, they find that every rung of a multiple of five (the fifth, tenth, fifteenth, and so on), are coated in a slipper substance made from the fungal gardens.
The rungs which are a multiple of seven (seven, fourteen, twenty-one) are set to break once anyone with 7+ \glspl{hp} puts their weight on it.
These breakable sections are actually solid illusions, so they appear again shortly after breaking, but will resist any markings such as paint, as those marking disappear as the illusion resets every so often.
Meanwhile, any rung which is both a multiple of 5 and 7 indicates that the next section is entirely illusory -- the real ladder is on the left, but disguised as more wall with clever paint-work.

Despite the ladder switching places, it appears to be one solid piece, reaching upwards.

The nura have since abandoned this path, and have forgotten the door at the top.

\paragraph{If the players climb up the staircase without understanding the trap,}
have them make a Group roll of Dexterity + Athletics, \gls{tn} 10.%
If they fall, they receive $1D6-2$ Damage.

\paragraph{If they continue climbing,}
give additional Dexterity rolls and explain what happens, but only give out specific rung-numbers if the player asks.

\paragraph{If the players want to make a roll,}
character can then make an Intelligence + Academics check to understand the trap, \gls{tn} 12.
They cannot re-roll this result, but they can try to climb again, in which case:

\begin{itemize}

  \item
  The Damage they get for failing increases by 1.
  \item
  The \gls{tn} to understand the trap decreases by 1 (so they will succeed if they can endure long enough).

\end{itemize}

\paragraph{If a \emph{player} figures out the trap,}
all rolls can be dispensed with, and the \glspl{pc} can ascend to room \ref{laddertop} at the top.

\newRule{\Glsfmtlong{dr} \& Vitals Shots}{

  \Gls{dr} reduce Damage taken due to armour, or just a creature's thick hide.
  The nura rat has `{\scshape \gls{dr} 2}', so it removes 2 from any Damage Taken.

  We assume everyone is trying to target sensitive areas, like the throat, groin, and eyes, whenever they can, to hit more exposed areas.
  If anyone hits 5 points over what they need to hit the rat, they get a `\textit{Vitals Shot}', which ignores the \gls{dr} entirely, as the attack hits a sensitive area.
}

\mapentry[diningRoom]{Dining Room}

\begin{boxtext}

  As you climb the dark stairway, you see a tiny candle's light above, and hear shouting and thrashing.
  The floor at the top is strewn with pots and plates, knives, forks and candlesticks.
  It looks like a pack of ravenous dogs swept through here and licked ever morsel from every surface.

\end{boxtext}

\paragraph{If a \gls{pc} sneaks up quietly,}
have them roll Dexterity + Stealth, at a \gls{tn} of 7 plus the rat's Wits + Vigilance.

They should get a +2 bonus for leaving any lights behind.

\begin{boxtext}

  In the middle of the room you can see two creatures wrestling on a table -- a goblin and the other you would call a rat, but only if the rat had been mated with a Great Dane.
  The two creatures are fighting over a human leg.

\end{boxtext}

\paragraph{If the \glspl{pc} wait patiently and silently,}
the nura rat and goblin will leave.
The goblin will go up the lift, and the rat will wander the hall, searching for more food.

\paragraph{If the \glspl{pc} make a sound},

\begin{itemize}

  \item
  the goblin grabs the candle on the table (which puts it out),
  \item
  then he runs out of the room, shouting for a nuramancer in room \ref{spellCasters}.
  \item
  The rat attacks the \glspl{pc} immediately.
  \iftoggle{oneshot}{(As before, players roll 2D6 + Dexterity + Combat roll, at \gls{tn} equal to the creature's `{\scshape Att}', and the winner deals Damage)}{}
  \item
  The goblin nuramancers observe them for a moment, then flee to the life (room \ref{lift}). 
\end{itemize}

\paragraph{If the \glspl{pc} enter combat,}
they take 1 Fatigue Point for each round of combat they were in.

\paragraph{Searching the room}
reveals two large knives which can be used as \emph{daggers}.
\iftoggle{oneshot}{%
  Daggers grant +1 Damage in combat.
}{}

\mapPic{b}{Dyson_Logos/lower}{
  \ref{entrycell}/81/04,
  \ref{escapeShaft}/86/15,
  \ref{diningRoom}/72/27,
  \ref{spellCasters}/58/29,
  \ref{desertPortal}/47/32,
  \ref{kitchen}/51/07,
  \ref{lift}/32/33,
  \ref{dragonApproach}/08/33,
  \ref{dragon}/13/47,
  \ref{treasureRoom}/25/53,
  \ref{nursery}/425/45,
  \ref{slugHall}/73/34,
  \ref{greatHall}/71/69,
  \ref{workshop}/90/62,
  \ref{grandLibrary}/905/81,
  \ref{windingStairs}/58/64,
  \ref{secondPrison}/43/68,
  \ref{armoury}/64/84,
  \ref{trappedHall}/54/84,
  \ref{topShaft}/25/87,
  \ref{fungusGarden}/09/77,
  \ref{lockedDoor}/35/94,
  \ref{lowerExit}/62/97,
}

\goblin[\npc{\N\F\Gn}{Hungry Goblin}]

\nurarat[\npc{\N\A}{Nura Rat}]

  \paragraph{Once the \glspl{pc} leave the room,}
  let them regenerate 40\% of their Fate Points, because a new scene has ended.


\mapentry[spellCasters]{Spellcaster Arguments}

\begin{boxtext}

  At the end of the tunnel, in an alcove to the right, goblins wearing long black robes bicker with each other over a little table.
  \iftoggle{hardcore}{%
    Other goblins stand at the side, apparently bored of the argument.
  }{}%
  Over on the left, you can see a light so strong it must be daylight.

\end{boxtext}

The two spell casters cannot figure out what the scrolls on the table do, despite the fact that one is a Scroll of uprising, brought here by the nura.

\paragraph{If the \glspl{pc} try to sneak past,}
they make a Group Roll with Dexterity + Stealth, \gls{tn} 10.
\iftoggle{oneshot}{As before, one player rolls 2D6, and everyone adds their own Dexterity + Stealth.
If a single \gls{pc} fails, everyone fails.}{}%
\exRef{core}{core rules}{grouproll}

\paragraph{If the goblins spot the party,}
they attempt to flee.
They cannot call the lift quickly enough to escape.

Two rounds later, the goblins in the kitchen join the attack.

\paragraph{If the \glspl{pc} corner the goblins,}
they respond with aggressive spells.

\iftoggle{oneshot}{
  \begin{itemize}

    \item
    Anyone with Invocation 2 or more can cast a fireball by spending 3 \glspl{ap} and 2 \glspl{ap}.
    \begin{itemize}
      \item
      Players must roll to avoid the magical flames at \gls{tn} 7 plus the caster's Intelligence + Projectiles.
      \item
      Players can spend 1 \glspl{ap} to \textit{Keep Edgy} against missile attacks such as fireballs, for the rest of combat.
      This allows them to add their Speed + Vigilance Bonus to the roll, but without Keeping Edgy, they gain no bonus to the roll.
    \end{itemize}
    \item
    If the caster hits, they deal $1D6 + 1 + Int$ Damage (so casters with Intelligence +2 deal $1D6+3$ Damage).

  \end{itemize}
}{}

\goblinnuramancer[\npc{\F}{Screamer the Nuramancer}]

\goblinnuramancer[\npc{\M}{Brock the Nuramancer}]

\iftoggle{hardcore}{
  \goblin[\npc{\T}{3 Goblins}]
}{}

\iftoggle{oneshot}{
  \paragraph{After the fight,}
  give each player 1 Fatigue Point for each round of combat.
}{}

\paragraph{If the characters investigate the table,}
they find a Portal Scroll, and a Scroll of Uprising.

\magicitem{Scroll of Uprising}{Fast Ghoul Calling}{Saurecanta}{Instant}{Pocket Spell}{3}{7}

Once the spell is cast, all creatures with 11 \glspl{hp} or fewer in the surrounding 3 areas, instantly raise from the dead as ghouls.
See \autoref{creaturesAndItems} for stats on the various ghouls.

Unlike other magical items in this area, this scroll was made by a nuramancer (not a gnome), so the command word is written plainly in the Trade Tongue (which many nura speak).
The scroll contains the epic of Logan, an Elvish poem.
Once the last line is spoken, the scroll activates (the rest does not need to be spoken).

\magicitem{Portal Scroll}% NAME
  {Eternal, Ranged, Clairvoyance, Ranged Gate}% SPELL
  {Alchemy}% PATH
  {2 Scenes}% DURATION
  {Pocket Spell \& Pocket Spell}% TYPE
  {2}% Potency
  {3}% MP

\iftoggle{oneshot}{
  \paragraph{If the \glspl{pc} try to examine a scroll,}
  ask if they think their character should speak Gnomish.
  If any say `yes', have them mark off a Story Point (on the back of their character sheet) and explain where they learnt the language.

  Every time players spend a Story Point, the character gains 1 \glspl{xp} for every Story Point marked off so far (so the second gives 2 \glspl{xp}, the third gives 3, and so on).%
  \exRef{core}{core rules}{stories}

  \paragraph{Understanding the scrolls}
  requires more than being literate.
  They will have to roll 
}{}

\paragraph{If the \glspl{pc} want to identify what a scroll does,}
have them roll Intelligence + Academics, \gls{tn} 12.

\paragraph{If the \glspl{pc} want to activate a Portal Scroll,}
they will have to answer its riddle.
Go to \autoref{riddles} and select any riddle to activate the scroll.

\paragraph{Once the Portal Scroll is activated,}
it begins to shimmer with golden flecks, and four rounds later the portal destroys itself, creating a magical portal to another realm.
See \autoref{creaturesAndItems}: \nameref{creaturesAndItems} for details on the calamities which ensue after reading a Portal Scroll.

\paragraph{When leaving this alcove,}
the \glspl{pc} head towards the blinding light.

\begin{boxtext}

  Turning away from the table in the little alcove, the hallway ahead drops into two shadowy openings, and on the right daylight floods in through a passage, almost blinding you.

\end{boxtext}

\begin{baseNote}{The Raiding Party}
\label{raidingParty}

\paragraph{If the \glspl{pc} decide to stop and rest,}
\iftoggle{hardcore}{%
  they are immediately interrupted by a returning raiding party of nura, with fresh prisoners.
}{
  they will be able to recovery for a while,\footnote{As usual, they recover Fatigue Points equal to half their current \glspl{hp}, and 40\% of their base \glspl{fp}.} before a raiding party descends with freshly captured prisoners.
}

The lift (room \ref{lift}) cannot take the full weight of an ogre plus the prisoners, so one goblin nuramancer and an ogre go to the bottom, then the prisoners get shoved in the lift by ogres at the top.

\begin{boxtext}
  A creaking noise comes from the two wooden doors, then you hear loud voices from inside.

  ``\textit{My guts! That old lady didn't agree with me}'',
  you hear a deep voice saying.

  ``\textit{Did you do an argument?}'', a shrill voice replies.
  
  You hear a loud `\textbf{bang!}', then the wooden doors open.

\end{boxtext}

\iftoggle{hardcore}{%
  \goblinnuramancer[\npc{\M\N}{Goblin Nuramancer}]
}{
  \iftoggle{oneshot}{
    \goblin[\npc{\T\F}{Goblin}]
  }{
    \goblin[\npc{\T\N}{2 Goblins}]
  }
}

\iftoggle{oneshot}{
  \ogre[\npc{\T\N\Hu}{Ogre}]
}{
  \armouredOgre
}

\paragraph{If any \glspl{pc} have died,}
this is a good time to give out new characters from the recently captured prisoners.

\paragraph{If the \glspl{pc} try to hide,}
they will find it easy.

\paragraph{If they try to ambush the nura,}
give them a Wits + Stealth (\gls{tn} 8) roll to determine the best location to stand.

Success means they can each make one attack roll at \gls{tn} 7 (not the creature's usual attack score) with +2 Damage.

\paragraph{Once the farmers come down,}
they are tired, and have 8 Fatigue Points each, so any Damage they receive will quickly rack up penalties.

Take the farmer statblocks from the handout, cut them apart from each other, and hand \iftoggle{hardcore}{2}{3} to the players to keep track of.
\iftoggle{oneshot}{
  The farmers will refuse to go first into any combat situation, if the characters fight first then they can make a Morale Check to join.
}{}

Shortly after, two more ogres come down to find out where the last people went.

\end{baseNote}

\mapentry[desertPortal]{Magic Portal}

\begin{boxtext}
  Shielding your eyes, you turn right and head towards the light.

  Just four steps down the passage, you find the Sunlight coming from a massive opening on the right.
  A warm wind blows through the portal and on the other side you can see an endless desert of bright, yellow rocks.
  Strange symbols and gemstones sit around the opening's perimeter.

  On the left, double doors made of wood stare into the light.
  Then passageway continues, but becomes quickly lost in darkness.

\end{boxtext}

\iftoggle{oneshot}{
  \paragraph{Ask the players if they can speak Gnomish,}
  if none got an opportunity to spend Story Points on the scrolls (above, in room \ref{spellCasters}).
}{}

\paragraph{If any of the \glspl{pc} are literate and speak the Gnomish tongue,}
they can read the runes above the magical portal; they read ``Realm of Bright Rocks''.

\paragraph{If the \glspl{pc} enter the portal and journey through the Realm of Bright Rocks,}
they won't find much -- not even a place to hide, given the desert is expansive and flat.
If they stay there long, give them 4 Fatigue Points (due to the heat) and roll an encounter for that Realm.%
\footnote{See \textit{Adventures in Fenestra} for more on the Realm of Bright Rocks\iftoggle{aif}{%
  , page \pageref{brightrocks}}%
  {}%
.}

\paragraph{If anyone tries to work out the portal's activation words,}
the player can roll Intelligence + Academics, \gls{tn} 12, as long as they speak Gnomish.
Success indicates that they can activate and deactivate the portal.

\paragraph{If they enter the portal to flee from the nura,}
\iftoggle{hardcore}{%
  the nura give chase\ldots for a while, however nura without food tire very quickly.
  If they run through 3 or more areas, the nura must turn back due to exhaustion.
}{%
  the nura leave them alone -- their hyperactive metabolism tires too quickly without food.
  They can only journey through that realm with a well-stocked supply line and heavy Sun-proof cloaks.
}

\iftoggle{oneshot}{
  \paragraph{Fatigue Points}
  don't do much at first, but every Fatigue Point above the character's \emph{current} \glspl{hp} gives a -1 penalty.
  So a character with 6 Fatigue Points who gets reduced to 3 \glspl{hp} would take a -3 penalty to all actions.
}{}

\iftoggle{oneshot}{
  \paragraph{Staying to stare at the portal}
  counts as a new area, and therefore a new scene, so the \glspl{pc} can regenerate 40\% of their \glspl{fp} and some \glspl{mp} again.
}{}

\mapentry[kitchen]{Kitchen}

\paragraph{If the \glspl{pc} have avoided making a lot of noise nearby,}
then the goblins are asleep in the kitchen.

\begin{boxtext}

The door creaks open and as you peer into the darkness you can just make out the four figures sprawled over tables and chairs or curled up on the oven, snoring quietly.
From the embers of the hearth you can see the ovens and cooking utensils that make up a substantial kitchen. 
Opening the door further, the light from the hall's sconces fills the room, and you can see that the four figures are more goblins with little fat bellies.
At the end of the kitchen you can see a large door to a cold store with a lock on it and some large cleavers stuck into a butchers block.

\end{boxtext}

\paragraph{If the \glspl{pc} try to sneak in,}
have them make a Group Roll of Dexterity + Stealth, \gls{tn} 8.
\iftoggle{oneshot}{%
  As before, the Group Roll only requires one player to roll, and any character which enters the room produces a different result.
}{}

\paragraph{If the roll fails,}
all goblins wake up
\iftoggle{oneshot}{
  but must take a morale check before attacking.
  The Morale check only uses the Combat Skill at \gls{tn} 5, with some modifiers.

  \begin{itemize}
    \item
    -2 if any \gls{pc} is larger has a higher Strength than the goblin.
    \item
    -2 if there are more \glspl{pc} \emph{in the room} than goblins.
    \item
    -2 if the goblin is wounded.
    \item
    -1 if any \gls{pc} has displayed magical powers.
  \end{itemize}

  Make a Group Roll for all the goblins (one roll, but each goblins adds their own Combat Skill to get a new result).

  \paragraph{If the goblins fail the Morale Check,}
  they attempt to flee into the prison cell the \glspl{pc} just came from.

  Their Morale check remains throughout the combat, without needing to re-roll it.
  If they roll a `7' from the start, then a Goblin with Combat +1 will have a total of `8', and one with Combat +2 would have `9'.
  If the \glspl{pc} outnumber the goblins, and one is stronger than them, then their total would become `5' and `6'.
  If a goblin became wounded, they would flee, but the others might stay.
}{
  and take a morale check.

  Two more goblins are sleeping in a corner, making six in total.
}

\pic{Decky/screech}

\goblin[\npc{\T\N}{2 Goblins on the table}]

\goblin[\npc{\T\N}{2 Goblins on the oven}]

\iftoggle{hardcore}{
  \goblin[\npc{\T\N}{2 Goblins on the floor}]
}{}

\newRule{Weapons}{
 \begin{itemize}
    \item
    The \textit{Attack Bonus} adds to their Dexterity + Combat roll when fighting.
    \item
    The \textit{Damage Bonus} adds to their Damage.
    Anyone with +4 Damage replaces their Damage Bonus with a $D6$, so $1D6+4 = 2D6$ Damage, and $1D6+5 = 2D6+1$ Damage.
    \item
    The \textit{\glspl{ap} Cost} shows how many \glspl{ap} the character must spend to use the weapon (no matter who instigated the attack).
    \begin{itemize}
      \item
      Characters cannot declare actions when they have less than 1 \glspl{ap} left, but they can (and must) still spend \glspl{ap} to defend themselves.
      \item
      Characters can get pushed into negative \glspl{ap} when they are forced to Defend themselves, which then become a penalty (being on -2 \glspl{ap} means a -2 penalty to all actions).
    \end{itemize}
    \item
    The \textit{Weight Rating} tells you what level of Strength Bonus you need to use the weapon unencumbered.
    If a weapon's Weight Rating is higher than the character's Strength penalty, then every additional point gives them -1 \glspl{ap} at the start of each combat round, and +1 Fatigue Points at the end of combat.
  \end{itemize}

\end{multicols}

\begin{nametable}[YXYYYY]{Kitchen Weapons}

  \textbf{No.} & \textbf{Name} & \textbf{Attack Bonus} & \textbf{Damage Bonus} & \textbf{\glspl{ap} Cost} & \textbf{Weight Rating} \\\hline

  1 & Broom & 1 & 0 & 2 & -1 \\
  4 & Small\chair
  2 & \Dagger 
  2 & \skillet
  4 & \Log
  1 & \woodaxe

  \end{nametable}
\begin{multicols}{2}

}

\paragraph{If the players try to find weapons,}%
\iftoggle{oneshot}{
  they can find plenty of make-shift weapons around the kitchen.

  These weapons may not be good quality, but they can still improve the \glspl{pc}' situation immensely.

  \paragraph{If the players select weapons for their characters,}
  check out the Kitchen Weapons table and have them write down the stats on their character sheet one at a time.
 }{
  they can use either of the two cast iron skillets.
}

\paragraph{If the party raid the room for food,}
they'll find a few canteens of water, one of wine, and some spare sacks of vegetables.

The players can unlock the cold store%
\footnote{A cold store is an old fashioned fridge where people put ice to keep food from going bad.}
  with a Dexterity + Larceny roll at \gls{tn} 10 to pick the lock, at which point they discover that the larder is still full of food.

\iftoggle{oneshot}{
  \paragraph{Resting a while}
  in the kitchen regenerates a number of Fatigue Points equal to half the player's current \glspl{hp} (rounded up).
  And of course, at the end of the scene each character also regenerates 40\% of their \glspl{fp}.
}{}

\mapentry[lift]{The Gnomish Lift}

\begin{boxtext}

  The great double doors swing open, revealing a wide, empty room.
  Your torchlight stretches far above, and well out of reach you can see a ceiling covered in gemstones, inlaid into the wood.
  The room appears otherwise empty.

\end{boxtext}

The gnomes created the lift with the Force sphere.
It can lift a combined Weight Rating of 16, and safely descend with a total Weight Rating of 26 standing on it.%
\footnote{A creature's Weight Rating is equal to its \glspl{hp}.
Some items count towards this total, but items with a negative Weight Rating do not.}
If the party step on it with a greater Weight Rating than this, each additional point inflicts 2 Fatigue Points on everyone in the lift upon impact with the ground.

The lift responds to magical passwords -- the gnomish words for `farm' (for the top), `sleep' (for the middle), and `food' (for the bottom).
The various ogres and goblins who use the lift only know the passwords for the bottom and middle section.
Only the nuramancer goblins know the word for the top.

The lift responds to any password within earshot.

\paragraph{If anyone tries to cling onto the walls,}
they will find less purchase than a Sun-screen salesman in a Scotland.

\magicitem{Gnomish Lift}%
{Wide, Sentient, Levitation}%
{Alchemy}%
{5 scenes}%
{Talisman}%
{5}%
{7}

\paragraph{Understanding this talisman}
requires the usual Intelligence + Academics roll at \gls{tn} 12.
However, a successful roll will only allow them to understand the word to take them to where they are, not the word to take them to another floor.

\paragraph{If the \glspl{pc} dawdle too long here,}
a Raiding party return with more prisoners to place in the cell (room \ref{entrycell}).
See \autopageref{raidingParty} for details.

\widePic[t]{Roch_Hercka/dragon}

\mapentry[dragonApproach]{Dragon's Approach}

\begin{exampletext}

  Once the gnomes turned into nura, they neglected to de-activate the portal, as it's easier to see when it's open.
  A couple of hours ago, a dragon named `Makil' wandered in, and went up the stairs towards the treasure room, but could not fit through the little gnomish door.
  He managed to pull a chest full of copper coinage through the door, but could not get anything else as he could not reach.

  The Makil wanted to barter with the goblins, but found that they responded to him either by trying to fight him, or by running away, and none spoke elvish (the only language he knows spoken in these parts).

  The cautious beast won't approach the nura to attack when he has no idea how many there are around him.
  He may be powerful, but a room full of ogres alongside a spell caster could seriously injure or kill him, and he doesn't want to take the chance.
  Similarly, the nura don't want to take any chances, so they've decided to just leave him where he is.

  The nura and dragon have ended with a stalemate, so he has taken a nap.

\end{exampletext}

\begin{boxtext}

  Ahead, three charred goblin corpses lie on the ground.
  A strange scent wanders down from above, something like a chicken cooked in sulphur.

\end{boxtext}

\paragraph{If the \glspl{pc} loot the bodies}
they find a portal scroll wrapped safely in a scroll case.

\mapentry[dragon]{The Dragon's Lair}

The dragon will happily talk with anyone who approaches, but only understands Elvish.
\iftoggle{hardcore}{}{%
  Remind the players that they can spend a Story Point to say that they know another language, as long as they say when they learnt the language.
}
Despite his linguistic deficits, he can communicate using the Enchantment sphere to send ideas to characters.

The dragon's eventual goal is to obtain the rest of the treasure, then return through the portal to the Realm of Bright Rocks.

\paragraph{If the party request he kill nura for them}
then he agrees to kill any in the current area, but will not journey to another floor.
In return, he wants all the treasure out of the treasure room.

\paragraph{If the party offer to split the treasure,}
he refuses, unless they give him a good reason.

\iftoggle{hardcore}{
  The treasure chinks noisily of course, so the party will then receive a -1 penalty to all Stealth checks.}{}

\paragraph{If the party push for more treasure,}
the dragon asks if they would like to challenge him to a game of riddles.
Each point anyone scores allows them to demand a single item, such as a chest, or quiver.

If they say `yes', then he accepts their challenge, and asks what their riddle is.%
\footnote{Use of the internet is prohibited by trans-dimensional law, common sense, basic decency, and the Geneva Convention.}
If they can think of none,
then the dragon declares that he has won the first round.

See \autoref{riddles} for riddles.

\paragraph{If the players ask why he wants treasure,}
he explains that he wants to attract a mate; when his flame becomes hot enough to melt the gold, he will carve a golden statue of the most deadly dragon in his area in order to attract her attention.
He will then decorate the statue with magical items.
\footnote{Would you prefer a mate is beautiful, or one that can kill anything in an instant?
Dragons think the answer is obvious.}

\paragraph{If the dragon parts on good terms,}
he blesses them all with a \textit{Wide Blessing} (which heals $2D6$ \glspl{fp}), and casts a \textit{Fortune} spell on the weakest member, allowing them to receive +1 to their Combat Skill for
\iftoggle{hardcore}{%
  three scenes.
}{
  the rest of the adventure.
}

\N\gls{pc}{\M}{Makil the Dragon}{Inquisitive}{Drums Fingers}{Acquisition}
\person{7}% STRENGTH
{2}% DEXTERITY 
{5}% SPEED
{{4}% INTELLIGENCE
{3}% WITS
{-2}}% CHARISMA
{5}% DR
{1}% COMBAT
{Aggression~2, Projectiles~2, Academics~3, Athletics~1, Deceit~3, Tactics~2, Vigilance~4
\knacks{\bloodCaster}
\Path{\enchantment~2, \fate~3, \invocation~4}}% SKILLS
{Nothing}% EQUIPMENT
{}

\iftoggle{oneshot}{
  \paragraph{If the \glspl{pc} attack the dragon,}
  he kills the first to attack.

  His {\scshape \gls{dr} 5} means he reduces all Damage by 5, unless the attacker hits 5 over the \gls{tn} to attack him (a total of 15), achieving a `Vitals Shot'.
}{}

\newRule{Projectiles}{
  To use shortbows, crossbows, or just throwing a rock at someone, the player rolls $2D6$ plus their character's Dexterity + Projectiles.

  The basic \gls{tn} is 7, but can go up for targets who are farther away.

  If \glspl{npc} decide to \textit{Keep Edgy} (watching out for missile attacks), they can add their Speed + Vigilance to the \gls{tn}.
  This costs 1 \glspl{ap}, as they need to spend time looking around the room to see where archers (or spellcasters) are.

  So a goblin with Speed 2, Vigilance 1, would be \gls{tn} 7 to hit normally, but once they spend 1 \glspl{ap} to Keep Edgy, their \gls{tn} would be 10.

  \begin{itemize}
    \item
    Shortbows deal $1D6-1$ Damage.
    \item
    Improvised projectiles take a -2 penalty to hit and -2 Damage.
  \end{itemize}
}

\mapentry[treasureRoom]{The Treasure Room}

\begin{boxtext}

  Through the door, two locked chests lie on the ground.
  Above them, a shortbow and two beautiful short swords stand affixed to the wall, with a quiver of arrows with gemstones used as arrow tips.
  \iftoggle{hardcore}{%
    The scroll cabinet has two scrolls left.
  }{The scroll cabinet has three scrolls inside.}

\end{boxtext}

\begin{itemize}

  \item{A short bow}
  \item{A buckler shield made of pure silver, worth 30sp (it breaks after one use)}
  \item{Two gem-encrusted shortswords (each are mana stones carrying 2 \glspl{mp})}
  \item{Four magical arrows which explode, dealing $2D6$ damage upon striking}
  \item{A chest containing 432 \glspl{cp}}
  \item{A chest containing 300 \glspl{sp}}
  \item{A Scroll of Stone}
  \item{Lock Scroll}
  \iftoggle{hardcore}{}{
    \item
    Portal Scroll
    \footnote{See \autoref{creaturesAndItems} for these scrolls.}
  }

\end{itemize}

\magicitem{Scroll of Stone}{Fast Transmutation}{Alchemy}{1 Scene}{Pocket Spell}{6}{1}
One spoken, the scroll rolls $2D6+6$ at a \gls{tn} of 7 plus the target's Weight Rating.
If the spell succeeds, the target turns to stone.

This spell was designed to be used on the caster; the gnome would turn themself to stone in order to become uninteresting to oozes and such for a short while.
However, people can try to use it against other targets.

Like any other scroll, the activation word is a riddle.
Unlike some of the others, it works \emph{instantly}.

\magicitem{Lock Scroll}{Fast Lock}{Alchemy}{2 Scenes}{Pocket Spell}{4}{3}

This scroll locks a door, which increases the \gls{tn} to open it by 4.
It can also be cast on a passageway, which obstructs it by adding a magical barrier with 9 Shield Points (`SP' work roughly the same as \glspl{hp} for magical fields).
\exRef{core}{core rules}{spellLock}

\end{multicols}

\section{Mid Levels}

\begin{multicols}{2}

\mapentry[nursery]{Nursery}

\begin{boxtext}

  The sound of crying emanates from the door as you swing it open.
  A dozen baby creatures with fat little noses lie in a crib with hay, looking up at you in terror.
  The entire crib and the floor around stink of shit.

\end{boxtext}

\begin{exampletext}

  When the nura arrived, the gnomes put their children in here, then left enough food for a few days, locked the door, and destroyed the key.
  Unfortunately, those gnomes have turned into goblins, and if they entered the room they would probably eat their own children.

\end{exampletext}

The door is locked, but can be picked with a Dexterity + Larceny check, \gls{tn} 9.
It's far too strong to be broken into by force.
If the \glspl{pc} enter the room, they may think these little gnomes are little goblins and kill them.
If any of them try to do so, have them roll a Wits + Medicine check to realize their mistake, \gls{tn} 8.
Any gnomes in the party will automatically pass this check.

\iftoggle{oneshot}{
  Anyone following Laiqu\"e who successfully rescues a child from the warren gains 3 \glspl{xp} at the end of the session.
}{}

\paragraph{If the party save the children,}
the children know to know to keep quiet and to follow, but they will be at the back of the party when moving alone.

There are six children in total, and they each have a Weight Rating between 1 and 2.
\iftoggle{oneshot}{
  Characters can only carry a Weight Rating equal or less than their Strength Bonus, or they will take penalties to their \glspl{ap}, and gain Fatigue Points at the end of each scene of activity.

  \paragraph{If the \glspl{pc} rest,}
  Remember to have them heal Fatigue Points equal to half their \glspl{hp}.
}{}

\mapentry[slugHall]{Slug Hall}

\begin{boxtext}

  The doorway reveals a massive hallway of sparkling, floating, lights.
  Across the earthy floor, pieces of paper lie everywhere.
  Giant slugs lazily wander, masticating them.
  Half of the slugs wear the pages, as if someone had thrown the paper in the room from above.
  Around those pages, little grubs eat into the massive slugs.
  The moment you enter, their eye-stalks perk up, and they begin to slide off the corpses they were feasting on.
  To your left, stairs head upwards into darkness.

\end{boxtext}

\begin{exampletext}

Gnomes grew mushrooms throughout this room in order to grow slugs, so that they could feed fireflies.
While torches work best, having omnipresent fireflies around the warren makes sure that people can coordinate between rooms without worrying about light.

Since then, the nura turned those slugs into nura slugs.
The fireflies have survived on the enlarged slug, which the nura keep alive with the scraps they don't want to eat, such as mushrooms and bones and faeces.

\end{exampletext}

\paragraph{If the \glspl{pc} run up the stairs,}
have them roll initiative against the slugs.

From that point until the slugs loose sight of them, they are in combat.
The slugs will spray acid at them, follow them up the stairs, and pester them for as long as they remain in sight.

\paragraph{If the \glspl{pc} throw in some food,}
the nura slugs chase it rather than them.

\paragraph{If the \glspl{pc} manage to investigate the corpses somehow,}
they find two dead gnomes, one with a torch, the other with a Portal Scroll.
\iftoggle{oneshot}{%
  As with the others, the Portal Scroll requires a riddle to be answered, takes four rounds before it activates, and vanishes once used.
}{}

\nuraslug[\npc{\T\N}{20 Nura Slugs}]

\mapentry[greatHall]{The Great Hallway}

\begin{boxtext}

  At the top of the stairs, this massive chamber lies empty, except for the fireflies darting about.
  To the right, there are two short tunnels.
  Ahead of you, two grand staircases lead up into a dim but unwavering light.
  Below you, the entire floor is sticky and greasy.
  Somewhere close, you hear snoring.

\end{boxtext}

If the \glspl{pc} have indeed been quiet enough in the previous room to not raise an alarm, they find everyone in nearby rooms napping.
A single sound means they will be in serious trouble.

The greasy floor results from a mixture of faeces, drool, blood and leftover mushroom-juice.
Anyone who can use Conjuration magic will be able to turn the offal on the floor into a slippery substance.

\paragraph{If the \glspl{pc} have come from room \ref{slugHall},}
they will not immediately see the staircase on their left, but will see it after doing literally anything (fighting, searching, et c.).

\paragraph{If the \glspl{pc} tarry or talk,}
have them make a Dexterity + Stealth check, \gls{tn} 6.
They make this as a \textit{Group Roll}, so a single roll counts for the whole group.
Each margin on the roll allows them an additional round before the horde wakes.
\iftoggle{core}{%
  \footnote{See the core book, page \pageref{grouproll} for group rolls and page \pageref{margin} for roll margins.}
}{}

\mapentry[workshop]{The Workshop}

\label{laddertop}

\begin{exampletext}

  Frightened gnomes fled their bedrooms from the nura who had rushed through the portal.
  Most were caught, but some managed to run up the trapped ladders, having memorized the sequence perfectly.
  As the last one got to the top of the ladders, he turned to cast an illusion of a solid wall in order to fool anyone coming up behind him.

\end{exampletext}

\begin{boxtext}

  Some goblins and three ogres lie sleeping on the floor between workbenches.
  The place is so full, you can't make out how many lie here, but the snoring indicates more than you can see.
  On the benches, most of the equipment lies broken, but obviously delicate gnomish hands once used these tables to polish gems, craft magical items, and forge digging equipment.
  On one table, you can see a pile of weapons -- short swords and spears -- piled on a table.

\end{boxtext}

Picks, shovels, wood, short swords, shortbows, and all manner of crafting and mining equipment litter the room.

\paragraph{If the \glspl{pc} have come up from room \ref{entrycell},}
they find the exit has been covered by an illusion of a wall.
If they investigate the area at all, the illusion fades, revealing the sleeping nura.

\paragraph{If the \glspl{pc} attempt to take either a short sword or a spear,}
each attempt requires a Dexterity + Stealth roll, \gls{tn} 7.
Failure will awaken the entire horde.

\iftoggle{oneshot}{
  You can find stats for the weapons in \autoref{creaturesAndItems}.
}{}

\pic{Decky/armoury}

\goblin[\npc{\N\T}{\arabic{enemyNo} Goblins}]

\addtocounter{enemyNo}{-2}

\ogre[\npc{\N\T}{\arabic{enemyNo} Ogres}]

\mapentry[grandLibrary]{The Grand Library}

\begin{boxtext}

  At the top of the stairs you find the ruins of a massive library.
  Book cases lie in a smashed heap on the ground, others appear to be used as a makeshift bed for an ogre.
  The books themselves are gone, except for a few scrolls, now tightly clutched by a goblin in a black cowl.
  His hand shines with a thumb-ring, showing three massive gemstones.

\end{boxtext}

In total, the room contains the following magical items:

\magicitem{Bowl of Water}{Conjuration}{Alchemy}{2 Scenes}{Talisman}{4}{3}

This bowl is always full of water.
If it empties, then the air around it quickly forms into more water.

\magicitem{Lock Scroll}{Lock}{Alchemy}{2 Scenes}{Pocket Spell}{4}{3}

This lock scroll will lock any door, increasing the \gls{tn} to break through it by 4.

As with any Portal Scroll, it leads to the Realm of Shifting Corridors.

\magicitem{Ring of Wishes}%
{Conjuration 4}%
{Alchemy}%
{5 Scenes}%
{Talisman $\times 3$}%
{5}%
{7}

This ring is actually three pocket spells -- each ruby in the ring is a separate magical item, and each use will make one ruby go dark.

The caster can summon anything available for Conjuration level 4.
With 7 \glspl{mp}, the ring has a Metamagic score of 7, allowing it to cast \textit{Wide} spells, and other Metamagic enhancements.

\iftoggle{oneshot}{%
  The spell can broadly summon any item up to the size of a person.
}{}

\magicitem{Scroll of Insight}%
{Massive Mage Sight}%
{Alchemy}%
{4 Scenes}%
{Pocket Spell}%
{4}%
{3}

The spell allows the user to feel 4 areas around (meaning `rooms'), though not with much detail.
The effect feels very disorienting at first, as most people are not used to having their consciousness spread apart so far.

\magicitem{Staff of Light}%
{Wide Light}%
{Alchemy}%
{Continuous}%
{Talisman}%
{3}%
{3}

This `staff' will probably feel more like a wand in the hands of a human, as it's about the height of a short gnome.
The gnomish word for `morning', activates the light, while the gnomish word for `night' stops it.
The nura do not know, or have forgotten, these activation words, so they simply keep it under blankets as they find the light irritating.

If wrapped up in heavy sheets and suddenly displayed, the light can blind opponents, as usual.

\goblinnuramancer

\iftoggle{hardcore}{
  \ogre[\npc{\N\T}{2 Docile Ogres}]
}{
  \ogre[\npc{\N\T}{Sleeping Ogre}]
}

\paragraph{If any of the \glspl{pc} attempt to sneak in,}
have them roll Dexterity + Stealth, \gls{tn} 8.

Failure will, of course, spell disaster, but success will allow them to steal a magical item.
If the player wants to steal multiple magical items, describe them and see how many they decide to take.
Each item taken increases the roll's \gls{tn} by 1, so taking 3 items would mean a \gls{tn} of 11.
The player should not roll again -- the original roll remains, but increasing the \gls{tn} may well turn success into awful failure.

\end{multicols}

\iftoggle{oneshot}{%
  \section{Last Level}
}{
  \section{Upper Warren}
}

\begin{multicols}{2}

\mapentry[windingStairs]{Winding Stairs}

\begin{boxtext}

  As you round the stairs' third turn, you see a massive ogre crouching by a door.
  It blocks the path completely.

\end{boxtext}

\paragraph{If the party have made a reasonable attempt at staying quiet,}
they can avoid alerting this ogre with a Wits + Stealth roll, \gls{tn} 9.
Whoever is at the front makes the roll.
If it's unclear who's at the front, the character with the highest Speed + Athletics is in the lead.
With a successful roll,
\iftoggle{hardcore}{
  the ogre is resting, and must take a round to gather what's left of his wits, but it will still wake if approached.
}{%
 the party find the ogre sleeping.
}

\ogre[\npc{\N\M}{Rick, the Ogre Guard}]

\paragraph{If the party try to talk with the ogre,}
Rick has only recently been turned into an ogre, and he has eaten his fill of mushrooms, so a Wits + Empathy roll, \gls{tn} 10, will allow the party to convince him to let them go.
However, if Rick lets them go, he will insist on joining them so he can be free.
While he genuinely wants to escape and become human again,%
\footnote{Nura who have recently turned can change back to their original forms if they are starved for twice as long as they have been nura.}
the moment the party enter battle with other nura his instincts will kick in, and he will turn on the party.%

\mapentry[secondPrison]{Second Prison}

\begin{exampletext}

  This little room once housed a full family of gnomes, but now serves only as another prison.

\end{exampletext}

The prisoners require no locks or handcuffs -- the ogre waiting outside suffices to terrify them into staying put.

\paragraph{If any of the \glspl{pc} have died,}
another player should spend 2 Story Points to introduce an old friend.
\iftoggle{hardcore}{}%
  \iftoggle{core}{%
    Pick another pre-made character sheet from the pile and hand it to any players without characters.
  }{}%
The players should agree on how they know each other.

Take the last two farmers from the handout to join the troupe.

\iftoggle{hardcore}{%
  \paragraph{If the party get a moment to ask about the outside world,}
  the villagers tell them that they last saw hundreds of nura swarming around town, and she was on her horse to get away.
  They suspect that the entire place will have been overrun.
}{}

\iftoggle{oneshot}{% This weapon chart goes next to the armoury
  \begin{figure*}[t!]
  \begin{nametable}[YXYYYY]{Armoury Weapons}

  \textbf{Quantity} & \textbf{Name} & \textbf{Attack Bonus} & \textbf{Damage Bonus} & \textbf{\glspl{ap} Cost} & \textbf{Weight Rating} \\\hline

  3 & Bucklar Shield & +2 & None & 1 & 0 \\
  3 & \shortsword
  7 & \woodaxe

  \end{nametable}
  \end{figure*}
}{}

\mapentry[armoury]{Armoury}

\begin{boxtext}

  At the top of the stairway, three dying fireflies wander pointlessly.
  Behind them, you see shadows, with a glimpse of metal, lying in an alcove.
  To the left, a wooden door waits, with dirty footprints visible on the floor.

\end{boxtext}

\begin{exampletext}

  The gnomes once stashed their little weapons here.
  The nura horde have added to it considerably.

\end{exampletext}

\begin{itemize}

  \item{3 buckler shields}
  \item{1 crossbow (unstrung but usable with an Wits + Crafts roll, \gls{tn} 7)}
  \item{3 quivers, each with 20 arrows}
  \iftoggle{hardcore}{%
    \item{3 crossbow bolts}
    \item{2 shortbows (also unstrung)}
  }{
    \item{8 crossbow bolts}
    \item{2 shortbows}
  }
  \item{3 shortswords}
  \item{7 wood axes}

\end{itemize}

\iftoggle{oneshot}{
  \paragraph{Bucklar Shields}
  These shields work like a weapon, except that they deal no damage.
  They require only 2 \glspl{ap} to use in combat, but add a +2 Bonus.

  \paragraph{The crossbow}
  (if repaired) deals $1D6+3$ Damage, but requires at least 4 rounds to reload.

  \paragraph{Shortbows}
  Require 1 \glspl{ap} to loose an arrow, and 1 \glspl{ap} to reload.
  However, they deal only $1D6-1$ Damage.
}{}

\mapentry[trappedHall]{Trapped Hallway}

\begin{boxtext}

  Your lantern illuminates dirty footprints leading out the door, to the left, and down a dark corridor.
  To your right, two statues of ogres stand, like stooping gargoyles, blocking the path.
  All around, the floor glistens with tiny gemstones carved into the centre of flagstones.

\end{boxtext}

The squares between the exit door and the party's path can turn anyone stepping upon them into stone.
One goblin was unfortunate enough to trigger the trap.
The second goblin did not believe the original goblins about why the first was there, and ignored the statue, so he was turned to stone too.

This magical trap didn't look quite right when the flagstones had to be inlaid with gems in order to complete the alchemical spell -- that kind of shine gives the game away.
The gnomes' solution was to make sure \textit{all} flagstones had a magical amulet in them, so nobody could tell which might hold a deadly spell.

\magicitem{Petrifying Flagstones}% NAME
  {Stonespell}% SPELL
  {Alchemy}% PATH
  {8 Scenes}% DURATION
  {Talisman}% TYPE
  {8}% POTENCY
  {7}% MP

Anyone standing on the stone rolls their Weight Rating at \gls{tn} 15 to avoid being turned to stone.
\iftoggle{oneshot}{%
  Each character's Weight Rating is equal to their \glspl{hp}.
}{}

\paragraph{If anyone steps into the trap,}
they must roll their Weight Rating at \gls{tn} 15, but anyone failing may spend 5 \glspl{fp} to ignore the effects of the trap.
If this happens, explain that their foot or hand begins to harden, then turn to stone for half a second before they retract it.

\paragraph{If the players figure out that the area is trapped and try to jump over it,}
remind them that they cannot see where the trap starts and where it ends.
If they avoid stepping into the trap, they can make a Speed + Athletics roll, \gls{tn} 9, to jump over the afflicted area.

\widePic{Roch_Hercka/garden}

\mapentry[topShaft]{The Top of the Shaft}

\begin{boxtext}

  Ahead lies to massive double-doors.
  To the right, the hallway extends into a massive room, where you can hear the sound of sawing.

\end{boxtext}

Grank the nuramancer goblin has heard the \glspl{pc} coming, and has no intention of fighting them alone.
He knows he has the only key to the exit door, so he intends to loose the hell-hounds he has tied up at the room's side before fleeing into the fungal gardens.

\iftoggle{hardcore}{%
  By the time the party reach round the corner, he will have cut the ropes disappeared, leaving unchained nura wolves.
}{
  The party arrive to see him still sawing through the ropes, and have only a single round to stop him before he runs.
}

\paragraph{No matter what the party do,}
the wolves go straight for the kill.

\paragraph{If the party attempt to run through the double doors,}
they will suddenly find an empty lift-shaft, unless they have taken the lift to the top themselves.
Have them make a Wits + Athletics check (\gls{tn} 7) to back off before they fall.

\nurawolf[\npc{\A\N\T}{4 Nura Wolves}]

\paragraph{If the party fall into the lift,}
they end up in the mid-section of the warren.

\mapentry[fungusGarden]{Fungal Gardens}

\begin{boxtext}

  As the doors open, your torchlight shows a crowded mess of fungus.
  Some reach up to your knees, others tower above the light and so far that you cannot see the top.
  The maze of unkempt mushrooms shows random, meandering paths between the taller fungi.

\end{boxtext}

\begin{exampletext}

  This beautiful fungal garden took dripping rain from above, and sieved it through the roof then the soil below, until it distributed nutrients for a forest of mushrooms, big and small.
  The fungal garden was regularly invaded by oozes which can creep into small cracks when young, and grow massive quickly.
  The nura never really kept up with the garden's maintenance, so the room festered with dangerous jellies.

\end{exampletext}

While the place looks serene, it is inhabited by
\iftoggle{hardcore}{%
  dangerous oozes.

}{%
  a dangerous ooze.
}

\paragraph{Once the players enter the room,}
\iftoggle{hardcore}{%
  the oozes begin to stalk them.
  If multiple oozes chase them, the smaller ones will always back away from the larger ones, so no more than one ooze should follow them at a time (always the largest one).
}{%
  the ooze begins to stalk them.
}

\paragraph{If the \glspl{pc} approach Grank,}
he will hide while casting a \textit{Wide, Raging Fireball}.
Finding him in the darkness before he finishes his spell requires a Wits + Vigilance roll, \gls{tn} 10.
If he is successful, the party make a group roll against \gls{tn} 9 to avoid the fireball ($1D6+3$ Damage).
\iftoggle{core}{%
See the core book, page \pageref{edgy} for more on Keeping Edgy to avoid missile attacks.}%
{
Anyone \textit{Keeping Edgy} can add their Speed + Combat to the roll, otherwise, they gain no bonus.
}

\paragraph{If Grank ever feels like his life is under threat,}
then he will taunt the \glspl{pc} with the key to the outside world he has in is position, and throw it into the nearest ooze.
He then lets out a giggle and dashes off into the fungal undergrowth, leaving the players to face the hulking pulsating mass.

\label{grank}
\npc{\M\N}{Grank}
\person{-2}% STRENGTH
{1}% DEXTERITY 
{4}% SPEED
{{1}% INTELLIGENCE
{2}% WITS
{-4}}% CHARISMA
{0}% DR
{1}% COMBAT
{Projectiles~1, Athletics~2, Medicine~1, Stealth~2, Tactics~2%
\knacks{\nuraCaster}
\Path{\invocation~2, \necromancy~3, \saurecanta~2}}% SKILLS
{\iftoggle{hardcore}{Seeing Stone}{\Dagger}}% EQUIPMENT
{}

\iftoggle{hardcore}{%
  \jelly

  \jelly

}{}

\jelly

\iftoggle{hardcore}{
  \magicitem{Seeing Stone}{Clairvoyance}{Alchemy}{Continuous}{Talisman}{3}{1}

  The Seeing Stone is a simple stone with a hole.
  Staring into the hole allows one to watch the Realm of Bright Rocks.
  Particularly perceptive watchers will notice that those watching in the stone can see exactly the spot where the portal on the bottom level of the warren opens.
}{}

\mapentry[lockedDoor]{The Locked Door}

The locked door has a bronze border and cannot be broken without a Strength + Crafts roll, \gls{tn} 18.
The exit key, however, fits in nicely.

If the party rest here long enough (perhaps because they have failed to get any key), a raiding party of nura return.
See page \pageref{raidingParty} for the returning raiding party.

\mapentry[lowerExit]{The Exit}

\iftoggle{oneshot}{
  \begin{boxtext}

    As the key turns, the door swings open, and daylight floods in.
    Green trees cover the road down the hill, and in the far distance, chimney-fires from little hamlets wander into the sky.

    You have finally escaped from the warren.

  \end{boxtext}
}{
  This is where the party exit the lower portion of the warren, and enter the upper part.

  \paragraph{If the \glspl{pc} don't have a key,}
  they might try tricking the goblin on the other side, who has a key to let people in and out.
  See area 1, \autoref{upper}.

  \paragraph{If the \glspl{pc} examine the door,}
  they can see the goblin in the next area by a glimmer of candle-light.
}

\end{multicols}
