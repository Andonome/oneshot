\chapter{The Warren}

\begin{figure*}[t]
	\includesvg{images/Dyson_Logos/lower}
\end{figure*}

\section{The Lowest Level}

\begin{multicols}{2}
\setcounter{list}{0}

\mapentry{Cells}
\label{entrycell}

\begin{exampletext}

	Gnomes erected cots, cribs and hammocks here to be used as a communal sleeping area.
	Since then, goblins have placed a bar over the outside of the door to house prisoners.

\end{exampletext}

This is the room where the PCs awaken.
Run the first event in the time line -- \textit{the Escape} (page \pageref{escape}).

\paragraph{If the PCs exit quietly,}
they will likely be spotted in the next room.

\paragraph{If the party do not manage to escape}
then add 3 Fatigue Points for their time starving in prison, and throw in a load of new prisoners.

Once new prisoners are in, the players can spend Story Points to introduce old friends, and have \textit{them} roll to escape from bondage.

\mapentry{Exit Shaft}

\begin{exampletext}

	This exit shaft was used by gnomes to travel up quickly.
	Aware of the dangers below, they trapped the ladder and hid the door at the top.
	Since then, the goblins have forgotten how the ladder works, and stopped using it.

\end{exampletext}

If the players attempt to climb, they find that every rung of a multiple of five (the fifth, tenth, fifteenth, and so on), are coated in a slipper substance made from the fungal gardens.
The rungs which are a multiple of 7 are set to break once anything with a Weight Rating 7 or more will break the rung.%
\footnote{A creature's Weight Rating is equal to its HP. \iftoggle{core}{%
	See the core book, page \pageref{weightrating}.
}{}}
These breakable sections are actually solid illusions, so they appear again shortly after breaking, but will resist any markings such as paint, as those marking disappear as the illusion resets every so often.
Meanwhile, any rung which is both a multiple of 5 and 7 indicates that the next section is invisible and has switch to the wall on the left.

Despite the ladder switching places, it appears to be one solid piece, reaching upwards.

The nura have since abandoned this path, and have forgotten the door at the top.

\paragraph{If the players climb up the staircase without understanding the trap,}
have them roll Dexterity + Athletics, TN 10.
If they fall, they receive $1D6-2$ Damage.

If they continue climbing, give additional Dexterity rolls.
The character can then make an Intelligence + Academics roll to understand the trap, TN 12.
Whatever number they roll, keep that result; it will tell you how long the character takes to figure out the trap.
Every time they make a new roll:

\begin{itemize}

	\item{The Damage they get for failing increases by 1.}
	\item{The roll to understand the trap decreases by 1.}

\end{itemize}

Of course, if the player figures out the trap, all rolls can be dispenses with, and the player can ascend to room at the top, on page \pageref{laddertop}.

\mapentry{Dining Room}

\begin{boxtext}

	As you climb the stairs away from your cell you enter a room full of benches and tables, all overturned and broken. 
	The floor is strewn with pots and plates, knives, forks and candlesticks, but no candles and not a scrap of food remains.
	It looks like a pack of ravenous dogs swept through here and licked ever morsel from every surface.
	In the middle of the room you can see upon a table two creatures -- a goblin and the other you would call a rat, but only if the rat had been mated with a Great Dane.
	The two creatures seem to be fighting over something and as you focus on the scene before you it becomes clear that what they are fighting over is a bone (possibly human).
	It looks at first like they are pulling at it, one on each end, but then you hear the sound of grinding and snapping and you realise that they are both just chewing at either end.

\end{boxtext}

\paragraph{If the players make a sound}
the goblin will stop fighting with the rat for the ownership of the bone and run out of the room shouting for the shaman who had left them in the cell previously.
There are two knives in this room that can be wielded as daggers.

\paragraph{If the party have crept in quietly, then leave}
the nura rat and goblin will leave.
The goblin will go up the lift, and the rat will leave to wander the hall, searching for more food.

\goblin[\npc{\F\N}{Hungry Goblin}]

\nurarat[\npc{\A\N}{Nura Rat}]

\mapentry{Spellcasters}

\begin{boxtext}

	Two goblins wearing long black robes bicker with each other over a little table.
	\iftoggle{hardcore}{%
		Other goblins stand at the side, apparently bored of the argument.
	}{}

\end{boxtext}

The two spell casters cannot figure out what the scrolls on the table do, despite the fact that one is a Scroll of uprising, brought here by the nura.

\paragraph{If the players try to sneak past,}
the roll is Dexterity + Stealth, TN 10.
They will also have to put out their torch, if they have one.

\paragraph{If the characters investigate the table,}
they find a Portal Scroll, and a Scroll of Uprising.

\magicitem{Scroll of Uprising}{Ghoul Calling}{Saurecanta}{Instant}{Pocket Spell}{+3}{7}

Once the spell is cast, all creatures with 11 HP or fewer in the surrounding 3 areas, raise from the dead as ghouls.
See Appendix \ref{undeadstats} for stats on the various ghouls.

\magicitem{Portal Scroll}{Unrestrained, Open Teleport}{Alchemy}{2 Scenes}{Pocket Spell}{4}{5}

The scroll takes 5 rounds to speak properly, and opens a doorway to the Realm of Shifting Corridors.
The portal remains open for 2 scenes.

\goblinnuramancer[\npc{\F}{Screamer}]

\goblinnuramancer[\npc{\M}{Brock}]

\iftoggle{hardcore}{
\goblin[\npc{\T}{3 Goblins}]
}{}

\mapentry{Magic Portal}

\begin{boxtext}

	In this room there stands before you a Magical portal 8' high and 4' wide.
	A warm wind blows through the portal and on the other side you can see an endless desert of bright, yellow rocks.
	Runic writing sits above the portal.
	The room seems empty except for the scuff marks leading to and from the portal, and the massive double-doorway behind you. 

	The passageway continues, but becomes quickly lost in darkness as it loses the Sunlight through the portal.

\end{boxtext}

\paragraph{If any of the PCs are literate and speak the Gnomish tongue,}
they can read the runes above the magical portal; they read ``Desert Realm''.

\paragraph{If the PCs enter the portal and journey through the Realm of Bright Rocks,}
they won't find much -- not even a place to hide, given the desert is expansive and flat.
If they stay there long, give them the usual 4 Fatigue Points and roll an encounter for that Realm.%
\footnote{See Adventures in Fenestra for more on the Realm of Bright Rocks\iftoggle{aif}{%
	, page \pageref{brightrocks}}%
	{}
.}

\mapentry{Kitchen}

\begin{boxtext}

The door creaks open and as you peer into the darkness you can just make out the 5 figuers sprawled over tables and chairs or curled up on the floor snooring quietly. 
From the embers of the hearth you can see the ovens and cooking utensils that make up a substantial kitchen. 
Opening the door further, the light from the hall's sconces fills the room, and you can see that the five figures are more goblins with little fat bellies.
At the end of the kitchen you can see a large door to a cold store with a lock on it and some large cleavers stuck into a butchers block.

\end{boxtext}

\noindent
In the room the players can find two cleavers which would be the equivalent to short swords and two large pot lids which could be used as bucklers.
There is also a sixth goblin in the room hiding in the far corner in the dark; players can make a Wits + Vigilance roll at TN 8 to spot him.
Should they make to much noise or light up the room, all goblins will wake up and attack.
The players can unlock the cold store
\footnote{A cold store is an old fashioned fridge where people put ice to keep food from going bad.}%
with a Dexterity + Larceny roll at TN 10 to pick the lock, at which point they discover that the larder is still full of food.


\goblin[\npc{\T\N}{2 Goblins on the table}]

\goblin[\npc{\T\N}{2 Goblins on the floor}]

\goblin[\npc{\T\N}{2 Goblins on the oven}]

\mapentry{The Gnomish Lift}

\begin{boxtext}

	The great double doors swing open, revealing a wide, empty room.
	Your torchlight stretches far above, and well out of reach you can see a ceiling covered in gemstones, inlaid into the wood.
	The room appears otherwise empty.

\end{boxtext}

The gnomes created the lift with the Force sphere.
It can lift a combined Weight Rating of 13, and safely descend with a total Weight Rating of 26 standing on it.%
\footnote{A creature's Weight Rating is equal to its HP.
Some items count towards this total, but items with a negative Weight Rating do not.}
If the party step on it with a greater Weight Rating than this, each additional point inflicts 2 Fatigue Points on everyone in the lift upon impact with the ground.

The lift responds to magical passwords -- the gnomish words for `farm' (for the top), `sleep' (for the middle), and `food' (for the bottom).
The various ogres and goblins who use the lift only know the passwords for the bottom and middle section.
Only the nuramancer goblins know the word for the top.

The lift responds to any password within earshot.

\paragraph{If a party member goes up the lift without the party,}
player the encounter quickly.
They will most likely die at the top.

\paragraph{If anyone tries to cling onto the walls,}
they will find less purchase than a Sun-screen salesman in a Scotland.

\mapentry{Dragon's Approach}

\begin{boxtext}

	Ahead, three charred goblin corpses lie on the ground.
	A strange scent wanders down from above, something like a chicken cooked in sulphur.

\end{boxtext}

\begin{exampletext}

	These three goblins came up the stairs with a plan to use a portal scroll next to the dragon, then throw javelins at it until it went away.
	The dragon, however, was faster.
	It incinerated them without a pause.

\end{exampletext}

\paragraph{If the party loot the bodies}
they find only the charred remains of a portal scroll.
They will be able to figure out that it has the same markings as any other portal scroll they find, but will probably not figure out what this scroll does just from its remains (TN 22).

\mapentry{The Dragon's Lair}

\begin{exampletext}

	Shortly after the nura arrived from the Realm of Bright Rocks, a dragon followed.
	He wandered up the stairs towards the treasure room, but could not fit through the little gnomish door.
	He managed to pull a chest full of copper coinage through the door, but could not get anything else as he could not reach.

	The nura wandered up the stairs, and found out too late they could not get any treasure.
	The dragon wanted to barter with them, and make a full army, but found that all the goblins and ogres responded to him either by trying to fight him, or by running away.

	The cautious beast won't approach the nura to attach when he has no idea how many there are around him.
	He may be powerful, but a room full of ogres alongside a spell caster could seriously injure or kill him, and he doesn't want to take the chance.
	Similarly, the nura don't want to take any chances, so they've decided to just leave him where he is.

	The nura and dragon have ended with a stalemate.

\end{exampletext}

The dragon will happily talk with anyone who approaches, but only understands Quenya (the Elvish language).
Remind the players that they can spend a Story Point to say that they know another language, as long as they say when they learnt the language.
Despite his linguistic deficits, he can communicate using the Enchantment sphere to send ideas to characters.

The dragon's eventual goal is to obtain the rest of the treasure, then return through the portal to the Realm of Bright Rocks.

\NPC{\M}{Makil the Dragon}{Inquisitive}{Drums Fingers}{Acquisition}
\person{7}% STRENGTH
{2}% DEXTERITY 
{5}% SPEED
{{4}% INTELLIGENCE
{3}% WITS
{-2}}% CHARISMA
{5}% DR
{2}% COMBAT
{Aggression 2, Projectiles 2, Academics 3, Athletics 1, Deceit 3, Tactics 2, Vigilance 4\Path{Blood}{Enchantment 2, Fate 3, Invocation 4}}% SKILLS
{Nothing}% EQUIPMENT
{\mana{8}}

\paragraph{If the party request he kill nura for them}
then he obliges, in return for getting all treasure out of the treasure room, and leaving it with him.
\paragraph{If the party offer to split the treasure,}
he agrees to letting them take half the coinage, while he takes the other half and all other items.
\paragraph{If the party push for more treasure,}
the dragon asks if they would like to challenge him to a game of riddles.
Each point anyone scores allows them to demand a single item, such as a chest, or quiver.

If they say `yes', then he accepts their challenge, and asks what their riddle is.%
\footnote{Use of the internet is prohibited by trans-dimensional law, common sense, basic decency, and the Geneva Convention.}
If they can think of none,
then the dragon declares that he has won the first round.

See Appendix \ref{riddles} for riddles.

If the dragon parts on good terms, he blesses them all with \textit{Fortune}, allowing them to receive +1 to their Combat Skill for the remainder of the adventure.

\mapentry{The Treasure Room}

\begin{boxtext}

	Through the door, two locked chests lie on the ground.
	Above them, a shortbow and two beautiful short swords stand affixed to the wall, with a quiver of arrows with gemstones used as arrow tips.
	\iftoggle{hardcore}{%
		The scroll cabinet has two scrolls left.
	}{The scroll cabinet has three scrolls inside.}

\end{boxtext}

\begin{itemize}

	\item{A short bow}
	\item{A buckler shield made of pure silver, worth 30sp (it breaks after one use)}
	\item{Two gem-encrusted shortswords (each are mana stones carrying 2 MP)}
	\item{Four magical arrows which explode, dealing $2D6$ damage upon striking}
	\item{A chest containing 432cp}
	\item{A chest containing 300sp}
	\item{A Scroll of Stone}
	\item{Lock Scroll}
	\iftoggle{hardcore}{}{
		\item{Portal Scroll}
	}

\end{itemize}

\magicitem{Scroll of Stone}{Transmutation}{Alchemy}{1 Scene}{Pocket Spell}{6}{1}
The spell rolls at a TN of 1 plus the caster's Weight Rating (which is equal to their HP).
If the spell succeeds, the reader turns to stone.

This spell was designed to evade oozes or other unintelligent creatures by becoming an uninteresting object for a short while.

\magicitem{Lock Scroll}{Lock}{Alchemy}{2 Scenes}{Pocket Spell}{4}{3}



\end{multicols}


\section{Mid Levels}

\begin{multicols}{2}

\mapentry{Nursery}

\begin{boxtext}

	The sound of crying emanates from the door as you swing it open.
	A dozen baby creatures with fat little noses lie in a crib with hay, looking up at you in terror.
	The entire crib and the floor around stink of shit.

\end{boxtext}

\begin{exampletext}

	When the nura arrived, the gnomes put their children in here, then left enough food for a few days, locked the door, and destroyed the key.
	Unfortunately, those gnomes have turned into goblins, and if they entered the room they would probably eat their own children.

\end{exampletext}

The door is locked, but can be picked with a Dexterity + Larceny check, TN 9.
If the PCs enter the room, they may think these little gnomes are little goblins and kill them.
If any of them try to do so, have them roll a Wits + Medicine check to realize their mistake, TN 8.
Any gnomes in the party will automatically pass this check.

\mapentry{Slug Hall}

\begin{boxtext}

	The doorway reveals a massive hallway of sparkling, floating, lights.
	Across the earthy floor pieces of paper lie everywhere.
	Giant slugs lazily wander, masticating them.
	Half of the slugs wear the pages, as if someone had thrown the paper in the room from above.
	Around those pages, little grubs eat into the massive slugs.
	The moment you enter, their eye-stalks perk up, and they begin to slide off the corpses they were feasting on.
	To your left, stairs head upwards into darkness.

\end{boxtext}

Gnomes grew mushrooms throughout this room in order to grow slugs, so that they could feed fireflies.
While torches work best, having omnipresent fireflies around the warren makes sure that people can coordinate between rooms without worrying about light.

Since then, the nura turned those slugs into nura slugs.
The fireflies have survived on the enlarged slug, which the nura keep alive as they prefer the human flesh and fresher mushrooms, rather than the grass, corpses, and faeces which the slugs will eat.

\paragraph{If the PCs run up the stairs,}
have them roll initiative against the slugs.

From that point until the slugs loose sight of them, they are in combat.
The slugs will spray acid at them, follow them up the stairs, and pester them for as long as they remain in sight.

\paragraph{If the PCs throw in some food,}
the nura slugs chase it rather than them.

\paragraph{If the PCs manage to investigate the corpses somehow,}
they find two dead gnomes, one with a torch, the other with a Portal Scroll.

\nuraslug[\npc{\T\N}{20 Nura Slugs}]

\mapentry{The Great Hallway}

\begin{boxtext}

	At the top of the stairs, this massive chamber lies empty, except for the fireflies darting about.
	To the right, there are two short tunnels.
	Ahead of you, two grand staircases lead up into a dim but unwavering light.
	Below you, the entire floor is sticky and greasy.
	Somewhere close, you hear snoring.

\end{boxtext}

If the PCs have indeed been quiet enough in the previous room to not raise an alarm, they find everyone in nearby rooms napping.
A single sound means they will be in serious trouble.

The greasy floor results from a mixture of faeces, drool, blood and leftover mushroom-juice.
Anyone who can use Conjuration magic will be able to turn it into a slipper surface.

\setcounter{enc}{\value{list}}
\addtocounter{enc}{-1}

\paragraph{If the PCs have come from room \arabic{enc},}
they will not immediately see the staircase on their left, but will see it after doing literally anything (fighting, searching, et c.).

\paragraph{If the PCs tarry or talk,}
have them make a Dexterity + Stealth check, TN 9.
They make this as a \textit{Group Roll}, so a single roll counts for the whole group.%
\iftoggle{core}{%
	\footnote{See the core book, page \pageref{grouproll} for group rolls.}
}{}

\mapentry{The Workshop}

\label{laddertop}

\begin{exampletext}

	Frightened gnomes fled their bedrooms from the nura who had rushed through the portal.
	Most were caught, but some managed to run up the stairs, memorizing the sequence perfectly.
	As the last one got to the top of the ladders, he turned to cast an illusion of a solid wall in order to fool intruders.

\end{exampletext}

\begin{boxtext}

	Some goblins and three ogres lie sleeping on the floor between workbenches.
	The place is so full, you can't make out how many lie here, but the snoring indicates more than you can see.
	On the benches, most of the equipment lies broken, but obviously delicate gnomish hands once used these tables to polish gems, craft magical items, and forge digging equipment.
	On one table, you can see a pile of weapons -- short swords and spears -- piled on a table.

\end{boxtext}

Picks, shovels, wood, short swords, shortbows, and all manner of \textit{Adenturing Equipment} lies around the room.
The players can find whatever they need here.

\paragraph{If the PCs have come up from their cells below}
(room 1, page \pageref{entrycell}), they find the exit has been covered by an illusion of a wall.
Upon stepping through it, the illusion fades, revealing the sleeping nura.

\paragraph{If the PCs attempt to take either a short sword or a spear,}
each attempt requires a Dexterity + Stealth roll, TN 7.
Failure will awaken the entire horde.

\goblin[\npc{\N\T}{\arabic{enemyNo} Goblins}]

\addtocounter{enemyNo}{-4}

\ogre[\npc{\N\T}{\arabic{enemyNo} Ogres}]

\mapentry{The Grand Library}

\begin{boxtext}

	At the top of the stairs you find the ruins of a massive library.
	Book cases lie in a smashed heap on the ground, others appear to be used as a makeshift bed for an ogre.
	The books themselves are gone, except for a few scrolls, now tightly clutched by a goblin in a black cowl.
	His hand shines with a thumb-ring, showing three massive gemstones.

\end{boxtext}

In total, the room contains the following magical items:

\magicitem{Bowl of Water}{Conjuration}{Alchemy}{2 Scenes}{Talisman}{4}{3}

This bowl is always full of water.
If it empties, then the air around it quickly forms into more water.

\magicitem{Lock Scroll}{Lock}{Alchemy}{2 Scenes}{Pocket Spell}{4}{3}

This lock scroll will lock any door, increasing the TN to break through it by 4.

\magicitem{Portal Scroll}{Unrestrained, Open Teleport}{Alchemy}{2 Scenes}{Pocket Spell}{4}{5}

As with any Portal Scroll, it leads to the Realm of Shifting Corridors.
Once activated, you should roll three encounters for that area, as usual, and see if anything comes out.

\iftoggle{hardcore}{}{
	\magicitem{Ring of Wishes}{Conjuration 4}{Alchemy}{3 Scenes}{Talisman $\times 3$}{5}{8}
}

This ring is actually three pocket spells -- each ruby in the ring is a separate magical item, and each use will make one ruby go dark.

The caster can summon anything available for Conjuration level 4.

\magicitem{Scroll of Insight}{Mage Sight}{Alchemy}{3 Scenes}{Pocket Spell}{4}{3}

The spell allows the user to feel the entire area (meaning `room'), though not with much detail.

\magicitem{Staff of Light}{Light}{Alchemy}{Continuous}{Talisman}{4}{3}

The gnomish word for `morning', activates the light, while the gnomish word for `night' stops it.
The nura do not know, or have forgotten, these activation words, so they simply keep it under blankets as they find the light irritating.

If wrapped up in heavy sheets and suddenly displayed, the light can blind opponents, as usual.

\goblinnuramancer

\iftoggle{hardcore}{
	\ogre[\npc{\N\T}{Sleeping Ogre}]
}{
	\ogre[\npc{\N\T}{2 Ogres}]
}

\end{multicols}

\section{Upper Levels}

\begin{multicols}{2}

\mapentry{Winding Stairs}

\begin{boxtext}

	As you round the stairs' third turn, you see a massive ogre crouching by a door.
	It blocks the path completely.

\end{boxtext}

If the party have made a reasonable attempt at staying quiet, they can avoid alerting this ogre with a Wits + Stealth roll, TN 9.
Whoever is at the front makes the roll.
If it's unclear who's at the front, the character with the highest Speed + Athletics is in the lead.
With a successful roll,
\iftoggle{hardcore}{
	the ogre is resting, and must take a round to pick up his weapon and gather what's left of his wits, but it will still wake if approached.
}{%
 the party find the ogre sleeping.
}

\ogre[\npc{\N\M}{Rick, the Ogre Guard}]

Rick has only recently been turned into an ogre, and he has eaten his fill of mushrooms, so a Wits + Empathy roll, TN 10, will allow the party to convince him to let them go.
However, if Rick lets them go, he will insist on joining them so he can be free.
While he genuinely wants to escape and become human again,%
\footnote{Nura who have recently turned can change back to their original forms if they are starved for twice as long as they have been nura.}
the moment the party enter battle with other nura his instincts will kick in, and he will turn on the party.%

\mapentry{Second Prison}

This little room once housed a full family of gnomes, but now serves only as another prison.
The prisoners require no locks or handcuffs -- the ogre waiting outside suffices to terrify them into staying put.

\paragraph{If any of the PCs have died,}
a new PC can be found here once another has spent the requisite Story Points to bring someone back.

Otherwise, only a farmer and a wandering noble, unlucky enough to be journeying on horseback at the wrong time, lie in the cells.

\npc{\F}{Anna of the Twisted Glen}
\person{0}% STRENGTH
{-1}% DEXTERITY 
{0}% SPEED
{{2}% INTELLIGENCE
{0}% WITS
{-1}}% CHARISMA
{0}% DR
{1}% COMBAT
{Academics 2, Deceit 3, Empathy 1, Vigilance 2}% SKILLS
{Nothing}% EQUIPMENT
{}

\npc{\M}{Elliot Smith}
\person{2}% STRENGTH
{0}% DEXTERITY 
{0}% SPEED
{{0}% INTELLIGENCE
{-1}% WITS
{1}}% CHARISMA
{0}% DR
{0}% COMBAT
{Beast Ken 1, Crafts 1, Empathy 1}% SKILLS
{Nothing}% EQUIPMENT
{}

\mapentry{Armoury}

\begin{boxtext}

	At the top of the stairway, three dying fireflies wander pointlessly.
	Behind them, you see shadows, with a glimpse of metal, lying in an alcove.
	To the left, a wooden door waits, with dirty footprints visible on the floor.

\end{boxtext}

\begin{exampletext}

	The gnomes once stashed their little weapons here.
	The nura horde have added to it considerably.

\end{exampletext}

\begin{itemize}

	\item{3 buckler shields}
	\item{1 crossbow (unstrung)}
	\item{3 quivers, each with 20 arrows}
	\item{shortbows (also unstrung)}
	\item{3 shortswords}
	\item{7 wood axes}

\end{itemize}

\mapentry{Trapped Hallway}

\begin{boxtext}

	Your lantern illuminates dirty footprints leading out the door, to the left, and down a dark corridor.
	To your right, two statues of ogres stand, like stooping gargoyles, blocking the path.
	All around, the floor glistens with tiny gemstones carved into the centre of flagstones.

\end{boxtext}

The squares between the exit door and the party's path can turn anyone stepping upon them into stone.
One ogre was unfortunate enough to trigger the trap.
The second ogre did not believe the goblins about why the first was there, and ignored the statue, so he was turned to stone too.

This magical trap didn't look quite right when the flagstones had to be inlaid with gems in order to complete the alchemical spell -- that kind of shine gives the game away.
The gnomes' solution was to make sure \textit{all} flagstones had a magical amulet in them, so nobody could tell which might hold a deadly spell.

\paragraph{If anyone steps into the trap,}
they may spend 5 FP to ignore the effects of the trap.
If this happens, explain that their foot or hand begins to harden, then turn to stone for half a second before they retract it.

\paragraph{If the players figure out that the area is trapped and try to jump over it,}
remind them that they cannot see where the trap starts and where it ends.
If they avoid stepping into the trap, they can make a Speed + Athletics roll, TN 9, to jump over the afflicted area.

\mapentry{The Top of the Shaft}

\begin{boxtext}

	Ahead lies to massive double-doors.
	To the right, the hallway extends into a massive room, where you can hear the sound of sawing.

\end{boxtext}

Grank the nuramancer goblin has heard the PCs coming, and has no intention of fighting them alone.
He knows he has the only key to the exit door, so he intends to untie the hell-hounds he has tied up at the room's side before fleeing into the fungal gardens.

By the time the party reach round the corner, he will have disappeared, leaving unchained nura wolves.

\nurawolf[\npc{\A\N\T}{4 Nura Wolves}]

\paragraph{No matter what the party do,}
the wolves go straight for the kill.

\paragraph{If the party attempt to run through the double doors,}
they will suddenly find an empty lift-shaft, unless they have taken the lift to the top themselves.
Have them make a Wits + Athletics check (TN 7) to back off before they fall.

\paragraph{If the party fall into the lift,}
they end up in the mid-section of the warren.

\mapentry{Fungal Gardens}

\begin{boxtext}

	As the doors open, your torchlight shows a crowded mess of fungus.
	Some reach up to your knees, others tower above the light and so far that you cannot see the top.
	The maze of unkempt mushrooms shows random, meandering paths between the taller fungi.

\end{boxtext}

\begin{exampletext}

	This beautiful fungal garden took dripping rain from above, and sieved it through the roof then the soil below, until it distributed nutrients for a forest of mushrooms, big and small.
	The fungal garden was regularly invaded by oozes which can creep into small cracks when young, and grow massive quickly.
	The nura never really kept up with the garden's maintenance, so the room festered with dangerous jellies.

\end{exampletext}

While the place looks serene, it is inhabited by
\iftoggle{hardcore}{%
	dangerous oozes.

	\jelly
	\jelly

}{
	a dangerous ooze.
}

\jelly

\paragraph{Once the players enter the room,}
\iftoggle{hardcore}{%
	the oozes begin to stalk them.
	If multiple oozes chase them, the smaller ones will always back away from the larger ones, so no more than one ooze should follow them at a time (always the largest one).
}{%
	the ooze begins to stalk them.
}

\label{grank}
\npc{\M\N}{Grank}
\person{-2}% STRENGTH
{1}% DEXTERITY 
{4}% SPEED
{{1}% INTELLIGENCE
{2}% WITS
{-4}}% CHARISMA
{0}% DR
{1}% COMBAT
{Projectiles 1, Athletics 2, Medicine 1, Stealth 2, Tactics 2%
\Path{Nura}{Invocation 2, Necromancy 3, Saurecanta 2}}% SKILLS
{Nothing}% EQUIPMENT
{\mana{4}}

\paragraph{If the PCs approach Grank,}
he will hide while casting a \textit{Wide, Raging Fireball}.
Finding him in the darkness before he finishes his spell requires a Wits + Vigilance roll, TN 10.
If he is successful, the party make a group roll against TN 9.
\iftoggle{core}{%
See the core book, page \pageref{edgy} for more on Keeping Edgy to avoid missile attacks.}%
{
Anyone \textit{Keeping Edgy} can add their Speed + Combat to the roll, otherwise, they gain no bonus.
}

\paragraph{If Grank ever feels like his life is under threat of death,}
then he will taunt the PCs with the key to the outside world he has in is position, and throw it into the nearest ooze.
He then lets out a giggle and dashes off into the fungal undergrowth, leaving the players to face the hulking pulsating mass.

\mapentry{The Locked Door}

The locked door has a bronze border and cannot be broken without a Strength + Crafts roll, TN 18.
The exit key, however, fits in nicely.

If the party rest here long enough (perhaps because they have failed to get any key), a raiding party of nura return.
See page \pageref{raidingParty} for the returning raiding party.

\mapentry{The Exit}

\iftoggle{hardcore}{%

This is where the party exit the lower portion of the warren, and enter the upper part.

\paragraph{If the PCs don't have a key,}
they might try tricking the goblin on the other side, who has a key to let people in and out.
See area 1, chapter \ref{upper}.

\paragraph{If the PCs examine the door,}
they can see the goblin in the next area by a glimmer of candle-light.

}{

\begin{boxtext}

	As the key turns, the door swings open, and daylight floods in.
	Green trees cover the road down the hill, and in the far distance, chimney-fires from little hamlets wander into the sky.

	You have finally escaped from the warren.

\end{boxtext}

Whatever the party attempt now, the nura will notice their absence, and start chasing after them, down the hillside, and perhaps into the hamlets.
\iftoggle{core}{See the core book, page \pageref{chases} for specifics on chase scenes.}{}

The party have eight areas to run through before getting to civilization, so this will be a long chase if they decide to run the entire way.

\paragraph{If the party hide,}
have them roll Intelligence + Stealth to find the best spot and best time to hide, TN 8.

\paragraph{If the party fight,}
have them roll Intelligence + Tactics, TN 8, to ambush the nura.
A successful roll means the entire party gain an Initiative Bonus equal to the Tactics Skill being used.

\paragraph{If the nura wolves from before lived,}
then goblins will mount them, and use them to chase after the party, then alert other nura about where they are.
Hiding moves from TN 8 to TN 11 for as long as those wolves live.

\goblin[\npc{\N\T}{12 Scout Goblins}]

}

\end{multicols}

