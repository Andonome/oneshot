\subsection{The Lowest Level}

\newRule{\Glsfmtplural{bandAct}}{

  \Gls{bandAct}
  \glsdesc{bandAct}

  \paragraph{Example 1:}
  The \glspl{pc} all try to get a feel for their surroundings in the darkness of room \vref{entrycell}.
  They make a \roll{Wits}{Caving} roll together.
  If the highest Bonus among the troupe is +3, the second-highest is +2 and the last is +1, then the total is ($3 + \frac{2}{2} + \frac{1}{4} = 5.25$)~`5'.
  So one player rolls the dice, and adds the troupe's +5 Bonus.

  \paragraph{Example 2:}
  The troupe want to search for the key in room \vref{slugHall} (\nameref{slugHall}) as quickly as they can.
  The \gls{gm} decides the roll should be \roll{Speed}{Vigilance}.
  Their Bonuses are +4, +3, +3, +1, +0, and -1.

  The later numbers can be ignored, so the troupe roll with a ($4 + \frac{3}{2} + \frac{3}{4} = 5.8$) +6 Bonus.
}

\iftoggle{hardcore}{}{
  \mapentry[entrycell]{Cells}

  \begin{exampletext}

    Gnomes erected cots, cribs and hammocks here to be used as a communal sleeping area.
    Since then, goblins have placed a bar over the outside of the door to house prisoners.

  \end{exampletext}

  \noindent
  Here the \glspl{pc} awaken to their hopeless situation.

  \begin{itemize}
    \iftoggle{hardcore}{%
      \item
      Ask the players to mark down $1D6$ \glspl{ep} on their character sheets from the trials they've experienced getting here, and of course from their hunger.
    }{}%
    \item
    Give the players a moment to get to know each others' characters and their surroundings.
  \end{itemize}

  \paragraph{If anyone tries to wriggle free of their ropes,}
  have them roll \roll{Dexterity}{Larceny},
  \iftoggle{hardcore}{%
    \tn[11].
    Freeing another character requires a \roll{Dexterity}{Crafts} roll, \tn[7], over the space of a round.
  }{%
    \tn[9].
    Freeing another character requires a full round.
  }

  \paragraph{After a moment,}
  a small voice from the corner of the room says

  \begin{quotation}
    ``No use in struggling.
    They will eat us soon no matter what we do''.
  \end{quotation}

  \noindent
  The \glspl{ogre} captured \gls{sadMan} from a nearby \gls{village}, along with others.
  They ate the others, while \gls{sadMan} hid under debris at the side of the room.
  He's learnt that cowardice means survival quickly, but won't unlearn it as fast.

  \firstPrisoner
  \warrenMapLower

  \paragraph{If anyone asks \gls{sadMan} for help,}
  he refuses, and explains why they are powerless, and may as well wait to be eaten.
  If they say they want to fight, he explains how large \glspl{ogre} stand, and how many wait above.
  If they say that he should untie them so they can die fighting, then he explains that he can't trust them, and that they might throw him out the door first.
  He would rather take his chances hiding in the dark.

  \paragraph{Every time the \glspl{pc} give \gls{sadMan} a reason to hope,}
  have them roll \roll{Charisma}{Empathy} at \iftoggle{oneshot}{\tn[9]}{\tn[11]}.
  Success means that \gls{sadMan} will untie them.
  A tie means he asks a follow-up question (allowing another re-roll if they can answer him well).
  A failure means he disengages, muttering ``no, it won't work''.

  \subsubsection*{Shortly after,}
  Blara the goblin druid, and an ogre walk towards the cell.
  The players should have around 5 actions (or dice-rolls) before trouble arrives.

  \begin{boxtext}
    Heavy footsteps pad down the hall, you hear the door's bar being lifted, and a little goblinoid face peeps in with a torch.
    Behind her, an ogre stoops to the height of a man to avoid the low ceiling.
  \end{boxtext}

  The ogre then tries to pick up a character, and take it up to the kitchen.

  \paragraph{If anyone wants to fight while still tied up,}
  they can do so with a -4 Penalty to the roll.


  \Person{\npc{\N\F}{Blara the Goblin Druid}}%
    {{-2}{2}{2}}% BODY
    {{0}{0}{-4}}% MIND
    {
      \setcounter{Athletics}{1}
      \setcounter{Deceit}{1}
      \setcounter{Stealth}{2}
      \setcounter{Brawl}{1}
      \setcounter{Caving}{2}
      \setcounter{Fire}{2}
      \setcounter{Air}{2}
      \Dagger
    }% SKILLS
    {\snapcaster}% KNACKS
    {torch, human foot\iftoggle{oneshot}{}{, map}}% EQUIPMENT
    {}% ABILITIES

  \showStdSpells

  \ogre[\npc{\N\M}{Olf the Ogre}]

  \paragraph{If anyone attacks Blara,}
  the \gls{ogre} immediately starts to guard her.%
  \exRef{core}{Core Rules}{guarding}

  Blara will attempt to run away, but must wait for the \gls{ogre} to move out of the doorway.

  \iftoggle{oneshot}{
    Chasing after Blara requires \gls{resistedaction} roll of \roll{Speed}{Athletics}
    (the player rolls 2D6 plus their \roll{Speed}{Athletics} against Blara's \roll{Speed}{Athletics}, \tn[10]).%
  }{}%

  \paragraph{If Blara flees,}
  she runs out, taking her torch with her.
  \iftoggle{oneshot}{%
    If the fire has gone out and Blara leaves with the torch, the room becomes pitch-black.
    This gives the \glspl{pc} a bonus to any attack roll equal to their \roll{Wits}{Vigilance} +3 (the \gls{ogre} cannot coordinate well in the dark).
  }{}

  \paragraph{If the party win the fight,}
  he will accompany them out, but his nerves are too shot to be of much use.
  He will not join any fights, but can hold a torch.

  \iftoggle{oneshot}{}{
    \paragraph{If the \glspl{pc} search Blara's body,}
    they find a map of the upper level (see the handouts).
  }

  \newRule{End of \Glsentrytext{interval} Regeneration}{
    \Gls{interval}
    \glsdesc{interval}

    Being underground, the \glspl{pc} will only regenerate 1 \gls{mp} per interval, which goes to whichever \gls{pc} has the most \glspl{mp} lost.

    Of course the \glspl{pc} won't know exactly how long they've spent underground, but they will at least be able to count the number of resting periods they take.
  }
}

\boxPair[t]{
  \goblin[\npc{\N\M}{Hungry Goblin}]
}{
  \morphrat[\npc{\R\A}{Morph Rat}]
}

\mapentry[escapeShaft]{The Chimney}

\begin{exampletext}
  The gnomes tunnelled down this shaft for mining, then made a ladder to easily reach the lower levels.

  More recently, the magical lift (room \ref{lift}) made it redundant, so they've placed a forge in front of the entrance to the workshop (room \ref{workshop}), since this shaft ascends above, as a chimney.
\end{exampletext}

\iftoggle{hardcore}{
  The players should make a \roll{Wits}{Crafts} (\tn[10]) to deduce that a furnace like this must have a wide chimney, and that tunnel may lead elsewhere.
}{
  The \glspl{pc} will probably encounter the ladder, carved into the stone, while feeling around in the dark.
  Perhaps one or more will find themselves at the top of the shaft when Blara enters.

  \paragraph{Anyone at the top of the ladder,}
  will hear an argument between a goblin and \pgls{ogre}.

  \paragraph{Pushing the furnace aside}
  demands a \roll{Strength}{Crafts} roll (\tn[6]), but doing so quietly requires a \roll{Strength}{Stealth} roll (\tn[13]).
  A tie means the horde stop bickering to focus on the noise, while failure indicates that they realize that the forge has a tunnel behind it, and two \glspl{ogre} come out to investigate (and likely try to eat a character).

  \paragraph{If someone wants to continue climbing up,}
  they can't.
  The rest of the ascent has nothing to hold onto, and becomes smaller as it ascends.
}

\iftoggle{hardcore}{
  \mapentry[entrycell]{Cells}
  Half a dozen farmers sit here with gloomy chat.
  Any remaining \gls{pc} allies in the \gls{warren} sit among them.

  None have weapons, but a couple may still have any armour they might normally wear.
}{}

\mapentry[diningRoom]{Dining Room}

Every morsel has been licked clean from this little dining room.
Pots, smashed plates, forks, and candlesticks litter the floor.
A goblin and a giant rat bicker over a human bone.

On the table rest two large knives which can be used as \emph{daggers}.
\iftoggle{oneshot}{%
  \Dagger
  \weaponName s grant \absNum{weaponDamage} Damage in combat, but no Bonus to hit.
}{}

\begin{boxtext}
  Smashed-up chairs and broken cutlery surround a low dining table, lit by a dribbling candle.
  On the table, a little bleach-white goblin wrestles with a rat-like creature, the size of a dog.
  The ugly pair fight over a human leg.
\end{boxtext}

\paragraph{If \pgls{pc} sneaks up quietly,}
have them roll \roll{Dexterity}{Stealth} (\tn[9]).
They should get a -2 Penalty for taking a torch.

A tie means they must retreat, while the goblin investigates their room.
Failure means the goblin and rat attack together.

\iftoggle{oneshot}{
  \paragraph{If the \glspl{pc} fight,}
  they take 1 \gls{ep} for the strenuous activity.
}{}

\mapentry[spellCasters]{Druidical Debates}

Two druids argue about how to use \pgls{deep} Scroll to return home, but can't figure it out.
If they hear the \glspl{pc} coming, they will shout for the goblins in room \vref{kitchen} to go investigate.

\begin{boxtext}
  From the top of the stairs, you see the silhouettes of a goblin, clothed only in leather satchels, holding another goblin on the ground, and yelling at him while pointing to the darkness behind him.
\end{boxtext}

\goblincaster[\npc{\T[2]\F\M\N}{Goblin Druids}]

\iftoggle{oneshot}{
  \paragraph{If the \glspl{pc} fight,}
  give each \pgls{ep}.
}{}

\paragraph{If the characters investigate the table,}
they find a tinder box, an unlit candle, and the \gls{talisman}, `\lootTalisman', left by the goblin druids.

\showTalisman

\labyrinthScroll

\showTalisman

\paragraph{If the \glspl{pc} want to use a scroll,}
they will have to answer its riddle.
Go to \autoref{riddles} and select a riddle which activates that scroll.

\paragraph{If the \textit{\Gls{deep} Scroll} is activated,}
it begins to shimmer with golden flecks, then blocks a section of the hall, turning into a gateway to a deep underground world.
See \nameref{readingScrolls} (\vpageref{readingScrolls}) for details on the calamities which ensue after reading the \Gls{deep} Scrolls.

\mapentry[kitchen]{Kitchen}

\begin{exampletext}
  Goblins with a curious side tried to operate the kitchen, and cook a feast.
  Half-cooked human limbs testify that goblins like their men rare.
\end{exampletext}

\begin{boxtext}
  Torchlight floods out of the twin entrances to the kitchen, illuminating barrels, a cold cooker, a long table covered in the skin-scraps of minced humans, and scattered bones.

  Two goblins lie on top of the cookers, with their fat bellies breathing gently in and out.
\end{boxtext}

More goblins lie sleeping, on the floor and out of sight.

\paragraph{If the \glspl{pc} have avoided making a lot of noise nearby,}
then the goblins are asleep in the kitchen, and they can sneak in with a \roll{Dexterity}{Stealth} roll (\tn[8]).%
\exRef{core}{Core Rules}{sneakattack}

\iftoggle{oneshot}{%
  As before, only one player rolls, and other characters use the same \gls{natural} to produce their own result.
  If the roll fails, all goblins wake up with hungry stomachs.
}{}

\goblin[\npc{\N}{Goblin on the table}]

\paragraph{If the players try to find weapons,}%
\iftoggle{oneshot}{
  they can find plenty of make-shift weapons around the kitchen.

  These weapons may not be good quality, but they can still improve the \glspl{pc}' situation immensely.

  \paragraph{If the players select weapons for their characters,}
  check out the Kitchen Weapons table (\vpageref{weaponsTable} and have them write down the stats on their character sheet one at a time.

 }{
  they can use either of the two cast iron skillets.
}

\paragraph{If the party raid the room for food,}
they'll find a few canteens of water, one of wine, and sacks vegetables (enough for 4 meals).
The table also holds a tinder box and three mushroom-based candles (unlit).

\iftoggle{hardcore}{
  \goblin[\npc{\T[2]\N\M}{\arabic{noAppearing} Goblins on the floor}]
}{
  \goblin[\npc{\N\M}{Goblin on the floor}]
}

\goblin[\npc{\T[2]\N\F}{\arabic{noAppearing} Goblins on the oven}]

\widePic[t]{Roch_Hercka/dragon}

\newRuleWeapons

\mapentry[lift]{The Gnomish Lift}

The gnomes created the lift with the Force Sphere.
It can lift a combined \gls{weight} of 17, and safely descend with a total \gls{weight} of 34 standing on it.%
\iftoggle{oneshot}{\footnote{A creature's \gls{weight} equals its \glspl{hp}, plus half the value of its equipment.}}{\footnote{For full details, see \autopageref{Lift}.}}
If the party step on it with a greater \gls{weight} than this, each additional point inflicts 2~\glspl{ep} on everyone in the lift upon impact with the ground.

The lift responds to magical passwords -- the gnomish words for `farm' (for the top), `work' (for the middle), and `cook' (for the bottom).
The \glspl{ogre} and goblins who use the lift only know the passwords for the bottom and middle sections (Blara extracted this information from a gnome before he died).%
\footnote{One goblin druid extracted the password for the top, but has kept this secret.}

The lift responds to any password spoken while standing on it.

\begin{boxtext}
  The great double doors swing open, revealing a wide, empty room.
  Your torchlight stretches far above, and well out of reach you can see a wooden ceiling.
  The room appears otherwise empty.
\end{boxtext}

\paragraph{If anyone tries to climb the walls,}
they will find less purchase than a Sun-screen salesman in a Scotland.

\paragraph{If the \glspl{pc} dawdle too long here,}
a `raiding party' return with more prisoners to place in the lower cell (room \ref{entrycell}).
See \autopageref{raidingParty} for details on the raiding party.

\mapentry[dragonApproach]{Dragon's Approach}

\begin{exampletext}
  When the horde attacked, one gnome decided to activate \pgls{deep} scroll in order to flee.
  Like the others, it targeted an unknown area, full of powerful magic, deep underground.
  Unlike the others, it held a sleeping dragon, who woke, and sauntered through the magical gateway.
  He stomped on a couple of goblins, then followed the smell of gold and magic, to the treasure room.
  However, the massive monster could not fit through the door.

  When goblins came up the stairs to investigate, the dragon incinerated them with his fiery breath.
  So now he waits, with the legendary patience of a dragon.
  He has no intention of fighting, since he could get hurt, and the goblin horde have no intention of bothering him, so they just pass by that corridor.

  The dragon and the horde have ended in a kind of stalemate.
  And just like the horde, he cannot leave, as he does not know how to read the \Gls{deep} Scrolls\ldots
\end{exampletext}

\begin{boxtext}
  A strange scent wanders down from above, something like a chicken cooked in rotten eggs.
  At the top of the staircase, your flame illuminates three charred goblin corpses on the ground, their charred hands turned upwards like dead spiders.
  The next staircase turns a sharp right, and up into darkness.
\end{boxtext}

\paragraph{If the \glspl{pc} loot the bodies}
they find one \Gls{deep} Scroll wrapped safely in a scroll case.

\newRuleProjectiles

\mapentry[dragon]{The Dragon's Lair}

The dragon will happily talk with anyone who approaches%
\iftoggle{hardcore}{, although he only speaks Gnollish and Elvish.}{
  The dragon's eventual goal is to obtain the rest of the treasure, then leave the \gls{warren}, and go somewhere he can spread his wings.
  Like all dragons, he has more cunning than ferocity, and \emph{plenty} of both.
  He will look for opportunities to turn the \glspl{pc} against each other, and reframe all conversations around the assumption that he will soon leave, with most of the treasure, through \pgls{deep} scroll, and that the \glspl{pc} should view this as their best possible outcome.

  Despite his cunning, he will agree to worse terms if he has to.
}

If the \glspl{pc} have no light, but still ascend, \gls{warrenDragon}'s booming voice with speak to them from \emph{above}.

\paragraph{Revealing \pgls{deep} scroll}
gains no reaction.
\Gls{warrenDragon} knows he has to play it cool, and not let anyone know that he really needs that spell to leave this place.
He \emph{might} fit through the narrow passages up, but he would be in a dangerous situation, being cramped in a narrow corridor, with \glspl{ogre} everywhere.

\paragraph{If the party request he kill goblins for them}
then he agrees to kill any in the current area, but will not journey to another floor.
In return, he wants all the treasure out of the treasure room.

\paragraph{If the party offer to split the treasure,}
he refuses, unless they give him a good reason.

\iftoggle{hardcore}{
  The treasure chinks noisily of course, so the party will then receive a -1 Penalty to all Stealth checks while carrying it.}{}

\paragraph{If the party push for more treasure,}
the dragon asks if they would like to challenge him to a game of riddles.
Each point anyone scores allows them to demand a single item, such as a chest, or quiver.

If they say `yes', then he accepts their challenge, and asks what their riddle is.%
\footnote{Use of the internet is prohibited by trans-dimensional law, common sense, basic decency, and the Geneva Convention.}
If they can think of none,
then the dragon declares that he has won the first round (`best of three!').

\paragraph{The rules for riddles}
are simple -- any question which someone has the knowledge to answer is a fair riddle.
Asking `how many letters in the Greek word for ``mushroom''?', is not a fair riddle, because someone may not know.

Any possible answer to a riddle is `the correct one'.
If someone asks `what is black and white and read all over', anything which fits all descriptions must be accepted as an answer.

See \autoref{riddles} for riddles.

\warrenDragon

\showStdSpells[
  \setcounter{diceNo}{1}
  \input{config/spells/Mind3.tex}
]

\paragraph{If the players ask why he wants treasure,}
he explains that he wants to attract a mate; when his flame becomes hot enough to melt the gold, he will carve a golden statue of the most deadly dragon in his area in order to attract her attention.
He will then decorate the statue with magical items.%
\footnote{Should the ideal mate be pretty, or have the power and aggression to destroy an entire town?
Dragons think the answer is obvious.}

\paragraph{If the dragon parts on good terms,}
he blesses them all, restoring any lost \glspl{fp}.

\iftoggle{oneshot}{
  \paragraph{If the \glspl{pc} attack the dragon,}
  he kills the first to attack.

  His {\scshape \gls{dr} 5} means he reduces all Damage by 5, unless the attacker hits 5 over the \gls{tn} to attack him (a total of 15), achieving a `Vitals Shot'.
}{}

\mapentry[treasureRoom]{Treasure Room}

\begin{itemize}
  \item
  Four wax candles (they burn for 1 hour each) with a tinder box.
  \item
  A chest containing 432 \glspl{cp}.
  \item
  A chest containing 300 \glspl{sp}.
  \item
  An ivory short bow (with string, but no arrows).
  \item
  A small backpack which can hold up to \pgls{weight} of 3.
  \item
  A buckler shield made of pure silver, worth 30~\glspl{sp} (it breaks after one use).
  \item
  Two gem-encrusted shortswords worth 4~\glspl{gp}.
\end{itemize}

\begin{boxtext}
  Through the door, two locked chests lie on the ground.
  Above them, a shortbow and two beautiful short swords stand affixed to the wall, with a quiver of arrows with gemstones used as arrow tips.
\end{boxtext}

