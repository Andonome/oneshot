\subsection{Mid Levels}

\iftoggle{hardcore}{
  \mapentry[goblinSentry]{The Sentry}

  Svart, the goblin, stands at the door to room \ref{nursery} (below), attempting to pick the lock.
  \paragraph{When the \glspl{pc} exit the lift door,}
  he runs over, hoping to beg some food from the \glspl{ogre} he expects to emerge.
  Once he sees the \glspl{pc}, he runs for the stairs, shouting for help.

  \goblin[\npc{\M\N}{Svart}]
}{}

\noteRaidingParty

\mapentry[nursery]{Nursery}

\begin{exampletext}
  When the horde arrived, a gnome locked the children in here with enough food for a few days.
\end{exampletext}

The door is locked, but can be picked with a \roll{Dexterity}{Larceny} roll (\tn[9]) if anyone has some lock-picking tools.
It's far too strong to be broken into by force.

If the \glspl{pc} enter the room, they may think these little gnomes are little goblins and kill them.
If any of them try to do so, have them roll a \roll{Intelligence}{Medicine} (\tn[8]) check to realize their mistake.
Of course, any gnome can identify the children immediately.

\begin{boxtext}
  The sound of crying emanates from the door as you swing it open.
  In this cramped room, a dozen infants with fat little noses lie in a crib with hay, looking up at you in terror.
  The tiny room stinks of shit.
\end{boxtext}

\paragraph{The moment the \glspl{pc} approach the nursery door,}
a goblin falls onto room \ref{lift}'s lift and states the command word to bring it up to the top floor.
This goblin is Grank, and he is the only goblin who knows the password for the top.

\paragraph{If the \glspl{pc} take the children,}
they know to know to keep quiet and follow, but will only keep quiet if the party succeed on an \roll{Intelligence}{Empathy} roll (\tn[10]).

\swarm{Gnomish Children}{5}{-2}{-3}{-5}

There are five children in total, and they each have \pgls{weight} between 1 and 2, but can generally be treated as a single `\gls{swarm}'.

None can walk, so the troupe must carry an additional \gls{weight} of 5, distributed as they please.
To further complicate matters, the gnome-children must go \emph{somewhere}, such as a bag.
Characters might also construct a baby-wrap out of clothes with an \roll{Intelligence}{Crafts} roll (\tn[10]).
Rolling a tie indicates the wrap does not work, while rolling a failure means the wrap works until the character tries to run\ldots

\mapentry[slugHall]{Slug Hall}

\begin{exampletext}
Gnomes grew mushrooms throughout this room in order to grow slugs, which fed fireflies.
While torches work better than fireflies, having omnipresent little lights wandering around the \gls{warren} made sure that people could get about easier.

Once the goblin druids arrived, they cast life-engorging spells to grow the slugs to monstrous proportions, in order to ensure that prisoners don't escape past this point.
Salt covers the stairs, preventing the slugs from moving upwards.
The fireflies continue nipping at them, and have quadrupled their population.
The slugs don't make good guardians, but at least the halls have plenty of light.
\end{exampletext}

\begin{boxtext}
  The doorway reveals a massive hallway of sparkling, floating, gently-buzzing, lights.
  Over to the left, a massive staircase leads up into the darkness.
  And ahead, the little lights gently illuminate giant slugs, feasting on corpses and torn-up books.

  The moment you enter, the slugs' eye-stalks perk up, and they begin to slide off the corpses they were feasting on, and approach\ldots
\end{boxtext}

\paragraph{If the \glspl{pc} run up the stairs,}
have them roll \roll{Speed}{Athletics} (\gls{tn} \arabic{track}) to avoid the acidic spray form the slugs.

From that point until the slugs loose sight of them, they are in combat.
The slugs will spray acid at them, follow them up the stairs, and pester them for as long as they remain in sight.

\morphslug[\npc{\T[9]\A\R}{20 Morph Slugs}]

\paragraph{If the \glspl{pc} throw in some food,}
the morph-slugs head\ldots slowly\ldots towards it rather than fight.

\paragraph{If the \glspl{pc} want to investigate the corpses,}
they find a dead gnome with \pgls{deep} Scroll, and a discarded, half-eaten cloak, with a key to room \ref{nursery} in the pocket.%
\footnote{See \autopageref{saving_the_children} for how the key arrived here.}
\iftoggle{oneshot}{%
  As with the others, the \Gls{deep} Scroll requires a riddle to be answered, takes four rounds before it activates, and vanishes once used.
}{}

\paragraph{If the \glspl{pc} remove all the salt from the stairs,}
the slugs ascend\ldots slowly.
However, the resulting battle will kill three \glspl{ogre}, and all morph-slugs.

\boxPair{
  \label{workshopGoblins}
  \goblin[\npc{\T[\arabic{enemyNo}]\N}{\arabic{enemyNo} Goblins}]

  \addtocounter{enemyNo}{-2}

  \ogre[\npc{\T[\arabic{enemyNo}]\N}{\arabic{enemyNo} Ogres}]
}{
  \pic{Decky/armoury}

}

\mapentry[greatHall]{The Great Hallway}

If the \glspl{pc} have indeed been quiet enough in the previous room to not raise an alarm, they find everyone in the neighbouring rooms napping.
A single sound means they will be in serious trouble.

The greasy floor results from a mixture of faeces, drool, blood and leftover mushroom-juice.

\begin{boxtext}
  At the top of the stairs, this massive chamber lies empty, except for the fireflies darting about, and some human bones on the filthy floor.
  To the right, there two short tunnels reverberate with assorted snoring sounds.
  Ahead of you, two grand staircases lead up into a dim but unwavering light.
  Below you, the entire floor is sticky and greasy.
\end{boxtext}

\paragraph{If the \glspl{pc} have come from room \ref{slugHall},}
they will not see the staircase on their left immediately, but \emph{will} see it after doing literally anything (fighting, searching, et~c.).

\paragraph{If the \glspl{pc} tarry or talk,}
have them roll \roll{Intelligence}{Stealth} (\tn[8]), to get across the sticky floor without squelching too much.
\iftoggle{oneshot}{%
  They make only one \gls{natural}, which counts for the whole group.
  Each margin on the roll allows them an additional round before the horde wakes.
}{}

\iftoggle{hardcore}{
  If a single \emph{player} clearly gives in-character advice to another, the horde awaken.
}{}

\mapentry[workshop]{The Workshop}

Picks, shovels, backpacks, wood, short swords, shortbows, and all manner of crafting and mining equipment litter the room.

\begin{boxtext}
  Some goblins and three \glspl{ogre} lie sleeping on the floor between workbenches.
  The place is so full, you can't make out how many lie here, but the snoring indicates more than you can see.
  On the benches, most of the equipment lies broken, but delicate gnomish hands once used these tables to polish gems, craft magical items, and forge digging equipment.
  On one table, you can see a pile shortswords and spears.
\end{boxtext}

\paragraph{If the \glspl{pc} have come up from room \ref{escapeShaft},}
the reaction depends entirely on how much noise they made while pushing the great forge aside.

\paragraph{If the \glspl{pc} attempt to take either a short sword or a spear,}
each attempt requires a \roll{Dexterity}{Stealth} roll, \tn[7].
Failure will awaken the entire horde, while a tie allows them a good head start.

Find the goblins' stats \vpageref{workshopGoblins}.

\mapentry[grandLibrary]{The Grand Library}

The goblin druid had been investigating \pgls{talisman} -- \lootTalisman.
It remains on the ground beside him.

\showTalisman

\goblincaster

\labyrinthScroll 
The goblin druid holds a \spellName\ in her sleepy hands.

\iftoggle{hardcore}{
  \ogre[\npc{\T[2]\N}{2 Docile Ogres}]
}{
  \ogre[\npc{\F\N}{Sleeping Ogre}]
}

\begin{boxtext}
  At the top of the stairs you find the ruins of a massive library.
  Book cases lie in a smashed heap on the ground, others appear to be used as a makeshift bed for an ogre.
  The books themselves are gone, except for a scroll, now tightly clutched by a goblin in a black cowl.
\end{boxtext}

\paragraph{If any of the \glspl{pc} attempt to sneak in,}
have them roll \roll{Dexterity}{Stealth}, \tn[8].

Failure will, of course, spell disaster, but success will allow them to steal a magical item.
If the player wants to steal multiple magical items, describe them and see how many they decide to take.
Each item taken increases the roll's \gls{tn} by 1, so taking 3 items would mean \pgls{tn} of 11.
The player should not roll again -- the original roll remains, but increasing the \gls{tn} may well turn success into awful failure.

