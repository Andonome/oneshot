\iftoggle{intro}{%
  \subsection{Last Level}

  The troupe emerge by the doorway, but the door has been locked.
  The only key sits in the pocket of a nearby goblin, who hides behind mutated wolve and a garden full of acidic oozes.
}{
  \needspace{12\baselineskip}
  \subsection{Upper Warren}
}

\iftoggle{hardcore}{
  \mapentry[lowerExit]{The Locked Door}

  \begin{exampletext}
    Grank, the goblin druid, keeps this door locked to make sure the prisoners don't escape.
  \end{exampletext}

  \goblincaster[\npc{\M\N}{Grank}\label{grank}]

  \setcounter{diceNo}{1}
  \showStdSpells

  \paragraph{If Grank hears strange noises,}
  (like human speech, or battle) he refuses to open the door, and shouts questions through.

  \paragraph{If the \glspl{pc} simply knock}
  he unlocks the door, and once he sees who they are, flees through the wide door (\vref{inDoor}) to the \nameref{fungusGarden} (\vpageref{fungusGarden}).

}{}

\mapentry[armoury]{Armoury}
\iftoggle{hardcore}{
  \begin{exampletext}
    The gnomes once stashed their little weapons here.
    The horde have added to it considerably.
  \end{exampletext}
}{
  \begin{boxtext}
    Up the stairway, three dying fireflies wander pointlessly.
    At the top, a stand for little weapons.
    To the right, a door with an ornate bronze handle.
  \end{boxtext}
}

The rack holds the following \glspl{weapon}:

\begin{multicols}{2}
\begin{itemize}
  \item
  3 buckler shields
  \item
  1 crossbow (unstrung but usable with an \roll{Intelligence}{Projectiles} roll, \tn[7])
  \item
  3 quivers, each with 20 arrows (for a shortbow)
  \item
  8 crossbow bolts
  \item
  2 shortbows
  \item
  3 shortswords
  \item
  7 wood axes
\end{itemize}
\end{multicols}

\iftoggle{intro}{
  \paragraph{Shields}
  work like any other \gls{weapon}.

  \paragraph{The crossbow}
  (if repaired) deals $1D6+3$ Damage, but requires at least 4 rounds to reload.

  \paragraph{Shortbows}
  Require 1~\glspl{ap} to loose an arrow, and 1~\glspl{ap} to reload.
  However, they deal only $1D6-1$ Damage.

  \begin{boxtable}[XYYc]
  \label{armouryWeapons}

  \textbf{Name} & \textbf{Attack Bonus} & \textbf{Damage Bonus} & \textbf{\gls{weight}} \\\hline

  \showWeapon{\buckler} \\
  \showWeapon{\shortsword} \\
  \showWeapon{\spear} \\
  \showWeapon{\woodaxe} \\

  \end{boxtable}
}{}

\newRule{\Glsfmtlong{dr} \& \Glsfmtplural{vitalShot}}{

  \paragraph{\Gls{armour}}
  \glsdesc{armour}
  \paragraph{\Gls{covering}}
  \glsdesc{covering}
  \paragraph{\Gls{dr}}
  \glsdesc{dr}
  \paragraph{\Glspl{vitalShot}}
  \glsdesc{vitalShot}

}

\mapentry[inDoor]{The In Door}

\iftoggle{hardcore}{}{
  \begin{boxtext}
    The door opens into a wide hallway.
    On the right hand wall, a smaller corridor begins four \glspl{step} ahead.
    On the left hand wall, a small door, six \glspl{step} ahead, barely visible in the shadows.
    On the immediate left, perfectly smooth, man-sized, double-doors stand without any handle, lock or other feature.
  \end{boxtext}
}

The double-doors open with a simple push, then swing shut.
They cannot be opened from the other side.
\footnote{The gnomes kept knocking into each other at the front door, so they made the double-doors an `in door', and the other an `out door'.}

\paragraph{Opening either door from the wrong side}
requires a \roll{Strength}{Larceny} roll, \tn[12].

\mapentry[outDoor]{The Out Door}

This little door opens the opposite way from the double doors in area \ref{inDoor}, and also demands a roll of \roll{Strength}{Larceny} (\tn[12]) to open from the wrong side.

\Glspl{ogre} cannot run through this door -- they must stop, and squeeze.
\iftoggle{intro}{%
  Anyone waiting on the other side can roll \roll{Intelligence}{Melee} against the ogre's \roll{Speed}{Melee}; success means the character can make \pgls{vitalShot} against the giant head squirming through the door.%
  \exRef{core}{Core Rules}{sneakattack}}{%
  It's a good place for \pgls{ambush}.
}

\iftoggle{hardcore}{}{
  \warrenMapUpper
}

\mapentry[topShaft]{The Top of the Shaft}

\begin{boxtext}
  Ahead of you stand the lift's double-doors.
  To the far right eyes of monstrous wolves blazing with feral intensity stare at you, straining against the taut ropes that bind them.
  The metal ring attached to the wall groans under the tension, as if it too feels the weight of the beasts' fury.

  Behind the wolves you spot a goblin.
  Its torch casts a glow on its face, illuminating a mischievous grin.
  The creature's eyes sparkle with malevolent glee as it slowly raises the flame higher, watching with rapt attention as the ropes begin to fray under the strain.

  The wolves' restraints rip off in a whipping snap as the beasts burst free from their bonds, and rush straight to you.
\end{boxtext}

Grank the goblin druid has heard the \glspl{pc} coming, and has no intention of fighting them alone.
He knows he has the only key to the exit door (room \ref{lowerExit}), so he intends to loose the morph-wolves he has tied up at the room's side before fleeing into the fungal gardens.

\widePic[t]{Roch_Hercka/garden}

\iftoggle{hardcore}{}{%
  \goblincaster[\npc{\M\N}{Grank}\label{grank}]

  \setcounter{diceNo}{1}
  \showStdSpells
}

\paragraph{If the \glspl{pc} run through the lift's double doors,}
they will find a sudden drop.
The players roll \roll{Wits}{Athletics} (\tn[7]) to have the \glspl{pc} catch themselves before falling.

\paragraph{Falling to the mid-way point}
inflicts $1D6+1$ Damage, plus the character's Strength Bonus.

\paragraph{Falling to the bottom}
inflicts $2D6+2$ Damage, plus the character's Strength Bonus.%

\paragraph{If the \glspl{pc} abandoned \gls{kalama} and the children}
(room \vref{nursery}),
consider what he might have done in the mean-time.
Perhaps he could use \pgls{talisman} to engineer a distraction and get up the lift.
Or perhaps \pgls{ogre} at him.

\morphwolf[\npc{\T[4]\A\R}{\arabic{noAppearing} Morph Wolves}]

\mapentry[fungusGarden]{Fungal Gardens}

\begin{boxtext}
  A gentle green glow in the distance silhouettes ten thousand mushrooms of every size.
  Some as big as a barn and as wide as a cart.
  Others only as thick as a man's neck, but just as tall.%
  \footnote{This beautiful fungal garden took dripping rain from above, and sieved it through the roof then the soil below, until it distributed nutrients for a forest of mushrooms, big and small.}
\end{boxtext}

\paragraph{Once the players enter the room,}
an ooze begin to stalk them.%
\footnote{%
  During \ifnum\value{cycle}=4\gls{cThree}\else\gls{cFive}\fi, acidic oozes crept in through tiny cracks, so the gnomes had to kill the little pests before they grew too large.
  The goblins haven't kept up with the maintenance, so the semi-sentient gelatinous \glspl{swarm} quickly grew massive, so the goblins simply closed the door and left them alone.
}
\iftoggle{hardcore}{%
  If multiple oozes chase them, the smaller ones will always back away from the larger ones, so no more than one ooze should follow them at a time (always the largest one).
}{}

\paragraph{Investigating the green glows}
reveals little patches of shiny-green \glspl{glowshroom}%
\iftoggle{intro}{(find them in the glossary, \vpageref{fenestraGlosPreamble})}{}.

\paragraph{If the \glspl{pc} approach Grank,}
he will hide while casting spells.

\paragraph{If Grank ever feels like his life is under threat,}
then he will taunt the \glspl{pc} with the key to the outside world he has in is position, and throw it into the nearest ooze.
He then lets out a giggle and dashes off into the fungal undergrowth, leaving the players to face the hulking, pulsating, mass.

\iftoggle{intro}{}{
  \jelly
}

\iftoggle{hardcore}{
  \jelly
}{}

\jelly

\iftoggle{hardcore}{}{
  \mapentry[lowerExit]{The Exit}

  \iftoggle{intro}{%
    \begin{boxtext}[]
      As the key turns, the door swings open, and daylight floods in.
      Green trees cover the road down the hill, and in the far distance, chimney-fires from little hamlets wander into the sky.

      You have finally escaped from the \gls{warren}.
    \end{boxtext}

    \subsection{Rewind}

    Remember to congratulate your players on a tough journey.
    Give them a moment to breathe.
    Summarize any clever plans or unexpected outcomes that happened.

    And if you've just finished reading the module for the first time, remember to come back later, and give it a second read-through before running it.%
    \footnote{Did you spot the key hiding in an image?}

    \pic{Decky/screech}
  }{
    This is where the party exit the lower portion of the \gls{warren}, and enter the upper.

    \paragraph{If the \glspl{pc} don't have a key,}
    they might try tricking the goblin on the other side, who has a key to let people in and out.
    See room \vref{downstairs}.

    \paragraph{If the \glspl{pc} examine the door,}
    they can see the goblin in the next area by a glimmer of candle-light.

  }

}
